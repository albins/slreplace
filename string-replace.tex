\documentclass{llncs}

\usepackage{latexsym}
\usepackage{setspace}
\usepackage{cancel}
\usepackage{listings}
\usepackage{graphicx}
\usepackage{appendix}
\usepackage{amssymb}
\usepackage{stmaryrd}
\usepackage{amsmath}
\usepackage{leftidx}
\usepackage{mathtools}
\usepackage[linesnumbered,noend]{algorithm2e}
\usepackage{paralist}
\usepackage{color}
\usepackage{mathrsfs}
\usepackage{tikz}
\usetikzlibrary{shapes}
%\usepackage{times}

%!TEX root = popl2018.tex

\newcommand{\set}[1]{\{ #1 \}}
\newcommand{\sequence}[2]{(#1, \ldots, #2)}
\newcommand{\couple}[2]{(#1,#2)}
\newcommand{\pair}[2]{(#1,#2)}
\newcommand{\triple}[3]{(#1,#2,#3)}
\newcommand{\quadruple}[4]{(#1,#2,#3,#4)}
\newcommand{\tuple}[2]{(#1,\ldots,#2)}
\newcommand{\Nat}{\ensuremath{\mathbb{N}}}
\newcommand{\Rat}{\ensuremath{\mathbb{Q}}}
\newcommand{\Rea}{\ensuremath{\mathbb{R}}}
\newcommand{\Zed}{\ensuremath{\mathbb{Z}}}
%\newcommand{\true}{\top}
%\newcommand{\false}{\perp}
\newcommand{\bottom}{\perp}
%% \newcommand{\powerset}[1]{{\cal P}(#1)}
\newcommand{\npowerset}[2]{{\cal P}^{#1}(#2)}
\newcommand{\finitepowerset}[1]{{\cal P}_f(#1)}
\newcommand{\level}[2]{L_{#1}(#2)}
\newcommand{\card}[1]{\mbox{card}(#1)}
\newcommand{\range}[1]{\mathtt{ran}(#1)}
\newcommand{\astring}{s}

\newcommand{\Cc}{\mathcal{C}}


\newcommand {\notof}{\ensuremath{\neg}}
\newcommand {\myand}{\ensuremath{\wedge}}
\newcommand {\myor}{\ensuremath{\vee}}
\newcommand {\mynext}{\mbox{{\sf X}}}
\newcommand {\until}{\mbox{{\sf U}}}
\newcommand {\sometimes}{\mbox{{\sf F}}}
\newcommand {\previous}{\mynext^{-1}}
\newcommand {\since}{\mbox{{\sf S}}}
\newcommand {\fminusone}{\mbox{{\sf F}}^{-1}}
\newcommand {\everywhere}[1]{\mbox{{\sf Everywhere}}(#1)}



\newcommand{\aatomic}{{\rm A}}
\newcommand{\aset}{X}
\newcommand{\asetbis}{Y}
\newcommand{\asetter}{Z}

\newcommand{\avarprop}{p}
\newcommand{\avarpropbis}{q}
\newcommand{\avarpropter}{r}
\newcommand{\varprop}{{\rm PROP}} % Set of atomic propositions (for a given logic)

% formulae

\newcommand{\aformula}{\astateformula} % a formula
\newcommand{\aformulabis}{\astateformulabis} % another formula (when at least 2 are present)
\newcommand{\aformulater}{\astateformulater} % another formula (when at least 3 are present)
\newcommand{\asetformulae}{X}
\newcommand{\subf}[1]{sub(#1)}

\newcommand{\aautomaton}{{\mathbb A}}
\newcommand{\aautomatonbis}{{\mathbb B}}

\newcommand {\length}[1] {\ensuremath{|#1|}}



% Equivalences
\newcommand{\egdef}{\stackrel{\mbox{\begin{tiny}def\end{tiny}}}{=}} % =def=
\newcommand{\eqdef}{\stackrel{\mbox{\begin{tiny}def\end{tiny}}}{=}} % =def=
\newcommand{\equivdef}{\stackrel{\mbox{\begin{tiny}def\end{tiny}}}{\equivaut}} % <=def=>
\newcommand{\equivaut}{\;\Leftrightarrow\;}

\newcommand{\ainfword}{\sigma}

\newcommand{\amap}{\mathfrak{f}}
\newcommand{\amapbis}{\mathfrak{g}}

\newcommand{\step}[1]{\xrightarrow{\!\!#1\!\!}}
\newcommand{\backstep}[1]{\xleftarrow{\!\!#1\!\!}}

\newcommand {\aedge}[1] {\ensuremath{\stackrel{#1}{\longrightarrow}}}
\newcommand {\aedgeprime}[1] {\ensuremath{\stackrel{#1}{\longrightarrow'}}}
\newcommand {\afrac}[1] {\ensuremath{\mathit{frac}(#1)}}
\newcommand {\cl}[1] {\ensuremath{\mathit{cl}(#1)}}
\newcommand {\sfc}[1] {\ensuremath{\mathit{sfc}(#1)}}
\newcommand {\dunion} {\ensuremath{\uplus}}
\newcommand {\edge} {\ensuremath{\longrightarrow}}
\newcommand {\emptyword}{\ensuremath{\epsilon}}
\newcommand {\floor}[1] {\ensuremath{\lfloor #1 \rfloor}}
\newcommand {\intersection} {\ensuremath{\cap}}
\newcommand {\union} {\ensuremath{\cup}}
\newcommand {\vals}[2] {\ensuremath{\mathit{val}_{#2}(#1)}}



\newcommand {\pspace} {\textsc{pspace}}
\newcommand {\nlogspace} {\textsc{nlogspace}}
\newcommand {\logspace} {\textsc{logspace}}
\newcommand {\expspace} {\textsc{expspace}}
\newcommand {\np} {\textsc{np}}
\newcommand {\threeexptime} {\textsc{3exptime}}
\newcommand {\polytime} {\textsc{p}}
\newcommand{\twoexpspace}{\textsc{2expspace}}
\newcommand{\threeexpspace}{\textsc{3expspace}}
\newcommand {\nexptime} {\textsc{nexptime}}



\newcommand{\aalphabet}{\Sigma}     % an alphabet, A is already used for atoms
\newcommand{\aword}{\mathfrak{u}}
\newcommand{\awordbis}{\mathfrak{v}}



\newcommand{\aassertion}{P}
\newcommand{\aassertionbis}{Q}
\newcommand{\aexpression}{e}
\newcommand{\aexpressionbis}{f}
\newcommand{\avariable}{\mathtt{x}}
\newcommand{\uniquevar}{\mathtt{u}}
\newcommand{\uniquevarbis}{\mathtt{v}}
\newcommand{\avariablebis}{\mathtt{y}}
\newcommand{\avariableter}{\mathtt{z}}
\newcommand{\nullconstant}{\mathtt{null}}
\newcommand{\nilvalue}{nil}
\newcommand{\emptyconstant}{\mathtt{emp}}
\newcommand{\infheap}{\mathtt{inf}}
\newcommand{\saturated}{\mathtt{Saturated}}

\newcommand{\astateformula}{\phi}
\newcommand{\astateformulabis}{\psi}
\newcommand{\astateformulater}{\varphi}
%%
\newcommand{\separate}{\ast}
\newcommand{\sep}{\separate}
\newcommand{\size}{\mathtt{size}}
\newcommand{\sizeeq}[1]{\mathtt{size} \ = \ #1}
\newcommand{\alloc}[1]{\mathtt{alloc}(#1)}
\newcommand{\allocb}[2]{\mathtt{alloc}^{-1}[#2](#1)}
\newcommand{\isol}[1]{\mathtt{isoloc}(#1)}
\newcommand{\icell}{\mathtt{isocell}}
\newcommand{\malloc}{\mathtt{malloc}}
\newcommand{\cons}{\mathtt{cons}}
\newcommand{\new}{\mathtt{new}}
\newcommand{\free}[1]{\mathtt{free} \ #1}
\newcommand{\maxform}[1]{\mathtt{maxForms}(#1)}
\newcommand{\locations}[1]{\mathtt{loc}(#1)}
\newcommand{\values}{\mathtt{Val}}
\newcommand{\aheap}{\mathfrak{h}}
\newcommand{\avaluation}{\mathfrak{V}}
\newcommand{\heaps}{\mathcal{H}}
\newcommand{\astore}{\mathfrak{s}}
\newcommand{\stores}{\mathcal{S}}
\newcommand{\amodel}{\mathfrak{M}}
\newcommand{\alabel}{\ell}

\newcommand{\aprogram}{\mathtt{PROG}}
\newcommand{\programs}{\mathtt{P}}
\newcommand{\ctprograms}{\programs^{ct}}
\newcommand{\aninstruction}{\mathtt{instr}}
\newcommand{\ainstruction}{\mathtt{instr}}
\newcommand{\instructions}{\mathtt{I}}
\newcommand{\aguard}{\ensuremath{g}}
\newcommand{\guards}{\ensuremath{G}}
\newcommand{\domain}[1]{\mathtt{dom}(#1)}
\newcommand{\memory}{\stores\times\heaps}
\newcommand{\skipinstruction}{\mathtt{skip}}

\newcommand{\execution}{\mathtt{comp}}
\newcommand{\aux}{\mathtt{embd}}
\newcommand{\runof}{run}
\newcommand{\anexecution}{e}


\newcommand{\aletter}{\ensuremath{a}}
\newcommand{\aletterbis}{\ensuremath{b}}
\newcommand{\alocation}{\mathfrak{l}}

\newcommand{\pointsl}[1]{\stackrel{#1}{\hookrightarrow}}
\newcommand{\ppointsl}[1]{\stackrel{#1}{\mapsto}}
\newcommand{\ourhook}[1]{\stackrel{#1}{\hookrightarrow}}
\newcommand{\ltrue}{{\sf true}}
\newcommand{\lfalse}{{\sf false}}


\newcommand{\variables}{\mathtt{FVAR}}
\newcommand{\pvariables}{\mathtt{PVAR}}
\newcommand{\secvariables}{\mathtt{SVAR}}
\newcommand{\logique}[1]{\mathtt{FO}(#1)}



\newcommand{\atranslation}{\mathfrak{t}}
\newcommand{\nbpred}[1]{\widetilde{\sharp #1}}
\newcommand{\nbpredstar}[1]{\widetilde{\sharp #1}^{\star}}
\newcommand{\isolated}{\mathtt{isol}}
\newcommand{\stdmarks}{\mathtt{envir}}
\newcommand{\relation}[1]{\mathtt{relation}_{#1}}
\newcommand{\freevar}{\mathtt{FV}}
\newcommand{\notonmark}{\mathtt{notonenv}}
\newcommand{\InVal}[1]{\mathtt{InVal}\!\left(#1\right)}
\newcommand{\NotOnEnv}[1]{\mathtt{NotOnEnv}\!\left(#1\right)}
\newcommand{\PartOfVal}[1]{\mathtt{PartOfVal}\!\left(#1\right)}
%\newcommand{\nbpreds}[3]{\sharp #1 \geq #2}
\newcommand{\defstyle}[1]{{\emph{#1}}}

\newcommand{\cut}[1]{}
\newcommand{\interval}[2]{[#1,#2]}
\newcommand{\buniquevar}{\overline{\uniquevar}}
\newcommand{\bbuniquevar}{\overline{\overline{\uniquevar}}}
\newcommand{\magicwand}{\mathop{\mbox{$\mbox{$-~$}\!\!\!\!\ast$}}}
\newcommand{\wand}{\magicwand}
\newcommand{\septraction}{\stackrel{\hsize0pt \vbox to0pt{\vss\hbox to0pt{\hss\raisebox{-6pt}{\footnotesize$\lnot$}\hss}\vss}}{\magicwand}}
%% \newcommand{\reach}{\mathtt{reach}}
\mathchardef\mhyphen="2D % hyphen while in math mode

\newcommand{\adataword}{\mathfrak{dw}}
\newcommand{\adatum}{\mathfrak{d}}

\newcommand{\collectionknives}{\mathtt{ks}}
\newcommand{\collectionknivesfork}[1]{\mathtt{ksfs}_{=#1}}
\newcommand{\collectionknivesforks}{\mathtt{ksfs}}
\newcommand{\collectionkniveslargeforks}{\mathtt{kslfs}}


\newcommand{\acounter}{\mathtt{C}}

\newcommand{\fotwo}[3]{{\mbox{FO2}_{#1,#2}(#3)}}
\newcommand{\mtrans}[1]{t\!\left(#1\right)^{\Box}}
\newcommand{\mbtrans}[2]{\mtrans{#2}_{#1}}


\newcommand{\alogic}{\mathfrak{L}}


\newcommand{\semantics}[1]{\ensuremath{[ #1 ]}}


\newcommand{\adomino}{\mathfrak{d}}
\newcommand{\atile}{\mathfrak{d}}
\newcommand{\atiling}{\mathfrak{t}}

\newcommand{\hori}{\mathtt{h}}
\newcommand{\verti}{\mathtt{v}}
\newcommand{\domi}{\mathtt{d}}

\newcommand{\cpyrel}{\mathfrak{cp}}

\newcommand{\cntcmp}{\mathfrak{C}}

\newcommand{\heapdag}{\mathfrak{G}}

\newcommand{\onmainpath}{\mathtt{mp}}

\newcommand{\tree}{\mathtt{tree}}

%\newcommand{\tile}{\mathtt{tile}}

\newcommand{\type}{\mathtt{type}}

\newcommand{\ptype}{\mathtt{ptype}}

\newcommand{\exttype}{\mathtt{exttype}}

\newcommand{\anctypes}{\mathtt{AncTypes}}

\newcommand{\destypes}{\mathtt{DesTypes}}

\newcommand{\inctypes}{\mathtt{IncTypes}}

\newcommand{\treeic}{\mathtt{treeIC}}

\newcommand{\trs}{\mathfrak{trs}}


\newcommand{\nin}{\not \in}
\newcommand{\cupplus}{\uplus}
\newcommand{\aunarypred}{\mathtt{P}}


\newcommand{\hide}[1]{}

\newcommand{\eval}[2]{\llbracket#1\rrbracket_{#2}}
\newcommand\cur{\mathsf{cur}}
\newcommand\dom{\mathsf{dom}}
\newcommand\rng{\mathsf{rng}}

\newcommand\dd{\mathbb{D}}
\newcommand\nat{\mathbb{N}}


\newcommand\cA{\mathcal{A}}
\newcommand\cB{\mathcal{B}}
\newcommand\cC{\mathcal{C}}
\newcommand\cE{\mathcal{E}}
\newcommand\cG{\mathcal{G}}
\newcommand\Ll{\mathcal{L}}
\newcommand\cM{\mathcal{M}}
\newcommand\cP{\mathcal{P}}
\newcommand\cR{\mathcal{R}}
\newcommand\cS{\mathcal{S}}
\newcommand\cT{\mathcal{T}}

\newcommand\vard{\mathfrak{d}}

\newcommand\replaceall{\mathsf{replaceAll}}
\newcommand\indexof{\mathsf{IndexOf}}


\newcommand\strline{\mathsf{SL}}

\newcommand\pstrline{\mathsf{SL_{pure}}}

\newcommand\search{\mathsf{search}}

\newcommand\verify{\mathsf{verify}}

\newcommand\searchleft{\mathsf{searchLeft}}

\newcommand\searchlong{\mathsf{searchLong}}


\newcommand\pref{\mathsf{Pref}}

\newcommand\wprof{\mathsf{WP}}

\newcommand\vars{\mathsf{Vars}}

\newcommand\dep{\mathsf{Dep}}
\newcommand\ptn{\mathsf{Ptn}}

\newcommand\src{\mathsf{src}}
\newcommand\strtorep{\mathsf{strToRep}}

\newcommand\rpleft{\mathsf{l}}
\newcommand\rpright{\mathsf{r}}


\newcommand\srcnd{\mathsf{srcND}}

\newcommand\ctxt{\mathsf{ctxt}}


\newcommand\ctxts{\mathsf{Ctxts}}

\newcommand\sprt{\mathsf{sprt}}

\newcommand\val{\mathsf{val}}

\newcommand\srclen{\mathsf{srcLen}}

\newcommand\rpleftlen{\mathsf{lLen}}


\newcommand\dfs{\mathsf{DFS}}

\newcommand\repr{\mathsf{rep}}

\newcommand\red{\mathsf{red}}

\newcommand\gfun{\mathcal{F}}


\newcommand{\leftmost}{{\sf leftmost}}
\newcommand{\longest}{{\sf longest}}

\newcommand{\arbidx}{{\sf Idx_{arb}}}
\newcommand{\dmdidx}{{\sf Idx_{dmd}}}
\newcommand{\lftlen}{{\sf Len_{lft}}}

%\newtheorem{remark}[theorem]{Remark}



\newcommand{\yan}[1]{\color{green} {YA: #1 :AY} \color{black}}
\newcommand{\zhilin}[1]{\color{cyan} {ZL: #1 :LZ} \color{black}}
\newcommand{\tl}[1]{\color{blue} {TL: #1 :LT} \color{black}}

\title{Solving String Constraints with\\
 Complex Functions}
\titlerunning{}
\author{}
\institute{}


\author{Taolue Chen, Yan Chen, Anthony W. Lin, Zhilin Wu}
\begin{document}

\maketitle


\begin{abstract}
Our goal in this note is to investigate whether we can find a fragment of string constraints which, on the one hand, includes the complex string manipulating functions $\replaceall(x, u, y)$ (where $u$ is a string, and $x,y$ are string variables), and on the other hand, enjoys a sound and complete decision procedure.
\end{abstract}

\newcommand{\concat} {\circ}

\newcommand{\replace} {{\sf replace}}

\newcommand{\str} {{\sf Str}}

\newcommand{\intnum} {{\sf Int}}

\newcommand{\regexp} {{\sf RegExp}}

\newcommand{\strarr} {{\sf StringArray}}

\newcommand{\dtypes} {{\sf DataTypes}}

\newcommand{\anarr} {{\mathbb{A}}}

\section{Introduction}

\cite{LB16,TCJ16,YABI14}

\section{Killer examples}

%https://github.com/kimmobrunfeldt/progressbar.js/blob/master/src/utils.js

\begin{verbatim}

// Renders templates with given variables. Variables must be surrounded with
// braces without any spaces, e.g. {variable}
// All instances of variable placeholders will be replaced with given content
// Example:
// render('Hello, {message}!', {message: 'world'})
function render(template, vars) {
    var rendered = template;

    for (var key in vars) {
        if (vars.hasOwnProperty(key)) {
            var val = vars[key];
            var regExpString = '\\{' + key + '\\}';
            var regExp = new RegExp(regExpString, 'g');

            rendered = rendered.replace(regExp, val);
        }
    }

    return rendered;
}
\end{verbatim}

\section{Preliminaries}

For $k \in \Nat$, let $[k] = \{1,\dots, k\}$.

For a vector $\vec{x}=(x_1,\dots, x_n)$, let $|\vec{x}|$ denote $n$ and for each $i \in [n]$, let $\vec{x}(i)$ denote $x_i$.

An alphabet $\Sigma$ is fixed.

We consider the following data types: String data type $\str$, Integer data type $\intnum$, and Array of strings $\str [\ ]$.


We assume a countably set of variables, of data types form $\dtypes$. We will use $x, y, z, \dots$ to denote the variables of data type $\str$, and $n, n', \dots$ to denote the variables of type $\intnum$. In addition, we use $X, Y, Z, \dots$ to denote the variables of data type $\str[\ ]$.
We use $u, v, w, \dots$ to denote the constant strings, and $c, c',\dots$ to denote the constant integers.


For a string $u \in \Sigma^+$, let $|u|$ denote the length of $u$, in addition, for $i \in [|u|]$, let $u[i]$ the $i$-th letter of $u$.

\begin{definition}[Regular expressions $\regexp$]
\[e \eqdef a \mid e + e \mid e \concat e \mid e^\ast, \mbox{ where } a \in \Sigma. \]
We also use the abbreviations $\Sigma \equiv \cup_{a \in \Sigma}\ a$ and $\Sigma^\ast \equiv (\cup_{a \in \Sigma}\ a)^\ast$.
\end{definition}

%%%%%%%%%%%%%%%%%%%%%%%%%%%%%%%%%%%%%%%%%%%%
\hide{
A regular expression $e$ is said to be \emph{bounded} if $e$ is defined by the rules, $e \eqdef a \mid u^\ast \mid e \concat e$, where $u$ is a constant string. For instance, $a (bc)^\ast$ is bounded, while $(ab^\ast)^\ast$ is not. A regular expression $e$ is a union of bounded regular expressions if $e = e_1 + \dots + e_k$ such that each $e_i$ is a bounded regular expression. We use UBR to abbreviate the set of regular expressions which are a union of bounded regular expressions.
}
%%%%%%%%%%%%%%%%%%%%%%%%%%%%%%%%%%%%%%%%%%%%

A nondeterministic finite automata (NFA) $\cA$ on $\Sigma$ is a tuple $(Q, \delta, q_0, F)$, where $Q$ is a finite set of states, $q_0 \in Q$ is the initial state, $F \subseteq Q$ is the set of final states, and $\delta \subseteq Q \times \Sigma \times Q$ is the transition relation. An NFA $\cA$ is deterministic if for each $(q, \sigma) \in Q \times \Sigma$, there is at most one $q' \in Q$ such that $(q, a, q') \in \delta$. An NFA $\cA$ is complete if for each $(q, \sigma) \in Q \times \Sigma$, there is at least one $q' \in Q$ such that $(q, a, q') \in \delta$. We assume that all NFA considered in this paper are complete. For a string $w = \sigma_1 \dots \sigma_n$, a run of $\cA$ on $w$ is a sequence $q_0 \dots q_n$ such that for each $i \in [n]$, $(q_{i-1}, \sigma_i, q_i) \in \delta$. A run $q_0 \dots q_n$ is accepting if $q_n \in F$. A string $w$ is accepted by $\cA$ if there is an accepting run of $\cA$ on $w$. An NFA $\cA$ is \emph{unambiguous} if for each word $w$, there is \emph{at most one accepting} run of $\cA$ on $w$.
We also use the notation $q_1 \xrightarrow[\cA]{w} q_{n+1}$ to denote the fact that there are $q_2,\dots, q_n \in Q$ such that for each $i \in [n]$, $(q_i, \sigma_i, q_{i+1}) \in \delta$.  


\section{Semantics of the function $\replaceall$}

The semantics of $\replaceall(x, u, y)$ is defined inductively as follows: let $u = u_1 \dots u_k$ (where $k \ge 1$),
\begin{itemize}
\item if $x \not \in \Sigma^\ast u \Sigma^\ast$, then $\replaceall(x, u, y) = x$, 
%
\item otherwise, let $x = x' u z$ such that $x' u_1 \dots u_{k-1} \not \in \Sigma^\ast u \Sigma^\ast$, then $\replaceall(x, u, y) = x' v \cdot \replaceall(z, u, y)$.
\end{itemize}
For instance, $\replaceall(aaaaa, aaa, y) = yaa$, and $\replaceall("Jeve", e, a) = Java$.

\tl{Anthony mentioned that there might be alternatives, for instance,  $\replaceall$ can be made nondeterministic ($\replaceall(aaaaa, aaa, y) = \{yaa, aya, aay\}$). This MIGHT be considered as well.}

%%%%%%%%%%%%%%%%%%%%%%%%%%%%%%%%%%%%%%%%%%%%%%
%%%%%%%%%%%%%%%%%%%%%%%%%%%%%%%%%%%%%%%%%%%%%%
\hide{
\smallskip

\noindent The semantics of $split(x, u)$ is defined inductively as follows:
\begin{itemize}
\item if $x \not \in \Sigma^\ast u \Sigma^\ast$, then $split(x,u) = x$, 
%
\item otherwise, let $x = y u z$ such that $y \not \in \Sigma^\ast u \Sigma^\ast$, then $split(x, u) = concat([y], split(z, u))$, where $[y]$ is the array comprising one element $y$ and $concat$ operation concatenates two arrays into one.  
\end{itemize} 

For instance, $split("Java", a) = [``J", ``v", ``"]$.

\smallskip

\noindent The semantics of $join(X)$ is defined inductively as follows: 
\begin{itemize}
\item if $X$ is the empty array, then $join(X) = ``''$,  
\item otherwise, let $X= concat([x], Y)$, then $join(X) = x\ \cdot\ ``," \ \cdot\ join(Y)$.
\end{itemize}
For instance, $join([``Tom", ``John", ``Henry"]) =``Tom,John,Henry"$.
}
%%%%%%%%%%%%%%%%%%%%%%%%%%%%%%%%%%%%%%%%%%%%%%
%%%%%%%%%%%%%%%%%%%%%%%%%%%%%%%%%%%%%%%%%%%%%%

\section{Straight-line string constraint with $\replaceall$  function}


\begin{definition}[Relational and regular constraints with $\replaceall$ function]
Relational constraints and regular constraints are defined by the following rules,
\[
\begin{array}{r c l cr}
s &\eqdef & x \mid u & \ \ & \mbox{(string terms)}\\
%t &\eqdef & s \mid e & \ \ & \mbox{(terms)}\\
\varphi &\eqdef & y = s_1 \concat \dots \concat s_n (n \in \Nat) \mid  x = \replaceall(y, u, s) \mid \varphi \wedge \varphi & \ \ & \mbox{(relational constraints)}\\
\psi & \eqdef & x \in e \mid \psi \wedge \psi \mid \psi \vee \psi \mid \neg \psi   & \ \ & \mbox{(regular constraints)} \\
\end{array}
\]
where $u \in \Sigma^\ast$ and $e$ is a regular expression. 
%A regular constraint $\psi$ is a UBR constraint if for each atom $x \in e$ occurring in $\psi$, $e$ is in UBR.
\end{definition}
For a formula $\varphi$, let $\vars(\varphi)$ denote the set of variables occurring in $\varphi$. Similarly for $\vars(\psi)$.

\tl{can the regular constraints be simplified to just a conjunction of $x\in e$?}

Given a relational constraint $\varphi$, a variable $x$ (of type $\str$) is called a \emph{source variable} of $\varphi$ if $\varphi$ \emph{does not} contain a conjunct of the form $x = s_1 \concat \dots \concat s_n$ or $x = \replaceall(\dots)$.

\tl{note that here the pattern $u$ is a constant, but we may also introduce a variable here, though its decidability is not clear.} 

%%%%%%%%%%%%%%%%%%%%%%%%%%%%%%%%%%%%%%%%%%
\hide{
\begin{definition}[Dependency graph]
Let $\varphi$ be a relational constraint with $\replaceall$ function. Then the dependency graph of $\varphi$, denoted by $\cG_\varphi = (\vars(\varphi), E_\varphi)$, where $E_\varphi$ comprises the following edges,
\begin{itemize}
\item for each atomic formula $y= s_1 \concat \dots \concat s_n$ and each $i \in [n]$ such that $s_i$ is a variable, $(s_i, y) \in E_\varphi$,
\item for each atomic formula $x = \replaceall(y, u, s)$, $(y, x) \in E_\varphi$, in addition, if $s$ is a variable, then $(s, x) \in E_\varphi$, $(s, y) \in E_\varphi$, 
\end{itemize}
\end{definition}
}
%%%%%%%%%%%%%%%%%%%%%%%%%%%%%%%%%%%%%%%%%%%

\begin{definition}[Straight-line relational constraints with $\replaceall$ function]
A relational constraint $ \varphi$ with $\replaceall$ function is straight-line, if $\varphi \eqdef \bigwedge \limits_{1 \le i \le m} x_i = P_i$ such that
\begin{itemize}
\item $x_1,\dots, x_m$ are mutually distinct,
\item for each $i \in [m]$, all the variables in $P_i$ are either source variables, or variables from $\{x_1,\dots, x_{i-1}\}$,
\end{itemize}
\end{definition}
Intuitively, in a straight-line relational constraint, the dependency graph of the string variables is acyclic.


\begin{definition}[Straight-line string constraints with $\replaceall$ function]
A straight-line string constraint $C$ with $\replaceall$ function (denoted by $\strline[\replaceall]$)  is defined as $ \varphi \wedge \psi$,  where 
\begin{itemize}
\item $\varphi$ is a straight-line relational constraint,  and
%
\item $\psi$ is a regular constraint.
%
\end{itemize}
%Let us use $\Cc$ to denote the set of straight-line string constraints with $\replaceall$ function.
An $\strline[\replaceall]$ constraint $\varphi \wedge \psi$ is called \emph{pure} if the concatenation symbol $\concat$ does not occur in $\varphi$. Let us use $\pstrline[\replaceall]$ to denote the set of pure $\strline[\replaceall]$ constraints.
\end{definition}
%An SL[$replace$] constraint $\Cc=\varphi \wedge \psi$ is \emph{recursion-free} if $\varphi$ is.


\noindent {\bf Satisfiability problem}: Given an $\strline[\replaceall]$ constraint $C$, decide whether $C$ is satisfiable.

The questions we plan to consider is summarlised in the following table 

\[
\begin{tabular}{c|c|c|c}
                             pattern (y)  &   replacement (z)        &   pure (without concatenation) &  full (with concatenation) \\
                              \hline
                              constant  &   constant                       & \cite{LB16}    &  \cite{LB16}   \\
 \hline
                              constant  &   variable                       &  Section \ref{sec:replaceallpure}       &    \\
 \hline
                              variable  &   constant                       &         &    \\
\hline
                              variable  &   variable                       &        &    \\
\end{tabular}
\]


\section{A decision procedure for $\pstrline[\replaceall]$} \label{sec:replaceallpure}


\subsection{The automata $\cA_u$}

Let $u \in \Sigma^+$ and $k=|u| \ge 2$.
Our goal is to construct an NFA $\cA_u=(Q_u, \delta_u, q_{0,u}, F_u)$ which parses a string $v \in \Sigma^\ast u \Sigma^\ast$ into $v_1 u v_2 u \dots v_l u v_{l+1}$ such that $v_j u[1] \dots u[k-1] \not \in \Sigma^\ast u \Sigma^\ast$ for each $1 \le j \le l$, in addition, $v_{l+1} \not \in \Sigma^\ast u \Sigma^\ast$. 
In order to check that a substring of $v$ is \emph{not} in $\Sigma^\ast u \Sigma^\ast$, we introduce a concept of $k$-window profiles w.r.t. $u$.


%If $u = \sigma$ for $\sigma \in \Sigma$, then $\cA_u=(Q_u, \delta_u, q_{0,u}, F_u)$, where $Q_u = \{(q_0, \bot), (q_0, \top) \}$, $\delta_u = \{(q_0, \bot) \xrightarrow{\sigma} (q_0, \top), (q_0, \bot) \xrightarrow{\Sigma \setminus \sigma} (q_0, \bot), (q_0, \top) \xrightarrow{ \sigma} (q_0, \top), (q_0, \top) \xrightarrow{\Sigma \setminus \sigma} (q_0, \bot)\}$, $q_{0,u} = (q_0, \bot)$, and $F_u = Q_u$.

%In the following, we assume that $|u| = k \ge 2$. Let $u = u_1 \dots u_k$, where each $u_i \in \Sigma$.

%We construct an NFA $\cA_u=(Q_u, \delta_u, q_{0,u}, F_u)$ which over a string $v$, parses $v \in \Sigma^\ast u \Sigma^\ast$ into $v_1 u v_2 u \dots v_l u v_{l+1}$ such that $v_j u_1 \dots u_{k-1} \not \in \Sigma^\ast u \Sigma^\ast$ for each $1 \le j \le l$, and $v_{l+1} \not \in \Sigma^\ast u \Sigma^\ast$. 
%Let $u = u_1\dots u_k$ such that $u_i \in \Sigma$ for each $i \in [k]$.  

A $k$-\emph{window profile $\vec{W}$ w.r.t. $u$} is an element of $\{\bot,\top\}^{k-1}$. Intuitively, in the position $i$ of a string $v$, $\vec{W}$ is an abstraction of the substring $v[i-k+2] \dots v[i]$ such that for each $j \in [k-1]$, $\vec{W}(j) = \top$ iff $v[i-j+1] \dots v[i] = u[1] \dots u[j]$. Let $\wprof_{u, k}$ denote the set of $k$-window profiles w.r.t. $u$. In particular, if $k = 1$, then $\wprof_{u, k} = \emptyset$.

The NFA $\cA_u$ is constructed as follows.
\begin{itemize}
\item  $Q_u = \{(\search, \vec{W}) \mid \vec{W} \in \wprof_{u, k}\} \cup \{(\verify, j, \vec{W}) \mid j \in [k-1], \vec{W} \in \wprof_{u,k}\}$, where $\search$ and $\verify$ are used to denote whether $\cA_u$ is in the ``search''-mode to search the next occurrence of $u$, or in the ``verify'' mode to verify that the current position is a part of an occurrence of $u$.
%
\item $q_{0,u}=(\search,\bot^k)$.

\item $\delta_{u}$ comprises the following transitions,
%guesses over each position, one of the following holds, the substring comprising the next $k$-symbols (including the current one) is $u$ or not.
\begin{itemize}
\item for each state $(\search, \vec{W})$ and $\sigma \in \Sigma$ such that $\vec{W}(k-1) = \bot$ or $\sigma \neq u[k]$,
\begin{itemize}
\item $(\search, \vec{W}) \xrightarrow{\sigma} (\search, \vec{W}')$, where $\vec{W}'(1) = \top$ iff $\sigma = u[1]$, and for each $i: 2 \le i \le k-1$, $\vec{W}'(i) =\top$ iff $\vec{W}({i-1}) = \top$ and $\sigma = u[i]$,
%
\item if $\sigma = u[1]$, then $(\search, \vec{W}) \xrightarrow{\sigma} (\verify, 1, \vec{W}')$,  where $\vec{W}'(1)=\top$,  and for each $i: 2 \le i \le k-1$, $\vec{W}'(i) =\top$ iff $\vec{W}({i-1}) = \top$ and $\sigma = u[i]$,
%
\end{itemize}
%
\item for each state $(\verify, i-1, \vec{W})$ such that
\begin{itemize}
\item $2 \le i \le k-1$,
\item $\vec{W}(i-1)=\top$, $\sigma = u[i]$, and
\item either $\vec{W}(k-1)=\bot$ or $\sigma \neq u[k]$, 
\end{itemize}
we have $(\verify, i-1, \vec{W}) \xrightarrow{\sigma} (\verify, i, \vec{W}')$, where for each $j: 2 \le j \le k-1$, $\vec{W}'(j) = \top$ iff $\vec{W}(j-1)=\top$ and $\sigma = u[j]$, 
%
\item for each state $(\verify, k-1, \vec{W})$ such that $\vec{W}(k-1)=\top$, we have $(\verify, k-1, \vec{W}) \xrightarrow{u[k]} (\search, \bot^k)$, where $\bot^k$ in $(\search, \bot^k)$ is used to \emph{reinitialise} the $k$-window profile w.r.t. $u$.
%
\end{itemize}
Note that the constraint $\vec{W}(k-1) = \bot$ or $\sigma \neq u[k]$ is used to guarantee that when parsing a string $v$ into $v_1 u v_2 u \dots v_{l} u v_{l+1}$, $v_j u[1] \dots u[k-1] \not \in \Sigma^\ast u \Sigma^\ast$ for each $j \in [l]$, in addition, $v_{l+1} \not \in  \Sigma^\ast u \Sigma^\ast$.
%
\item $F_u=\{(\search, \vec{W}) \mid \vec{W} \in \wprof_{u, k} \} $. Note that the states $(\verify, j, \vec{W})$ are not final states, since when in these states, the verification of the next occurrence of $u$ has not yet been complete.
\end{itemize}
In an accepting run $r$ of $\cA_u$ on a string $v = v_1 u v_2 u \dots v_l u v_{l+1}$, the state sequence in the run is of the form 
$$(\search, \bot^k)\ r_1\ (\search, \bot^k)\ r_2\ (\search, \bot^k)\ \dots\ r_l\ (\search, \bot^k)\ r_{l+1}$$ 
such that  for each $j \in [l]$, $r_j \in (F_u \setminus \{(\search, \bot^k)\})^+ Q_{\verify, 1}  \dots  Q_{\verify, k-1}$, and $r_{l+1} \in (F_u)^+$, where $Q_{\verify, 1} = \{(\verify, 1, \vec{W}) \mid \vec{W} \in \wprof_{u,k}\}, \dots, Q_{\verify, k-1}=\{(\verify, k-1, \vec{W}) \mid \vec{W} \in \wprof_{u, k}\}$. Intuitively, each occurence of $(\search, \bot^k)$, except the first one, witnesses the \emph{first} occurrence of $u$ after its previous occurrence or starting from the beginning.

The NFA $\cA_u$ constructed above is \emph{unambiguous} in the sense that for each string $v \in \Sigma^+$, there is \emph{exactly one accepting run} of $\cA_u$ on $v$.


\subsection{The decision procedure}

Let $C= \varphi \wedge \psi$ be an $\pstrline[\replaceall]$ formula such that $\varphi = \bigwedge \limits_{1 \le i \le m} x_i = \replaceall(z, u_i, z')$, where $z, z'$ are either variables from $\{x_1,\dots, x_{i-1}\}$ or source variables. Suppose $\vars(\varphi) = \{x_1,\dots, x_m, y_1, \dots, y_n\}$, where $y_1,\dots, y_n$ are the source variables. 

We first utilise the fact that regular languages are closed under all the Boolean operations to normalise $\psi$ into the form $\bigwedge \limits_{i \in [m]} x_i \in \Ll(\cA_{x_i}) \wedge \bigwedge \limits_{j \in [n]} y_j \in \Ll(\cA_{y_j})$, where $\cA_{x_i}$ for $i \in [m]$ and $\cA_{y_j}$ for $j \in [n]$ are DFA. 


Our goal is to construct a two-way DFA $\cC_C$ to check the satisfiability of $C=\varphi \wedge \psi$. The inputs of $\cC_C$ are strings of the form $\triangleright \#_1 v_1 \#_2 \dots \#_n v_n\triangleleft$, where $v_1,\dots, v_n$ represent an assignment of the variables $y_1,\dots,y_n$.

\hide{
Define the \emph{dependency graph} of $C$ as $G_C= (\vars(\varphi), E_C)$, such that for each $i \in [m]$, if $x_i = \replaceall(x_j, u_i, z')$, then $(x_i, \replaceall_1(u_i), x_j) \in E_C$ and $(x_i, \replaceall_3, z') \in E_C$. For each variable $y_j$, let $\dep(y_j)$ denote the set of variables that are dependent on $y_j$, that is, the set of variables that are reachable from $y_j$ through \emph{nonempty} paths in $G_C$, in addition, let $\ptn(y_j)$ denote the set of strings $u_i$ such that an edge labeled by $\replaceall_1(u_i)$ is reachable from $y_j$ (an edge is reachable from $y_j$ if its source vertex is reachable from $y_j$). In particular, $y_j \not \in \dep(y_j)$. Similarly, we can define $\dep(x_i)$ and $\ptn(x_i)$ for the variables $x_i$. Note that $\dep(y_j), \dep(x_i) \subseteq \{x_1,\dots, x_m\}$ for each $j \in [n]$ and $i \in [m]$.
}

\begin{definition}[Dependency graph]
Define the \emph{dependency graph} of $C$ as $G_C= (\vars(\varphi), E_C)$, such that for each $i \in [m]$, if $x_i = \replaceall(x_j, u_i, z')$, then $(x_i, \src_i, x_j) \in E_C$ and $(x_i, \strtorep_i, z') \in E_C$. A final (resp. initial) vertex in $G_C$ is a vertex in $G_C$ without successors (resp. predecessors). The edges labeled by $\src_i$ and $\strtorep$ are called the $\src$-edges and $\strtorep$-edges respectively. The $\src$-length of a path $\pi$, denoted by $\srclen(\pi)$, is the number of $\src$-edges on $\pi$. A path of $G_C$ is a sequence $z_1 \ell_1 z_2 \dots \ell_{k-1} z_k$ such that for each $i \in [k-1]$, $(z_i, \ell_i, z_{i+1}) \in E_C$. A path is initial (resp. final) if the path starts from an initial vertex (resp. stops at a final vertex).
% e the $\src$-nesting-depth of $z$ in $G_C$, denoted by $\srcnd_{G_C}(z)$,  as the maximum number of $\src$-edges in paths from source variables to $z$.
\end{definition}
Note that $G_C$ satisfies that the out-degree of each vertex is two or zero. 

For each tuple of strings $\vec{v}=(v_1,\dots, v_n)$, if for each $j \in [n]$, $y_j$ is assigned with the string $v_j$, then for each vertex $z$ in $G_C$, the value of $z$, denoted by $\val_{\vec{v}}(z)$, is determined and can be computed from $\vec{v}$ in bottom-up by using $G_C$. 

%\begin{definition}[$\src$-length of forward-maximal paths]
%A backward-maximal path in $G_C$ is a path of $G_C$ that ends in a sink vertex. The $\src$-length of a forward-maximal $\pi$, denoted by $\srclen(\pi)$, is the number of $\src$-edges on $\pi$. 
%\end{definition}
%Note that if $G_C$ is nonempty, then there must be forward-maximal paths of zero $\src$-length in $G_C$.

%\begin{definition}[Contexts]
%Let $z$ be a vertex in $G_C$. A context $\theta$ of $z$ in $G_C$ is the sequence of the labels of edges on a path from $z$ to some sink node in $G_C$. %where the enumeration follows the direction of edges on the path. 
%Let $\ctxts_{G_C}(z)$ denote the set of contexts of $z$ in $G_C$.\zhilin{not so correct}
%\end{definition}


%By an induction on the $\src$-lengths of forward-maximal paths $\pi$, we construct an NFA $\cB_{\pi}$ as follows.

We first assume that there is \emph{exactly  one} initial vertex in $G_C$, say $x_i$. 


\begin{definition}[$\pi$-automaton]\label{def-pi-aut}
Let $\pi=z_1 \ell_1 \dots z_{k-1} \ell_{k-1} z_k$ be an initial path.  Then a $\pi$-automaton $\cB_\pi$ is inductively defined as follows. 
\begin{itemize}
\item If $k=1$, then $\cB_\pi$ is $\cA_{z_1}$.
%
\item Otherwise, let $\pi' = z_1 \ell_1 \dots \ell_{k-2}z_{k-1}$.
\begin{itemize} 
\item If the last edge of $\pi$ is an $\strtorep$-edge, then $\cB_\pi$ is the product of $\cB_{\pi'}$ and $\cA_{z_k}$.

\item otherwise, let $\src_j$ be the label of the $\src$-edge out of $z_{k-1}$, the $\strtorep$-edge out of $z_{k-1}$ be from $z_{k-1}$ to $z'$, and $\pi'' = z_1 \dots z_{k-1} z'$. Then the definition of $\cB_\pi$ is parameterized by an element $f_{\pi''} \in \cM(\cB_{\pi''})$. From the definition, we know that $\cB_{\pi''}$ is the product of $\cB_{\pi'}$ and $\cA_{z'}$. The function $f_{\pi''}$ determines a function $f'_{\pi''} \in \cM(\cB_{\pi'})$. Then $\cB_\pi$ is defined as the DFA obtained by determinizing the product automaton of the DFA $\cA_{z_k}$ and the NFA $\cB'_{\pi}$ which includes the following transitions,
\begin{itemize}
\item $((\search, \vec{W}), q) \xrightarrow{\sigma} ((\search, \vec{W}'), q')$ such that $(\search, \vec{W}) \xrightarrow{\sigma} (\search, \vec{W}')$ is a transition in $\cA_{u_j}$ and $q \xrightarrow{\sigma} q'$ is a transition in $\cB_{\pi'}$,
\item $((\search, \vec{W}), q) \xrightarrow{u_j[1]} ((\verify, 1, \vec{W}'), q')$ such that  $(\search, \vec{W}) \xrightarrow{u_j[1]} (\verify,1,  \vec{W}')$ is a transition in $\cA_{u_j}$ and $f'_{\pi''}(q)=q'$,
\item $((\verify, j'-1, \vec{W}), q) \xrightarrow{u_j[j']} ((\verify, j', \vec{W}'), q)$ such that $2 \le j' \le |u_j |-1$ and $(\verify, j'-1, \vec{W}) \xrightarrow{u_j[j']} (\verify,j',  \vec{W}')$ is a transition in $\cA_{u_j}$,
\item $((\verify, |u_j|-1, \vec{W}),q) \xrightarrow{u_j[|u_j|]} ((\search, \vec{W}'),q)$ such that $(\verify, |u_j|-1, \vec{W}) \xrightarrow{u_j[|u_j|]} (\search, \vec{W}')$ is a transition in $\cA_{u_j}$.
%Over an input $w$, run $\cA_{u_j}$ on $w$, run $\cB_{\pi'}$ at the same time, and replace each occurrence of $u_j$ with $f_{\pi'}$.
\end{itemize}
\end{itemize}
Note that since $f'_{\pi''} \in  \in \cM(\cB_{\pi'})$, the domain and range of $f'_{\pi''}$ are subsets of the state space of $\cB_{\pi'}$. 
\end{itemize}
\end{definition}

%\begin{remark}
%Let $\pi=z_1 \dots z_k$ be an initial path. In addition, let $\src_{i_1}, \dots, \src_{i_l}$ be an enumeration of the labels of $\src$-edges on $\pi$. From Definition~\ref{def-pi-aut}, we deduce that each state of $\cB_\pi$ is of the form $(S_{i_1}, \dots, S_{i_l}, q_1, \dots, q_k)$, where $S_{i_1},\dots, S_{i_l}$ are the state subsets of $\cA_{u_{i_1}},\dots, \cA_{u_{i_l}}$ respectively, and $q_1,\dots, q_k$ are the states in $\cA_{z_1},\dots, \cA_{z_k}$ respectively. In addition, for a prefix $\pi'$ of $\pi$, the state set of $\cB_{\pi'}$ is a subset of the state set of $\cB_{\pi}$.
%\end{remark}


%%%%%%%%%%%%%%%%%%%%%%%%%%%%%%%%%%%%%%%%%%%%%%%
%%%%%%%%%%%%%%%%%%%%%%%%%%%%%%%%%%%%%%%%%%%%%%%
\hide{
\begin{definition}[projection of monoid elements]
Let $\pi = z_1 \dots z_k$ be an initial path, and $\src_{i_1}, \dots, \src_{i_l}$ be an enumeration of the labels of $\src$-edges on $\pi$. In addition, let $\pi' = z_1 \dots z_j$ be a prefix of $\pi$, and $f \in \cM(\cB_{\pi})$. Suppose $\src_{i_r}$ is the label of the last $\src$-edge on $\pi'$ (where $1 \le r \le l$). Then the projection of $f$ to $\pi'$, denoted by $f |_{\pi'}$, is defined as follows: for each state $(S_{i_1},\dots, S_{i_r}, q_1, \dots, q_j)$ of $\cB_{\pi'}$, 
$$(f|_{\pi'})(S_{i_1},\dots, S_{i_r}, q_1, \dots, q_j) = (S'_{i_1},\dots, S'_{i_r}, q'_1, \dots, q'_j)$$ 
if there are $S_{i_{r+1}}, \dots, S_{i_l}, S'_{i_{r+1}}, \dots, S'_{i_l}, q_{j+1}, \dots, q_k, q'_{j+1}, \dots, q'_k$, such that 
\begin{itemize}
\item for each $s: r+1 \le s \le l$, $S_{i_{s}}, S'_{i_{s}}$ are the state subsets of $\cA_{u_{i_{s}}}$, 
\item for each $s: j+1 \le s \le k$, $q_{j}, q'_j $ is a state of $\cA_{z_j}$, and
\item $f(S_{i_1},\dots, S_{i_l}, q_1, \dots, q_k)=(S'_{i_1},\dots, S'_{i_l}, q'_1, \dots, q'_k)$.
\end{itemize}
\end{definition}
Note that 

\begin{definition}[$\pi$-automaton]\label{def-pi-aut}
Let $\pi=z_1 \dots z_k$ be an initial path.  Then a $\pi$-automaton $\cB_\pi$ is inductively defined as follows.
\begin{itemize}
\item If $\srclen(\pi)=0$, then $\cB_\pi$ is the product of $\cA_{z_1},\dots, \cA_{z_k}$.
%
\item Otherwise, 
 suppose for each initial path $\pi'$ such that $\srclen(\pi') < \srclen(\pi)$, $\cB_{\pi'}$ have been constructed.  
 Let $z_{i-1} \xrightarrow{\src_j} z_{i}$ be the last $\src$-edge on $\pi$. 
%In addition, let $z_{i-1} \xrightarrow{\strtorep_j} z'_i$ be the other edge out of $z_{i-1}$. 
Let $\pi' = z_1\dots z_{i-1}$. Then $\pi'$ is an initial path in $G_C$ such that $\srclen(\pi') < \srclen(\pi)$. The construction of $\cB_\pi$ is parameterized by an element $f_{\pi'} \in \cM(\cB_{\pi'})$. $\cB_\pi$ is DFA obtained by determinizing the product automaton of DFA $\cA_{z_i},\dots, \cA_{z_k}$ and the NFA $\cB'_{\pi}$ which includes the following transitions,
\begin{itemize}
\item $((\search, \vec{W}), q) \xrightarrow{\sigma} ((\search, \vec{W}'), q')$ such that $(\search, \vec{W}) \xrightarrow{\sigma} (\search, \vec{W}')$ is a transition in $\cA_{u_j}$ and $q \xrightarrow{\sigma} q'$ is a transition in $\cB_{\pi'}$,
\item $((\search, \vec{W}), q) \xrightarrow{u_j[1]} ((\verify, 1, \vec{W}'), q')$ such that  $(\search, \vec{W}) \xrightarrow{u_j[1]} (\verify,1,  \vec{W}')$ is a transition in $\cA_{u_j}$ and $f_{\pi'}(q)=q'$,
\item $((\verify, j'-1, \vec{W}), q) \xrightarrow{u_j[j']} ((\verify, j', \vec{W}'), q)$ such that $2 \le j' \le |u_j |-1$ and $(\verify, j'-1, \vec{W}) \xrightarrow{u_j[j']} (\verify,j',  \vec{W}')$ is a transition in $\cA_{u_j}$,
\item $((\verify, |u_j|-1, \vec{W}),q) \xrightarrow{u_j[|u_j|]} ((\search, \vec{W}'),q)$ such that $(\verify, |u_j|-1, \vec{W}) \xrightarrow{u_j[|u_j|]} (\search, \vec{W}')$ is a transition in $\cA_{u_j}$.
%Over an input $w$, run $\cA_{u_j}$ on $w$, run $\cB_{\pi'}$ at the same time, and replace each occurrence of $u_j$ with $f_{\pi'}$.
\end{itemize}
\end{itemize}
\end{definition}
}
%%%%%%%%%%%%%%%%%%%%%%%%%%%%%%%%%%%%%%%%%%%%%%%
%%%%%%%%%%%%%%%%%%%%%%%%%%%%%%%%%%%%%%%%%%%%%%%

%\begin{remark}
%Suppose $\pi = z_1 \dots z_k$ and $z_{i_1}, \dots, z_{i_l}$ is an enumeration of the destination vertices of the $\src$-edges on $\pi$, with $i_1 < \dots < i_l$. For each $j \in [l]$, let $\pi'_j = z_1 \dots z_{i_j-1}$. Then by induction, the construction of $\cB_{\pi}$ is essentially parameterized by $f_{\pi_j} \in \cB_{\pi'_j}$ for $j \in [l]$.
%\end{remark}

%A run of $\cC_C$ comprises two phases.

%In the first phase, $\cC_C$ checks that $v_j$ is accepted by $\cA_{y_j}$ for each $j \in [n]$.

The two-way DFA $\cC_C$ traverses all the initial paths in $G_C$ in a \emph{depth-first} manner. The traversal starts at the initial vertex $x_i$, and for each vertex $z$ with two outgoing edges, the $\strtorep$-edge is traversed before the $\src$-edge.  When the traversal reaches a vertex $z$, the currently-traversed initial path from $x_i$ to $z$, denoted by $\pi_z$, is recorded for $z$.
A vertex $z$ is said to be \emph{entered} (resp. \emph{backtracked to}) when the traversal reaches $z$ by going through an edge with $z$ as the target vertex (resp. source vertex). When entering each vertex $z$, the $\pi_z$-automaton $\cB_{\pi_z}$ is constructed. When entering a final vertex $z$ or backtracking to a non-final vertex $z$, the monoid element used to construct $\pi_z$-automata  (cf. Definition~\ref{def-pi-aut}), denoted by $f_{\pi_z}$, is computed. 

\paragraph*{When entering a final vertex $y_j$,} the reading head of $\cC_C$ is moved to the position immediately after $\#_j$.
% and $\cC_C$ first checks that $v_j$ is accepted by $\cA_{y_j}$. 
%
Let the last edge of $\pi_{y_j}$ be from $z$ to $y_j$.  By induction, $z$ has been entered and $\cB_{\pi_z}$ has been constructed. 
\begin{itemize}
\item If the edge from $z$ to $y_j$ is an $\strtorep$-edge, then $\cB_{\pi_{y_j}}$ is the product of $\cB_{\pi_z}$ and $\cA_{y_j}$.  In addition,  $\cB_{\pi_{y_j}}$ is run on $v_j$ to compute $f_{\pi_{y_j}}$ which comprises the state pairs $((q, p), (q', p'))$ such that $(q, p)  \xrightarrow[\cB_{\pi_{y_j}}]{v_j} (q', p')$, where $q,q'$ are states of $\cA_{y_j}$ and $p, p'$ are the states of $\cB_{\pi_z}$ respectively. 
%
\item If the edge from $z$ to $y_j$ is an $\src$-edge, let $\src_{i'}$ be the label of the edge from $z$ to $y_j$ and the $\strtorep$-edge out of $z$ be from $z$ to $z'$. By induction, $z'$ has been backtracked to and $f_{\pi_z z'} \in \cM(\cB_{\pi_z})$ has been computed. Then $\cB_{\pi_{y_j}}$ is constructed  as in Definition~\ref{def-pi-aut}, with $\cA_{z_k}, \cA_{u_j}, \cB_{\pi'}, f_{\pi''}$ replaced by $\cA_{y_j}, \cA_{u_{i'}}, \cB_{\pi_z}, f_{\pi_z z'}$ respectively. In addition, $f_{\pi_{y_j}} \in \cM(\cB_{\pi_{y_j}})$ is computed by running  $\cB_{\pi_{y_j}}$ on $v_j$. 
%Moreover, $\cC_C$  asserts that $f_{\pi_{y_j}}$ contains a state pair $((q_0, -), (q', -))$ for some final state $q'$ of $\cA_{y_j}$.
\end{itemize}
Moreover, $\cC_C$ asserts that $f_{\pi_{y_j}}$ contains a state pair $((q_0, -), (q', -))$ for some final state $q'$ of $\cA_{y_j}$.



\paragraph*{When entering a non-final vertex $x_{i'}$,} let the last edge of $\pi_{x_{i'}}$ be from $z$ to $x_{i'}$.  By induction, $z$ has been entered and $\cB_{\pi_z}$ has been constructed. 
\begin{itemize}
\item If the edge from $z$ to $x_{i'}$ is an $\strtorep$-edge, then $\cB_{\pi_{x_{i'}}}$ is the product of $\cB_{\pi_z}$ and $\cA_{x_{i'}}$.
%
\item Otherwise, let $\src_{i''}$ be the label of the edge from $z$ to $x_{i'}$ and the $\strtorep$-edge out of $z$ be from $z$ to $z'$. By induction, $z'$ has been backtracked to and $f_{\pi_z z'}$ has been computed. Then $\cB_{\pi_{x_{i'}}}$ is constructed as in Definition~\ref{def-pi-aut}, with $\cA_{z_k}, \cA_{u_j}, \cB_{\pi'}, f_{\pi''}$ replaced by $\cA_{x_{i'}}, \cA_{u_{i''}}, \cB_{\pi_z}, f_{\pi_z z'}$ respectively. 
\end{itemize}

\paragraph*{When backtracking to a non-final vertex $x_{i'}$,}  suppose the backtracking is from $z$ to $x_{i'}$.
\begin{itemize}
\item If the backtracking is through an $\strtorep$-edge, then $f_{\pi_{x_{i'}}}$ is obtained from $f_{\pi_z}$ as follows: for each state $p$ of $\cB_{\pi_{x_{i'}}}$, $f_{\pi_{x_{i'}}}(p) = p'$ such that $f_{\pi_z}(q_0, p)= (q, p')$ for the initial state $q_0$ of $\cA_{z}$ and a state $q$ of $\cA_z$. 

%1) the state $q$ such that $q_0 \xrightarrow[\cA_{y_j}]{v_j} q$, where $q_0$ is the initial state of $\cA_{y_j}$, 2) the function $f_{\pi_{y_j}}$ such that for each state $p$ in $\cB_{\pi_z}$, $f_{\pi_{y_j}}(p)=p'$ with $p \xrightarrow[\cB_{\pi_z}]{ v_j} p'$. If $q$ is not a final state of $\cA_{y_j}$, then $\cC_C$ enters a special state $q_{\bot}$ and stays in $q_\bot$ forever. Otherwise, the depth-first traversal  continues.

%$\cB_{\pi_z}$ is the product of $\cB_{\pi_{x_{i'}}}$ and $\cA_{z}$, and $f_{\pi_{x_{i'}}} \in \cM(\cB_{\pi_{x_{i'}}})$ is obtained from $f_{\pi_z} \in \cM(\cB_{\pi_z})$ as follows: for each state $q$ in $\cB_{\pi_{x_{i'}}}$, $f_{\pi_{x_{i'}}}(q) = q'$ if there are states $p, p'$ of $\cA_z$ such that $f_{\pi_z}(q, p) = (q', p')$.
%
\item Otherwise,  
 let $\src_{i''}$ be the label of the edge used in the backtracking from $z$ to $x_{i'}$. 
% \begin{itemize}
% \item If $z = y_j$ for some $j \in [n]$, then from the construction of $\cB_{\pi_{y_j}}$, we know that the run of $\cB_{\pi_{y_j}}$ on $v_j$ is relative to the runs of $\cA_{u_{i''}}$ on $v_j$. If we focus on the unique accepting run of $\cA_{u_{i''}}$ on $v_j$, then the run of $\cB_{\pi_{y_j}}$ on $v_j$ determines a function $f_{\pi_{x_{i'}}} \in \cM(\cB_{\pi_{x_{i'}}})$. Such a function $f_{\pi_{x_{i'}}}$ can be computed by running $\cB_{\pi_{y_j}}$ on $v_j$. In addition, during the run of $\cB_{\pi_{y_j}}$ on $v_j$, we assert that $v_j$ is accepted by $\cA_{y_j}$.
%
% \item Otherwise, $z$ is a non-final vertex.
 By induction, $f_{\pi_z} \in \cM(\cB_{\pi_{z}})$ has been computed. From the construction of $\cB_{\pi_{z}}$, we know that the function $f_{\pi_z}$ determines a unique state pair of $\cA_{u_{i''}}$, say $((\search, \bot^{|u_{i''}|}), (\search, \vec{W}'))$, which intuitively means that $(\search, \bot^{|u_{i''}|}) \xrightarrow[\cA_{u_{i''}}]{\val_{\vec{v}}(z)} (\search, \vec{W}')$. The function $f_{\pi_z}$ together with the state pair $((\search, \bot^{|u_{i''}|}), (\search, \vec{W}'))$ determines a function $f_{\pi_{x_{i'}}} \in \cM(\cB_{\pi_{x_{i'}}})$. Such a function $f_{\pi_{x_{i'}}} $ can be computed effectively from $f_{\pi_z}$ and $((\search, \bot^{|u_{i''}|}), (\search, \vec{W}'))$. 
 %In addition, we assert that $f_{\pi_z}$ determines a state pair $(q_0, q)$ such that $q_0$ is an initial state of $\cA_z$ and $q$ is a final state of $\cA_z$ (intuitively, this means that $\val_{\vec{v}}(z)$ is accepted by $\cA_z$). \zhilin{to be made more specific}
%\end{itemize}
\end{itemize}
In addition, $\cC_C$ asserts that $f_{\pi_{x_{i'}}}$ contains a state pair $((q'_0, -), (q', -))$ such that $q'_0$ is the initial state of $\cA_{x_{i'}}$ and $q'$ is a final state of $\cA_{x_{i'}}$.

%Finally, for the initial vertex $x_i$, it is asserted that $f_{\pi_{x_i}}$  determines a state pair $(q_0, q)$ such that $q_0$ is an initial state of $\cA_{x_i}$ and $q$ is a final state of $\cA_{x_i}$  (intuitively, this means that $\val_{\vec{v}}(x_i)$ is accepted by $\cA_{x_i}$). 

Finally, for the more general case that there are multiple initial vertices in $G_C$, $\cC_C$ is constructed by doing a aforementioned depth-first traversal on $G_C$, for every initial vertex.


\begin{example}
\end{example}



%$x_2 = \replaceall(x_4, v, x_5)  \wedge x_1 = \replaceall(x_2, u, x_3) \wedge x_1 \in e_1 \wedge x_2 \in e_2 \wedge x_3 \in e_3 \wedge x_4 \in e_4 \wedge x_5 \in e_5$.
%$x_3 \# x_5 \# x_4$
%
%$x_1 = \replaceall(x_2, u, x_3) \wedge x_3 = \replaceall(x_4, v, x_5) \wedge x_1 \in e_1 \wedge x_2 \in e_2 \wedge x_3 \in e_3 \wedge x_4 \in e_4 \wedge x_5 \in e_5$.
%
%$x_5 \# x_4 \# x_2$


\section{A decision procedure for $\strline[\replaceall]$}





%%%%%%%%%%%%%%%%%%%%%%%%%%%original texts removed%%%%%%%%%%%%%%%%%%%%%%%%%%%%%
%%%%%%%%%%%%%%%%%%%%%%%%%%%original texts removed%%%%%%%%%%%%%%%%%%%%%%%%%%%%%
\hide{
\begin{proposition}
Suppose $y = replace(x, u, v)$, where $u, v \in \Sigma^+$, and $e$ is a regular expression. Then a regular expression $rep_{post}(e, u, v)$ (resp. $rep_{pre}(e, u, v)$) can be computed to define 
 $\{w'  \in \Sigma^+ \mid \exists w \in \Ll(e).\ w' = replace(w, u, v)\}$ (resp. $\{w  \in \Sigma^+ \mid \exists w' \in \Ll(e).\ w' = replace(w, u, v)\}$).
\end{proposition}

\begin{proof}
Let $u = u_1 \dots u_k$ and $v = v_1 \dots v_l$ (where $k \ge 1$ and $l \ge 0$).

Suppose $\cA_e = (Q_e, \delta_e, q_{0,e}, F_e)$ is an NFA to define $\Ll(e)$. In addition, suppose that $\cA_u = (Q_u, \delta_u, q_{0,u}, F_u)$ is a DFA to define $\Sigma^\ast u \Sigma^\ast$.

We first construct an NFA $\cA_{post}$ to define $\{w'  \in \Sigma^+ \mid \exists w \in \Ll(e).\ w' = replace(w, u, v)\}$.

When reading a position $i$ of $w'$, let $w'_1$ be a prefix of $w'$ up to the position $i-1$ and $\cA_{post}$ has guessed a word $w_1$ such that $w'_1 = replace(w_1, u, v)$, let $q$ be a state of $\cA_e$ after reading $w_1$ and the current state of $\cA_u$ be $q'$, then $\cA_{post}$ nondeterministically chooses to do one of the following,
\begin{itemize}
\item checks that $v$ occurs in the next $l$ positions (including the position $i$, if $l =0$, then this checking is skipped), chooses a state $q'' \in \delta_e(q, u)$ and updates the state of $\cA_e$ into $q''$, for each $j \in [k-1]$, checks that $\delta_u(q', u_1 \dots u_j)$ is not a final state, finally, resets the state of $\cA_u$ to the initial state,
%
\item updates the state of $\cA_e$ into a state from $\delta_e(q, w'_i)$,  and the state of $\cA_u$ into $\delta_u(q', w'_i)$, move the reading head one position to the right.
\end{itemize}
If after reading $w'$, the state of $\cA_e$ is a final state, then $\cA_{post}$ accepts.

We then construct an NFA $\cA_{pre}$ to define $\{w  \in \Sigma^+ \mid \exists w' \in \Ll(e).\ w' = replace(w, u, v)\}$.

When reading the position $i$ of a word $w$, let $w_1$ be a prefix of $w$ up to the position $i-1$ and $w'_1 = replace(w_1, u, v)$, let $q$ be a state of $\cA_e$ after reading $w'_1$ and $q'$ be a state of $\cA_u$ after reading $w_1$, then $\cA_{pre}$ nondeterministically chooses to do one of the following
\begin{itemize}
\item checks that $u$ occurs in the next $k$ positions (including the position $i$), chooses a state $q'' \in \delta_e(q, v)$ and updates the state of $\cA_e$ into $q''$, for each $j \in [k-1]$, checks that $\delta_u(q', u_1 \dots u_j)$ is not a final state, finally, resets the state of $\cA_u$ to the initial state,
%
\item updates the state of $\cA_e$ into a state from $\delta_e(q, w'_i)$,  makes sure that $\delta_u(q', w'_i)$ is not a final state, updates the state of $\cA_u$ into $\delta_u(q', w'_i)$,  moves the reading head one position to the right.
\end{itemize}
\qed
\end{proof}

Next, we present a decision procedure for SL[$replace$] formulae.

Let $\varphi \wedge \psi$ be an SL[$replace$] formula and $G_\varphi = (V, E)$ be the dependency graph of $\varphi$ defined as follows: $V$ is the set of variables occurring in $\varphi$, and $(x, y) \in E$ iff $\varphi$ contains a conjunct $y = x'_1 \concat \dots \concat x'_k$ such that $x = x'_i$ for some $i$, or $y =replace(x, u, v)$ for some $u \in \Sigma^+$ and $v \in \Sigma^\ast$.  Then $G_\varphi$ is a directed acyclic graph. Let $x_1 \dots x_m$ be a topological sorting of the vertices in $G_\varphi$, that is, for each $x_i, x_j \in V$, $(x_i, x_j) \in E$ implies that $i < j$. In addition, let $\psi = \bigwedge \limits_{i \in [m]} x_i \in e_i$ (for each variable, there is exactly one membership constraint). 

\begin{enumerate}
\item Forward analysis: For each $i \in [m]$, compute a regular expression $e'_i$ as follows. Let $i \in [m]$. Suppose $e'_1, \dots, e'_{i-1}$ have been computed. Let $e'_i : = e_i$.
\begin{itemize}
\item If $x_i = x_{j_1} \dots x_{j_r}$, then $e'_i := e'_i \cap e'_{j_1} \concat \dots \concat e'_{j_r}$.
%
\item If $x_i = replace(x_j, u, v)$, where $j \in [i-1]$, then $e'_i := e'_i \cap  rep_{post}(e'_j, u, v)$.
\end{itemize}
%
\item Backward analysis: For each $i \in [m]$, (nondeterministically) compute a regular expression $e''_i$ and checks that $\Ll(e''_i) \neq \emptyset$. 
Let $i \in [m]$. Suppose $e''_{i+1}, \dots, e''_{m}$ have been computed.  Then $e''_i$ is computed by the following procedure.
\begin{enumerate}
\item Let $e''_i := e'_i$.  
%
\item For each conjunct of $\varphi$ of the form $x_{i'} = x_{j_1} \concat \dots \concat x_{j_r}$ such that $i' > i$ and $i = j_s$ for some $s \in [r]$, nondeterministically guesses $e'''_{j_1},\dots, e'''_{j_r}$ such that $e''_{i'} = e'''_{j_1} \concat \dots \concat e'''_{j_r}$, let $e''_i := e''_i\ \cap \bigcap \limits_{s \in [r], i = j_s} e'''_{j_s}$.
%
\item For each conjunct of $\varphi$ of the form $x_{i'} = replace(x_i, u, v)$ such that $i' > i$, let $e''_i: = e''_i \cap reg_{pre}(e''_{i'}, u, v)$. 
\end{enumerate}
\end{enumerate}

\noindent {\bf Questions}: 1. Is the above decision procedure complete ?  2. Can we find a way to implement the decision procedure ?

\tl{I am thinking the following question but without a sasfactory answer: suppose that there is only one source variable, and the constraints are satisfiable, can the solution of the source variable be bounded in poly space, or must in exp space. I understand that this question has been answered \cite{LB16} for general transducers, but here we only have replace, and somehow I want to avoid the approach of "encoding $\circ$ as transducers".}  

\section{Playing area}

Essentially we have a batch of equations, each of which is either $x= s_1\cdots s_n$, or $x=replace (y, u,v)$. Suppose there is no source variable. Then the RHS of the first equation must be a constant. The problem is trivial. 

Suppose that we only have one source variable $x$ which appear in the first equation, and $x$ must be from a given regular expression $r$. Then we can "solve" the equation such that each variable is associated with a regular expression. Suppose that $x_i=r_i$, and we further have constraints $x_i\in e_i$. 

The idea might be, for each equation $x_m=...$, assume that all of the variables from the RHS already have a "range" ...

Some questions we have discussed:
\begin{itemize}
\item in the case that each variable only appears on the RHS once, the problem should be simple;

\item in the case that variables occur multiple times, we have the issue of $x=yyy$, and we have to "guess". A question is whether this is avoidable?

\item for a general form of replace function where the third argument is a variable, what is the complexity? Zhilin mentioned that for the simple variant of replace, one can reduce this to word equations, which requires further investigations;

\item more generally, zhilin proposes parameterised string transducer, which is very interesting. 
\end{itemize}
}
%%%%%%%%%%%%%%%%%%%%%%%%%%%%%%%%%%%%%%%%%%%%%%%%%%%%%%%%%%%%%%%%%
%%%%%%%%%%%%%%%%%%%%%%%%%%%%%%%%%%%%%%%%%%%%%%%%%%%%%%%%%%%%%%%%%

\bibliographystyle{alpha}
\bibliography{string}
\end{document}