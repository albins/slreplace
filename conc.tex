%!TEX root = popl2018.tex

\section{Conclusion}

In this paper, we have investigated extensively the decidability boundary of the satisfiability problem for the string constraints involving the $\replaceall$ function and regular constraints. The $\replaceall$ functions are considered in their most general form, that is, $\replaceall(x, e, y)$, where $x,y$ can be string variables, and $e$ is either a string constant/variable, or a regular expression. As the satisfiability problem is undecidable in general, we focused on the straight-line fragment. We showed that while it remains to be undecidable if the second parameter of the $\replaceall$ function is a variable, it becomes decidable, more precisely in EXPSPACE, if the second parameter is a regular expression. The decision procedure was obtained by an automata-theoretic construction, which is modular and amenable to implementations. In addition, we proved that the decision procedure is in fact PSPACE-complete for several special cases that are meaningful in practice. Finally, we show that extending the decidable straight-line fragment with any of the integer constraints, character constraints, and constraints involving the $\indexof$ function, leads to the undecidability immediately. 
Our work clarified important fundamental issues surrounding the $\replaceall$ functions in string constraint solving and provided a novel decision procedure which paved a way to a string solver that is able to fully support the $\replaceall$ function. This would be the most direct future work. 