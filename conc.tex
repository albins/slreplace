%!TEX root = popl2018.tex

\section{Conclusion}

\OMIT{
Since any theory of strings containing the string-replace function (even the 
most restricted version where pattern/replacement strings are both constant 
strings) is undecidable unless we impose some kind of straight-line restriction 
on the formulas, 
}
We have initiated a systematic investigation of the decidability of 
the satisfiability problem for the straight-line fragments of string 
constraints involving the $\replaceall$ function and regular constraints.
The fragment is known to be appropriate for applications in symbolic execution 
of 
string-manipulating programs \cite{LB16}. Our main result is a decision 
procedure for a large fragment of the logic, wherein the pattern string is
a regular expression (which covers a large proportion of the usage of the 
$\replaceall$ function in practice). Concatenation is obtained for free since 
this restriction of $\replaceall$ can easily express concatenation.
%in expressing concatenation, we have obtained a strict generalisation of a
%partial result by \cite{LB16} 
We have shown that the decidability
of this fragment cannot be substantially extended. This is achieved by showing
that if either (1) the pattern string is a variable, or (2) the length
constraint is incorporated in the fragment, then undecidability results.
%boundary of string constraints with the $\replaceall$ function by show
%The $\replaceall$ functions are considered in their most general form, that is, $\replaceall(x, e, y)$, where $x,y$ can be string variables, and $e$ is either a string constant/variable, or a regular expression. As the satisfiability problem is undecidable in general, we focused on the straight-line fragment. We showed that while it remains to be undecidable if the second parameter of the $\replaceall$ function is a variable, it becomes decidable, more precisely in EXPSPACE, if the second parameter is a regular expression. The decision procedure was obtained by an automata-theoretic construction, which is modular and amenable to implementations. In addition, we proved that the decision procedure is in fact PSPACE-complete for several special cases that are meaningful in practice. Finally, we show that extending the decidable straight-line fragment with any of the integer constraints, character constraints, and constraints involving the $\indexof$ function, leads to the undecidability immediately. 
Our work clarified important fundamental issues surrounding the $\replaceall$
functions in string constraint solving and provided a novel decision procedure
which paved a way to a string solver that is able to fully support the
$\replaceall$ function. This would be the most immediate future work. 
