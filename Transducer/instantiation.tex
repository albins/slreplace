%!TEX root = main.tex

\section{Instantiations of the generic decision procedure}\label{sec-instan}

%We instantiate the generic decision procedure presented previously. 
%As in Section~\ref{sec:framework}, 
In our framework
each string function $f(x_1, \cdots, x_\arity)$ ($\arity\geq 1$) from the symbolic execution $S$ is modelled by a parametric transducer $\Transducer$  
with $\arity-1$ parameters, and the size of $f$, $|f|$, is naturally defined by the size of $\Transducer$ (i.e., $|\Transducer|$). 
%We shall usually refer to them as \emph{string constraints with transducers}. 
For the pre-image of an FA $\Aut$,  %$\Pre_{R_f}(\Aut)$ 
$f^{-1}(\Aut)$ is instantiated by $\Pre_\Transducer(\Aut)$, the set of tuples $(w_1,\ldots, w_\arity)$ such that there is an accepting run of $\Transducer$ on $w_1$ with output $w'\in \Lang(\Aut)$ when the parameters are equipped with $w_2,\ldots, w_\arity$. To apply Theorem~\ref{thm-generic-dec}, one needs to verify the \prerec{} assumption respectively for different classes of parametric transducers under consideration. 
%While the latter trivially holds as we only consider assertions being regular constraints, the former is not trivial. 
%For the rest of this section, we consider three classes of parametric transducers, viz. two-way transducers, one-way transducers, and reversal-bounded two-way transducers. We shall show that in each case, the pre-image is a recognisable relation with various size bounds $\ell(|\Transducer|, |\Aut|)$. This would yield tight complexity bounds on the path feasibility problem for string manipulating programs $S$ when the string functions therein are given in parametric transducers.  



%the set of tuples $(w, w_1,\ldots, w_k)$ such that there is an accepting run of $\Transducer$ on $w$ with output $w'\in \Lang(\Aut)$ when equipped with the parameters $w_1,\ldots, w_k$. 
 
\subsection{Two-way transducers}\label{sec-2way}

%Let $\Transducer$ be a 2NPT with $k$ parameters $y_1,\cdots, y_k$ and $\Aut$ be an NFA. Then the \emph{preimage} of $\Aut$ w.r.t. $\Transducer$, denoted by $\Pre_\Transducer(\Aut)$, is the set of tuples $(w, w_1,\cdots, w_k)$ such that there is an accepting run of $\Transducer$ on $w$ producing an output $w' \in \Lang(\Aut)$, equipped with the parameters $w_1,\cdots, w_k$.


\begin{lemma}\label{lem-2pt}
The \prerec{} assumption holds for symbolic executions with \PPT{}s, with $\ell(|\Transducer|, |\Aut|) = 2^{\bigO(|\Transducer||\Aut|\log (|\Transducer||\Aut|))}$.
%Let $\Transducer$ be a \PPT{} and $\Aut$ be a conjunctive \FA{} $\Aut$. Then $\Pre_\Transducer(\Aut)$ is a recognisable relation. Moreover, a representation of $\Pre_\Transducer(\Aut)$, whose atom size is bounded by $2^{O(|\Transducer||\Aut|\log (|\Transducer||\Aut|))}$, can be computed effectively. Moreover, each disjunct of the representation can be nondeterministically computed in $2^{O(|\Transducer||\Aut|\log (|\Transducer||\Aut|))}$ space.
\end{lemma}

\begin{proof}
Assume a \PPT{} $\Transducer=(X, Q, q_0, F, \delta)$ with $X = \{x_1,\cdots, x_\arity\}$ and a conjunctive \FA{} $\Aut = ((Q', \delta'), S')$ with $S' \subseteq Q' \times Q'$. %We first introduce some notations. 
Let $\oalphabet = \ialphabet \cup X$.

To compute $\Pre_\Transducer(\Aut)$, we construct a conjunctive \FFA{} %\footnote{Conjunctive \FFA{}s are defined similarly to conjunctive \FA{}s.} 
$\cB_{\Transducer, \Aut, S_{x_1}, \cdots, S_{x_\arity}, q}$, %defined as follows. 
%and $\Aut[S]$ defined below.
%
where  $q \in F$ and  $S_{x_1}, \cdots, S_{x_\arity} \subseteq Q' \times Q'$.   
%The conjunctive \FFA{} $\cB_{\Transducer, \Aut, S_{x_1}, \cdots, S_{x_k}, q}$ is constructed 
Intuitively this is done from $\Transducer$  by checking that there is an accepting run of $T$ on $w$ ending at the state $q$, and simulating an accepting run of $\Aut$ on the output of $\Transducer$. Therefore, each state of $\cB_{\Transducer, \Aut, S_{x_1}, \cdots, S_{x_k}, q}$ is of the form $(q,q') \in Q \times Q'$. 
%In particular, 
When a transition of $\Transducer$ outputs a string $w'$ from $\oalphabet^*$ where some parameter $x_i$ appears (instead of a constant string), for each occurrence of $x_i$ in $w'$, a pair of states $(q_1', q_2')$ will be nondeterministically selected from $S_{x_i}$, and the second component of the state of $\cB_{\Transducer, \Aut, S_{x_1}, \cdots, S_{x_k}, q}$ will be updated from $q_1'$ to $q_2'$ to mimic a possible instantiation of $x_i$.  

%\tl{zhilin, I am not quite clear about the para, $y_1$ is a typo of $x_1$? I wrote a para above; please check}
%(instead of materialising the output), where $S_{x_1}$ is used to jump over $y_1$ when $x_1$ is the output of a transition of $\Transducer$, similarly for $S_{x_2}, \cdots, S_{x_k}$. 


More precisely, $\cB_{\Transducer, \Aut, S_{x_1}, \cdots, S_{x_k}, q}$ is defined as
%\begin{enumerate} 
%\item 
%the conjunctive \FFA{} 
$\Aut'' = ((\controls'', \transrel''), S'')$, where $Q'' = Q \times Q'$, $S'' = \{((q_0, p), (q, p')) \mid (p, p') \in S'\}$, and the transition relation $\transrel''$ comprises the tuples $((q_1, q'_1), a, dir, (q_2, q'_2))$ such that there is $w' \in \oalphabet^*$ satisfying that $(q_1, a, dir, q_2, w') \in \transrel$, and one of the following constraints holds, 
\begin{itemize}
\item $w' \in \ialphabet^*$  and $q'_1 \xrightarrow[\Aut]{w'} q'_2$, 
%
\item suppose $w' = w'_1 x_{i_1} \ldots w'_{r} x_{i_{r}} w'_{r+1}$ such that $r \ge 1$, $w'_j \in \ialphabet^*$ for each $j \in [r+1]$,  and $i_j \in [k]$ for each $j \in [r]$, then there are states $p_0, p_1, \ldots, p_{2r+1} \in \controls'$ satisfying that $p_0 = q'_1$, $p_{2r+1} = q'_2$, and 
$$p_0 \xrightarrow[\Aut]{w'_1} p_1 \xrightarrow{S_{x_{i_1}}} p_2\ \ldots\ p_{2r-2} \xrightarrow[\Aut]{w'_r}p_{2r-1} \xrightarrow{S_{x_{i_r}}} p_{2r} \xrightarrow[\Aut]{w'_{r+1}} p_{2r+1},$$ 
where $p_{2j-1} \xrightarrow{S_{x_{i_j}}} p_{2j}$ means $(p_{2j-1}, p_{2j}) \in S_{x_{i_j}}$ for each $j \in [r]$.
\end{itemize}
%
%
%\item Define $\cB_{\Transducer, \Aut, S_{x_1}, \cdots, S_{x_k}, q}$ as $((Q''', \delta'''), S')$, where $S' = \{(\vec{\rho}_1, \vec{\rho}_2) \mid  (q'_1,q'_2) \in S, \vec{\rho}_1[1] =(q_0,q'_1), \vec{\rho}_2[|\vec{\rho}_2|] = (q, q'_2) \}$.
%\end{enumerate} 
It is easy to see that  the size of $\cB_{\Transducer, \Aut, S_{x_1}, \cdots, S_{x_k}, q}$ is at most $|\Transducer||\Aut|$ and $|S''| = |S'|$.   
%$2^{O(|T||\Aut| \log(|T||\Aut|))}$.   

%For $S \subseteq Q' \times Q'$, let $\Aut[S]$ denote the product of $\Aut(q'_1, \{q'_2\})$ for $(q'_1,q'_2) \in S$ (cf. Section~\ref{sec:prelim}, {Operations of NFAs.}). Since $S$ contains at most $|Q'|^2=|\Aut|^2$ many elements, the size of $\Aut[S]$ is bounded by $|\Aut|^{|\Aut|^2} \approx 2^{O(|\Aut|^2 \log |\Aut|)}$.
\smallskip
\noindent
\emph{Claim.} %For each tuple of strings $(w, w_1,\cdots, w_k)$, 
$(w, w_1,\cdots, w_k) \in \Pre_\Transducer(\Aut)$ iff there are $S_{x_1}, \cdots, S_{x_k} \subseteq Q' \times Q'$ and $q \in F$ such that $w \in \Lang(\cB_{\Transducer, \Aut, S_{x_1}, \cdots, S_{x_k}, q})$ and $w_i \in \Lang(\Aut[S_{x_i}])$ for each $i \in [k]$, where $\Aut[S_{x_i}]=((Q', \delta'), S_{x_i})$. [The proof the claim is given in Appendix~\ref{appendix:sec-2way}.]
\smallskip

Thus, $\Pre_\Transducer(\Aut)$ is a recognisable relation equal to
\[
\bigcup_{S_{x_1}, \cdots, S_{x_k} \subseteq Q' \times Q', q\in F} \Lang(\cB_{\Transducer, \Aut, S_{x_1}, \cdots, S_{x_k},q}) \times \Lang(\Aut[S_{x_1}]) \times \cdots  \times \Lang(\Aut[S_{x_k}]).\]
%

To represent $\Pre_\Transducer(\Aut)$, we transform the conjunctive \FFA{} $\cB_{\Transducer, \Aut, S_{x_1}, \cdots, S_{x_k},q}$ to a union of conjunctive \FA{}s as follows: Transform $\Aut''$ into a one-way transition graph $(\controls''',\delta''')$, where $\controls'''$ are vectors of states of $\Aut''$ of lengths at most $2|\controls''|-1$ (cf. Proposition~\ref{prop-2nfa-nfa}). Then $\Lang(\cB_{\Transducer, S_{x_1}, \cdots, S_{x_k},q})$ is the union of $\Lang(((\controls''',\delta'''), S''))$ for $S'' \subseteq \controls''' \times \controls''',$ which comprises nondeterministically selected pairs $(\vec{\rho}_1, \vec{\rho}_2) \in \controls''' \times \controls'''$, one for each $(p, p') \in S'$, such that $\vec{\rho}_1[1] =(q_0, p)$ and  $\vec{\rho}_2[|\vec{\rho}_2|] = (q, p')$.
%that is computed nondeterministically from $\cB_{\Transducer, S_{x_1}, \cdots, S_{x_k},q}$ and $S'$ as follows:  For each $(p, p') \in S$, nondeterministically select one pair $(\vec{\rho}_1, \vec{\rho}_2) \in \controls''' \times \controls'''$ satisfying $\vec{\rho}_1[1] =(q_0, p)$ and  $\vec{\rho}_2[|\vec{\rho}_2|] = (q, p')$, then put it into $S''$.

Since the size of each conjunctive \FA{} $((\controls''',\delta'''), S'')$ is $ 2^{\bigO(|\Transducer||\Aut|\log (|\Transducer||\Aut|))}$, we conclude that 
the \prerec{} assumption holds for string functions definable in \PPT{}s with $\ell(|\Transducer|, |\Aut|) = 2^{O(|\Transducer||\Aut|\log (|\Transducer||\Aut|))}$.
%$\Pre_\Transducer(\Aut)$ is a recognisable relation and a representation of $\Pre_\Transducer(\Aut)$, whose atom size is bounded by $2^{O(|\Transducer||\Aut|\log (|\Transducer||\Aut|))}$, can be constructed effectively.
\end{proof}

From Lemma~\ref{lem-2pt} and Theorem~\ref{thm-generic-dec}, %by some simple calculation, we deduce that 
we have that, when instantiated by \PPT{}s, %given a string manipulating program $S$ in SSA form where string functions are modelled by \PPT{}s, 
the path feasibility of a string constraint $S$ can be decided in 
%nondeterministic 
%$$O((\rcdim(\varphi)+2)^{\rcdep(\varphi)} (f^{\langle \rcdep(\varphi) \rangle}(|\varphi|, |\psi|))^2 \log f^{\langle \rcdep(\varphi) \rangle}(|\varphi|, |\psi|))$$ 
%space, where $f(j, k) = 2^{jk \log (j k)}$. 
nondeterministic $\tower(\rcdep(S), \bigO(\rcphi(S) \rcpsi(S) \log(\rcphi(S) \rcpsi(S)))$ space (which can be made deterministic space by Savitch's theorem \cite{Savitch70}). Here $\tower(1, k)= 2^k$ and $\tower(j+1, k) = 2^{\tower(j, k)}$. 
After the result we sketch the lower bound.
%We conclude that the path feasibility of such string manipulating programs  can be decided in $\rcdep(S)$-exponential space.
%Moreover, we have a matching lower bound whose proof is sketched below.
%
\begin{theorem} \label{thm:two-way}
Given a symbolic execution $S$ with \PPT{}s, the path feasibility of $S$ can be decided in $\rcdep(S)$-exponential space. Moreover,  for any $\expheight$, path feasibility is
 $\expheight$-$\expspace$-hard.
\end{theorem}

\paragraph{Lower-bound.}
For each $\expheight$ we reduce from a tiling problem that is hard for $\expheight$-$\expspace$.
For this we need to use large numbers that act as indices.
Similar encodings of large numbers appear in the study of higher-order programs (e.g.~\cite{J01,CW07}) except quite different machinery is needed to enforce the encoding.
The complete reduction is given in 
\shortlong{the full version of this article}
          {Appendix~\ref{sec:two-way-lower}},
with some intuition given here.

A \emph{tiling problem} consists of a finite set of tiles $\tiles$ as well as horizontal and vertical tiling relations 
$\hrel, \vrel \subseteq \tiles \times \tiles$.
Given a tiling \emph{corridor} of a certain width, as well as initial and final tiles
$\inittile, \fintile \in \tiles$
the task is to find a tiling where the first (resp. last) tile of the first (resp. last) row is $\inittile$ (resp. $\fintile$), and horizontally (resp. vertically) adjacent tiles $\tile, \tile'$ have
$\tup{\tile, \tile'} \in \hrel$
(resp. $\vrel$).
Corridor width can be considered equivalent to the space of a Turing machine.

Solving a tiling problem of width $n$ can be reduced to checking whether a \FFT\ of size 
$\ap{\bigoh}{n}$ 
can output a specified symbol $\top$.
Equivalently, we could use a \FFA.
A solutions is a word
$t_{1,1} t_{1,2} \cdots t_{1,n} \numsep \cdots \numsep t_{m,1} t_{m,2} \cdots t_{m,n}$
where $\numsep$ separates rows.
The \FFT\ performs $n+1$ passes.
During the first pass it checks that the tiling begins with $\inittile$, ends with $\fintile$, and
$\tup{t_{i,j}, t_{i,j+1}} \in \hrel$
for all
$1 \leq j < n$.
In $n$ more passes we verify that $\vrel$ is obeyed; the $j$th pass verifies the $j$th column.

Now consider two \FFT{}s and a tiling problem of exponential width.
Intuitively, we precede each tile with its column number in $n$ binary bits.
That is
\[
    0\ldots00\ \tile_{1,1}\ %
    0\ldots01\ \tile_{1,2}\ %
    \cdots
    1\ldots11\ \tile_{1,2^n}\ %
    \numsep
    \cdots
    \numsep
    0\ldots00\ \tile_{m,1}\ %
    0\ldots01\ \tile_{m,2}\ %
    \cdots
    1\ldots11\ \tile_{m,2^n} \ .
\]
The first \FFT{} checks the solution similarly to the width $n$ problem, but needs to handle the large width when checking $\vrel$.
For this it will use a second \FFT.
For each column, the first \FFT{} nondeterministically selects all the tiles in this column 
(verifying $\vrel$ on-the-fly).
The addresses of the selected tiles are output to the second \FFT{} which checks that they are equal.
The first \FFT{} goes through a non-deterministic number of such passes and the second \FFT{} enforces that there are $2^n$ of them (in column order).
To do this, the second \FFT{} checks that after the addresses of the $i$-th column are output by the first \FFT{}, then the addresses of the $(i+1)$-th column are output next.
\mat{@Zhilin: is above better?}

To increase the corridor width by another exponential, we need another \FFT.
To reach doubly-exponentially large numbers, we need to precede each tile with a binary sequence of exponential length.
For this we precede each bit with another binary sequence, this time of length $n$.
The first \FFT\ then outputs queries to the second, which outputs queries to the third \FFT, each time removing one exponential.
With $(n+1)$ \FFT, we can search for a solution of a tiling problem over an $n$-fold exponentially wide corridor.



%========================================one-way transducers==========================================

%\subsection{One-way transducers} \label{sec-one-way}


%\begin{lemma}\label{lem-1pt}
%The \prerec{} assumption holds for string functions definable in \PT{}s, with $\ell(|\Transducer|, |\Aut|) = |\Transducer||\Aut|$.
%%Let $\Transducer$ be a 1PT and $\Aut$ be a conjunctive NFA. Then $\Pre_\Transducer(\Aut)$ is a recognisable relation. Moreover, a representation of $\Pre_\Transducer(\Aut)$, whose atom size is bounded by $|\Transducer||\Aut|$, can be computed effectively.
%\end{lemma}

%Lemma~\ref{lem-1pt} can be proved in the same way as Lemma~\ref{lem-2pt}. The only difference is that the construction of $\cB_{\Transducer, \Aut, S_{x_1}, \ldots, S_{x_k}, q}$ is simpler and its size is at most $|\Transducer| |\Aut|$, since the transformation from  conjunctive \FFA{} to conjunctive \FA{} is not needed.

%From Lemma~\ref{lem-1pt} and Theorem~\ref{thm-generic-dec}, we deduce the following result.
%%
%\begin{theorem} \label{thm-1pt}
%%Given an $\straightline[\owpt]$ formula $\varphi \wedge \psi$, 
%The path feasibility problem of string manipulating programs in SSA form where string functions are modelled by \PT{}s is $\expspace$-complete. 
%\end{theorem}

%The lower bound in Theorem~\ref{thm-1pt} is deduced from the result in \cite{LB16} that the path feasibility of straight-line string constraints with concatenation and one-way transducers, which are both special cases of \PT{}s, is $\expspace$-complete.

%====================================

\subsection{$k$-Reversal-bounded two-way transducers} \label{subsec:krb}
 
In practice, many string functions need only a small number of reversals for the reading head, rather than full two-way transducers. 
E.g., the reading head for one-way \PT{} (resp. the string reverse function) needs $0$ (resp. $2$) reversal(s).

Let $k \in \Nat$ a $k$-reversal bounded \PPT{} $\Transducer$ ($k$-\RBPPT{}) is such that in every run of $\Transducer$, the reading direction is reversed at most $k$ times. 
Notice that $k$ must be even since accepting runs must stop on the right end of input strings. 
Evidently, $0$-reversal bounded \PPT{}s are precisely one-way \PT{}s.   
Crucially, string constraints with $k$-\RBPPT{} can be translated to string constraints with only \PT{} and a special transducer $\Transducer_{\sf reverse}$ for the reverse function  (cf. Example~\ref{exmp-ft} in Section~\ref{sec:framework}).  

\begin{proposition} \label{prop:trans}
	Any string constraint $S$ with $k$-\RBPPT{} can be transformed to $S'$ with only \PT{}s and $\Transducer_{\sf reverse}$, and $ \rcdep(S')\in \mathcal{O}(k\cdot \rcdep(S))$. 
\end{proposition}

Proposition~\ref{prop:trans} is shown in
\shortlong{the full version}{Appendix~\ref{app-rb-2pt}}.
Intuitively, a run of a $k$-\RBPPT{} can be split into $k$ runs, each travelling in a single direction.
We can simulate each run with $k$ one-way \PT{}s whose outputs are then concatenated.
Right-to-left runs need to be simulated backwards (left-to-right), and thus the output needs to be reversed before concatenation.


To instantiate the generic procedure, one only needs to verify the  \prerec{} assumption for \PT{}s and $\Transducer_{\sf reverse}$. Indeed, although $\Transducer_{\sf reverse}$ is a two-way transducer, its pre-image does \emph{not} bring an exponential blowup as general \PPT{}s would have. 

\begin{lemma}\label{lem-1pt}
	The \prerec{} assumption holds for string functions definable in \PT{}s or $\Transducer_{\sf reverse}$, with $\ell(|\Transducer|, |\Aut|) = |\Transducer||\Aut|$.
\end{lemma}

Lemma~\ref{lem-1pt} can be proved in the same way as Lemma~\ref{lem-2pt}. The only difference is that the construction of $\cB_{\Transducer, \Aut, S_{x_1}, \ldots, S_{x_k}, q}$ is simpler and its size is at most $|\Transducer| |\Aut|$, since the transformation from  conjunctive \FFA{} to conjunctive \FA{} is not needed. For $\Transducer_{\sf reverse}$, to compute the pre-image one only needs to reverse the direction of each transition of $\Aut$ and swap the initial and final states.   

Combing Theorem~\ref{thm-generic-dec}, Proposition~\ref{prop:trans}, and Lemma~\ref{lem-1pt} together, we have:

\begin{theorem} \label{thm-rb-2pt}
Let  $k \in \Nat$. The path feasibility of string constraints with $k$-\RBPPT{}s is $\expspace$-complete. 
\end{theorem}

The lower bound is from the $\expspace$-hardness of path feasibility of straight-line string constraints with concatenation and one-way (non-parametric) transducers,  both of which are special cases of \PT{}s~\cite{LB16}.
