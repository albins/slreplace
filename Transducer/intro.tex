\section{Introduction}
\label{sec:intro}

Strings are a fundamental data type in virtually all programming languages.
Their generic nature can, however, lead to many subtle programming 
bugs, some with serious security consequences, e.g., cross-site scripting
(XSS) and code injections, both of which are ranked top 10 by Open Web
Application Security Projects (OWASP) in 2017 \cite{owasp17}. 

One effective
method for software testing (and, in particular, automatic test-case 
generation with a good coverage) is based on \emph{symbolic executions} 
\cite{king76} and combinations with dynamic analysis
called \emph{dynamic symbolic execution} \cite{jalangi,DART,EXE,CUTE,KLEE}.
See \cite{symbex-survey} for an excellent survey. Unlike purely random testing,
which runs only \emph{concrete} (i.e. actual) program executions on different 
inputs, the techniques of symbolic executions analyse \emph{static} paths
(also called symbolic executions) through the software system under test.
Since such a path is simply a sequence of assignments and
conditionals/assertions, it can be viewed as a constraint $\varphi$ (over 
appropriate domains corresponding to the data types used in the path) and a 
constraint solver could be invoked on $\varphi$ to check the \emph{feasibility}
of the static path, which can be used for generating inputs that lead to 
an undesirable program behaviour or an exploration of a new part of the
system under test.

Symbolic executions rely on constraint solvers at their core. When the system
under test is a program that manipulates strings, the techiques crucially need
constraint solvers over the string domain (a.k.a. \emph{string solvers}).

%Talk about symbolic execution. Dynamic symbolic execution.
