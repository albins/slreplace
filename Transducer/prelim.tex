%!TEX root = main.tex

\section{Preliminaries}
\label{sec:prelim}

\paragraph{General Notations.}
Let $\mathbb{Z}$ and $\Nat$ denote the set of integers and natural numbers
respectively. For $i \le j \in \Nat$, let $[i, j]$ denote $\{i,i+1,\ldots,j\}$ and
$[i] = [0, i]$. \zhilin{The only place I notice that $[i] = [0,i]$ is used is the definition of semantics of \FFA{}(T). In most places, $[i]$ is used for $\{1,\ldots,i\}$. I would suggest to define $[i]=\{1,\ldots,i\}$, and if $[0,i]$ should be used, then just use $[0,i]$ there. I will assume this notation through the paper.} For a vector
$\vec{x}=(x_1,\cdots, x_n)$, let $|\vec{x}|$ denote the length of $\vec{x}$
(i.e., $n$) and  $\vec{x}[i]$ denote $x_i$ for each $i \in [n]$. For a set
$S$, we use $S^*$ (resp.~$S^+$) to denote the set of all finite (resp.~finite
and nonempty) sequences over $S$. We use $\epsilon$ for the empty sequence.

%=======================================================================
\OMIT{
\paragraph{Graph-Theoretical Notation.} \tl{not sure whether it is really needed; will see}
A DAG (\emph{directed acyclic graph}) $G$ is a finite directed graph $(V, E)$ with
no directed cycles, where $V$ (resp.~$E \subseteq V \times V$) is a set of vertices (resp.~edges).
%. That is, each DAG consists of finitely many vertices and edges, with each edge directed from one vertex to another, such that there is no way to start at any vertex $\mathit{v}$ and follow a consistently-directed sequence of edges that eventually loops back to $\mathit{v}$ again.
Equivalently, a DAG is a directed graph that has a topological ordering, which
is a sequence of the vertices such that every edge is directed from an earlier
vertex to a later vertex in the sequence. An edge $(\mathit{v},\mathit{v'})$ in
$G$ is called an \emph{incoming} edge of $\mathit{v'}$ and an \emph{outgoing}
edge of $\mathit{v}$. If $(\mathit{v},\mathit{v'}) \in E$, then $\mathit{v'}$ is
called a \emph{successor} of $\mathit{v}$ and $\mathit{v}$ is called a
\emph{predecessor} of $\mathit{v'}$. A \emph{path} $\pi$ in $G$ is a sequence
$\mathit{v}_0 \mathit{e}_1 \mathit{v}_1 \cdots \mathit{v}_{n-1} \mathit{e}_n
\mathit{v}_n$ such that for each $i \in [n]$, we have $\mathit{e}_i =
(\mathit{v}_{i-1},\mathit{v}_i) \in E$. The \emph{length} of the path $\pi$
%$\mathit{v}_0 e_1 \mathit{v}_1 \cdots \mathit{v}_{n-1} e_n \mathit{v}_n$ in $G$
is the number $n$ of edges in $\pi$. If there is a path from
$\mathit{v}$ to $\mathit{v'}$ (resp. from $\mathit{v'}$ to $\mathit{v}$) in $G$,
then $\mathit{v'}$ is said to be \emph{reachable} (resp. \emph{co-reachable})
from $\mathit{v}$ in $G$. If $\mathit{v}$ is reachable from $\mathit{v'}$ in
$G$, then $\mathit{v'}$ is also called an \emph{ancestor} of $\mathit{v}$ in
$G$. In addition, an edge $(\mathit{v'},\mathit{v''})$ is said to be reachable
(resp. co-reachable) from $\mathit{v}$ if $\mathit{v'}$ is reachable from $\mathit{v}$ (resp. $\mathit{v''}$ is co-reachable from $\mathit{v}$). The \emph{in-degree} (resp. \emph{out-degree}) of a vertex $\mathit{v}$ is the number of incoming (resp. outgoing) edges of $\mathit{v}$.
%A vertex $\mathit{v}$ in $G$ is said to be a \emph{join} vertex if the in-degree of $\mathit{v}$ is at least two.
%A DAG $G$ is called an \emph{arborescence} if there is a vertex $v_0$ such that all the vertices are reachable from $v_0$ in $G$, in addition, there are no join vertices in $G$.
A \emph{subgraph} $G'$ of $G=(V,E)$ is a directed graph $(V', E')$ with
$V' \subseteq V$ and $E' \subseteq E$. Let $G'$ be a subgraph of $G$. Then $G \setminus G'$ is the graph obtained from $G$ by removing all the edges in $G'$.
}
%=============================================================================

\paragraph{Automata.} We review some necessary background from automata theory;
for more, see \cite{Kozen-automata,HU79}. Let $\ialphabet$ be a finite set (called
\defn{alphabet}). We usually define  $\overline{\ialphabet} := \ialphabet \cup \{\EndLeft,\EndRight\}$, where $\EndLeft,\EndRight\notin \Sigma$ are tape end markers for 2-way automata/transducer models. \tl{I made  $\overline{\ialphabet}$ here as it used in multiple places}
 A sequence over $\ialphabet$ is called a \defn{string}. A \defn{language} is set of strings over $\ialphabet$.
We will review regular languages and necessary models of finite-state automata below.

\begin{definition}[Two-way finite-state automata] \label{def:2nfa}
    A \emph{(nondeterministic) two-way finite-state automaton}
(\FFA{}) over a finite alphabet $\ialphabet$ is a tuple $\Aut =
(\ialphabet, \EndLeft, \EndRight, \controls, q_0, \finals, \transrel)$ where 
    $\controls$ is a finite set of 
    states, $\EndLeft$ (resp.~$\EndRight$) a left (resp.~right) input tape end 
    marker, $q_0\in \controls$ is
the initial state, $\finals\subseteq \controls$ is a set of final states, and 
    $\transrel$ is the
transition relation  $\transrel\subseteq \controls \times 
    \overline{\ialphabet}\times \{\Left, \Stay, \Right\}\times \controls$. 
    %with $\overline{\ialphabet} := \ialphabet \cup \{\EndLeft,\EndRight\}$.
    Here, we assume %$\EndLeft, \EndRight \notin \ialphabet$, and 
    that
    there are no transitions that take the head of the tape past the left/right
    end marker (i.e.~$(p,\EndLeft,\Left,q), (p,\EndRight,\Right,q) \notin
    \transrel$ for every $p, q \in \controls$).
%\tl{not sure you want to use $\leftarrow, \rightarrow$?}
%    \anthony{Just defined macros. Please use generic macros in the definitions
%    guys.}

A (nondeterministic one-way) finite-state automaton (\FA{})
is a \FFA{} such that $\transrel \subseteq \controls \times \overline{\ialphabet} \times
    \{\Right,\Stay\} \times \controls $.
\end{definition}
In the sequel, whenever understood we will only tacitly mention $\ialphabet$, 
$\EndLeft$, and $\EndRight$ in $\Aut$. 

%A \emph{nondeterministic finite automaton} (NFA) $\cA$ on $\Sigma$ is a tuple $(Q, \delta, q_0, F)$, where $Q$ is a finite set of \emph{states}, $q_0 \in Q$ is the \emph{initial} state, $F \subseteq Q$ is the set of \emph{final} states, and $\delta \subseteq Q \times \Sigma \times Q$ is the \emph{transition relation}.

The notion of runs of \FFA{} on an input string is exactly the same as that of
Turing machines on a read-only input tape. More precisely, for a string 
$w = a_1 \dots a_n$, a \emph{run} of $\Aut$ on $w$
%(with $a_0 = \EndLeft$ and $a_{n+1} = \EndRight$)
is a sequence of pairs $(q_0,i_0),\ldots, (q_m,i_m) \in \controls \times [0, n+1]$ 
defined as follows. Let $a_0 = \EndLeft$ and $a_{n+1} = \EndRight$. The
following conditions then have to be satisfied:
\begin{itemize}
    \item $i_0 = 0$, and
    \item for every $j \in [m-1]$, we have $(q_j,a_{i_j}, dir, q_{j+1}) \in
        \transrel$ and $i_{j+1} = i_j + dir$ for some $dir \in  \{\Left, \Stay, \Right\}$.
\end{itemize}
The run is said to be \defn{accepting} if $i_m = n+1$ and $q_m \in \finals$.
A string $w$ is \defn{accepted} by $\Aut$ if there is an accepting run of
$\Aut$ on $w$. The set of strings accepted by $\Aut$ is denoted by $\Lang(\Aut)$,
a.k.a., the language \defn{recognised} by $\Aut$.
%Since we deal with computational complexity in the sequel, we define
The \defn{size} $|\Aut|$ of $\Aut$ is defined to be $|\controls|$; this will
be needed when we talk about computational complexity.

For convenience, we will also refer to an \FA{} without initial and final states, that is, a pair $(Q, \delta)$, as a \emph{transition graph}.

\OMIT{
state sequence $q_0 \dots q_n$ such that for each $i \in [n]$, $(q_{i-1}, a_i, q_i) \in \delta$. A run $q_0 \dots q_n$ is \emph{accepting} if $q_n \in F$. A string $w$ is \emph{accepted} by $\cA$ if there is an accepting run of $\cA$ on $w$. We use $\Ll(\cA)$ to denote the language defined by $\cA$, that is, the set of strings accepted by $\cA$. We will use $\cA, \cB, \cdots$ to denote NFAs.
%An NFA $\cA$ is \emph{deterministic} if for each $(q, \sigma) \in Q \times \Sigma$, there is at most one $q' \in Q$ such that $(q, a, q') \in \delta$. An NFA $\cA$ is \emph{complete} if for each $(q, \sigma) \in Q \times \Sigma$, there is at least one $q' \in Q$ such that $(q, a, q') \in \delta$. We assume that all NFA considered in this paper are complete.  An NFA $\cA$ is \emph{unambiguous} if for each string $w$, there is \emph{at most one accepting} run of $\cA$ on $w$.
For a string $w= a_1 \dots a_n$, we also use the notation $q_1 \xrightarrow[\cA]{w} q_{n+1}$ to denote the fact that there are $q_2,\dots, q_n \in Q$ such that for each $i \in [n]$, $(q_i, a_i, q_{i+1}) \in \delta$.  For an NFA $\cA=(Q, \delta, q_0, F)$ and $q, q' \in Q$, we use $\cA(q,q')$ to denote the NFA obtained from $\cA$ by changing the initial state to $q$ and the set of final states to $\{q'\}$. The \emph{size} of an NFA $\cA=(Q, \delta, q_0, F)$, denoted by $|\cA|$, is defined as $|Q|$, the number of states.

For convenience, we will also call an NFA without initial and final states, that is, a pair $(Q, \delta)$, as a \emph{transition graph}.
}

\FFA{} and \FA{} recognise precisely the same class of languages, i.e., 
\emph{regular languages}. The following result is standard and can be found in textbooks on automata theory
(e.g. \cite{Kozen-automata}). 

\begin{proposition}[\cite{Kozen-automata}]\label{prop-2nfa-nfa}
	Every \FFA{} $\Aut$ is equivalent to an \FA{} of size $2^{\bigO(|\Aut| \log |\Aut|)}$. 
%	Moreover, the equivalent
%    NFA can be computed in exponential time. 
    Moreover, every \FA{} can be transformed in polynomial time into an $\varepsilon$-free \FA{}, that is, an \FA{} where $\transrel\subseteq \controls \times \overline{\ialphabet} \times   \{\Right\} \times \controls $.
\end{proposition}

In the rest of this paper, \FA{}s refer to $\varepsilon$-free \FA{}s where directions in transitions are omitted. Moreover, for simplicity of notations, in \FA{}s, we omit the two end markers $\EndLeft, \EndRight$ and assume that $\transrel \subseteq \controls \times \ialphabet \times    \{\Right,\Stay\} \times \controls$. \zhilin{I assume that the two end markers are removed in FAs.}

\paragraph{Operations of \FA{}s.} For an FA $\Aut=(Q, q_0, F, \delta)$, $q \in Q$ and $P \subseteq Q$, we use $\Aut(q, P)$ to denote the FA $(Q, q, P, \delta)$, that is, the FA obtained from $\Aut$ by changing the initial state and the set of final states to $q$ and $P$ respectively. We use $q \xrightarrow[\Aut]{w} q'$ to denote the fact that a string $w$ is accepted by $\Aut(q, \{q'\})$. 


Given two \FA{}s $\Aut_1 = (Q_1, q_{0,1}, F_{1}, \delta_1)$ and $\Aut_2 = (Q_2, q_{0,2}, F_2, \delta_2)$, the \emph{product} of $\Aut_1$ and $\Aut_2$, denoted by $\Aut_1 \times \Aut_2$, is defined as $(Q_1 \times Q_2, (q_{0,1}, q_{0,2}), F_1 \times F_2, \delta_1 \times \delta_2)$, where $\delta_1 \times \delta_2$ is the set of tuples $((q_1,q_2), a, (q'_1, q'_2))$ such that $(q_1, a, q'_1) \in \delta_1$ and $(q_2, a, q'_2) \in \delta_2$. Evidently, we have $\Lang(\Aut_1 \times \Aut_2) = \Lang(\Aut_1) \cap \Lang(\Aut_2)$.


\anthony{I've removed the ``regular languages'' paragraph because it seems
unnecessary. I recommend that operation on sets can be defined in General
Notation above on general sets.}
%\tl{Other relevant models such as SST, when appropriate, will be put here.}

\zhilin{Let us fix the abbreviations here: \\
--\FFA{}(2FT): two-way finite-state automata(transducer),\\
--FA(FT): finite-state automata(transducer),\\
--2PT: two-way parametric transducer,\\
--PT: parametric transducer,\\
--2SA(2ST): two-way symbolic automata(transducer),\\
--SA(ST): symbolic automata(transducer) \\
--2SPT: two-way symbolic parametric transducer,\\
--SPT: symbolic parametric transducer 
}
\anthony{Can you please introduce macros?}

 
%===================================================================
 
\OMIT{
\paragraph{Regular Languages.}
Fix a finite \emph{alphabet} $\Sigma$. Elements in $\Sigma^*$ are called \emph{strings}. Let $\varepsilon$ denote the empty string and  $\Sigma^+ = \Sigma^* \setminus \{\varepsilon\}$. We will use $a,b,\cdots$ to denote letters from $\Sigma$ and $u, v, w, \cdots$ to denote strings from $\Sigma^*$. For a string $u \in \Sigma^*$, let $|u|$ denote the \emph{length} of $u$ (in particular, $|\varepsilon|=0$). A \emph{position} of a nonempty string $u$ of length $n$ is a number $i \in [n]$ (Note that the first position is $1$, instead of  0). In addition, for $i \in [|u|]$, let $u[i]$ denote the $i$-th letter of $u$.
For two strings $u_1, u_2$, we use $u_1 \cdot u_2$ to denote the \emph{concatenation} of $u_1$ and $u_2$, that is, the string $v$ such that $|v|= |u_1| + |u_2|$ and for each $i \in [|u_1|]$, $v[i]= u_1[i]$ and for each $i \in |u_2|$, $v[|u_1|+i]=u_2[i]$. Let $u, v$ be two strings. If $v = u \cdot v'$ for some string $v'$, then $u$ is said to be a \emph{prefix} of $v$. In addition, if $u \neq v$, then $u$ is said to be a \emph{strict} prefix of $v$. If $u$ is a prefix of $v$, that is, $v = u \cdot v'$ for some string $v'$, then
we use $u^{-1} v$ to denote $v'$. In particular, $\varepsilon^{-1} v = v$.

A \emph{language} over $\Sigma$ is a subset of $\Sigma^*$. We will use $L_1, L_2, \dots$ to denote languages. For two languages $L_1, L_2$, we use $L_1 \cup L_2$ to denote the union of $L_1$ and $L_2$, and $L_1 \cdot L_2$ to denote the concatenation of $L_1$ and $L_2$, that is, the language $\{u_1 \cdot u_2 \mid u_1 \in L_1, u_2 \in L_2\}$. For a language $L$ and $n \in \Nat$, we define $L^n$, the \emph{iteration} of $L$ for $n$ times, inductively as follows: $L^0=\{\varepsilon\}$ and $L^{n} =L \cdot L^{n-1}$ for $n > 0$. We also use $L^*$ to denote the iteration of $L$ for arbitrarily many times, that is, $L^* = \bigcup \limits_{n \in \Nat} L^n$. Moreover, let $L^+ = \bigcup \limits_{n \in \Nat \setminus \{0\}} L^n$.

\begin{definition}[Regular expressions $\regexp$]
	\[e \eqdef \emptyset \mid \varepsilon \mid a \mid e + e \mid e \concat e \mid e^*, \mbox{ where } a \in \Sigma. \]
	Since $+$ is associative and commutative, we also write $(e_1 + e_2) + e_3$ as $e_1 + e_2 + e_3$ for brevity. We use the abbreviation $e^+ \equiv e \concat e^*$. Moreover, for $\Gamma = \{a_1, \cdots, a_n\}\subseteq \Sigma$, we use the abbreviations $\Gamma \equiv a_1 + \cdots + a_n$ and $\Gamma^\ast \equiv (a_1 + \cdots + a_n)^\ast$.
\end{definition}
We define $\Ll(e)$ to be the language defined by $e$, that is, the set of strings that match $e$, inductively as follows: $\Ll(\emptyset) =\emptyset$,
%\begin{itemize}
%\item
$\Ll(\varepsilon) =\{\varepsilon\}$,
%
%\item
$\Ll(a)= \{a\}$,
%
%\item
$\Ll(e_1 + e_2) = \Ll(e_1) \cup \Ll(e_2)$,
%
%\item
$\Ll(e_1 \concat e_2) = \Ll(e_1) \cdot \Ll(e_2)$,
%
%\item
$\Ll(e_1^*)=(\Ll(e_1))^*$.
%\end{itemize}
In addition, we use $|e|$ to denote the number of symbols occurring in $e$.

%A \emph{nondeterministic finite automaton} (NFA) $\cA$ on $\Sigma$ is a tuple $(Q, \delta, q_0, F)$, where $Q$ is a finite set of \emph{states}, $q_0 \in Q$ is the \emph{initial} state, $F \subseteq Q$ is the set of \emph{final} states, and $\delta \subseteq Q \times \Sigma \times Q$ is the \emph{transition relation}. For a string $w = a_1 \dots a_n$, a \emph{run} of $\cA$ on $w$ is a state sequence $q_0 \dots q_n$ such that for each $i \in [n]$, $(q_{i-1}, a_i, q_i) \in \delta$. A run $q_0 \dots q_n$ is \emph{accepting} if $q_n \in F$. A string $w$ is \emph{accepted} by $\cA$ if there is an accepting run of $\cA$ on $w$. We use $\Ll(\cA)$ to denote the language defined by $\cA$, that is, the set of strings accepted by $\cA$. We will use $\cA, \cB, \cdots$ to denote NFAs.
%%An NFA $\cA$ is \emph{deterministic} if for each $(q, \sigma) \in Q \times \Sigma$, there is at most one $q' \in Q$ such that $(q, a, q') \in \delta$. An NFA $\cA$ is \emph{complete} if for each $(q, \sigma) \in Q \times \Sigma$, there is at least one $q' \in Q$ such that $(q, a, q') \in \delta$. We assume that all NFA considered in this paper are complete.  An NFA $\cA$ is \emph{unambiguous} if for each string $w$, there is \emph{at most one accepting} run of $\cA$ on $w$.
%For a string $w= a_1 \dots a_n$, we also use the notation $q_1 \xrightarrow[\cA]{w} q_{n+1}$ to denote the fact that there are $q_2,\dots, q_n \in Q$ such that for each $i \in [n]$, $(q_i, a_i, q_{i+1}) \in \delta$.  For an NFA $\cA=(Q, \delta, q_0, F)$ and $q, q' \in Q$, we use $\cA(q,q')$ to denote the NFA obtained from $\cA$ by changing the initial state to $q$ and the set of final states to $\{q'\}$. The \emph{size} of an NFA $\cA=(Q, \delta, q_0, F)$, denoted by $|\cA|$, is defined as $|Q|$, the number of states. For convenience, we will also call an NFA without initial and final states, that is, a pair $(Q, \delta)$, as a \emph{transition graph}.

It is well-known (e.g. see \cite{HU79}) that regular expressions and \FA{}s are
expressively equivalent, and generate precisely all \emph{regular languages}.
In particular, from a regular expression, an equivalent \FA{} can be constructed
in linear time. Moreover, regular languages are closed under Boolean
operations, i.e., union, intersection, and complementation.
In particular, given two \FA{} $\cA_1=(Q_1, \delta_1, q_{0,1}, F_1)$ and
$\cA_2=(Q_2, \delta_2, q_{0,2}, F_2)$ on $\Sigma$, the intersection $\Ll(\cA_1)
\cap \Ll(\cA_2)$ is recognised by the \emph{product automaton} $\cA_1 \times
\cA_2$ of $\cA_1$ and $\cA_2$ defined as $(Q_1 \times Q_2, \delta, (q_{0,1}, q_{0,2}), F_1 \times F_2)$, where $\delta$ comprises the transitions $((q_1, q_2), a, (q'_1, q'_2))$ such that $(q_1, a, q'_1) \in \delta_1$ and $(q_2, a, q'_2) \in \delta_2$.
}
%==========================================================================================
