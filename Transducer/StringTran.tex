\documentclass{llncs}


%% Some recommended packages.
\usepackage{booktabs}   %% For formal tables:
                        %% http://ctan.org/pkg/booktabs
%\usepackage{subcaption} %% For complex figures with subfigures/subcaptions
                        %% http://ctan.org/pkg/subcaption
\usepackage{latexsym}
\usepackage{setspace}
\usepackage{cancel}
\usepackage{listings}
\usepackage{graphicx}
\usepackage{appendix}
\usepackage{amssymb}
\usepackage{stmaryrd}
\usepackage{amsmath}
\usepackage{leftidx}
\usepackage{mathtools}
\usepackage{paralist}
\usepackage{color}
\usepackage{mathrsfs}
\usepackage{tikz}
%\usepackage[draft]{minted}
\usetikzlibrary{shapes}
\usepackage[linesnumbered,ruled]{algorithm2e}

%==========================================================
%!TEX root = popl2018.tex

\newcommand{\set}[1]{\{ #1 \}}
\newcommand{\sequence}[2]{(#1, \ldots, #2)}
\newcommand{\couple}[2]{(#1,#2)}
\newcommand{\pair}[2]{(#1,#2)}
\newcommand{\triple}[3]{(#1,#2,#3)}
\newcommand{\quadruple}[4]{(#1,#2,#3,#4)}
\newcommand{\tuple}[2]{(#1,\ldots,#2)}
\newcommand{\Nat}{\ensuremath{\mathbb{N}}}
\newcommand{\Rat}{\ensuremath{\mathbb{Q}}}
\newcommand{\Rea}{\ensuremath{\mathbb{R}}}
\newcommand{\Zed}{\ensuremath{\mathbb{Z}}}
%\newcommand{\true}{\top}
%\newcommand{\false}{\perp}
\newcommand{\bottom}{\perp}
%% \newcommand{\powerset}[1]{{\cal P}(#1)}
\newcommand{\npowerset}[2]{{\cal P}^{#1}(#2)}
\newcommand{\finitepowerset}[1]{{\cal P}_f(#1)}
\newcommand{\level}[2]{L_{#1}(#2)}
\newcommand{\card}[1]{\mbox{card}(#1)}
\newcommand{\range}[1]{\mathtt{ran}(#1)}
\newcommand{\astring}{s}

\newcommand{\Cc}{\mathcal{C}}


\newcommand {\notof}{\ensuremath{\neg}}
\newcommand {\myand}{\ensuremath{\wedge}}
\newcommand {\myor}{\ensuremath{\vee}}
\newcommand {\mynext}{\mbox{{\sf X}}}
\newcommand {\until}{\mbox{{\sf U}}}
\newcommand {\sometimes}{\mbox{{\sf F}}}
\newcommand {\previous}{\mynext^{-1}}
\newcommand {\since}{\mbox{{\sf S}}}
\newcommand {\fminusone}{\mbox{{\sf F}}^{-1}}
\newcommand {\everywhere}[1]{\mbox{{\sf Everywhere}}(#1)}



\newcommand{\aatomic}{{\rm A}}
\newcommand{\aset}{X}
\newcommand{\asetbis}{Y}
\newcommand{\asetter}{Z}

\newcommand{\avarprop}{p}
\newcommand{\avarpropbis}{q}
\newcommand{\avarpropter}{r}
\newcommand{\varprop}{{\rm PROP}} % Set of atomic propositions (for a given logic)

% formulae

\newcommand{\aformula}{\astateformula} % a formula
\newcommand{\aformulabis}{\astateformulabis} % another formula (when at least 2 are present)
\newcommand{\aformulater}{\astateformulater} % another formula (when at least 3 are present)
\newcommand{\asetformulae}{X}
\newcommand{\subf}[1]{sub(#1)}

\newcommand{\aautomaton}{{\mathbb A}}
\newcommand{\aautomatonbis}{{\mathbb B}}

\newcommand {\length}[1] {\ensuremath{|#1|}}



% Equivalences
\newcommand{\egdef}{\stackrel{\mbox{\begin{tiny}def\end{tiny}}}{=}} % =def=
\newcommand{\eqdef}{\stackrel{\mbox{\begin{tiny}def\end{tiny}}}{=}} % =def=
\newcommand{\equivdef}{\stackrel{\mbox{\begin{tiny}def\end{tiny}}}{\equivaut}} % <=def=>
\newcommand{\equivaut}{\;\Leftrightarrow\;}

\newcommand{\ainfword}{\sigma}

\newcommand{\amap}{\mathfrak{f}}
\newcommand{\amapbis}{\mathfrak{g}}

\newcommand{\step}[1]{\xrightarrow{\!\!#1\!\!}}
\newcommand{\backstep}[1]{\xleftarrow{\!\!#1\!\!}}

\newcommand {\aedge}[1] {\ensuremath{\stackrel{#1}{\longrightarrow}}}
\newcommand {\aedgeprime}[1] {\ensuremath{\stackrel{#1}{\longrightarrow'}}}
\newcommand {\afrac}[1] {\ensuremath{\mathit{frac}(#1)}}
\newcommand {\cl}[1] {\ensuremath{\mathit{cl}(#1)}}
\newcommand {\sfc}[1] {\ensuremath{\mathit{sfc}(#1)}}
\newcommand {\dunion} {\ensuremath{\uplus}}
\newcommand {\edge} {\ensuremath{\longrightarrow}}
\newcommand {\emptyword}{\ensuremath{\epsilon}}
\newcommand {\floor}[1] {\ensuremath{\lfloor #1 \rfloor}}
\newcommand {\intersection} {\ensuremath{\cap}}
\newcommand {\union} {\ensuremath{\cup}}
\newcommand {\vals}[2] {\ensuremath{\mathit{val}_{#2}(#1)}}



\newcommand {\pspace} {\textsc{pspace}}
\newcommand {\nlogspace} {\textsc{nlogspace}}
\newcommand {\logspace} {\textsc{logspace}}
\newcommand {\expspace} {\textsc{expspace}}
\newcommand {\np} {\textsc{np}}
\newcommand {\threeexptime} {\textsc{3exptime}}
\newcommand {\polytime} {\textsc{p}}
\newcommand{\twoexpspace}{\textsc{2expspace}}
\newcommand{\threeexpspace}{\textsc{3expspace}}
\newcommand {\nexptime} {\textsc{nexptime}}



\newcommand{\aalphabet}{\Sigma}     % an alphabet, A is already used for atoms
\newcommand{\aword}{\mathfrak{u}}
\newcommand{\awordbis}{\mathfrak{v}}



\newcommand{\aassertion}{P}
\newcommand{\aassertionbis}{Q}
\newcommand{\aexpression}{e}
\newcommand{\aexpressionbis}{f}
\newcommand{\avariable}{\mathtt{x}}
\newcommand{\uniquevar}{\mathtt{u}}
\newcommand{\uniquevarbis}{\mathtt{v}}
\newcommand{\avariablebis}{\mathtt{y}}
\newcommand{\avariableter}{\mathtt{z}}
\newcommand{\nullconstant}{\mathtt{null}}
\newcommand{\nilvalue}{nil}
\newcommand{\emptyconstant}{\mathtt{emp}}
\newcommand{\infheap}{\mathtt{inf}}
\newcommand{\saturated}{\mathtt{Saturated}}

\newcommand{\astateformula}{\phi}
\newcommand{\astateformulabis}{\psi}
\newcommand{\astateformulater}{\varphi}
%%
\newcommand{\separate}{\ast}
\newcommand{\sep}{\separate}
\newcommand{\size}{\mathtt{size}}
\newcommand{\sizeeq}[1]{\mathtt{size} \ = \ #1}
\newcommand{\alloc}[1]{\mathtt{alloc}(#1)}
\newcommand{\allocb}[2]{\mathtt{alloc}^{-1}[#2](#1)}
\newcommand{\isol}[1]{\mathtt{isoloc}(#1)}
\newcommand{\icell}{\mathtt{isocell}}
\newcommand{\malloc}{\mathtt{malloc}}
\newcommand{\cons}{\mathtt{cons}}
\newcommand{\new}{\mathtt{new}}
\newcommand{\free}[1]{\mathtt{free} \ #1}
\newcommand{\maxform}[1]{\mathtt{maxForms}(#1)}
\newcommand{\locations}[1]{\mathtt{loc}(#1)}
\newcommand{\values}{\mathtt{Val}}
\newcommand{\aheap}{\mathfrak{h}}
\newcommand{\avaluation}{\mathfrak{V}}
\newcommand{\heaps}{\mathcal{H}}
\newcommand{\astore}{\mathfrak{s}}
\newcommand{\stores}{\mathcal{S}}
\newcommand{\amodel}{\mathfrak{M}}
\newcommand{\alabel}{\ell}

\newcommand{\aprogram}{\mathtt{PROG}}
\newcommand{\programs}{\mathtt{P}}
\newcommand{\ctprograms}{\programs^{ct}}
\newcommand{\aninstruction}{\mathtt{instr}}
\newcommand{\ainstruction}{\mathtt{instr}}
\newcommand{\instructions}{\mathtt{I}}
\newcommand{\aguard}{\ensuremath{g}}
\newcommand{\guards}{\ensuremath{G}}
\newcommand{\domain}[1]{\mathtt{dom}(#1)}
\newcommand{\memory}{\stores\times\heaps}
\newcommand{\skipinstruction}{\mathtt{skip}}

\newcommand{\execution}{\mathtt{comp}}
\newcommand{\aux}{\mathtt{embd}}
\newcommand{\runof}{run}
\newcommand{\anexecution}{e}


\newcommand{\aletter}{\ensuremath{a}}
\newcommand{\aletterbis}{\ensuremath{b}}
\newcommand{\alocation}{\mathfrak{l}}

\newcommand{\pointsl}[1]{\stackrel{#1}{\hookrightarrow}}
\newcommand{\ppointsl}[1]{\stackrel{#1}{\mapsto}}
\newcommand{\ourhook}[1]{\stackrel{#1}{\hookrightarrow}}
\newcommand{\ltrue}{{\sf true}}
\newcommand{\lfalse}{{\sf false}}


\newcommand{\variables}{\mathtt{FVAR}}
\newcommand{\pvariables}{\mathtt{PVAR}}
\newcommand{\secvariables}{\mathtt{SVAR}}
\newcommand{\logique}[1]{\mathtt{FO}(#1)}



\newcommand{\atranslation}{\mathfrak{t}}
\newcommand{\nbpred}[1]{\widetilde{\sharp #1}}
\newcommand{\nbpredstar}[1]{\widetilde{\sharp #1}^{\star}}
\newcommand{\isolated}{\mathtt{isol}}
\newcommand{\stdmarks}{\mathtt{envir}}
\newcommand{\relation}[1]{\mathtt{relation}_{#1}}
\newcommand{\freevar}{\mathtt{FV}}
\newcommand{\notonmark}{\mathtt{notonenv}}
\newcommand{\InVal}[1]{\mathtt{InVal}\!\left(#1\right)}
\newcommand{\NotOnEnv}[1]{\mathtt{NotOnEnv}\!\left(#1\right)}
\newcommand{\PartOfVal}[1]{\mathtt{PartOfVal}\!\left(#1\right)}
%\newcommand{\nbpreds}[3]{\sharp #1 \geq #2}
\newcommand{\defstyle}[1]{{\emph{#1}}}

\newcommand{\cut}[1]{}
\newcommand{\interval}[2]{[#1,#2]}
\newcommand{\buniquevar}{\overline{\uniquevar}}
\newcommand{\bbuniquevar}{\overline{\overline{\uniquevar}}}
\newcommand{\magicwand}{\mathop{\mbox{$\mbox{$-~$}\!\!\!\!\ast$}}}
\newcommand{\wand}{\magicwand}
\newcommand{\septraction}{\stackrel{\hsize0pt \vbox to0pt{\vss\hbox to0pt{\hss\raisebox{-6pt}{\footnotesize$\lnot$}\hss}\vss}}{\magicwand}}
%% \newcommand{\reach}{\mathtt{reach}}
\mathchardef\mhyphen="2D % hyphen while in math mode

\newcommand{\adataword}{\mathfrak{dw}}
\newcommand{\adatum}{\mathfrak{d}}

\newcommand{\collectionknives}{\mathtt{ks}}
\newcommand{\collectionknivesfork}[1]{\mathtt{ksfs}_{=#1}}
\newcommand{\collectionknivesforks}{\mathtt{ksfs}}
\newcommand{\collectionkniveslargeforks}{\mathtt{kslfs}}


\newcommand{\acounter}{\mathtt{C}}

\newcommand{\fotwo}[3]{{\mbox{FO2}_{#1,#2}(#3)}}
\newcommand{\mtrans}[1]{t\!\left(#1\right)^{\Box}}
\newcommand{\mbtrans}[2]{\mtrans{#2}_{#1}}


\newcommand{\alogic}{\mathfrak{L}}


\newcommand{\semantics}[1]{\ensuremath{[ #1 ]}}


\newcommand{\adomino}{\mathfrak{d}}
\newcommand{\atile}{\mathfrak{d}}
\newcommand{\atiling}{\mathfrak{t}}

\newcommand{\hori}{\mathtt{h}}
\newcommand{\verti}{\mathtt{v}}
\newcommand{\domi}{\mathtt{d}}

\newcommand{\cpyrel}{\mathfrak{cp}}

\newcommand{\cntcmp}{\mathfrak{C}}

\newcommand{\heapdag}{\mathfrak{G}}

\newcommand{\onmainpath}{\mathtt{mp}}

\newcommand{\tree}{\mathtt{tree}}

%\newcommand{\tile}{\mathtt{tile}}

\newcommand{\type}{\mathtt{type}}

\newcommand{\ptype}{\mathtt{ptype}}

\newcommand{\exttype}{\mathtt{exttype}}

\newcommand{\anctypes}{\mathtt{AncTypes}}

\newcommand{\destypes}{\mathtt{DesTypes}}

\newcommand{\inctypes}{\mathtt{IncTypes}}

\newcommand{\treeic}{\mathtt{treeIC}}

\newcommand{\trs}{\mathfrak{trs}}


\newcommand{\nin}{\not \in}
\newcommand{\cupplus}{\uplus}
\newcommand{\aunarypred}{\mathtt{P}}


\newcommand{\hide}[1]{}

\newcommand{\eval}[2]{\llbracket#1\rrbracket_{#2}}
\newcommand\cur{\mathsf{cur}}
\newcommand\dom{\mathsf{dom}}
\newcommand\rng{\mathsf{rng}}

\newcommand\dd{\mathbb{D}}
\newcommand\nat{\mathbb{N}}


\newcommand\cA{\mathcal{A}}
\newcommand\cB{\mathcal{B}}
\newcommand\cC{\mathcal{C}}
\newcommand\cE{\mathcal{E}}
\newcommand\cG{\mathcal{G}}
\newcommand\Ll{\mathcal{L}}
\newcommand\cM{\mathcal{M}}
\newcommand\cP{\mathcal{P}}
\newcommand\cR{\mathcal{R}}
\newcommand\cS{\mathcal{S}}
\newcommand\cT{\mathcal{T}}

\newcommand\vard{\mathfrak{d}}

\newcommand\replaceall{\mathsf{replaceAll}}
\newcommand\indexof{\mathsf{IndexOf}}


\newcommand\strline{\mathsf{SL}}

\newcommand\pstrline{\mathsf{SL_{pure}}}

\newcommand\search{\mathsf{search}}

\newcommand\verify{\mathsf{verify}}

\newcommand\searchleft{\mathsf{searchLeft}}

\newcommand\searchlong{\mathsf{searchLong}}


\newcommand\pref{\mathsf{Pref}}

\newcommand\wprof{\mathsf{WP}}

\newcommand\vars{\mathsf{Vars}}

\newcommand\dep{\mathsf{Dep}}
\newcommand\ptn{\mathsf{Ptn}}

\newcommand\src{\mathsf{src}}
\newcommand\strtorep{\mathsf{strToRep}}

\newcommand\rpleft{\mathsf{l}}
\newcommand\rpright{\mathsf{r}}


\newcommand\srcnd{\mathsf{srcND}}

\newcommand\ctxt{\mathsf{ctxt}}


\newcommand\ctxts{\mathsf{Ctxts}}

\newcommand\sprt{\mathsf{sprt}}

\newcommand\val{\mathsf{val}}

\newcommand\srclen{\mathsf{srcLen}}

\newcommand\rpleftlen{\mathsf{lLen}}


\newcommand\dfs{\mathsf{DFS}}

\newcommand\repr{\mathsf{rep}}

\newcommand\red{\mathsf{red}}

\newcommand\gfun{\mathcal{F}}


\newcommand{\leftmost}{{\sf leftmost}}
\newcommand{\longest}{{\sf longest}}

\newcommand{\arbidx}{{\sf Idx_{arb}}}
\newcommand{\dmdidx}{{\sf Idx_{dmd}}}
\newcommand{\lftlen}{{\sf Len_{lft}}}

%\newtheorem{remark}[theorem]{Remark}

\newcommand{\OMIT}[1]{}
\newcommand{\defn}[1]{\emph{#1}}

\newcommand{\Left}{\ensuremath{-1}}
\newcommand{\Right}{\ensuremath{1}}
\newcommand{\Stay}{\ensuremath{0}}

\newcommand{\Aut}{\ensuremath{\mathcal{A}}}
\newcommand{\AutB}{\ensuremath{\mathcal{B}}}
\newcommand{\Transducer}{\ensuremath{T}}
\newcommand{\controls}{\ensuremath{Q}}
\newcommand{\finals}{\ensuremath{F}}
\newcommand{\transrel}{\ensuremath{\delta}}

\newcommand{\Lang}{\mathcal{L}}
\newcommand{\Tran}{\mathcal{T}}

\newcommand{\ialphabet}{\Sigma}
\newcommand{\oalphabet}{\Gamma}

\newcommand{\EndLeft}{\ensuremath{\vartriangleright}}
\newcommand{\EndRight}{\ensuremath{\vartriangleleft}}



%==========================================================

%\newcommand\shortlong[2]{#2}
\newcommand\shortlong[2]{#1}

\newif\ifdraft\drafttrue
%\newif\ifdraft\draftfalse
\ifdraft
\newcommand{\anthony}[1]{\color{red} {YA: #1 :AY} \color{black}}
\newcommand{\zhilin}[1]{\color{brown} {ZL: #1 :LZ} \color{black}}
\newcommand{\tl}[1]{\color{blue} {TL: #1 :LT} \color{black}}
\newcommand{\mat}[1]{\color{cyan} {MH: #1 :HM} \color{black}}
\else
\newcommand{\anthony}[1]{}
\newcommand{\zhilin}[1]{}
\newcommand{\tl}[1]{}
\newcommand{\mat}[1]{}
\fi

\newcommand{\concat} {\circ}
\newcommand{\replace} {{\sf replace}}
\newcommand{\str} {{\sf Str}}
\newcommand{\intnum} {{\sf Int}}
\newcommand{\regexp} {{\sf RegExp}}
\newcommand{\strarr} {{\sf StringArray}}
\newcommand{\dtypes} {{\sf DataTypes}}
\newcommand{\anarr} {{\mathbb{A}}}

%============================================================


\begin{document}

\title{String Constraint Solving with Transducers: A Unified Approach}

\author{}
\institute{}

\maketitle

\begin{abstract}
	The purpose of the project is to generalise the replaceall function considered recently in the POPL'18 to transducers, along the line of the POPL'16 paper.


	CAV instruction: Regular Papers should not exceed 16 pages in LNCS format, not counting references and appendices. Authors can include a clearly marked appendix at the end of their submissions that is exempt from the page limit restrictions.
\end{abstract}



\section{Introduction}

What we hope is two-fold :-
\begin{itemize}
	\item[(i)]  for a transducer $T$ encoding a string manipulating function $z=f(x;\vec{y})$, suppose that  the output $z$ is constrained by a regular language $\mathcal{A}$, we could compute the pre-image $f^{-1}(\mathcal{A)}$ \emph{as a recognisable relation}.

	\item[(ii)] we have a ``generic" way to solve straight-line string constraints when the ``semantics" of the string operator (function) is a recognisable.
\end{itemize}

In term of the replaceall function, both (i) and (ii) hold, and indeed the general algorithm framework therein can serve (ii). So the main job is to show that, for each transducer model, the pre-image can indeed give a recognisable relation.

\begin{tabular}{|c|c|c|}
	\hline

						 &  regular constraints    &  regular and length constraints \\
	\hline
	\hline
one-way transducer						 &  \cite{LB16}            &   \cite{LB16}                   \\
two-way transducer						 &  $n$-EXPSPACE-complete  &   ??                            \\
one-way transducer	with variables       &  EXPSPACE-complete (?)  &  undecidable \cite{CCHLW18}     \\
two-way transducer  with variables       &       ??               & undecidable \cite{CCHLW18}     \\
\hline
\end{tabular}

%===========================================================================================================

\section{Preliminaries}

Let $\mathbb{Z}$ and $\Nat$ denote the set of integers and natural numbers respectively. For $k \in \Nat$, let $[k] = \{1,\cdots, k\}$. For a vector $\vec{x}=(x_1,\cdots, x_n)$, let $|\vec{x}|$ denote the length of $\vec{x}$ (i.e., $n$) and  $\vec{x}[i]$ denote $x_i$ for each $i \in [n]$.

\paragraph{Graph-Theoretical Notation.} \tl{not sure whether it is really needed; will see}
A DAG (\emph{directed acyclic graph}) $G$ is a finite directed graph $(V, E)$ with
no directed cycles, where $V$ (resp.~$E \subseteq V \times V$) is a set of vertices (resp.~edges).
%. That is, each DAG consists of finitely many vertices and edges, with each edge directed from one vertex to another, such that there is no way to start at any vertex $\mathit{v}$ and follow a consistently-directed sequence of edges that eventually loops back to $\mathit{v}$ again.
Equivalently, a DAG is a directed graph that has a topological ordering, which
is a sequence of the vertices such that every edge is directed from an earlier
vertex to a later vertex in the sequence. An edge $(\mathit{v},\mathit{v'})$ in
$G$ is called an \emph{incoming} edge of $\mathit{v'}$ and an \emph{outgoing}
edge of $\mathit{v}$. If $(\mathit{v},\mathit{v'}) \in E$, then $\mathit{v'}$ is
called a \emph{successor} of $\mathit{v}$ and $\mathit{v}$ is called a
\emph{predecessor} of $\mathit{v'}$. A \emph{path} $\pi$ in $G$ is a sequence
$\mathit{v}_0 \mathit{e}_1 \mathit{v}_1 \cdots \mathit{v}_{n-1} \mathit{e}_n
\mathit{v}_n$ such that for each $i \in [n]$, we have $\mathit{e}_i =
(\mathit{v}_{i-1},\mathit{v}_i) \in E$. The \emph{length} of the path $\pi$
%$\mathit{v}_0 e_1 \mathit{v}_1 \cdots \mathit{v}_{n-1} e_n \mathit{v}_n$ in $G$
is the number $n$ of edges in $\pi$. If there is a path from
$\mathit{v}$ to $\mathit{v'}$ (resp. from $\mathit{v'}$ to $\mathit{v}$) in $G$,
then $\mathit{v'}$ is said to be \emph{reachable} (resp. \emph{co-reachable})
from $\mathit{v}$ in $G$. If $\mathit{v}$ is reachable from $\mathit{v'}$ in
$G$, then $\mathit{v'}$ is also called an \emph{ancestor} of $\mathit{v}$ in
$G$. In addition, an edge $(\mathit{v'},\mathit{v''})$ is said to be reachable
(resp. co-reachable) from $\mathit{v}$ if $\mathit{v'}$ is reachable from $\mathit{v}$ (resp. $\mathit{v''}$ is co-reachable from $\mathit{v}$). The \emph{in-degree} (resp. \emph{out-degree}) of a vertex $\mathit{v}$ is the number of incoming (resp. outgoing) edges of $\mathit{v}$.
%A vertex $\mathit{v}$ in $G$ is said to be a \emph{join} vertex if the in-degree of $\mathit{v}$ is at least two.
%A DAG $G$ is called an \emph{arborescence} if there is a vertex $v_0$ such that all the vertices are reachable from $v_0$ in $G$, in addition, there are no join vertices in $G$.
A \emph{subgraph} $G'$ of $G=(V,E)$ is a directed graph $(V', E')$ with
$V' \subseteq V$ and $E' \subseteq E$. Let $G'$ be a subgraph of $G$. Then $G \setminus G'$ is the graph obtained from $G$ by removing all the edges in $G'$.

\subsection{A collection of automata models}

\begin{definition}[two-way finite state automata]
A \emph{nondeterministic two-way finite state automaton}
(2NFA) over a finite alphabet $\Sigma$ is a tuple $\cA =
(Q, q_0, F, \Delta)$ where $Q$ is a finite set of states, $q_0\in Q$ is
the initial state, $F\subseteq Q$ is a set of final states, and $\Delta$ is the
transition relation  of type $\Delta\subseteq Q \times \Sigma\times \{+1, -1\}\times Q $.
\tl{not sure you want to use $\leftarrow, \rightarrow$?}

A nondeterministic (one-way) finite state automaton (NFA)
is a 2NFA such that $\Delta\subseteq Q \times \Sigma\times \{+1\} \times Q $, therefore we
will often write $\Delta\subseteq Q \times \Sigma \times Q$.
\end{definition}


%A \emph{nondeterministic finite automaton} (NFA) $\cA$ on $\Sigma$ is a tuple $(Q, \delta, q_0, F)$, where $Q$ is a finite set of \emph{states}, $q_0 \in Q$ is the \emph{initial} state, $F \subseteq Q$ is the set of \emph{final} states, and $\delta \subseteq Q \times \Sigma \times Q$ is the \emph{transition relation}.

For a string $w = a_1 \dots a_n$, a \emph{run} of $\cA$ on $w$ is a state sequence $q_0 \dots q_n$ such that for each $i \in [n]$, $(q_{i-1}, a_i, q_i) \in \delta$. A run $q_0 \dots q_n$ is \emph{accepting} if $q_n \in F$. A string $w$ is \emph{accepted} by $\cA$ if there is an accepting run of $\cA$ on $w$. We use $\Ll(\cA)$ to denote the language defined by $\cA$, that is, the set of strings accepted by $\cA$. We will use $\cA, \cB, \cdots$ to denote NFAs.
%An NFA $\cA$ is \emph{deterministic} if for each $(q, \sigma) \in Q \times \Sigma$, there is at most one $q' \in Q$ such that $(q, a, q') \in \delta$. An NFA $\cA$ is \emph{complete} if for each $(q, \sigma) \in Q \times \Sigma$, there is at least one $q' \in Q$ such that $(q, a, q') \in \delta$. We assume that all NFA considered in this paper are complete.  An NFA $\cA$ is \emph{unambiguous} if for each word $w$, there is \emph{at most one accepting} run of $\cA$ on $w$.
For a string $w= a_1 \dots a_n$, we also use the notation $q_1 \xrightarrow[\cA]{w} q_{n+1}$ to denote the fact that there are $q_2,\dots, q_n \in Q$ such that for each $i \in [n]$, $(q_i, a_i, q_{i+1}) \in \delta$.  For an NFA $\cA=(Q, \delta, q_0, F)$ and $q, q' \in Q$, we use $\cA(q,q')$ to denote the NFA obtained from $\cA$ by changing the initial state to $q$ and the set of final states to $\{q'\}$. The \emph{size} of an NFA $\cA=(Q, \delta, q_0, F)$, denoted by $|\cA|$, is defined as $|Q|$, the number of states.

For convenience, we will also call an NFA without initial and final states, that is, a pair $(Q, \delta)$, as a \emph{transition graph}.


\begin{proposition}
	Any 2NFA is effectively equivalent to an NFA.
\end{proposition}

\begin{definition}[Two-way Finite transducers]
  Nondeterministic two-way finite state \emph{transducers} (2NFTs) over $\Sigma$ and $\Gamma$ extend NFAs with a one-way left-to-right output tape. They are defined as 2NFAs except that the transition relation $\Delta$ is extended with outputs: $\Delta\subseteq Q \times \Sigma \times \Gamma \times \{-1, +1\} \times  Q $. If a transition $(q, a, b, q′, m)$ is enabled on a letter $a\in \Sigma$, the letter $b\in \Gamma$ is appended to the right of
	the output tape and the transducer goes to state $q'$.
\end{definition}



%\tl{Other relevant models such as SST, when appropriate, will be put here.}

\begin{definition}[Recognisable relation]
	Given a finite alphabet $\Sigma$, a $k$-ary relation $R\subseteq \Sigma^*\times \cdots\times \Sigma^*$ is \emph{recognisable}  if $R=\bigcup_{i=1}^n L^{(i)}_1\times \cdots\times L^{(i)}_k$ where $L^{(i)}_j$ a regular for each $j\in [k]$ .
%
%	[One can certainly generalise this to $n$-ary relations. ]
\end{definition}



\paragraph{Regular Languages.}
Fix a finite \emph{alphabet} $\Sigma$. Elements in $\Sigma^*$ are called \emph{strings}. Let $\varepsilon$ denote the empty string and  $\Sigma^+ = \Sigma^* \setminus \{\varepsilon\}$. We will use $a,b,\cdots$ to denote letters from $\Sigma$ and $u, v, w, \cdots$ to denote strings from $\Sigma^*$. For a string $u \in \Sigma^*$, let $|u|$ denote the \emph{length} of $u$ (in particular, $|\varepsilon|=0$). A \emph{position} of a nonempty string $u$ of length $n$ is a number $i \in [n]$ (Note that the first position is $1$, instead of  0). In addition, for $i \in [|u|]$, let $u[i]$ denote the $i$-th letter of $u$.
For two strings $u_1, u_2$, we use $u_1 \cdot u_2$ to denote the \emph{concatenation} of $u_1$ and $u_2$, that is, the string $v$ such that $|v|= |u_1| + |u_2|$ and for each $i \in [|u_1|]$, $v[i]= u_1[i]$ and for each $i \in |u_2|$, $v[|u_1|+i]=u_2[i]$. Let $u, v$ be two strings. If $v = u \cdot v'$ for some string $v'$, then $u$ is said to be a \emph{prefix} of $v$. In addition, if $u \neq v$, then $u$ is said to be a \emph{strict} prefix of $v$. If $u$ is a prefix of $v$, that is, $v = u \cdot v'$ for some string $v'$, then
we use $u^{-1} v$ to denote $v'$. In particular, $\varepsilon^{-1} v = v$.

A \emph{language} over $\Sigma$ is a subset of $\Sigma^*$. We will use $L_1, L_2, \dots$ to denote languages. For two languages $L_1, L_2$, we use $L_1 \cup L_2$ to denote the union of $L_1$ and $L_2$, and $L_1 \cdot L_2$ to denote the concatenation of $L_1$ and $L_2$, that is, the language $\{u_1 \cdot u_2 \mid u_1 \in L_1, u_2 \in L_2\}$. For a language $L$ and $n \in \Nat$, we define $L^n$, the \emph{iteration} of $L$ for $n$ times, inductively as follows: $L^0=\{\varepsilon\}$ and $L^{n} =L \cdot L^{n-1}$ for $n > 0$. We also use $L^*$ to denote the iteration of $L$ for arbitrarily many times, that is, $L^* = \bigcup \limits_{n \in \Nat} L^n$. Moreover, let $L^+ = \bigcup \limits_{n \in \Nat \setminus \{0\}} L^n$.

\begin{definition}[Regular expressions $\regexp$]
	\[e \eqdef \emptyset \mid \varepsilon \mid a \mid e + e \mid e \concat e \mid e^*, \mbox{ where } a \in \Sigma. \]
	Since $+$ is associative and commutative, we also write $(e_1 + e_2) + e_3$ as $e_1 + e_2 + e_3$ for brevity. We use the abbreviation $e^+ \equiv e \concat e^*$. Moreover, for $\Gamma = \{a_1, \cdots, a_n\}\subseteq \Sigma$, we use the abbreviations $\Gamma \equiv a_1 + \cdots + a_n$ and $\Gamma^\ast \equiv (a_1 + \cdots + a_n)^\ast$.
\end{definition}
We define $\Ll(e)$ to be the language defined by $e$, that is, the set of strings that match $e$, inductively as follows: $\Ll(\emptyset) =\emptyset$,
%\begin{itemize}
%\item
$\Ll(\varepsilon) =\{\varepsilon\}$,
%
%\item
$\Ll(a)= \{a\}$,
%
%\item
$\Ll(e_1 + e_2) = \Ll(e_1) \cup \Ll(e_2)$,
%
%\item
$\Ll(e_1 \concat e_2) = \Ll(e_1) \cdot \Ll(e_2)$,
%
%\item
$\Ll(e_1^*)=(\Ll(e_1))^*$.
%\end{itemize}
In addition, we use $|e|$ to denote the number of symbols occurring in $e$.

%A \emph{nondeterministic finite automaton} (NFA) $\cA$ on $\Sigma$ is a tuple $(Q, \delta, q_0, F)$, where $Q$ is a finite set of \emph{states}, $q_0 \in Q$ is the \emph{initial} state, $F \subseteq Q$ is the set of \emph{final} states, and $\delta \subseteq Q \times \Sigma \times Q$ is the \emph{transition relation}. For a string $w = a_1 \dots a_n$, a \emph{run} of $\cA$ on $w$ is a state sequence $q_0 \dots q_n$ such that for each $i \in [n]$, $(q_{i-1}, a_i, q_i) \in \delta$. A run $q_0 \dots q_n$ is \emph{accepting} if $q_n \in F$. A string $w$ is \emph{accepted} by $\cA$ if there is an accepting run of $\cA$ on $w$. We use $\Ll(\cA)$ to denote the language defined by $\cA$, that is, the set of strings accepted by $\cA$. We will use $\cA, \cB, \cdots$ to denote NFAs.
%%An NFA $\cA$ is \emph{deterministic} if for each $(q, \sigma) \in Q \times \Sigma$, there is at most one $q' \in Q$ such that $(q, a, q') \in \delta$. An NFA $\cA$ is \emph{complete} if for each $(q, \sigma) \in Q \times \Sigma$, there is at least one $q' \in Q$ such that $(q, a, q') \in \delta$. We assume that all NFA considered in this paper are complete.  An NFA $\cA$ is \emph{unambiguous} if for each word $w$, there is \emph{at most one accepting} run of $\cA$ on $w$.
%For a string $w= a_1 \dots a_n$, we also use the notation $q_1 \xrightarrow[\cA]{w} q_{n+1}$ to denote the fact that there are $q_2,\dots, q_n \in Q$ such that for each $i \in [n]$, $(q_i, a_i, q_{i+1}) \in \delta$.  For an NFA $\cA=(Q, \delta, q_0, F)$ and $q, q' \in Q$, we use $\cA(q,q')$ to denote the NFA obtained from $\cA$ by changing the initial state to $q$ and the set of final states to $\{q'\}$. The \emph{size} of an NFA $\cA=(Q, \delta, q_0, F)$, denoted by $|\cA|$, is defined as $|Q|$, the number of states. For convenience, we will also call an NFA without initial and final states, that is, a pair $(Q, \delta)$, as a \emph{transition graph}.

It is well-known (e.g. see \cite{HU79}) that regular expressions and NFAs are
expressively equivalent, and generate precisely all \emph{regular languages}.
In particular, from a regular expression, an equivalent NFA can be constructed
in linear time. Moreover, regular languages are closed under Boolean
operations, i.e., union, intersection, and complementation.
In particular, given two NFA $\cA_1=(Q_1, \delta_1, q_{0,1}, F_1)$ and
$\cA_2=(Q_2, \delta_2, q_{0,2}, F_2)$ on $\Sigma$, the intersection $\Ll(\cA_1)
\cap \Ll(\cA_2)$ is recognised by the \emph{product automaton} $\cA_1 \times
\cA_2$ of $\cA_1$ and $\cA_2$ defined as $(Q_1 \times Q_2, \delta, (q_{0,1}, q_{0,2}), F_1 \times F_2)$, where $\delta$ comprises the transitions $((q_1, q_2), a, (q'_1, q'_2))$ such that $(q_1, a, q'_1) \in \delta_1$ and $(q_2, a, q'_2) \in \delta_2$.

\section{String constraints} \label{sec-core}

In this section, we define a general string constraint language that supports
concatenation, transducers, and regular constraints.
\tl{for length constraints, let's decide later whether it should be put here}
Throughout this section, we fix an alphabet $\Sigma$.
We consider the String data type $\str$, and assume a countable set of variables
$x, y, z, \cdots$ of $\str$.


\begin{definition}[Relational and regular constraints]
	Relational constraints and regular constraints are defined by the following rules,
	\[
	\begin{array}{r c l cr}
	s &\eqdef & x \mid u & \ \ & \mbox{(string terms)}\\
%	p &\eqdef & x \mid e & \ \ & \mbox{(pattern terms)}\\
	%t &\eqdef & s \mid e & \ \ & \mbox{(terms)}\\
	\varphi &\eqdef & x = s \concat s  \mid  x = T(\vec{s}) \mid \varphi \wedge \varphi & \ \ & \mbox{(relational constraints)}\\
	\psi & \eqdef & x \in e \mid \psi \wedge \psi %\mid \psi \vee \psi \mid \neg \psi
	& \ \ & \mbox{(regular constraints)} \\
	\end{array}
	\]
	where $x$ is a string variable, $u \in \Sigma^\ast$ and $e$ is a regular expression over $\Sigma$.
\end{definition}
\tl{this is not optimal. For $T$ with multiple parameters, the concatenation is redundant}

For a formula $\varphi$ (resp. $\psi$), let $\vars(\varphi)$ (resp. $\vars(\psi)$) denote the set of variables occurring in $\varphi$ (resp. $\psi$). Given a relational constraint $\varphi$, a variable $x$ is called a \emph{source variable} of $\varphi$ if $\varphi$ \emph{does not} contain a conjunct of the form $x = s_1 \concat s_2$ or $x = T(\vec{s})$.

%We then notice that, with the $\replaceall$ function in its general form, the concatenation operation is in fact redundant.

%\begin{proposition}\label{prop-concat}
%	The concatenation  operation ($\concat$) can be simulated  by the $\replaceall$ function.
%\end{proposition}
%\begin{proof}
%	It is sufficient to observe that %the concatenation operator $s_1 \concat s_2$ is redundant in the sense that
%	a relational constraint $x = s_1 \concat s_2$ can be rewritten as
%	\[x' = \replaceall(ab, a, s_1) \wedge x = \replaceall(x', b, s_2),\] where $a,b$ are two fresh letters.
%\end{proof}

%In light of Proposition~\ref{prop-concat}, in the sequel, we will \emph{dispense the concatenation operator} mostly and focus on \textbf{the string constraints that involve  the $\replaceall$ function only}.

%Another example to show the power of the $\replaceall$ function is that it can simulate the extension of regular expressions with string variables, which is  supported by the mainstream scripting languages like Python, Javascript, and PHP. For instance, $x \in y^*$ can be expressed by $x =\replaceall(x', a, y) \wedge x' \in a^*$, where $x'$ is a fresh variable and $a$ is a fresh letter.



The generality of the constraint language makes it undecidable,
even in very simple cases. To retain decidability, we follow \cite{LB16} and focus on the ``straight-line fragment" of the language. This straight-line fragment captures the structure of straight-line string-manipulating
programs.

\begin{definition}[Straight-line relational constraints]
	A relational constraint $ \varphi$ with transducers is straight-line, if $\varphi \eqdef \bigwedge \limits_{1 \le i \le m} x_i = P_i$ such that
	\begin{itemize}
		\item $x_1,\dots, x_m$ are mutually distinct,
		\item for each $i \in [m]$, all the variables in $P_i$ are either source variables, or variables from $\{x_1,\dots, x_{i-1}\}$,
	\end{itemize}
	%Occasionally we refer to $x_m$ as the output variable.
\end{definition}
%Intuitively, in a straight-line relational constraint, the dependency graph (see Definition~\ref{def:dep-graph}) of the string variables is acyclic.
%\mat{forward reference!}

\begin{remark}
	Checking whether a relational constraint $\varphi$ is straight-line can be done in linear time.
\end{remark}

\begin{definition}[Straight-line string constraints]
	A straight-line string constraint $C$ with transducers (denoted by $\strline[T]$)  is defined as $ \varphi \wedge \psi$,  where
	\begin{itemize}
		\item $\varphi$ is a straight-line relational constraint with transducers,  and
		%
		\item $\psi$ is a regular constraint.
		%
	\end{itemize}
\end{definition}



 We first introduce a graphical representation of $\strline[T]$ formulae as follows.

 \begin{definition}[Dependency graph]
	\label{def:dep-graph}
	Suppose $C= \varphi \wedge \psi$ is an $\strline[\replaceall]$ formula where the pattern parameters of the $\replaceall$ terms are regular expressions. %Let $\vars(\varphi) = \{x_1,\dots, x_m, y_1, \dots, y_n\}$, where $y_1,\dots, y_n$ are  source variables.
	Define the \emph{dependency graph} of $C$ as $G_C= (\vars(\varphi), E_C)$, such that for each $i \in [m]$, if $x_i = \replaceall(z, e_i, z')$, then $(x_i, (\rpleft, e_i), z) \in E_C$ and $(x_i, (\rpright, e_i), z') \in E_C$. A final (resp. initial) vertex in $G_C$ is a vertex in $G_C$ without successors (resp. predecessors). The edges labelled by $(\rpleft, e_i)$ and $(\rpright, e_i)$ are called the $\rpleft$-edges and $\rpright$-edges respectively. The \emph{depth} of $G_C$ is the maximum length of the paths in $G_C$. In particular, if $\varphi$ is empty, then the depth of $G_C$ is zero.
	%The $\rpleft$-length of a path $\pi$, denoted by $\rpleftlen(\pi)$, is the number of $\rpleft$-edges on $\pi$. A path of $G_C$ is a sequence $z_1 \ell_1 z_2 \dots \ell_{k-1} z_k$ such that for each $i \in [k-1]$, $(z_i, \ell_i, z_{i+1}) \in E_C$. A path is initial (resp. final) if the path starts from an initial vertex (resp. stops at a final vertex).
	% e the $\src$-nesting-depth of $z$ in $G_C$, denoted by $\srcnd_{G_C}(z)$,  as the maximum number of $\src$-edges in paths from source variables to $z$.
 \end{definition}
 Note that $G_C$ is a DAG where the out-degree of each vertex is two or zero.

\subsection{The satisfiability problem} \label{sec-sat}
In this paper, we focus on the satisfiability problem of $\strline[T]$, which is formalised as follows.

%\smallskip

\begin{quote} \centering
	\framebox{Given an $\strline[T]$ constraint $C$, decide whether $C$ is satisfiable.}
\end{quote}
\smallskip

%To approach this problem, we identify several fragments of  $\strlineTT]$, depending on whether the pattern and the replacement parameters are constants or variables.  We shall investigate extensively the satisfiability problem of the fragments of $\strline[\replaceall]$. % (see Table~\ref{tab-sum}).  Note that for $x=\replaceall (y, p, z)$, $p$ is referred to as a \emph{pattern} and $z$ is referred to as a \emph{replacement}.

%=========================================================================================================

\section{Two-way transducers}

In this section, we focus on the case where the transducer $T$ has only one parameter, i.e., the relational constraint is of form $x=T(y)$ for example. We will give both upper and low bounds.

\tl{A question: here there are two ways to present the result, one is to follow \cite{LB16} to encode $\concat$, and the other is to deal with $\concat$ explicitly. Which way do you think is better?}

\subsection{Upper-bound}
We first show that, when the 2-way transducer $T$ and an NFA


\subsection{Lower-bound}


\section{Lower Bound For String Constraints with Two-Way Transducers}
\label{sec:two-way-lower}

\subsection{Tiling Problems}

\begin{definition}[Tiling Problem]
    A \emph{tiling problem} is a tuple
    $\tup{\tiles, \hrel, \vrel, \inittile, \fintile}$
    where
        $\tiles$ is a finite set of tiles,
        $\hrel \subseteq \tiles \times \tiles$ is a horizontal matching relation,
        $\vrel \subseteq \tiles \times \tiles$ is a vertical matching relation, and
        $\inittile, \fintile \in \tiles$ are initial and final tiles respectively.
\end{definition}

A solution to a tiling problem over a $\linlen$-width corridor is a sequence
\[
    \begin{array}{c}
        \tile^1_1 \ldots \tile^1_\linlen \\
        \tile^2_1 \ldots \tile^2_\linlen \\
        \ldots \\
        \tile^\tileheight_1 \ldots \tile^\tileheight_\linlen
    \end{array}
\]
where
$\tile^1_1 = \inittile$,
$\tile^\tileheight_\linlen = \fintile$,
and for all
$1 \leq i < \linlen$
and
$1 \leq j \leq \tileheight$
we have
$\tup{\tile^j_i, \tile^j_{i+1}} \in \hrel$
and for all
$1 \leq i \leq \linlen$
and
$1 \leq j < \tileheight$
we have
$\tup{\tile^j_i, \tile^{j+1}_i} \in \vrel$.
Note, we will assume that $\inittile$ and $\fintile$ can only appear at the beginning and end of the tiling respectively.

Tiling problems characterise many complexity classes~\cite{??}.
In particular, we will use the following facts.
\begin{itemize}
\item
    For any $\linlen$-space Turing machine, there exists a tiling problem of size polynomial in the size of the Turing machine, over a corridor of width $\linlen$, that has a solution iff the $\linlen$-space Turing machine has a terminating computation.

\item
    There is a fixed
    $\tup{\tiles, \hrel, \vrel, \inittile, \fintile}$
    such that for any width $\linlen$ there is a unique solution
    \[
        \begin{array}{c}
            \tile^1_1 \ldots \tile^1_\linlen \\
            \tile^2_1 \ldots \tile^2_\linlen \\
            \ldots \\
            \tile^\tileheight_1 \ldots \tile^\tileheight_\linlen
        \end{array}
    \]
    and moreover $\tileheight$ is exponential in $\linlen$.
    One such example is a Turing machine where the tape contents represent a binary number.
    The Turing machine starts from a tape containing only $0$s and finishes with a tape containing only $1$s by repeatedly incrementing the binary encoding on the tape.
    This Turing machine can be encoded as the required tiling problem.
\end{itemize}

\subsection{Large Numbers}

The crux of the proof is encoding large numbers that can take values between $1$ and $\expheight$-fold exponential.

A linear-length binary number could be encoded simply as a sequence of bits
\[
    b_0 \ldots b_\linlen \in \set{0,1}^\linlen \ .
\]
To aid with later constructions we will take a more oblique approach.
Let
$\tup{\tilesnum{1}, \hrelnum{1}, \vrelnum{1}, \inittilenum{1}, \fintilenum{1}}$
be a copy of the fixed tiling problem from the previous section for which there is a unique solution, whose length must be exponential in the width.
In the future, we will need several copies of this problem, hence the indexing here.
Fix a width $\linlen$ and let $\nmax{1}$ be the corresponding corridor length.
A \emph{level-1} number can encode values from $1$ to $\nmax{1}$.
In particular, for $1 \leq i \leq \nmax{1}$ we define
\[
    \tenc{1}{i} = \tile^i_1 \ldots \tile^i_\linlen
\]
where
$\tile^i_1 \ldots \tile^i_\linlen$
is the tiling of the $i$th row of the unique solution to the tiling problem.

A \emph{level-2} number will be derived from tiling a corridor of width $\nmax{1}$, and thus the number of rows will be doubly-exponential.
For this, we require another copy
$\tup{\tilesnum{2}, \hrelnum{2}, \vrelnum{2}, \inittilenum{2}, \fintilenum{2}}$
of the above tiling problem.
Moreover, let $\nmax{2}$ be the length of the solution for a corridor of width $\nmax{1}$.
Then for any
$1 \leq i \leq \nmax{2}$
we define
\[
    \tenc{2}{i} =
        \tenc{1}{1} \tile^i_1
        \tenc{1}{2} \tile^i_2
        \ldots
        \tenc{1}{\nmax{1}} \tile^i_{\nmax{1}}
\]
where
$\tile^i_1 \ldots \tile^i_{\nmax{1}}$
is the tiling of the $i$th row of the unique solution to the tiling problem.
That is, the encoding indexes each tile with it's column number, where the column number is represented as a level-1 number.

In general, a \emph{level-$\expheight$} number is of length $(\expheight-1)$-fold exponential and can encode numbers $\expheight$-fold exponential in size.
We use a copy
$\tup{\tilesnum{\expheight},
      \hrelnum{\expheight},
      \vrelnum{\expheight},
      \inittilenum{\expheight},
      \fintilenum{\expheight}}$
of the above tiling problem and use a corridor of width
$\nmax{\expheight-1}$.
We define $\nmax{\expheight}$ as the length of the unique solution to this problem.
Then, for any $1 \leq i \leq \nmax{\expheight}$ we have
\[
    \tenc{\expheight}{i} =
        \tenc{(\expheight-1)}{1} \tile^i_1
        \tenc{(\expheight-1)}{2} \tile^i_2
        \ldots
        \tenc{(\expheight-1)}{\nmax{(\expheight-1)}} \tile^i_{\nmax{(\expheight-1)}}
\]
where
$\tile^i_1 \ldots \tile^i_{\nmax{\expheight-1}}$
is the tiling of the $i$th row of the unique solution to the tiling problem.

\subsection{Recognising Large Numbers}

We first define a useful program
$\goodnums{\expheight}{x}$
with a single input string $x$ which can only be satisfied if $x$ is of the following form, where
$\numeq$, $\numplus$, and $\numsep$
are auxiliary symbols.
\[
    \begin{array}{c}
        \goodnums{\expheight}{x} \text{ is satisfiable} \\
        \iff \\
        x \in \brac{
            \brac{\tenc{\expheight}{1} \brac{\numeq \tenc{\expheight}{1}}^\ast}
            \numplus
            \brac{\tenc{\expheight}{2} \brac{\numeq \tenc{\expheight}{2}}^\ast}
            \numplus
            \ldots
            \numplus
            \brac{
                \tenc{\expheight}{\nmax{\expheight}}
                    \brac{\numeq \tenc{\expheight}{\nmax{\expheight}}}^\ast
            }
            \numsep
        }^\ast
    \end{array}
\]
That is, the string must contain sequences of strings that count from $1$ to $\nmax{\expheight}$.
This counting may stutter and repeat a number several times before moving to the next.
A separator $\numeq$ indicates a stutter, while $\numplus$ indicates that the next number must add one to the current number.
Finally, a $\numsep$ ends a sequence and may start again from $\tenc{\expheight}{1}$.

\paragraph{Base case $\expheight = 1$.}

We define
\[
    \goodnums{1}{x} = \left\{
        \begin{array}{l}
            y := \ap{T}{x}; \\
            \ASSERT{y \in \top}
        \end{array}
    \right.
\]
where $\top$ is a character output by $T$ when $x$ is correctly encoded.
Otherwise $T$ outputs $\bot$.

We describe how $T$ operates.
Because $T$ is two-way, it may perform several passes of the input $x$.
Recall that $T$ requires $x$ to contain a word of the form
\[
    \brac{
        \brac{\tenc{1}{1} \brac{\numeq \tenc{1}{1}}^\ast}
        \numplus
        \brac{\tenc{1}{2} \brac{\numeq \tenc{1}{2}}^\ast}
        \numplus
        \ldots
        \numplus
        \brac{
            \tenc{1}{\nmax{1}}
                \brac{\numeq \tenc{1}{\nmax{1}}}^\ast
        }
        \numsep
    }^\ast
\]
and each $\tenc{1}{i}$ is the $i$th row of the unique solution to the tiling problem of width $\linlen$.
The passes proceed as follows.
If a passes fails, the transducer outputs $\bot$ and terminates.
If all passes succeed, the transducer outputs $\top$ and terminates.
\begin{itemize}
\item
    During the first pass $T$ verifies that the input is of the form
    \[
        \brac{
            \tilesnum{1}^\linlen
            \brac{\set{\numeq,\numplus} \tilesnum{1}^\linlen}^\ast
            \numsep
        }^\ast \ .
    \]

\item
    During the second pass the transducer verifies that the first block of
    $\tilesnum{1}^\linlen$,
    and all blocks of
    $\tilesnum{1}^\linlen$
    immediately following a $\numsep$ have $\inittilenum{1}$ as the first tile.
    Simultaneously, it can verify that all blocks of
    $\tilesnum{1}^\linlen$
    immediately preceding a $\numsep$ finish with the tile $\fintilenum{1}$.
    Moreover, it checks that $\inittilenum{1}$ and $\fintilenum{1}$ do not appear elsewhere.

\item
    During the third pass $T$ verifies the horizontal tiling relation.
    That is, every contiguous pair of tiles $\tile, \tile'$ in $x$ must be such that
    $(\tile, \tile') \in \hrelnum{1}$.
    This can easily be done by storing the last character read into the states of $T$.

\item
    The vertical tiling relation and equality checks are verified using $\linlen$ more passes.
    During the $j$th pass, the $j$th column is tested.
    The transducer $T$ stores in its state the tile in the $j$th column of the first block of $\tilesnum{1}^\linlen$ or any block immediately following $\numsep$.
    (The transducer can count to $\linlen$ in its state.)
    It then moves to the $j$th column of the next block of $\tilesnum{1}^\linlen$, remembering whether the blocks were separated with $\numeq$, $\numplus$, or $\numsep$.
    If the separator was $\numeq$ the transducer checks that the $j$th tile of the current block matches the tile stored in the state (i.e.~is equal to the preceding block).
    If the separator was $\numplus$ the transducer checks that the $j$th tile of the current block is related by $\vrelnum{1}$ to the stored tile.
    In this case the current $j$th tile is stored and the previously stored tile forgotten.
    Finally, if the separator was $\numsep$ there is nothing to check.
    If any check fails, the pass will also fail.
\end{itemize}

If all passes succeed, we know that $x$ contains a word where blocks separated by $\numeq$ are equal
(since all positions are equal, as verified individually by the final $\linlen$ passes),
blocks separated by $\numplus$ satisfy $\vrelnum{1}$ in all positions,
the $\inittilenum{1}$ tile appears at the start of all sequences separated by $\numsep$ and each such sequence ends with $\fintilenum{1}$, and
finally the horizontal tiling relation is satisfied at all times.
Thus, $x$ must be of the form
\[
    \brac{
        \brac{\tenc{1}{1} \brac{\numeq \tenc{1}{1}}^\ast}
        \numplus
        \brac{\tenc{1}{2} \brac{\numeq \tenc{1}{2}}^\ast}
        \numplus
        \ldots
        \numplus
        \brac{
            \tenc{1}{\nmax{1}}
                \brac{\numeq \tenc{1}{\nmax{1}}}^\ast
        }
        \numsep
    }^\ast
\]
as required.


\paragraph{Inductive case $\expheight$.}

We define a program
$\goodnums{\expheight}{x}$
such that
\[
    \begin{array}{c}
        \goodnums{\expheight}{x} \text{ is satisfiable} \\
        \iff \\
        x \in \brac{
            \brac{\tenc{\expheight}{1} \brac{\numeq \tenc{\expheight}{1}}^\ast}
            \numplus
            \brac{\tenc{\expheight}{2} \brac{\numeq \tenc{\expheight}{2}}^\ast}
            \numplus
            \ldots
            \numplus
            \brac{
                \tenc{\expheight}{\nmax{\expheight}}
                    \brac{\numeq \tenc{\expheight}{\nmax{\expheight}}}^\ast
            }
            \numsep
        }^\ast \ .
    \end{array}
\]
Assume, by induction, we have a program
$\goodnums{\expheight-1}{x}$
which already satisfies this property (for $\expheight-1$).
We define
\[
    \goodnums{\expheight}{x} =
    \left\{
        \begin{array}{l}
            y = \ap{T}{x}; \\
            \goodnums{\expheight-1}{y}
        \end{array}
    \right.
\]
where $T$ is a transducer that behaves as described below.
Note, the reference to
$\goodnums{\expheight-1}{y}$
is not a procedure call, since these are not supported by our language.
Instead, the procedure is inlined, with its input variable $x$ replaced by $y$ and other variables renamed to avoid clashes.

The transducer will perform several passes to make several checks.
If a check fails it will halt and output a symbol $\bot$, which means that $y$ can no longer satisfy
$\goodnums{\expheight-1}{y}$.
During normal execution $T$ will make checks that rely on level-$(\expheight-1)$ numbers appearing in the correct sequence or being equal.
To ensure these properties hold, $T$ will write these numbers to $y$ and rely on these properties then being verified by
$\goodnums{\expheight-1}{y}$.
The passes behave as follows.
\begin{itemize}
\item
    During the first pass $T$ verifies that $x$ belongs to the regular language
    \[
        \brac{
            \brac{
                \brac{
                    \brac{
                        \brac{\tilesnum{1}^\linlen \tilesnum{2}}^\ast \tilesnum{3}
                    }^\ast
                    \cdots
                }^\ast
                \tilesnum{\expheight}
            }^\ast
            \brac{
                \set{\numeq,\numplus}
                \brac{
                    \brac{
                        \brac{
                            \brac{\tilesnum{1}^\linlen \tilesnum{2}}^\ast \tilesnum{3}
                        }^\ast
                        \cdots
                    }^\ast
                    \tilesnum{\expheight}
                }^\ast
            }^\ast
            \numsep
         }^\ast
         \ .
    \]
    This can be done with a polynomial number of states.

\item
    During the second pass $T$ will verify that the first instance of
    $\tilesnum{\expheight}$
    appearing in the word or after a $\numsep$ is $\inittilenum{\expheight}$.
    Similarly, the final instance of any
    $\tilesnum{\expheight}$
    before any $\numsep$ is $\fintilenum{\expheight}$.
    Moreover, it checks that
    $\inittilenum{\expheight}$
    and
    $\fintilenum{\expheight}$
    do not appear elsewhere.

\item
    During the third pass $T$ will verify the horizontal tiling relation
    $\hrelnum{\expheight}$.
    Each block (separated by $\numeq$, $\numplus$, or $\numsep$) is checked in turn.
    There are two components to this.
    \begin{itemize}
    \item
        The indexing of the tiles must be correct.
        That is, the first tile of the block must be indexed
        $\tenc{\expheight-1}{1}$,
        the second
        $\tenc{\expheight-1}{2}$,
        through to
        $\tenc{\expheight-1}{\nmax{\expheight-1}}$.
        Thus, $T$ copies directly the instance of
        $\brac{
            \brac{
                \brac{\tilesnum{1}^\linlen \tilesnum{2}}^\ast \tilesnum{3}
            }^\ast
            \cdots
        }^\ast$
        preceding each
        $\tilesnum{\expheight}$
        to the output tape, followed immediately by
        $\numplus$
        as long as the character after
        $\tilesnum{\expheight}$
        is not a separator from
        $\set{\numeq,\numplus,\numsep}$.
        Otherwise, it is the end of the block and $\numsep$ is written.

        Hence,
        $\goodnums{\expheight-1}{y}$
        will verify that the output for each block is
        $\tenc{\expheight-1}{1}
         \numplus \cdots \numplus
         \tenc{\expheight-1}{\nmax{\expheight-1}}$
        which enforces that the indexing of the tiles is correct.

    \item
        Horizontally adjacent tiles must satisfy
        $\hrelnum{\expheight}$.
        This is done by simply storing the last read tile from
        $\tilesnum{\expheight}$
        in the state of $T$.
        Then whenever a new tile from
        $\tilesnum{\expheight}$
        is seen without a separator $\numeq$, $\numplus$, or $\numsep$, then it can be checked against the previous tile and
        $\hrelnum{\expheight}$.
    \end{itemize}

\item
    The transducer $T$ then performs a non-deterministic number of passes to check the vertical tiling relation.
    We will use
    $\goodnums{\expheight-1}{y}$
    to ensure that $T$ in fact performs
    $\nmax{\expheight-1}$
    passes, the first checking the first column of the tiling over
    $\tilesnum{\expheight}$,
    the second checking the second column, and so on up to the
    $\nmax{\expheight-1}$th column.

    Note, we know from the previous pass that each row of the tiling is indexed correctly.
    In the sequel, let us use the term ``session'' to refer to the sequences of characters separated by $\numsep$.

    Each pass of $T$ checks a single column (across all sessions).
    At the start of each session, $T$ moves non-deterministically to the start of some block
    $\brac{
        \brac{
            \brac{\tilesnum{1}^\linlen \tilesnum{2}}^\ast \tilesnum{3}
        }^\ast
        \cdots
    }^\ast
    \tilesnum{\expheight}$
    (without passing $\numeq$, $\numplus$, or $\numsep$).
    It then copies the tiles from
    $\brac{
        \brac{
            \brac{\tilesnum{1}^\linlen \tilesnum{2}}^\ast \tilesnum{3}
        }^\ast
        \cdots
    }^\ast$
    to $y$ and saves the tile from
    $\tilesnum{\expheight}$
    in its state before moving to the next separator from
    $\set{\numeq,\numplus,\numsep}$.
    In the case of $\numsep$ nothing needs to be checked and $T$ continues to the next session or finishes the pass if there are no more sessions.
    In the case of $\numeq$ or $\numplus$ the transducer remembers this separator and moves non-deterministically to the start of some block (without passing another $\numeq$, $\numplus$, or $\numsep$).
    It then writes $\numeq$ to $y$ as it is intended that $T$ choose the same column as before.
    This will be verified by
    $\goodnums{\expheight-1}{y}$.
    To aid with this $T$ copies the tiles from
    $\brac{
        \brac{
            \brac{\tilesnum{1}^\linlen \tilesnum{2}}^\ast \tilesnum{3}
        }^\ast
        \cdots
    }^\ast$
    to $y$.
    It can then check the tile from
    $\tilesnum{\expheight}$.
    If the remembered separator was $\numeq$ then this tile must match the saved one.
    If it was $\numplus$ then this tile must be related by
    $\vrelnum{\expheight}$
    to the saved one.
    If this succeeds , $T$ stores the new tile and forgets the old and continues to the next separator to continue checking
    $\vrelnum{\expheight}$.

    At the end of the pass (checking a single column from all sessions) then $T$ will either have failed and written $\bot$ or written a sequence of level-$(\expheight-1)$ numbers to $y$ separated by $\numeq$.
    That is
    \[
        \tenc{\expheight-1}{i_1}
        \numeq
        \cdots
        \numeq
        \tenc{\expheight-1}{i_{\alpha}}
    \]
    for some $\alpha$.
    Since, by induction,
    $\goodnums{\expheight-1}{y}$
    is correct, then the program can only be satisfied if $T$ chose the same position in each row.
    That is
    $i_1 = \cdots = i_\alpha$.
    Thus, the vertical relation for the $i_1$th column has been verified.

    At this point $T$ can either write $\numsep$ and terminate or perform another pass (non-deterministically).
    In the latter case, it outputs $\numplus$, moves back to the beginning of the tape, and starts again.
    Thus, after a number of passes, $T$ will have written
    \[
        \brac{
            \tenc{\expheight-1}{i^1_1}
            \numeq
            \cdots
            \numeq
            \tenc{\expheight-1}{i^1_{\alpha_1}}
        }
        \numplus
        \cdots
        \numplus
        \brac{
            \tenc{\expheight-1}{i^\beta_1}
            \numeq
            \cdots
            \numeq
            \tenc{\expheight-1}{i^\beta_{\alpha_\beta}}
        }
        \numsep
    \]
    for some $\beta$, $\alpha_1$, \ldots, $\alpha_\beta$.
    Since
    $\goodnums{\expheight-1}{y}$
    will only accept such sequences of the form
    \[
        \brac{
            \tenc{\expheight-1}{1}
            \brac{
                \numeq \tenc{\expheight-1}{1}
            }^\ast
        }
        \numplus
        \cdots
        \numplus
        \brac{
            \tenc{\expheight-1}{\nmax{\expheight-1}}
            \brac{
                \numeq \tenc{\expheight-1}{\nmax{\expheight-1}}
            }^\ast
        }
        \numsep
    \]
    we know that $T$ must check each vertical column in turn, from $1$ to
    $\nmax{\expheight-1}$.
\end{itemize}

Thus, at the end of all passes, if $T$ has not output $\bot$ it has verified that $x$ is a correct encoding of a solution to
$\tup{\tilesnum{\expheight},
      \hrelnum{\expheight},
      \vrelnum{\expheight},
      \inittilenum{\expheight},
      \fintilenum{\expheight}}$.
That is, together with
$\goodnums{\expheight-1}{y}$
we know that
    the word is of the correct format,
    each row has a tile for each index and these indices appear in order,
    the horizontal relation is respected, and
    the vertical tiling relation is respected.
If $x$ is not a correct encoding then $T$ will not be able to produce a $y$ that satisfies
$\goodnums{\expheight-1}{y}$.


\subsection{Reducing from a Tiling Problem}

Now that we are able to encode large numbers, we can encode an $\expheight$-$\expspace$-hard tiling problem as a satisfiability problem of $\strline[T]$ with two-way transducers.
In fact, most of the technical work has been done.

Thus, fix a tiling problem
$\tup{\tiles, \hrel, \vrel, \inittile, \fintile}$
that is $\expheight$-$\expspace$-hard.
In particular, we allow a corridor $\nmax{\expheight}$ tiles wide.
We use the program
\[
    S = \left\{
        \begin{array}{l}
            y = \ap{T}{x}; \\
            \goodnums{\expheight}{y}
        \end{array}
    \right.
\]
where $T$ is defined exactly as in the inductive case of
$\goodnums{\expheight}{y}$
except the tiling problem used is
$\tup{\tiles, \hrel, \vrel, \inittile, \fintile}$
rather than
$\tup{\tilesnum{\expheight},
      \hrelnum{\expheight},
      \vrelnum{\expheight},
      \inittilenum{\expheight},
      \fintilenum{\expheight}}$.

A satisfying tiling
\[
    \begin{array}{c}
        \tile^1_1 \ldots \tile^1_{\nmax{\expheight}} \\
        \cdots \\
        \tile^\tileheight_1 \ldots \tile^\tileheight_{\nmax{\expheight}}
    \end{array}
\]
can be encoded
\[
    \tenc{\expheight}{1} \tile^1_1
    \cdots
    \tenc{\expheight}{\nmax{\expheight}} \tile^1_{\nmax{\expheight}}
    \numplus
    \cdots
    \numplus
    \tenc{\expheight}{1} \tile^\expheight_1
    \cdots
    \tenc{\expheight}{\nmax{\expheight}} \tile^\expheight_{\nmax{\expheight}}
    \numsep
\]
which will satisfy $S$ in the same way as a correct input to
$\goodnums{\expheight}{y}$.
To see this, note that
$\tenc{\expheight}{1} \tile^i_1
 \cdots
 \tenc{\expheight}{\nmax{\expheight}} \tile^i_{\nmax{\expheight}}$
acts like some
$\tenc{\expheight+1}{i}$.

In the opposite direction, assume some input satisfying $S$.
Arguing as in the encoding of large numbers, this input must be of the form
\[
    \brac{
        \brac{\tilerow_1 \brac{\numeq \tilerow_1}^\ast}
        \numplus
        \brac{\tilerow_2 \brac{\numeq \tilerow_2}^\ast}
        \numplus
        \ldots
        \numplus
        \brac{
            \tilerow_\tileheight
            \brac{\numeq \tilerow_\tileheight}^\ast
        }
        \numsep
    }^\ast
\]
where each $\tilerow_i$ is a row of a correct solution to the tiling problem.

Thus, with $\expheight+1$ transducers, we can encode a $\expheight$-$\expspace$-hard problem.


%==========================================================================================

\section{Two-way transducers with length constraints}

This section is dynamical: we hope for the best of Anothy's result; in case it does not work, we have two possible backups: (1) reversal-bounded 2-way transducers; (2) using reversal-bounded counter machines to represent (both regular and length) constraints

%===========================================================================================

\section{One-way transducers with variables}

Not quite sure whether we need this section, it might be just a simple generalisation of the popl'18 paper, or be subsumed by the next section ; we will see.

%========================================================================================

\section{Two-way transducers with variables}

\subsection{Pre-image computation of 2-way transducers}

Let $\vec{y}=\{y_1, \cdots, y_m\}$.

The general idea is to encode a general string manipulating function $f(x, \vec{y})$ as a NFT $T$ over $\Sigma$ and $\Sigma\cup\{\vec{y}\}$. The question for the pre-image computation is formalised as follows:
\begin{itemize}
	\item INPUT: A NFT $T$, a regular language $\mathcal{A}$.
	\item OUTPUT: $(L^{(0)}_i, L^{(1)}_i, \cdots, L^{(m)}_i )_{i=1}^\ell$, such that
	\[\exists z\in\mathcal{A} \wedge z=f(x, \vec{y})\mbox{ iff }\exists k. x\in L^{(0)}_k \wedge y_i\in L^{(i)}_k \]
\end{itemize}

%===========================================================================================

%\section{Matt's pet :-), maybe another paper}
%
%\begin{definition}[multi-tape transducer, with k input tapes, and one output tape.]
%	\tl{Matt, please elaborate}
%	The
%	k input tapes follow a stack discipline: tape i can only move if the
%	head position of all tapes $j > i$ is 0
%\end{definition}


\section{Conclusion}

%==============================================================================================
\newpage

% Bibliography
\bibliographystyle{plain}
\bibliography{string}

\appendix

\end{document}
