%!TEX root = main.tex

\section{Symbolic extensions}
\label{sec:symbolic}

In this section, we consider an extension of string manipulating programs to the symbolic setting, where finite-state automata and parametric transducers are replaced by \emph{symbolic automata} \cite{NG01,DV14} and \emph{symbolic parametric transducers} (to be introduced) respectively.
%
Intuitively, in symbolic automata %are an extension of finite-state automata in which 
the input letters are symbolically described by unary predicates from a  logical theory. %On the other hand, 
%
Likewise, in symbolic parametric transducers %are an extension of parametric transducers in which 
the output letters are given by terms in the logic theory (where the parameters may occur),  apart from the symbolic input letters. Symbolic parametric transducers can be seen as an extension of both the parametric transducer introduced in this paper and the symbolic transducer introduced in \cite{VHLMB12}.
%In the sequel, we first define symbolic automata and transducers, then introduce symbolic parametric transducers.
% and show how to extend the results in preceding sections to the symbolic setting.


%Let $\data$ be an infinite data domain.  We define data strings as elements of $\data^\ast$.
We assume familiarity with the 
%(many-sorted) 
first-order logic (cf.\ some textbook e.g. \cite{EFT94}).
A concept of background theories is used in the definition of symbolic automata and transducers\footnote{In literature, a symbolic automata is defined on a Boolean algebra, which is a background theory without requiring the set of formulae to be closed under substitutions. We define symbolic automata on background theories (rather than Boolean algebras) in Definition~\ref{def-2sa} to align to the symbolic parametric transducers defined later.}.
%and label theories. 
Intuitively, 
%a Boolean algebra is a set of unary first-order logic formulae closed under Boolean connectives, 
a  background theory is a set of unary first-order logic formulae closed under Boolean connectives and substitutions. 
 %Label theories extend background theories further by adding inequalities of terms into the set of formulae.
 
 
 \begin{definition}[Background theories]
A background theory $\Upsilon$ is a tuple $(\signature, \structure, \Psi)$ satisfying the following constraints:
\begin{itemize}
\item $\signature=(\functions, \predicates)$ is a signature, where $\functions$ (resp. $\predicates$) is a recursively enumerable set of function symbols (resp. relation symbols), 
%\begin{itemize}
%\item
%
%\item for each $\vec{s} = s_1 \times \dots \times s_n$ and $i \in [n]$, there is a function $\pi_{\vec{s}, i}$ of arity $s_1 \times \dots \times s_n \rightarrow s_i$ in $\functions$ such that $\pi_{\vec{s},i}(\vec{t}) = t_i$ for each term $\vec{t} = (t_1,\dots, t_n)$ of sort $s_1 \times \dots \times s_n$.
%\end{itemize}

\item $\structure$ is an $\signature$-structure $(\data, \interpretation)$, where $\data$ is a countably infinite set (the data domain), and $\interpretation$ is a function on $\data$ satisfying: for each $n$-ary function symbol $f \in \functions$ (resp. relation symbol $R \in \predicates$), $\interpretation(f)$ is an $n$-ary function on $\data$ (resp. $\interpretation(R)$ is an $n$-ary relation on $\data$),
%
\item $\Psi$ is a recursively enumerable set of \emph{unary} $\signature$-formulae closed under Boolean connectives $\vee, \wedge, \neg$ and substitutions, that is, for each $\signature$-term $t(x)$ and $\psi(x) \in \Psi$, we have $\psi[t(x)/x] \in \Psi$. 
%In addition, it is assumed that each formula $\psi \in \Psi$ of arity $s^\ell$ (where $\ell > 0$) has the same set of free variables $\{x_1,\dots, x_\ell\}$.
\end{itemize}
%A background theory is a Boolean algebra $(\signature, \structure, \Psi)$ such that $\Psi$ is closed under substitutions, that is, for each $\signature$-term $t(x)$ and $\psi(x) \in \Psi$, we have $\psi[t(x)/x] \in \Psi$. \\
%
For each $\psi(x) \in \Psi$, we use $\|\psi\|$ to denote the set $\{d \in \data \mid  \structure \models \psi(d)\}$. Elements of $\| \psi\|$ are called the \emph{witnesses} of $\psi$.
A formula $\psi \in \Psi$ is \emph{satisfiable}, denoted by $\issat(\psi)$, if $\|\psi\| \neq \emptyset$. A background theory $\Upsilon = (\signature, \structure, \Psi)$ is \emph{decidable} if  checking  $\issat(\psi)$ for a  given $\psi \in\Psi$ is decidable.
\end{definition}


In the rest of this section, we {\bf fix a decidable background theory} $\Upsilon=(\signature, \structure, \Psi)$ with $\signature=(\functions, \predicates)$ and $\structure = (\data, \interpretation)$. Moreover, we assume that given a formula $\psi \in \Psi$ of size $m$ in $\Upsilon$, $\issat(\psi)$ can be decided in $\bigO(\beta(m))$ space for some monotone function $\beta$ such that $\beta(m) \ge \log m$.
A \emph{data string} is an element of $\data^*$. A \emph{data language} is a set of data strings.



%%%%===========================================================
\hide{
\begin{definition}[Background theories]
A background theory  $\Upsilon$ is a tuple $(\signature, \structure, \Psi)$ satisfying the following constraints:
\begin{itemize}
\item $\signature=(\functions, \predicates)$ is a signature, where $\functions$ (resp. $\predicates$) is a recursively enumerable set of function symbols (resp. relation symbols), 
%\begin{itemize}
%\item
%
%\item for each $\vec{s} = s_1 \times \dots \times s_n$ and $i \in [n]$, there is a function $\pi_{\vec{s}, i}$ of arity $s_1 \times \dots \times s_n \rightarrow s_i$ in $\functions$ such that $\pi_{\vec{s},i}(\vec{t}) = t_i$ for each term $\vec{t} = (t_1,\dots, t_n)$ of sort $s_1 \times \dots \times s_n$.
%\end{itemize}

\item $\structure$ is an $\signature$-structure $(\data, \interpretation)$, where $\data$ is a countably infinite set (the data domain), and $\interpretation$ is a function on $\data$ satisfying: for each $n$-ary function symbol $f \in \functions$ (resp. relation symbol $R \in \predicates$), $\interpretation(f)$ is an $n$-ary function on $\data$ (resp. $\interpretation(R)$ is an $n$-ary relation on $\data$),
%
\item $\Psi$ is a recursively enumerable set of unary $\signature$-formulae closed under Boolean connectives $\vee, \wedge, \neg$. In addition, $\Psi$ is closed under substitutions, that is, for each $\signature$-term $t(x)$ and $\psi(x) \in \Psi$, we have $\psi[t(x)/x] \in \Psi$.
%In addition, it is assumed that each formula $\psi \in \Psi$ of arity $s^\ell$ (where $\ell > 0$) has the same set of free variables $\{x_1,\dots, x_\ell\}$.
For each $\psi(x) \in \Psi$, we use $\|\psi\|$ to denote the set $\{d \in \data \mid  \structure \models \psi(d)\}$. Elements of $\| \psi\|$ are called the \emph{witnesses} of $\psi$.
\end{itemize}
A formula $\psi \in \Psi$ is \emph{satisfiable}, denoted by $\issat(\psi)$, if $\|\psi\| \neq \emptyset$. In addition, $\Upsilon$ is \emph{decidable} if  checking  $\issat(\psi)$ for a  given $\psi \in\Psi$ is decidable.
\end{definition}
}
%%%%===========================================================


\begin{definition}[Symbolic automata]\label{def-2sa}
    A (nondeterministic two-way)  symbolic automata (\SSA) is a tuple $\Aut = (\EndLeft, \EndRight, \controls, q_0, \finals, \transrel)$, where  
\begin{itemize}
%\item $\Upsilon=(\signature, \structure, \Psi)$ is a decidable Boolean algebra, where $\signature = (\functions, \predicates)$ and $\structure = (\data, \interpretation)$,
%
\item $\EndLeft$, $\EndRight$, $\controls$, $q_0$, $\finals$ are defined precisely as in \FFA{}s, 
%
\item $\transrel$ is a finite set of transitions of one of the following forms,
\begin{itemize}
\item     symbolic transitions $(q, \psi(x), dir, q') \in \controls \times \Psi \times \{\Left, \Stay, \Right\} \times \controls$, where $q, q'$ are the source and target state, $dir$ is the direction, and $\psi(x)$ is the guard of the transition, 
%
\item     non-symbolic transitions $(q, \EndLeft, dir, q')$ with $dir \in \{\Right, \Stay\}$, or $(q, \EndRight, dir', q')$ with $dir' \in \{\Left,\Stay\}$. 
\end{itemize}
\end{itemize}
A \SSA{} is an \SA{} if there are no transitions with the direction ``$\Left$". 
%for each $(q, \psi(x), dir, q')$, we have $dir \in \{\Stay, \Right\}$, moreover, the transitions of the form $(q, \EndRight, \Left, q')$ are removed.
\end{definition}

\paragraph{Semantics of \SSA{} $\Aut = (\Upsilon, \EndLeft, \EndRight, \controls, q_0, \finals, \transrel)$.}
A transition $\tau=(q_1,\psi(x), dir, q_2) \in \transrel$ in the \SSA $\Transducer$ can be concretised
into a potentially infinite set of \emph{concrete} transitions $\|\tau\| \subseteq Q \times \data \times \{\Left, \Stay, \Right\} \times Q$, where $(q_1, d, dir, q_2)  \in \|\tau\|$ iff $d \in \|\psi\|$.
Intuitively, suppose $\Aut$ is at the state $q_1$ and reading the input data value $d \in \data$.
If there is a transition $(q_1, \psi(x), dir, q_2) \in \transrel$ such that $d \in \|\psi\|$, then $\Aut$ can move its reading head according to $dir$, and change the state from $q_1$ to $q_2$.

%$\Aut$ is interpreted on \emph{data strings}, namely, elements of $\data^*$.
Given a data string $w = d_1, \dots, d_n$, a \emph{run} of $\Aut$ on $w$
is a sequence of tuples $(q_0, i_0), \ldots, (q_m, i_m) \in \controls \times [0, n+1]$ 
such that
%let $d_0 = \EndLeft$ and $d_{n+1} = \EndRight$, %. The following conditions, then, have to be satisfied:
\begin{itemize}
    \item $i_0 = 0$, and
    \item for every $j \in [m-1]$, one of the following holds,
    \begin{itemize}
  	\item $0 < i_j < n+1$,  and $\tau=(q_j, \psi(x), dir, q_{j+1}) \in \transrel$ for some $\psi(x) \in \Psi$ and $dir \in \{\Left, \Stay, \Right\}$, such that $(q_j, d_{i_j}, dir, q_{j+1}) \in \|\tau\|$, and $i_{j+1} = i_j + dir$,
	%
	\item $i_j = 0$, $(q_j, \EndLeft, dir, q_{j+1}) \in \transrel$, and $i_{j+1} = i_j + dir$,
	%
	\item $i_j = n+1$, $(q_j, \EndRight, dir, q_{j+1}) \in \transrel$, and $i_{j+1} = i_j + dir$.
  \end{itemize}
\end{itemize}
The run is said to be \defn{accepting} if $i_m = n+1$ and $q_m \in \finals$.
A data string $w$ is said to be accepted by $\Aut$ if there is an accepting run of
$\Aut$ on $w$. We use $\Lang(\Aut)$ to denote the data language recognised by $\Aut$, that is,  the set of data strings accepted by $\Aut$.
%
For simplicity, we usually omit $\EndLeft$, and $\EndRight$ from the definition of \SSA{}s. 

To measure the size of a \SSA{} $\Aut$, we need take the guards of the transitions (from the background theory $\Upsilon$) into  consideration. Therefore, we assume that the formulae in $\Psi$ are encoded in a proper way. %e.g. encoded in binary. 
For $\psi \in \Psi$, we use $|\psi|$ to denote the number of symbols in the encoding of $\psi$. Moreover, we define the size of a symbolic transition $\tau = (q_j, \psi(x), dir, q_{j+1})$ to be $|\psi|$.
For a refined complexity analysis, we use the following two measures of $\Aut$: %numbers to capture the size of $\Transducer$:  Define 
the \emph{structural size} and the \emph{data size} of $\Aut$, denoted by $|\Aut|_s$ and $|\Aut|_d$ respectively, where $|\Aut|_s$ is defined as the number of transitions of $\Aut$, and $|\Aut|_d$ is defined as the maximum size of symbolic transitions of $\Aut$.


\begin{proposition}\label{prop-2nsa}
Assume a \SSA  $\Aut$. Then
\begin{itemize}
\item $\Aut$ can be  transformed into an equivalent \SA{} $\Aut'$ in time $|\Aut'|_s \cdot |\Aut'|_d$, where $|\Aut'|_s = 2^{\bigO\left(|\Aut|_s \log |\Aut|_s\right)}$ and $|\Aut'|_d = \bigO(|\Aut|_s |\Aut|_d)$;
%$|\Aut'| = 2^{O (|\Aut| \log |\Aut| )}$.
%
%
%\item SAs are under the intersection operator\footnote{As a matter of fact, SAs are closed under all Boolean operations. Since we only use the intersection in this paper, we restrict our attention to the intersection.}. \zhilin{complexity}
%
\item the nonemptiness of $\Aut$ can be decided in nondeterministic $\bigO( (\log |\Aut|_c) + \beta(|\Aut|_d))$ space.
\end{itemize}
\end{proposition}
The proof of the first result in Proposition~\ref{prop-2nsa} can be found in Appendix~\ref{app-sym}, while the second result is obtained by guessing a path in $\Aut$ and was shown in \cite{NG01,DV14}.

\smallskip

Next, we introduce symbolic transducers.


%we derive a new sort $s^*$ such that $\|s^* \| = \bigcup \limits_{i \in \Nat}\data_s^i$, where $\data^i_s = \{\varepsilon\}$. An element of $\|s^* \|$ is called a \emph{data string} of sort $s$.




% ======= label theories are removed, which are only useful for deterministic SA ================
% ======= label theories are removed, which are only useful for deterministic SA ================
\hide{
\begin{definition}[Label theories]
A label theory  $\Upsilon$ is a tuple $(\signature, \structure, \Psi, \Psi')$ satisfying the following constraints:
\begin{itemize}
\item $(\signature, \structure, \Psi)$ is a background theory with $\signature = (\functions, \predicates)$,
%
\item $\Psi'$ is the set of formulae of the form $\psi(x) \wedge t_1(x) \neq t_2(x)$, where $\psi(x) \in \Psi$ and $t_1(x), t_2(x)$ are $\signature$-terms.
\end{itemize}
A label theory is decidable if it is decidable to check $\issat(\psi)$ for $\psi \in \Psi \cup \Psi'$.
\end{definition}

Given a formula $\psi(x) \in \Psi$ and two $\signature$-terms $t_1(x), t_2(x)$,
$t_1$ and $t_2$ are \emph{equivalent up to $\psi$}, denoted by $t_1 \simeq_\psi t_2$, if
 $\issat(\psi(x) \wedge t_1(x) \neq t_2(x))$ does not hold.
%Two sequences of $s^i/ s$-terms $\vec{f}=f_1...f_n$ and $\vec{g}=g_1...g_m$ are \emph{equivalent up to $\psi$}, denoted by $\vec{f}\simeq_\psi \vec{g}$,
%iff $n=m$ and for every $j \in [n]$, $f_j \simeq_\psi g_j$.

%Given an $s^i/ s$-term sequence $\vec{f}=f_1...f_n$ and a sequence of data values $\vec{d} = (d_1, \dots, d_i) \in (\data_{s_{\sf in}})^i$,
%let $\|\vec{f}\|(\vec{d})$ denote the sequence $\|f_1\|(\vec{d})...\|f_n\|(\vec{d})$, that is, a data string of sort $s$.
}
% ======= label theories are removed, which are only useful for deterministic SA ================
% ======= label theories are removed, which are only useful for deterministic SA ================


\begin{definition}[Symbolic finite-state transducers]
    A nondeterministic two-way symbolic transducer (\SST) is an extension of an \SSA{} with outputs. More precisely, a \SST{} $\Transducer$ is a tuple $(\EndLeft, \EndRight, \controls, q_0, \finals, \transrel)$ where  
\begin{itemize}
%\item $\Upsilon=(\signature, \structure, \Psi)$ is a decidable background theory, where $\signature = (\functions, \predicates)$ and $\structure = (\data, \interpretation)$,
%
\item $\EndLeft$, $\EndRight$, $\controls$, $q_0$, $\finals$ are defined precisely as in \SSA{}s, 
%
\item $\transrel$ is a finite set of  transitions which are in one of the following forms,
\begin{itemize}
\item  symbolic transitions $(q, \psi(x), dir, q', \vec{t}(x))$ such that $q, q' \in Q$, $dir \in \{\Left, \Stay, \Right\}$, $\psi(x) \in \Psi$ and
$\vec{t}(x) = t_1(x) \ldots t_r(x)$ is a (possibly empty) sequence of $\signature$-terms, 
\item   $(q, \EndLeft, dir, q', \epsilon)$ or $(q, \EndRight, dir', q', \epsilon)$, with $dir \in \{\Right, \Stay\}$ and $dir' \in \{\Left, \Stay\}$,
\end{itemize}
\end{itemize}
A \SST{} is an \ST{} if there are no transitions with the direction ``$\Left$". 
%$(q, \psi(x), dir, q', \vec{t}(x))$ we have $dir \in \{\Stay, \Right\}$ and the transitions of the form $(q, \EndRight, \Left, q', c)$ are not present.
\end{definition}

\paragraph{Semantics of \SST{}.}
The semantics of \SST{}s is defined similarly to that of \SSA{}s. 
A transition $\tau=(q_1,\psi(x), dir, q_2, \vec{t}(x)) \in \transrel$ with $\vec{t}(x) = t_1(x) \ldots t_r(x)$ in the \SST{} $\Transducer$ can be concretised
into a potentially infinite set of \emph{concrete} transitions $\|\tau\| \subseteq Q \times \data \times \{\Left, \Stay, \Right\} \times Q \times \data^r$, where $(q_1, d, dir, q_2, d'_1\ldots d'_r)  \in \|\tau\|$ iff $d \in \|\psi\|$ and $d'_j = \interpretation(t_j)(d)$ for each $j \in [r]$.
%Intuitively, suppose $\Transducer$ is at the state $q_1$ and reading the input data value $d \in \data$.
%If there is a transition $(q_1, \psi(x), dir, q_2, t(x) )\in \transrel$ and $d \in \|\psi\|$, $\Transducer$ can move its reading head according to $dir$, update the current state $q_1$ to $q_2$,  and produce the data value $d' \in \data$.

Given a data string $w = d_1, \dots, d_n$, a \emph{run} of $\Transducer$ on $w$
is a sequence of tuples $(q_0, i_0, \vec{d}'_0), \ldots, (q_m, i_m, \vec{d}'_m) \in \controls \times [0, n+1] \times \data^*$ 
such that
%let $d_0 = \EndLeft$ and $d_{n+1} = \EndRight$, %. The following conditions, then, have to be satisfied:
\begin{itemize}
    \item $i_0 = 0$, and
    \item for every $j \in [m-1]$, one of the following holds,
    \begin{itemize}
  	\item  $0< i_j < n+1$, $\tau=(q_j, \psi(x), dir, q_{j+1}, \vec{t}(x)) \in \transrel$ for some $\psi(x) \in \Psi$, $dir \in \{\Left, \Stay, \Right\}$, and a sequence of  $\signature$-terms $\vec{t}(x)$, such that $(q_j, d_{i_j}, dir, q_{j+1}, \vec{d}'_j) \in \|\tau\|$, and $i_{j+1} = i_j + dir$,
	%
	\item $i_j = 0$, $(q_j, \EndLeft, \Right, q_{j+1}, \epsilon) \in \transrel$, and $\vec{d}'_j  = \epsilon$,
	%
	\item $i_j = n+1$, $(q_j, \EndRight, \Left, q_{j+1}, \epsilon) \in \transrel$, and $\vec{d}'_j  = \epsilon$.
  \end{itemize}
\end{itemize}
The run is said to be \defn{accepting} if $i_m = n+1$ and $q_m \in \finals$. When a run is accepting, $\vec{d}'_0, \ldots, \vec{d}'_m$ is the \emph{output} of the run.
A data string $w'$ is said to be an output of $\Transducer$ on $w$ if there is an accepting run of
$\Transducer$ on $w$ with output $w'$. We use $\Tran(\Transducer)$ to denote the \emph{transduction} defined by $\Transducer$, that is, the relation comprising the data-string pairs $(w, w')$ such that $w'$ is an output of $\Transducer$ on $w$.
%
We define $|\Transducer|_s$ and $|\Transducer|_d$, the structural and data size of $\Transducer$, by taking the outputs of transitions into consideration. Specifically, $|\Transducer|_s$ is defined as the product of the number of transitions and the maximum length of sequences of $\signature$-terms in  transitions of $\Transducer$. 
On the other hand, $|\Transducer|_d$ is defined as the maximum size of symbolic transitions of $\Transducer$, where the size of a symbolic transition $(q_j, \psi(x), dir, q_{j+1}, \vec{t}(x))$ with $\vec{t}=t_1 \ldots t_r$ is defined as $|\psi| + |t_1| + \ldots + |t_r|$.
%the number of transitions of $\Transducer$, and $\sizetrans(\Transducer)$, the maximum size of transitions of $\Transducer$.


%\begin{proposition}\label{prop-2nst-nst}
%Each 2ST can be turned into an equivalent ST. 
%\end{proposition}

We are ready to introduce symbolic parametric transducers.

\begin{definition}[Symbolic parametric transducers]
A nondeterministic two-way symbolic parametric transducer (\SSPT) is an extension of a \SST{} with parameters. More precisely,  a \SSPT{} $\Transducer$ is a tuple
$(\EndLeft, \EndRight, Y, \controls, q_0, \finals, \transrel)$ where:
\begin{itemize}
%\item $\Upsilon=(\signature,\structure, \Psi, \Psi')$ is a decidable background theory with $\signature = (\functions, \predicates)$ and $\structure = (\data, \interpretation)$,
%
\item $\EndLeft$, $\EndRight$, $\controls$, $q_0$, $\finals$ are defined precisely as in \SST{}s, 
%
\item $Y=\{y_1,\ldots, y_m\}$ is a finite set of parameters, 
%
\item $\transrel$ is a finite set of transitions in one of the following forms: 
\begin{itemize}
\item symbolic transitions $(q, \psi(x), q', \vec{t})$ such that $q,q' \in Q$, $\psi(x) \in \Psi$,
$\vec{t} = t_1 \ldots t_r$ such that for each $i \in [r]$, either $t_i$ is an $\signature$-term where only the variable $x$ occurs, or $t_i = y_j$ for some $y_j \in Y$,
%
\item non-symbolic transitions $(q, \EndLeft, dir, q', \epsilon)$, $(q, \EndRight, dir, q', \epsilon)$, with $dir \in \{\Right, \Stay\}$ and $dir' \in \{\Left, \Stay\}$. 
\end{itemize}
\end{itemize}
\end{definition}

Notice that, as in \PPT{}s, parameters are only allowed in the output track.
Intuitively, each instantiation of the parameters $y_1,\ldots, y_m$ with data strings 
$w_1,\ldots, w_m$ gives rise to a \SST{} $\Transducer[w_1/y_1,\ldots, w_m/y_m]$ by replacing each occurrence of $y_i$ in transitions with $w_i$. 
%outputs the data string $w_j$, whenever a transition of the form $(p, \psi, dir, q, y_j)$ is
%taken by the transducer. 
%This instantiation of the parameters is only done 
%\emph{once} before the symbolic parametric transducer is run.
%
We use $\transet(\Transducer)$ to denote the set of tuples $(w, w_1, \ldots, w_m, w')$ such that $(w, w') \in \transet(\Transducer[w_1/y_1,\ldots, w_m/y_m])$.
The two size measures of \SSPT{}s can be defined in a similar way as \SST{}s. 


%%%============================================
%%%============================================
\hide{
%To extend the string constraint language $\straightline[\transet]$ to the symbolic setting, 
To lift the result to the symbolic setting, we only need  \emph{symbolic automata}, the counterpart of \FA. We start with \emph{Boolean algebra}, 
\tl{I am not an expert, but technically say $\Psi$ is a boolean algebra? More precisely, \SA{} works not on a background theory, but some more relaxed thing.}
which is adapted from a background theory $\Upsilon=(\signature,\structure, \Psi)$ by dropping out the requirement that $\Psi$ is closed under substitutions.

\tl{as we donot have $\straightlinesym[\transet]$ anymore? to be updated}
For a class $\transet$ of symbolic parametric transducers, we use $\straightlinesym[\transet]$ to denote the extension of straight-line string constraint language by replacing parametric transducers with symbolic parametric transducers and finite-state automata with symbolic automata.
}
%%%============================================
%%%============================================


\paragraph{Symbolic extension of the generic decision procedure.} We consider the path feasibility problem of data string manipulating programs in SSA form where each assertion is a conjunction of atomic constraints of the form $x \in \Aut$ with $\Aut$ an \SA. 
To solve the path feasibility problem of such data string manipulating programs,
the generic decision procedure in Section~\ref{sec:algo} can be smoothly extended, by replacing recognisable relations and the \prerec{} assumption with symbolically recognisable relations  and the $\mathbb{S}$\prerec{} assumption presented below.

\begin{definition}[Symbolic recognisable relations]
%	Given a decidable Boolean algebra $\Upsilon=(\signature, \structure, \Psi)$ with $\signature = (\functions, \predicates)$ and $\structure = (\data, \interpretation)$, 
	A $k$-ary relation $R \subseteq \data^*\times \ldots\times \data^*$ is a \emph{symbolic recognisable} relation if $R=\bigcup_{i=1}^n L^{(i)}_1 \times \ldots \times L^{(i)}_k$ where $L^{(i)}_j$ is recognised by some \SA{} for each $j\in [k]$.
\end{definition}
Similar to recognisable relations, we also represent a  $k$-ary symbolic recognisable relation as a collection of tuples $(\Aut_1, \ldots, \Aut_k)$, where each atom $\Aut_i$ is a conjunctive representation of \SA{}, namely, of the form $((\controls, \transrel), S)$ such that $S \subseteq \controls \times \controls$. We will use a conjunctive \SA{} to mean a conjunctive representation of the \SA{}. For a conjunctive \SA{} $\cC= ((\controls, \transrel), S)$, the two size measures $|\cC|_s$ and $|\cC|_d$ are defined similarly to \SA{}s. Moreover, the structural (resp. data) size of a representation of a symbolic recognisable relation $R$  is defined as the maximum structural (resp. data) size of the atoms.

\begin{quote}
{\bf The $\mathbb{S}$\prerec{} assumption}. For each data string function $f$ in $S$ and each conjunctive \SA{} $\Aut$,  $\Pre_{R_f}(\Aut)$ is a symbolic recognisable relation. Furthermore, 
a representation of $\Pre_{R_f}(\Aut)$, whose control size (resp. data size) is bounded by  $\ell_s(|f|_s, |\Aut|_c)$ (resp. $\ell_d(|f|_s,  |f|_d, |\Aut|_s, |\Aut|_d)$) for some monotone functions $\ell_s$ and $\ell_d$, can be computed effectively. We also assume that each disjunct of the representation can be nondeterministically computed in $\ell_s(|f|_s, |\Aut|_s) \cdot \ell_d(|f|_s,  |f|_d, |\Aut|_s,  |\Aut|_d)$ space.
%the space to compute a representation of $\Pre_{R_f}(\Aut)$ and the atom size of such a representation\footnote{Usually the amount of space is at least as large as the size of the representation; here for simplicity we assume that they are asymptotically the same.} are both bounded by $\ell(|f|, |\Aut|)$ for some monotone function $\ell$.  
%
%\item[A2] Each assertion $g$ in $S$ is given as a conjunction of atomic regular constraint $x\in \Aut$. 
\end{quote} 
%
Here $|f|_s$ and $|f|_d$ are the structural and data sizes of a representation of $f$ respectively; the concrete definitions depend on the form of $f$, which will be given when the generic decision procedure is instantiated later. Furthermore, we define the size of $f$ (resp. $|\Aut|$), denoted by $|f|$ (resp. $\Aut$), as $|f|_s \cdot |f|_d$ (resp. $|\Aut|_s \cdot |\Aut|_d$).

Let $S$ be a data string manipulating program in SSA form.  We use $|S|$ to denote the sum of $|f|$ and $|\Aut|$ for data string functions $f$ and \SA{}s $\Aut$ occurring in $S$.  
For refined complexity analysis, we also define $\rcdep(S)$ and $\rcdim(S)$ similarly to the string setting. Moreover, we use $\rcphi_s(S)$ (resp. $\rcphi_d(S)$) to denote the maximum of $|f|_s$ (resp. $|f|_d$) for string functions $f$ occurring in the assignments of $S$, and $\rcpsi_s(S)$ (resp. $\rcpsi_d(S)$) to denote the maximum of $|\Aut|_s$ (resp. $|\Aut|_d$) for \SA{}s $\Aut$ occurring in $S$. Finally, we define $\rcsreg(S)$ to be the maximum number of atomic \SA{} constraints for one variable in $S$.


%Similarly, we can define $\numtrans(\psi)$ and $\sizetrans(\psi)$.


%We can adapt the definition of $\numtrans$ and $\sizetrans$ to the conjunctive representations of SA easily. We will use conjunctive SA to mean a conjunctive representation of the SA.

For a binary function $g(k, l)$, $h(i, j, k, l)$, and $n \in \Nat \backslash \{0\}$, we define $g^{\langle n \rangle}(k, l)$ and $h^{\langle n \rangle}_g(i, j, k, l)$ as follows: $g^{\langle 1 \rangle}( k, l) = g(k, l)$, $h^{\langle 1 \rangle}_g(i, j, k, l) = h(i, j, k, l)$, and 
$$g^{\langle n+1 \rangle}( k, l) = g(k, g^{\langle n \rangle}( k, l)), \ h^{\langle n+1 \rangle}_g (i, j, k, l) = h(i, j, g^{\langle n \rangle}(k, l), h^{\langle n \rangle}(i, j, k, l)).$$


\begin{theorem}\label{thm-generic-dec-symbolic}
	Given a data string manipulating program $S$ satisfying the $\mathbb{S}$\prerec{} assumption, the path feasibility problem of $S$ can be decided in \emph{nondeterministic} $M+ \beta(M)$ space, where  
	\[
	\begin{array}{l c l}
		M & = & |\vars(S)| \cdot (\rcdim(S)+1)^{\rcdep(S)}  \rcsreg(S) \cdot  (\ell_s^{\langle \rcdep(S) \rangle}(\rcphi_s(S), \rcpsi_s(S)))^{r}\ \cdot \\
		& &  \hspace{1cm} (\ell_d)^{\langle  \rcdep(S) \rangle}_{\ell_c}(\rcphi_s(S),\rcphi_d(S),  \rcpsi_s(S), \rcpsi_d(S))
\end{array}
	\]
for some constant $r > 0$. 
\end{theorem}



%%%=============================================================
\hide{
\begin{theorem}\label{thm-generic-dec-symbolic}
Let $\transet$ be a class of symbolic parametric transducers. Suppose that %$\transet$ satisfies that 
for each $\Transducer \in \transet$ and conjunctive \SA{} $\Aut$, $\Pre_\Transducer(\Aut)$ is a symbolic recognisable relation and a representation of which can be computed effectively, 
such that for every atom $\Aut'$ therein, 
$\numtrans(\Aut') = O(g(\numtrans(\Transducer),  \numtrans(\Aut)))$ and 
$\sizetrans(\Aut') = O(h(\numtrans(\Transducer), \sizetrans(\Transducer), \numtrans(\Aut), \sizetrans(\Aut)))$, for two monotone functions $g$ and $h$.
Then the path feasibility of a $\straightlinesym[\transet]$ formula $\varphi \wedge \psi$ can be decided in nondeterministic space
$$
O\left(
\begin{array}{l}
(\rcdim(\varphi)+2)^{\rcdep(\varphi)}|\psi| \cdot (\numtrans'')^2 \log \numtrans'' + \\
f((\rcdim(\varphi)+2)^{\rcdep(\varphi)} (\numtrans'')^2 \sizetrans'')  
\end{array}
\right),
$$ 
where $\numtrans'' = g^{\langle \rcdep(\varphi) \rangle}(\numtrans(\varphi), \numtrans(\psi))$ and 
$$\sizetrans'' = h^{\langle \rcdep(\varphi) \rangle}_g(\numtrans(\Transducer), \sizetrans(\Transducer), \numtrans(\Aut), \sizetrans(\Aut)).$$
\end{theorem}
}
%%%=============================================================

\paragraph{Instantiation of the generic decision procedure to \SSPT{}s and \SPT{}s.}

Note that for illustration purpose, we choose not to instantiate the decision procedure to $k$-RB2SPTs here.

%\noindent Finally, we instantiate the generic decision procedure to \SSPT{}s.

\begin{lemma}\label{lem-spt}
The $\mathbb{S}$\prerec{} assumption holds for \SSPT{}s, with $\ell_s(|\Transducer|_s, |\Aut|_s) = 2^{\bigO( |\Transducer|_s |\Aut|_s^{|\Transducer|_s} \log (|\Transducer|_s |\Aut|_c^{|\Transducer|_c}))}$ and  $\ell_d(|\Transducer|_s, |\Transducer|_d, |\Aut|_s, |\Aut|_d) = \bigO(|\Transducer|_s  |\Transducer|^2_d |\Aut|_s^{|\Transducer|_s} |\Aut|_d)$. 

The $\mathbb{S}$\prerec{} assumption holds for \SPT{}s, with $\ell_s(|\Transducer|_s, |\Aut|_s) =\bigO(|\Transducer|_s |\Aut|_s^{|\Transducer|_s})$ and  $\ell_d(|\Transducer|_s, |\Transducer|_d, |\Aut|_s, |\Aut|_d) = \bigO(|\Transducer|^2_d |\Aut|_d)$.
%
%Given a \SSPT{} $\Transducer$ and a conjunctive \SA{} $\Aut$, $\Pre_\Transducer(\Aut)$ is a symbolic recognisable relation and a representation of which can be computed effectively, such that for each atom $\Aut'$ therein, 
%$|\Aut'|_c = 2^{\bigO(|\Transducer|_c |\Aut|_c \log (|\Transducer|_c |\Aut|_c))},$ 
%and 
%$|\Aut'|_d = |\Transducer|_c |\Aut|_c |\Transducer|_d |\Aut|_d.$
%On the other hand, if $\Transducer$ is an \SPT, then $|\Aut'|_c$ and $|\Aut'|_d$ are reduced to $\bigO(|\Transducer|_c |\Aut|_c)$ and $\bigO(|\Transducer|_d |\Aut|_d)$ respectively.
\end{lemma}


From Theorem~\ref{thm-generic-dec-symbolic} and Lemma~\ref{lem-spt}, we have the following result.
\begin{theorem}\label{thm-spt}
The path feasibility of data string manipulating programs, where the data string functions are given by \SSPT{}s, can be decided in nondeterministic $\tower(\rcdep(S)+1, \bigO(|S|^2)) + \beta(\tower(\rcdep(S)+1, \bigO(|S|^2)))$  space. On the other hand,  the path feasibility for \SPT{}s can be decided in nondeterministic $|S|^{|S|^{\bigO(\rcdep(S))}} + \beta(|S|^{|S|^{\bigO(\rcdep(S))}})$ space.
\end{theorem}



%==========================================================================
%==========================================================================
%============================many sorted first-order logic=========================
%==========================================================================
%==========================================================================

\hide
{

\paragraph{Many-sorted first-order logic.}
We assume a signature $\signature=(\sorts, \functions, \predicates)$, where $\sorts$ is a countable set of \emph{sorts}, $\functions$ is a countable set of \emph{function symbols}, and $\predicates$ is a countable set of \emph{predicate symbols}. Each function or predicate symbol has an associated \emph{arity}, which is a tuple of sorts in $\sorts$.  A function symbol with a single sort is called a \emph{constant}. A predicate symbol with a single sort is called a \emph{set}, which intuitively denotes a set of elements of that sort.

An $\signature$-term is built from the function symbols in $\functions$ and variables taken from a set $\mathcal{X}$ that is disjoint from $\sorts$, $\functions$, and $\predicates$. Each variable $x \in \mathcal{X}$ has an associated sort in $\sorts$. In addition, we assume that the variables in $\mathcal{X}$ are linearly ordered $\preceq_{\mathcal{X}}$. When writing $t(\vec{x})$ for a vector of distinct variables $\vec{x}$ such that $\vec{x} = (x_1,\dots, x_n)$ follows the ascending order of the linear order $\preceq_{\mathcal{X}}$, we assume that the variables occurring in the term $t$ are from $\vec{x}$. For a term $t(\vec{x})$ of sort $s$ such that $\vec{x} = (x_1, \dots, x_n)$ and each $x_i$ for $i \in [n]$ is of sort $s_i \in \sorts$, the term $t$ is said to be \emph{of arity} $(s_1 \times \dots \times s_n) \rightarrow s$. In addition, for a vector of terms $(t_1, \dots, t_m)$ such that all the variables of $t_1 ,\dots, t_m$ are from $\vec{x} = (x_1, \dots, x_n)$, if $x_1 \preceq_{\mathcal{X}} x_2  \preceq_{\mathcal{X}} \dots  \preceq_{\mathcal{X}} x_n$, each $x_i$ for $i \in [n]$ is of sort $s_i$, and each $t_j$ for $j \in [m]$ is of sort $s'_j$, then $(t_1,\dots, t_m)$ is said to be a term of arity $(s_1,\dots, s_n) \rightarrow (s'_1,\dots, s'_m)$. For readability, a term of arity $(s_1,\dots, s_n) \rightarrow (s'_1,\dots, s'_m)$ is also called a $(s_1,\dots, s_n) \big/ (s'_1,\dots, s'_m)$-term. We use $(t_1,\dots, t_m)(\vec{x})$ to denote a vector of terms whose variables are from $\vec{x}$.  For convenience, we also write $t(\vec{x})$ as $\lambda \vec{x}.\ t$ and $(t_1,\dots, t_m)(\vec{x})$ as $\lambda \vec{x}.\ (t_1,\dots, t_m)$. 

We assume the standard notions of $\signature$-atoms, $\signature$-literals, and $\signature$-formulae, whose definitions can be found in some textbooks on mathematical logic (see e.g. \cite{Gal85}). The set of free variables of a $\signature$-formula $\psi$ is denoted by $\free(\psi)$. When writing $\psi(\vec{x})$, we assume that the free variables of $\psi$ are from $\vec{x}$. For a formula  $\psi(\vec{x})$ such that $\vec{x} = (x_1, \dots, x_n)$ and each $x_i$ for $i \in [n]$ is of sort $s_i \in \sorts$, the formula $\psi$ is said to be \emph{of arity} $s_1 \times \dots \times s_n$.
A formula $\psi$ that contains exactly one free variable (resp. two, $n \ge 3$ free variables) is called a \emph{unary} (resp. \emph{binary}, $n$-ary) $\signature$-formula. A formula $\psi$ contains no free variables is called a $0$-ary formula, aka a sentence. For $i, j \in \Nat \backslash \{0\}$, a formula $\psi(\vec{x})$ of arity $s^j$ (where $\vec{x}=(x_1, \dots, x_j)$), and an $s^i/s^j$-term $\vec{f}=(f_1,\dots, f_j)$, we use $\psi[\vec{f}/\vec{x}]$ to denote the formula obtained from $\psi$ by simultaneously replacing $x_1$ with $f_1$, $\dots$, and $x_j$ with $f_j$.

An $\signature$-interpretation $I$ maps: (i) each sort $s \in \sorts$  to a set $s^{I}$, (ii) each function symbol $f \in \functions$ of arity $s_1 \times \ldots \times s_n \rightarrow s$ to a total function $f^I: s_1^I \times \ldots \times s^I_n \rightarrow s^I$ if $n>0$, and to an element of $s^I$ if $n = 0$, and (iii) each predicate symbol $p \in \predicates$ of sort $s_1 \times \ldots \times s_n$ to a  subset of $p^I \subseteq s^I_1 \times \ldots s^I_n$.
An $\signature$-assignment $\eta$ maps each variable $x \in \mathcal{X}$ of sort $s \in \sorts$ to an element of $s^I$.
\begin{itemize}
\item For a term $t$, the interpretation of $t$ under $(I, \eta)$ for an $\signature$-interpretation $I$ and $\signature$-assignment $\eta$, denoted by $t^{(I,\eta)}$, can be defined inductively on the syntax of terms.
\item The satisfiability relation between pairs of an $\signature$-interpretation and an $\signature$-assignment, and $\signature$-formulae, written $I \models_{\eta} \psi$,
is defined inductively, as usual.
\end{itemize}
We say that $(I,\eta)$ is a model of $\psi$ if $I \models_{\eta} \psi$. For an $\signature$-sentence $\psi$, we also write $I \models \psi$ if there is an $\signature$-assignment $\eta$ such that $I \models_\eta \psi$.

Let $\signature$ be a signature and $\cI$ be a set of $\signature$-interpretations. Then $\theory(\cI)$, \emph{the $\signature$-theory associated with $\cI$}, is the set of  $\signature$-sentences $\psi$ such that for each $I \in \cI$, $I \models \psi$.


}
