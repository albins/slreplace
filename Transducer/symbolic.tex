%!TEX root = main.tex

\section{Symbolic extensions}
\label{sec:symbolic}


In this section, we consider an extension of parametric transducers to infinite data domains, called \emph{symbolic parametric transducers}, where the input letters are replaced by formulae over an infinite data domain and the output letters are replaced by terms. 
%
Symbolic parametric transducers can be seen as an extension of both parametric transducers introduced in this paper and symbolic transducers introduced in \cite{VHLMB12}.

%Let $\data$ be an infinite data domain.  We define data strings as elements of $\data^\ast$.
We assume the reader's familiarity with the first-order logic (cf. some textbook e.g. \cite{EFT94}).
For the definition of symbolic transducers, we introduce the concept of background theories and label theories. Intuitively, background theories are a set of unary first-order logic formulae closed under Boolean connectives and substitutions. Label theories extend background theories further by adding inequalities of terms into the set of formulae.

\begin{definition}[Background theories]
A background theory  $\Upsilon$ is a tuple $(\signature, \structure, \Psi)$ satisfying the following constraints:
\begin{itemize}
\item $\signature=(\functions, \predicates)$ is a signature, where $\functions$ (resp. $\predicates$) is a recursively enumerable set of function symbols (resp. relation symbols), 
%\begin{itemize}
%\item
%
%\item for each $\vec{s} = s_1 \times \dots \times s_n$ and $i \in [n]$, there is a function $\pi_{\vec{s}, i}$ of arity $s_1 \times \dots \times s_n \rightarrow s_i$ in $\functions$ such that $\pi_{\vec{s},i}(\vec{t}) = t_i$ for each term $\vec{t} = (t_1,\dots, t_n)$ of sort $s_1 \times \dots \times s_n$.
%\end{itemize}

\item $\structure$ is a $\signature$-structure $(\data, \interpretation)$, where $\data$ is a countable infinite set, called the data domain, 
and $\interpretation$ is a function on $\data$ satisfying: for each $n$-ary function symbol $f \in \functions$ (resp. relation symbol $R \in \predicates$), $\interpretation(f)$ is an $n$-ary function on $\data$ (resp. $\interpretation(R)$ is an $n$-ary relation on $\data$),
%
\item $\Psi$ is a recursively enumerable set of unary $\signature$-formulae closed under Boolean connectives $\vee, \wedge, \neg$. In addition, $\Psi$ is closed under substitutions, that is, for each $\signature$-term $t(x)$ and $\psi(x) \in \Psi$, we have $\psi[t(x)/x] \in \Psi$.
%In addition, it is assumed that each formula $\psi \in \Psi$ of arity $s^\ell$ (where $\ell > 0$) has the same set of free variables $\{x_1,\dots, x_\ell\}$.
For each $\psi(x) \in \Psi$, we use $\|\psi\|$ to denote the set $\{d \in \data \mid  \structure \models \psi(d)\}$. Elements of $\| \psi\|$ are called the \emph{witnesses} of $\psi$.
\end{itemize}
A formula $\psi \in \Psi$ is \emph{satisfiable}, denoted by $\issat(\psi)$, if $\|\psi\| \neq \emptyset$. In addition, $\Upsilon$ is \emph{decidable} iff it is decidable to check $\issat(\psi)$, given $\psi \in\Psi$.
\end{definition}
%The notion $\issat(\psi)$ for $\psi \in \Psi$ and the decidability of $\Upsilon$ can be defined similarly as effective Boolean algebra.

For a background theory $\Upsilon=(\signature, \structure, \Psi)$ with $\structure = (\data, \interpretation)$, a \emph{data string} is an element of $\bigcup \limits_{i \in \Nat} \data^i$.

%we derive a new sort $s^*$ such that $\|s^* \| = \bigcup \limits_{i \in \Nat}\data_s^i$, where $\data^i_s = \{\varepsilon\}$. An element of $\|s^* \|$ is called a \emph{data string} of sort $s$.

\begin{definition}[Label theories]
A label theory  $\Upsilon$ is a tuple $(\signature, \structure, \Psi, \Psi')$ satisfying the following constraints:
\begin{itemize}
\item $(\signature, \structure, \Psi)$ is a background theory with $\signature = (\functions, \predicates)$,
%
\item $\Psi'$ is the set of formulae of the form $\psi(x) \wedge t_1(x) \neq t_2(x)$, where $\psi(x) \in \Psi$ and $t_1(x), t_2(x)$ are $\signature$-terms.
\end{itemize}
A label theory is decidable if it is decidable to check $\issat(\psi)$ for $\psi \in \Psi \cup \Psi'$.
\end{definition}

Given a formula $\psi(x) \in \Psi$ and two $\signature$-terms $t_1(x), t_2(x)$,
$t_1$ and $t_2$ are \emph{equivalent up to $\psi$}, denoted by $t_1 \simeq_\psi t_2$, if
 $\issat(\psi(x) \wedge t_1(x) \neq t_2(x))$ does not hold.
%Two sequences of $s^i/ s$-terms $\vec{f}=f_1...f_n$ and $\vec{g}=g_1...g_m$ are \emph{equivalent up to $\psi$}, denoted by $\vec{f}\simeq_\psi \vec{g}$,
%iff $n=m$ and for every $j \in [n]$, $f_j \simeq_\psi g_j$.

%Given an $s^i/ s$-term sequence $\vec{f}=f_1...f_n$ and a sequence of data values $\vec{d} = (d_1, \dots, d_i) \in (\data_{s_{\sf in}})^i$,
%let $\|\vec{f}\|(\vec{d})$ denote the sequence $\|f_1\|(\vec{d})...\|f_n\|(\vec{d})$, that is, a data string of sort $s$.

\begin{definition}[Symbolic finite-state transducers]
    A \emph{nondeterministic two-way  symbolic \emph{transducer}} (2NST) is a tuple $\Transducer = (\Upsilon, \EndLeft, \EndRight, \controls, q_0, \finals, \transrel)$ where  
\begin{itemize}
\item $\Upsilon=(\signature, \structure, \Psi, \Psi')$ is a decidable label theory, where $\signature = (\functions, \predicates)$ and $\structure = (\data, \interpretation)$,
%
\item $\EndLeft$, $\EndRight$, $\controls$, $q_0$, $\finals$ are defined precisely as in 2NFT, 
%
\item $\transrel$ is a finite set of  transitions of one of the following forms,
\begin{itemize}
\item   transitions $(q, \psi(x), dir, q', t(x))$ such that $q, q' \in Q$, $dir \in \{\Left, \Stay, \Right\}$, $\psi(x) \in \Psi$ and
$t(x)$ is an $\signature$-term, 
\item   transitions $(q, \EndLeft, \Right, q', c)$ and $(q, \EndRight, \Left, q', c)$, where $c$ is an $\signature$-constant. 
\end{itemize}
\end{itemize}
A 2NST is an NST if for each $(q, \psi(x), dir, q', t(x))$, we have $dir \in \{\Stay, \Right\}$, moreover, the transitions of the form $(q, \EndRight, \Left, q', c)$ are removed.
\end{definition}

\paragraph{Semantics of 2NST.}
A transition $\tau=(q_1,\psi(x), dir, q_2, t(x)) \in \transrel$ in the 2NST $\Transducer$ can be concretised
into a potentially infinite set of \emph{concrete} transitions $\|\tau\| \subseteq Q \times \data \times \{\Left, \Stay, \Right\} \times Q \times \data$, where $(q_1, d, dir, q_2, d')  \in \|\tau\|$ iff $d \in \|\psi\|$ and $d' = \interpretation(t)(d)$.
Intuitively, suppose $\Transducer$ is at the state $q_1$ and reading the input data value $d \in \data$,
if there is a transition $(q_1, \psi(x), dir, q_2, t(x) )\in \transrel$ such that $d \in \|\psi\|$, then $\Transducer$ can move its reading head according to $dir$, and moves from the state
$q_1$ to the state $q_2$, moreover, it produces a data value $d' \in \data$.

Given a data string $w = d_1 \dots d_n$, a \emph{run} of $\Transducer$ on $w$
is a sequence of tuples $(q_0, i_0, d'_0), \ldots, (q_m, i_m, d'_m) \in \controls \times (\{0\} \cup [n+1]) \times \data$ 
such that
%let $d_0 = \EndLeft$ and $d_{n+1} = \EndRight$, %. The following conditions, then, have to be satisfied:
\begin{itemize}
    \item $i_0 = 0$, and
    \item for every $j \in [m-1]$, one of the following holds,
    \begin{itemize}
  	\item  $\tau=(q_j, \psi(x), dir, q_{j+1}, t(x)) \in \transrel$ for some $\psi(x) \in \Psi$, $dir \in \{\Left, \Stay, \Right\}$, and $\signature$-term $t(x)$, such that $(q_j, d_{i_j}, dir, q_{j+1}, d'_j) \in \|\tau\|$, and $i_{j+1} = i_j + dir$,
	%
	\item $i_j = 0$, $(q_j, \EndLeft, \Right, q_{j+1}, c) \in \transrel$, $d'_j = \interpretation(c)$,
	%
	\item $i_j = n+1$, $(q_j, \EndRight, \Left, q_{j+1}, c) \in \transrel$, $d'_j = \interpretation(c)$.
  \end{itemize}
\end{itemize}
The run is said to be \defn{accepting} if $i_m = n+1$ and $q_m \in \finals$. When a run is accepting, $d'_0 \ldots d'_m$ is said to be the \emph{output} of the run.
A data string $w'$ is said to be an output of $\Transducer$ on $w$ if there is an accepting run of
$\Transducer$ on $w$ with output $w'$. We use $\Tran(\Transducer)$ to denote the \emph{transduction} defined by $\Transducer$, that is, the relation comprising the data-string pairs $(w, w')$ such that $w'$ is an output of $\Transducer$ on $w$.

\begin{proposition}\label{prop-2nst-nst}
Each 2NST can be turned into an equivalent NST. 
\end{proposition}


\begin{definition}[Symbolic parametric transducers]
A \emph{nondeterministic two-way symbolic parametric transducer} (2NSPT) is a tuple
$\Transducer=(\Upsilon, \EndLeft, \EndRight, X, \controls, q_0, \finals, \transrel)$, where:
\begin{itemize}
\item $\Upsilon=(\signature,\structure, \Psi, \Psi')$ is a decidable label theory with $\signature = (\functions, \predicates)$ and $\structure = (\data, \interpretation)$,
%
\item $\EndLeft$, $\EndRight$, $\controls$, $q_0$, $\finals$ are defined precisely as in 2NST, 
%
\item $Y=\{y_1,\ldots, y_m\}$ is a finite set of parameters (variables), 
%
\item $\transrel$ is a finite set of transitions of one of the following forms: 
\begin{itemize}
\item transitions $(q, \psi(x), q', t(x))$ or $(q, \psi(x), q', y)$ such that $q,q' \in Q$, $\psi(x) \in \Psi$,
$t(x)$ is a $\signature$-term, and $y \in Y$,
%
\item transitions $(q, \EndLeft, \Right, q', c)$, $(q, \EndLeft, \Right, q', y)$, $(q, \EndRight, \Left, q', c)$,  or $(q, \EndRight, \Left, q', y)$, where $c$ is an $\signature$-constant, and $y \in Y$. 
\end{itemize}
\end{itemize}
\end{definition}

Notice that parameters are only allowed in the output track.
Intuitively, each instantiation of the parameters $y_1,\ldots, y_m$ with data strings 
$w_1,\ldots, w_m$ gives rise to a nondeterministic two-way symbolic transducer which outputs
the data string $w_j$, whenever a transition of the form $(p, \psi, dir, q, y_j)$ is
taken by the transducer. This instantiation of the parameters is only done 
\emph{once} before the symbolic parametric transducer is run.

In addition, to extend the string constraint language $\straightline[\transet]$ to infinite data domains, a counterpart for NFA, called \emph{symbolic automata}, is also needed. 

A \emph{Boolean algebra} is adapted from a background theory $\Upsilon=(\signature,\structure, \Psi)$ by \emph{dropping} the requirement that $\Psi$ is closed under substitutions.

\begin{definition}[Symbolic automata]
    A \emph{nondeterministic two-way  symbolic \emph{automata}} (2NSA) is a tuple $\Aut = (\Upsilon, \EndLeft, \EndRight, \controls, q_0, \finals, \transrel)$, where  
\begin{itemize}
\item $\Upsilon=(\signature, \structure, \Psi)$ is a decidable Boolean algebra, where $\signature = (\functions, \predicates)$ and $\structure = (\data, \interpretation)$,
%
\item $\EndLeft$, $\EndRight$, $\controls$, $q_0$, $\finals$ are defined precisely as in 2NFA, 
%
\item $\transrel$ is a finite set of  transitions of one of the following forms,
\begin{itemize}
\item   transitions $(q, \psi(x), dir, q')$ such that $q, q' \in Q$, $dir \in \{\Left, \Stay, \Right\}$, and $\psi(x) \in \Psi$, 
%
\item   transitions $(q, \EndLeft, \Right, q')$ and $(q, \EndRight, \Left, q')$. 
\end{itemize}
\end{itemize}
A 2NSA is an NSA if for each $(q, \psi(x), dir, q')$, we have $dir \in \{\Stay, \Right\}$, moreover, the transitions of the form $(q, \EndRight, \Left, q')$ are removed.
\end{definition}

The semantics of 2NSA is defined similarly to 2NST, with the output data values removed.

For a class $\transet$ of symbolic parametric transducers, we use $\straightlinesym[\transet]$ to denote the extension of straight-line string constraint language by replacing parametric transducers with symbolic parametric transducers and finite-state automata with symbolic automata.

For solving the satisfiability problem of $\straightlinesym[\transet]$, the generic decision procedure in Section~\ref{sec:algo} can be smoothly extended, by replacing recognisable relations with symbolically recognisable relations defined below.

\begin{definition}[Symbolically recognisable relations]
	Given a decidable Boolean algebra $\Upsilon=(\signature, \structure, \Psi)$ with $\signature = (\functions, \predicates)$ and $\structure = (\data, \interpretation)$, a $k$-ary relation $R \subseteq \data^*\times \ldots\times \data^*$ is \emph{symbolically recognisable} if $R=\bigcup_{i=1}^n L^{(i)}_1 \times \ldots \times L^{(i)}_k$ where $L^{(i)}_j$ is recognised by some NSA for each $j\in [k]$.
\end{definition}


\begin{theorem}\label{thm-generic-dec-symbolic}
Let $\transet$ be a class of symbolic parametric transducers. Suppose that %$\transet$ satisfies that 
for each $\Transducer \in \transet$ and NSA $\Aut$, $\Pre_\Transducer(\Aut)$ is a symbolically recognisable relation and a representation of which can be computed effectively. Then the satisfiability of $\straightlinesym[\transet]$ is decidable.
\end{theorem}

\begin{lemma}\label{lem-st}
Given a 2NSPT (resp. NSPT) $\Transducer$ and a NSA $\Aut$, $\Pre_\Transducer(\Aut)$ is a symbolically recognisable relation and a representation of which can be computed effectively.
\end{lemma}

From Theorem~\ref{thm-generic-dec-symbolic} and Lemma~\ref{lem-st}, we have the following result.
\begin{theorem}
Satisfiability of $\straightlinesym[\twspt]$ and $\straightlinesym[\owspt]$ is decidable.
\end{theorem}



%==========================================================================
%==========================================================================
%============================many sorted first-order logic=========================
%==========================================================================
%==========================================================================

\hide
{

\paragraph{Many-sorted first-order logic.}
We assume a signature $\signature=(\sorts, \functions, \predicates)$, where $\sorts$ is a countable set of \emph{sorts}, $\functions$ is a countable set of \emph{function symbols}, and $\predicates$ is a countable set of \emph{predicate symbols}. Each function or predicate symbol has an associated \emph{arity}, which is a tuple of sorts in $\sorts$.  A function symbol with a single sort is called a \emph{constant}. A predicate symbol with a single sort is called a \emph{set}, which intuitively denotes a set of elements of that sort.

An $\signature$-term is built from the function symbols in $\functions$ and variables taken from a set $\mathcal{X}$ that is disjoint from $\sorts$, $\functions$, and $\predicates$. Each variable $x \in \mathcal{X}$ has an associated sort in $\sorts$. In addition, we assume that the variables in $\mathcal{X}$ are linearly ordered $\preceq_{\mathcal{X}}$. When writing $t(\vec{x})$ for a vector of distinct variables $\vec{x}$ such that $\vec{x} = (x_1,\dots, x_n)$ follows the ascending order of the linear order $\preceq_{\mathcal{X}}$, we assume that the variables occurring in the term $t$ are from $\vec{x}$. For a term $t(\vec{x})$ of sort $s$ such that $\vec{x} = (x_1, \dots, x_n)$ and each $x_i$ for $i \in [n]$ is of sort $s_i \in \sorts$, the term $t$ is said to be \emph{of arity} $(s_1 \times \dots \times s_n) \rightarrow s$. In addition, for a vector of terms $(t_1, \dots, t_m)$ such that all the variables of $t_1 ,\dots, t_m$ are from $\vec{x} = (x_1, \dots, x_n)$, if $x_1 \preceq_{\mathcal{X}} x_2  \preceq_{\mathcal{X}} \dots  \preceq_{\mathcal{X}} x_n$, each $x_i$ for $i \in [n]$ is of sort $s_i$, and each $t_j$ for $j \in [m]$ is of sort $s'_j$, then $(t_1,\dots, t_m)$ is said to be a term of arity $(s_1,\dots, s_n) \rightarrow (s'_1,\dots, s'_m)$. For readability, a term of arity $(s_1,\dots, s_n) \rightarrow (s'_1,\dots, s'_m)$ is also called a $(s_1,\dots, s_n) \big/ (s'_1,\dots, s'_m)$-term. We use $(t_1,\dots, t_m)(\vec{x})$ to denote a vector of terms whose variables are from $\vec{x}$.  For convenience, we also write $t(\vec{x})$ as $\lambda \vec{x}.\ t$ and $(t_1,\dots, t_m)(\vec{x})$ as $\lambda \vec{x}.\ (t_1,\dots, t_m)$. 

We assume the standard notions of $\signature$-atoms, $\signature$-literals, and $\signature$-formulae, whose definitions can be found in some textbooks on mathematical logic (see e.g. \cite{Gal85}). The set of free variables of a $\signature$-formula $\psi$ is denoted by $\free(\psi)$. When writing $\psi(\vec{x})$, we assume that the free variables of $\psi$ are from $\vec{x}$. For a formula  $\psi(\vec{x})$ such that $\vec{x} = (x_1, \dots, x_n)$ and each $x_i$ for $i \in [n]$ is of sort $s_i \in \sorts$, the formula $\psi$ is said to be \emph{of arity} $s_1 \times \dots \times s_n$.
A formula $\psi$ that contains exactly one free variable (resp. two, $n \ge 3$ free variables) is called a \emph{unary} (resp. \emph{binary}, $n$-ary) $\signature$-formula. A formula $\psi$ contains no free variables is called a $0$-ary formula, aka a sentence. For $i, j \in \Nat \backslash \{0\}$, a formula $\psi(\vec{x})$ of arity $s^j$ (where $\vec{x}=(x_1, \dots, x_j)$), and an $s^i/s^j$-term $\vec{f}=(f_1,\dots, f_j)$, we use $\psi[\vec{f}/\vec{x}]$ to denote the formula obtained from $\psi$ by simultaneously replacing $x_1$ with $f_1$, $\dots$, and $x_j$ with $f_j$.

An $\signature$-interpretation $I$ maps: (i) each sort $s \in \sorts$  to a set $s^{I}$, (ii) each function symbol $f \in \functions$ of arity $s_1 \times \ldots \times s_n \rightarrow s$ to a total function $f^I: s_1^I \times \ldots \times s^I_n \rightarrow s^I$ if $n>0$, and to an element of $s^I$ if $n = 0$, and (iii) each predicate symbol $p \in \predicates$ of sort $s_1 \times \ldots \times s_n$ to a  subset of $p^I \subseteq s^I_1 \times \ldots s^I_n$.
An $\signature$-assignment $\eta$ maps each variable $x \in \mathcal{X}$ of sort $s \in \sorts$ to an element of $s^I$.
\begin{itemize}
\item For a term $t$, the interpretation of $t$ under $(I, \eta)$ for an $\signature$-interpretation $I$ and $\signature$-assignment $\eta$, denoted by $t^{(I,\eta)}$, can be defined inductively on the syntax of terms.
\item The satisfiability relation between pairs of an $\signature$-interpretation and an $\signature$-assignment, and $\signature$-formulae, written $I \models_{\eta} \psi$,
is defined inductively, as usual.
\end{itemize}
We say that $(I,\eta)$ is a model of $\psi$ if $I \models_{\eta} \psi$. For an $\signature$-sentence $\psi$, we also write $I \models \psi$ if there is an $\signature$-assignment $\eta$ such that $I \models_\eta \psi$.

Let $\signature$ be a signature and $\cI$ be a set of $\signature$-interpretations. Then $\theory(\cI)$, \emph{the $\signature$-theory associated with $\cI$}, is the set of  $\signature$-sentences $\psi$ such that for each $I \in \cI$, $I \models \psi$.


}
