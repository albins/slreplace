%!TEX root = main.tex

\section{Proof of Theorem~\ref{thm-rb-2pt}: $k$-\RBPPT{} is in EXPSPACE}\label{app-rb-2pt}

We first show that the $\expspace$ upper bound holds for the extension of
\PT{}s with the reverse function.
%
\mat{TODO: show this}
%
We then reduce in polynomial time the satisfiability problem for $k$-\RBPPT{}s
to that for \PT{}s plus reverse function.

Let
$y := \Transducer(x)$
be an assignment statement in $S$ such that
$\Transducer = (\controls, q_0, \finals, \transrel)$
is $(2k)$-reversal bounded for some $k > 0$.
%
We replace $y := \Transducer(x)$ by a program $S'$ as follows.

Each accepting run of $\Transducer$ on an input string can be split into $2k+1$
maximal subruns without reversals (i.e.\ split the run at the turning points).
%
Intuitively, we shall split $y$ into $2k+1$ substrings, corresponding to the
$2k+1$ maximal subruns, and for each substring of $y$, use one \PT{} and the
reverse function to produce it.

\newcommand\init{\mathsf{init}}
\newcommand\prefix{\mathsf{prefix}}
\newcommand\suffix{\mathsf{suffix}}
\newcommand\finalpass{\mathsf{finalpass}}
\newcommand\reverse{\mathsf{reverse}}
\newcommand\concatpt{\mathsf{concat}}

For preparation, we introduce the following \FT{}s:
    $\init$,
    $\prefix$,
    $\suffix$,
    $\finalpass$,
    $\reverse$,
    $\concatpt$.
These \FT{}s behave as follows.
\begin{itemize}
\item
    $\init$ begins by outputting $q_0$ and then directly copying its input to
    the output tape.

\item
    $\prefix$ marks the beginning and end of a forwards run of $\Transducer$,
    assuming that an initial or backwards marking has already taken place (that
    is, either one or two states from $\controls$ appear in the input).
    %
    It directly copies its input to the output tape until it the first instance
    of some
    $q \in \controls$.
    %
    This $q$ is also output.
    %
    After this, it will copy the rest of the input tape to the output tape with
    the exception that any occurrence of a state from $\controls$ is not copied
    (erased).
    %
    At some point during this second phase it will non-deterministically output
    a single
    $q' \in \controls$
    between the copies of two input characters and then continue to copy its
    input to the output until the end of the tape (ignoring any $\controls$).

\item
    $\suffix$ marks the end and beginning of a backwards run assuming $\prefix$
    has previously marked the beginning and end of the preceding forwards run.
    %
    It will do this by copying the input tape directly to the output tape,
    omitting the first instance of $\controls$.
    %
    Additionally, before the second occurrence of some $\controls$ it will
    non-deterministically output some single $q \in \controls$ between the
    copies of two input characters.
    %
    On encountering the second instance of $\controls$ it will copy this and
    continue to copy the remainder of the tape.

\item
    $\finalpass$ behaves as prefix, but instead of non-deterministically
    choosing a position to output some
    $q \in \controls$
    it will output some accepting $q$ after all input has been copied.
    %
    Note, it will still copy the first instance of some $\controls$ to the
    output tape.

\item
    $\reverse$ reverses its input string.

\item
    $\concatpt$ concatenates its arguments.
\end{itemize}

We illustrate the use of these \FT{}s with an example.  Let the input word be
$a b c a b c$
and consider the program below.
%
To the left of the program we give the output of an example run with the input
$x = a b c a b c$.
%
\[
    \begin{array}{rcll}
        x_0 &:=& \init(x);
            & \qquad (x_0 = q_0\ a\ b\ c\ a\ b\ c) \\
        x_1 &:=& \prefix(x_0);
            & \qquad (x_1 = q_0\ a\ b\ c\ a\ q_1\ b\ c) \\
        x_2 &:=& \suffix(x_1);
            & \qquad (x_2 = a\ b\ q_2\ c\ a\ q_1\ b\ c) \\
        x_3 &:=& \prefix(x_2);
            & \qquad (x_3 = a\ b\ q_2\ c\ a\ b\ q_3\ c) \\
        x_4 &:=& \suffix(x_3);
            & \qquad (x_4 = a\ b\ c\ a\ q_4\ b\ q_3\ c) \\
        x_5 &:=& \finalpass(x_4);
            & \qquad (x_5 = a\ b\ c\ a\ q_4\ b\ c\ q_5) \\
    \end{array}
\]
%
The values of
$x_0, \ldots, x_5$
correspond to a $4$-reversal run of $\Transducer$.
%
Beginning at $q_0$ we first read $abca$ and reach $q_1$; here we reverse
direction and read $ac$ to reach $q_2$; next we reverse direction again and
read $c a b$ to reach $q_3$; after another reversal $b$ is read and the state
$q_4$ is reached; finally, $bc$ is read and $q_5$ is reached at the end of the
tape.

We now define $\Transducer_f$ and $\Transducer_b$ which simulate a
single-direction run of $\Transducer$ between the two instances of $\controls$
in the forwards or backwards direction.
%
That is
\begin{itemize}
\item
    $\Transducer_f$ (from the beginning of the input tape) skips over all
    characters until the first instance of $\controls$ is found.
    %
    Let this first instance be $q$.
    %
    It then starts simulating $\Transducer$ in the left-to-right direction
    starting from $q$.
    %
    That is, it can only simulate moves of the form
    $(q, a, d, q', b)$
    where
    $d \in \{\Stay, \Right\}$.
    %
    Any output is written to the output tape.
    %
    This simulation continues until $\Transducer_f$ reads some
    $q' \in \controls$.
    %
    It accepts only if the simulation has also reached $q'$.

\item
    $\Transducer_b$ (from the beginning of the input tape) skips over all
    characters until the first instance of $\controls$ is found.
    %
    Let this first instance be $q$.
    %
    It then starts simulating a right-to-left run of $\Transducer$ for a run
    that would end at $q$.
    %
    To do this it has to perform moves \emph{in reverse};
    that is, for moves of $\Transducer$ of the form
    $(q, a, d, q', b)$
    where
    $d \in \{\Left, \Stay\}$
    the transducer $\Transducer_b$ can simulate
    $(q', a, d^{-1}, q, b)$
    where
        $(\Left)^{-1} = \Right$
        and
        $(\Stay)^{-1} = \Stay$.
    %
    Any output is written to the output tape.
    %
    This simulation continues until $\Transducer_b$ reads some
    $q' \in \controls$
    and accepts only if the simulation has also reached $q'$.
\end{itemize}

We can now apply $\Transducer_f$ to the output of each use of $\prefix$ or
$\finalpass$ and $\Transducer_b$ to the output of each use of $\suffix$.
%
Since $\Transducer_b$ performs its simulation in reverse, we need to apply
$\reverse$ to its output.
%
Then, we concatenate all outputs to obtain $y$.
%
Thus, the output $y$ of the following program on an input $x$ is the same as
$y := \Transducer(x)$
when $\Transducer$ performs $2k$ reversals.
\[
    \begin{array}{rcl}
        x_0 &:=& \init(x); \\
        x_1 &:=& \prefix(x_0); \\
        x_2 &:=& \suffix(x_1); \\
        &\vdots& \\
        x_{2k-1} &:=& \prefix(x_{2k-2}); \\
        x_{2k} &:=& \suffix(x_{2k-1}); \\
        x_{2k+1} &:=& \finalpass(x_{2k}) \\
        \\
        y_1 &:=& \Transducer_f(x_1); \\
        y'_2 &:=& \Transducer_b(x_2); \\
        y_2 &:=& \reverse(y'_2) \\
        &\vdots& \\
        y_{2k-1} &:=& \Transducer_f(x_{2k-1}); \\
        y'_{2k} &:=& \Transducer_b(x_{2k}); \\
        y_{2k} &:=& \reverse(y'_{2k}) \\
        y_{2k+1} &:=& \Transducer_f(x_{2k+1}); \\
        \\
        y &:=& \concatpt(y_1, \ldots, y_{2k+1})
    \end{array}
\]

