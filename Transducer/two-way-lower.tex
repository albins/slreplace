
\section{Lower Bound For String Constraints with Two-Way Transducers}
\label{sec:two-way-lower}

\mat{TODO: update to new notation, add note about turing machines as main text, argue that we really need n 2T and one 2NFA to tally with main body text}

\subsection{Tiling Problems}

\begin{definition}[Tiling Problem]
    A \emph{tiling problem} is a tuple
    $\tup{\tiles, \hrel, \vrel, \inittile, \fintile}$
    where
        $\tiles$ is a finite set of tiles,
        $\hrel \subseteq \tiles \times \tiles$ is a horizontal matching relation,
        $\vrel \subseteq \tiles \times \tiles$ is a vertical matching relation, and
        $\inittile, \fintile \in \tiles$ are initial and final tiles respectively.
\end{definition}

A solution to a tiling problem over a $\linlen$-width corridor is a sequence
\[
    \begin{array}{c}
        \tile^1_1 \ldots \tile^1_\linlen \\
        \tile^2_1 \ldots \tile^2_\linlen \\
        \ldots \\
        \tile^\tileheight_1 \ldots \tile^\tileheight_\linlen
    \end{array}
\]
where
$\tile^1_1 = \inittile$,
$\tile^\tileheight_\linlen = \fintile$,
and for all
$1 \leq i < \linlen$
and
$1 \leq j \leq \tileheight$
we have
$\tup{\tile^j_i, \tile^j_{i+1}} \in \hrel$
and for all
$1 \leq i \leq \linlen$
and
$1 \leq j < \tileheight$
we have
$\tup{\tile^j_i, \tile^{j+1}_i} \in \vrel$.
Note, we will assume that $\inittile$ and $\fintile$ can only appear at the beginning and end of the tiling respectively.

Tiling problems characterise many complexity classes~\cite{??}.
In particular, we will use the following facts.
\begin{itemize}
\item
    For any $\linlen$-space Turing machine, there exists a tiling problem of size polynomial in the size of the Turing machine, over a corridor of width $\linlen$, that has a solution iff the $\linlen$-space Turing machine has a terminating computation.

\item
    There is a fixed
    $\tup{\tiles, \hrel, \vrel, \inittile, \fintile}$
    such that for any width $\linlen$ there is a unique solution
    \[
        \begin{array}{c}
            \tile^1_1 \ldots \tile^1_\linlen \\
            \tile^2_1 \ldots \tile^2_\linlen \\
            \ldots \\
            \tile^\tileheight_1 \ldots \tile^\tileheight_\linlen
        \end{array}
    \]
    and moreover $\tileheight$ is exponential in $\linlen$.
    \mat{Not sure if this is folklore, but it can be seen by taking a $\linlen$ space Turing machine that deterministically increments from a tape with all 0 to all 1, then encoding this as a tiling problem.}
\end{itemize}

\subsection{Large Numbers}

The crux of the proof is encoding large numbers that can take values between $1$ and $\expheight$-fold exponential.

A linear-length binary number could be encoded simply as a sequence of bits
\[
    b_0 \ldots b_\linlen \in \set{0,1}^\linlen \ .
\]
To aid with later constructions we will take a more oblique approach.
Let
$\tup{\tilesnum{1}, \hrelnum{1}, \vrelnum{1}, \inittilenum{1}, \fintilenum{1}}$
be a copy of the fixed tiling problem from the previous section for which there is a unique solution, whose length must be exponential in the width.
In the future, we will need several copies of this problem, hence the indexing here.
Fix a width $\linlen$ and let $\nmax{1}$ be the corresponding corridor length.
A \emph{level-1} number can encode values from $1$ to $\nmax{1}$.
In particular, for $1 \leq i \leq \nmax{1}$ we define
\[
    \tenc{1}{i} = \tile^i_1 \ldots \tile^i_\linlen
\]
where
$\tile^i_1 \ldots \tile^i_\linlen$
is the tiling of the $i$th row of the unique solution to the tiling problem.

A \emph{level-2} number will be derived from tiling a corridor of width $\nmax{1}$, and thus the number of rows will be doubly-exponential.
For this, we require another copy
$\tup{\tilesnum{2}, \hrelnum{2}, \vrelnum{2}, \inittilenum{2}, \fintilenum{2}}$
of the above tiling problem.
Moreover, let $\nmax{2}$ be the length of the solution for a corridor of width $\nmax{1}$.
Then for any
$1 \leq i \leq \nmax{2}$
we define
\[
    \tenc{2}{i} =
        \tenc{1}{1} \tile^i_1
        \tenc{1}{2} \tile^i_2
        \ldots
        \tenc{1}{\nmax{1}} \tile^i_{\nmax{1}}
\]
where
$\tile^i_1 \ldots \tile^i_{\nmax{1}}$
is the tiling of the $i$th row of the unique solution to the tiling problem.
That is, the encoding indexes each tile with it's column number, where the column number is represented as a level-1 number.

In general, a \emph{level-$\expheight$} number is of length $(\expheight-1)$-fold exponential and can encode numbers $\expheight$-fold exponential in size.
We use a copy
$\tup{\tilesnum{\expheight},
      \hrelnum{\expheight},
      \vrelnum{\expheight},
      \inittilenum{\expheight},
      \fintilenum{\expheight}}$
of the above tiling problem and use a corridor of width
$\nmax{\expheight-1}$.
We define $\nmax{\expheight}$ as the length of the unique solution to this problem.
Then, for any $1 \leq i \leq \nmax{\expheight}$ we have
\[
    \tenc{\expheight}{i} =
        \tenc{(\expheight-1)}{1} \tile^i_1
        \tenc{(\expheight-1)}{2} \tile^i_2
        \ldots
        \tenc{(\expheight-1)}{\nmax{(\expheight-1)}} \tile^i_{\nmax{(\expheight-1)}}
\]
where
$\tile^i_1 \ldots \tile^i_{\nmax{\expheight-1}}$
is the tiling of the $i$th row of the unique solution to the tiling problem.

\subsection{Recognising Large Numbers}

We first define a useful formula
$\goodnums{\expheight}{x}$
with a single free-variable $x$ which can only be satisfied if $x$ is of the following form, where
$\numeq$, $\numplus$, and $\numsep$
are auxiliary symbols.
\[
    \begin{array}{c}
        \goodnums{\expheight}{x} \\
        \iff \\
        x \in \brac{
            \brac{\tenc{\expheight}{1} \brac{\numeq \tenc{\expheight}{1}}^\ast}
            \numplus
            \brac{\tenc{\expheight}{2} \brac{\numeq \tenc{\expheight}{2}}^\ast}
            \numplus
            \ldots
            \numplus
            \brac{
                \tenc{\expheight}{\nmax{\expheight}}
                    \brac{\numeq \tenc{\expheight}{\nmax{\expheight}}}^\ast
            }
            \numsep
        }^\ast
    \end{array}
\]
That is, the string must contain sequences of strings that count from $1$ to $\nmax{\expheight}$.
This counting may stutter and repeat a number several times before moving to the next.
A separator $\numeq$ indicates a stutter, while $\numplus$ indicates that the next number must add one to the current number.
Finally, a $\numsep$ ends a sequence and may start again from $\tenc{\expheight}{1}$.

\paragraph{Base case $\expheight = 1$.}

We define
\[
    \goodnums{1}{x} =
        (y = \ap{T}{x} \land y \in \top)
\]
where $\top$ is a character output by $T$ when $x$ is correctly encoded.
Otherwise $T$ outputs $\bot$.

We describe informally how $T$ operates.
\mat{We may or may not give the formal definition later.}
Because $T$ is two-way, it may perform several passes of the input $x$.
Recall that $T$ requires $x$ to contain a word of the form
\[
    \brac{
        \brac{\tenc{1}{1} \brac{\numeq \tenc{1}{1}}^\ast}
        \numplus
        \brac{\tenc{1}{2} \brac{\numeq \tenc{1}{2}}^\ast}
        \numplus
        \ldots
        \numplus
        \brac{
            \tenc{1}{\nmax{1}}
                \brac{\numeq \tenc{1}{\nmax{1}}}^\ast
        }
        \numsep
    }^\ast
\]
and each $\tenc{1}{i}$ is the $i$th row of the unique solution to the tiling problem of width $\linlen$.
The passes proceed as follows.
If a passes fails, the transducer outputs $\bot$ and terminates.
If all passes succeed, the transducer outputs $\top$ and terminates.
\begin{itemize}
\item
    During the first pass $T$ verifies that the input is of the form
    \[
        \brac{
            \tilesnum{1}^\linlen
            \brac{\set{\numeq,\numplus} \tilesnum{1}^\linlen}^\ast
            \numsep
        }^\ast \ .
    \]

\item
    During the second pass the transducer verifies that the first block of
    $\tilesnum{1}^\linlen$,
    and all blocks of
    $\tilesnum{1}^\linlen$
    immediately following a $\numsep$ have $\inittilenum{1}$ as the first tile.
    Simultaneously, it can verify that all blocks of
    $\tilesnum{1}^\linlen$
    immediately preceding a $\numsep$ finish with the tile $\fintilenum{1}$.
    Moreover, it checks that $\inittilenum{1}$ and $\fintilenum{1}$ do not appear elsewhere.

\item
    During the third pass $T$ verifies the horizontal tiling relation.
    That is, every contiguous pair of tiles $\tile, \tile'$ in $x$ must be such that
    $(\tile, \tile') \in \hrelnum{1}$.
    This can easily be done by storing the last character read into the states of $T$.

\item
    The vertical tiling relation and equality checks are verified using $\linlen$ more passes.
    During the $j$th pass, the $j$th column is tested.
    The transducer $T$ stores in its state the tile in the $j$th column of the first block of $\tilesnum{1}^\linlen$ or any block immediately following $\numsep$.
    (The transducer can count to $\linlen$ in its state.)
    It then moves to the $j$th column of the next block of $\tilesnum{1}^\linlen$, remembering whether the blocks were separated with $\numeq$, $\numplus$, or $\numsep$.
    If the separator was $\numeq$ the transducer checks that the $j$th tile of the current block matches the tile stored in the state (i.e.~is equal to the preceding block).
    If the separator was $\numplus$ the transducer checks that the $j$th tile of the current block is related by $\vrelnum{1}$ to the stored tile.
    In this case the current $j$th tile is stored and the previously stored tile forgotten.
    Finally, if the separator was $\numsep$ there is nothing to check.
    If any check fails, the pass will also fail.
\end{itemize}

If all passes succeed, we know that $x$ contains a word where blocks separated by $\numeq$ are equal
(since all positions are equal, as verified individually by the final $\linlen$ passes),
blocks separated by $\numplus$ satisfy $\vrelnum{1}$ in all positions,
the $\inittilenum{1}$ tile appears at the start of all sequences separated by $\numsep$ and each such sequence ends with $\fintilenum{1}$, and
finally the horizontal tiling relation is satisfied at all times.
Thus, $x$ must be of the form
\[
    \brac{
        \brac{\tenc{1}{1} \brac{\numeq \tenc{1}{1}}^\ast}
        \numplus
        \brac{\tenc{1}{2} \brac{\numeq \tenc{1}{2}}^\ast}
        \numplus
        \ldots
        \numplus
        \brac{
            \tenc{1}{\nmax{1}}
                \brac{\numeq \tenc{1}{\nmax{1}}}^\ast
        }
        \numsep
    }^\ast
\]
as required.


\paragraph{Inductive case $\expheight$.}

We define a formula
$\goodnums{\expheight}{x}$
such that
\[
    \begin{array}{c}
        \goodnums{\expheight}{x} \\
        \iff \\
        x \in \brac{
            \brac{\tenc{\expheight}{1} \brac{\numeq \tenc{\expheight}{1}}^\ast}
            \numplus
            \brac{\tenc{\expheight}{2} \brac{\numeq \tenc{\expheight}{2}}^\ast}
            \numplus
            \ldots
            \numplus
            \brac{
                \tenc{\expheight}{\nmax{\expheight}}
                    \brac{\numeq \tenc{\expheight}{\nmax{\expheight}}}^\ast
            }
            \numsep
        }^\ast \ .
    \end{array}
\]
Assume, by induction, we have a formula
$\goodnums{\expheight-1}{x}$
which already satisfies this property (for $\expheight-1$).
We define
\[
    \goodnums{\expheight}{x} =
    \brac{
        y = \ap{T}{x}
        \land
        \goodnums{\expheight-1}{y}
    }
\]
where $T$ is a transducer that behaves as described below.
\mat{We may or may not define it formally later.}
The transducer will perform several passes to make several checks.
If a check fails it will halt and output a symbol $\bot$, which means that $y$ can no longer satisfy
$\goodnums{\expheight-1}{y}$.
During normal execution $T$ will make checks that rely on level-$(\expheight-1)$ numbers appearing in the correct sequence or being equal.
To ensure these properties hold, $T$ will write these numbers to $y$ and rely on these properties then being verified by
$\goodnums{\expheight-1}{y}$.
The passes behave as follows.
\begin{itemize}
\item
    During the first pass $T$ verifies that $x$ belongs to the regular language
    \[
        \brac{
            \brac{
                \brac{
                    \brac{
                        \brac{\tilesnum{1}^\linlen \tilesnum{2}}^\ast \tilesnum{3}
                    }^\ast
                    \cdots
                }^\ast
                \tilesnum{\expheight}
            }^\ast
            \brac{
                \set{\numeq,\numplus}
                \brac{
                    \brac{
                        \brac{
                            \brac{\tilesnum{1}^\linlen \tilesnum{2}}^\ast \tilesnum{3}
                        }^\ast
                        \cdots
                    }^\ast
                    \tilesnum{\expheight}
                }^\ast
            }^\ast
            \numsep
         }^\ast
         \ .
    \]
    This can be done with a polynomial number of states.

\item
    During the second pass $T$ will verify that the first instance of
    $\tilesnum{\expheight}$
    appearing in the word or after a $\numsep$ is $\inittilenum{\expheight}$.
    Similarly, the final instance of any
    $\tilesnum{\expheight}$
    before any $\numsep$ is $\fintilenum{\expheight}$.
    Moreover, it checks that
    $\inittilenum{\expheight}$
    and
    $\fintilenum{\expheight}$
    do not appear elsewhere.

\item
    During the third pass $T$ will verify the horizontal tiling relation
    $\hrelnum{\expheight}$.
    Each block (separated by $\numeq$, $\numplus$, or $\numsep$) is checked in turn.
    There are two components to this.
    \begin{itemize}
    \item
        The indexing of the tiles must be correct.
        That is, the first tile of the block must be indexed
        $\tenc{\expheight-1}{1}$,
        the second
        $\tenc{\expheight-1}{2}$,
        through to
        $\tenc{\expheight-1}{\nmax{\expheight-1}}$.
        Thus, $T$ copies directly the instance of
        $\brac{
            \brac{
                \brac{\tilesnum{1}^\linlen \tilesnum{2}}^\ast \tilesnum{3}
            }^\ast
            \cdots
        }^\ast$
        preceding each
        $\tilesnum{\expheight}$
        to the output tape, followed immediately by
        $\numplus$
        as long as the character after
        $\tilesnum{\expheight}$
        is not a separator from
        $\set{\numeq,\numplus,\numsep}$.
        Otherwise, it is the end of the block and $\numsep$ is written.

        Hence,
        $\goodnums{\expheight-1}{y}$
        will verify that the output for each block is
        $\tenc{\expheight-1}{1}
         \numplus \cdots \numplus
         \tenc{\expheight-1}{\nmax{\expheight-1}}$
        which enforces that the indexing of the tiles is correct.

    \item
        Horizontally adjacent tiles must satisfy
        $\hrelnum{\expheight}$.
        This is done by simply storing the last read tile from
        $\tilesnum{\expheight}$
        in the state of $T$.
        Then whenever a new tile from
        $\tilesnum{\expheight}$
        is seen without a separator $\numeq$, $\numplus$, or $\numsep$, then it can be checked against the previous tile and
        $\hrelnum{\expheight}$.
    \end{itemize}

\item
    The transducer $T$ then performs a non-deterministic number of passes to check the vertical tiling relation.
    We will use
    $\goodnums{\expheight-1}{y}$
    to ensure that $T$ in fact performs
    $\nmax{\expheight-1}$
    passes, the first checking the first column of the tiling over
    $\tilesnum{\expheight}$,
    the second checking the second column, and so on up to the
    $\nmax{\expheight-1}$th column.

    Note, we know from the previous pass that each row of the tiling is indexed correctly.
    In the sequel, let us use the term ``session'' to refer to the sequences of characters separated by $\numsep$.

    Each pass of $T$ checks a single column (across all sessions).
    At the start of each session, $T$ moves non-deterministically to the start of some block
    $\brac{
        \brac{
            \brac{\tilesnum{1}^\linlen \tilesnum{2}}^\ast \tilesnum{3}
        }^\ast
        \cdots
    }^\ast
    \tilesnum{\expheight}$
    (without passing $\numeq$, $\numplus$, or $\numsep$).
    It then copies the tiles from
    $\brac{
        \brac{
            \brac{\tilesnum{1}^\linlen \tilesnum{2}}^\ast \tilesnum{3}
        }^\ast
        \cdots
    }^\ast$
    to $y$ and saves the tile from
    $\tilesnum{\expheight}$
    in its state before moving to the next separator from
    $\set{\numeq,\numplus,\numsep}$.
    In the case of $\numsep$ nothing needs to be checked and $T$ continues to the next session or finishes the pass if there are no more sessions.
    In the case of $\numeq$ or $\numplus$ the transducer remembers this separator and moves non-deterministically to the start of some block (without passing another $\numeq$, $\numplus$, or $\numsep$).
    It then writes $\numeq$ to $y$ as it is intended that $T$ choose the same column as before.
    This will be verified by
    $\goodnums{\expheight-1}{y}$.
    To aid with this $T$ copies the tiles from
    $\brac{
        \brac{
            \brac{\tilesnum{1}^\linlen \tilesnum{2}}^\ast \tilesnum{3}
        }^\ast
        \cdots
    }^\ast$
    to $y$.
    It can then check the tile from
    $\tilesnum{\expheight}$.
    If the remembered separator was $\numeq$ then this tile must match the saved one.
    If it was $\numplus$ then this tile must be related by
    $\vrelnum{\expheight}$
    to the saved one.
    If this succeeds , $T$ stores the new tile and forgets the old and continues to the next separator to continue checking
    $\vrelnum{\expheight}$.

    At the end of the pass (checking a single column from all sessions) then $T$ will either have failed and written $\bot$ or written a sequence of level-$(\expheight-1)$ numbers to $y$ separated by $\numeq$.
    That is
    \[
        \tenc{\expheight-1}{i_1}
        \numeq
        \cdots
        \numeq
        \tenc{\expheight-1}{i_{\alpha}}
    \]
    for some $\alpha$.
    Since, by induction,
    $\goodnums{\expheight-1}{y}$
    is correct, then the formula can only be satisfied if $T$ chose the same position in each row.
    That is
    $i_1 = \cdots = i_\alpha$.
    Thus, the vertical relation for the $i_1$th column has been verified.

    At this point $T$ can either write $\numsep$ and terminate or perform another pass (non-deterministically).
    In the latter case, it outputs $\numplus$, moves back to the beginning of the tape, and starts again.
    Thus, after a number of passes, $T$ will have written
    \[
        \brac{
            \tenc{\expheight-1}{i^1_1}
            \numeq
            \cdots
            \numeq
            \tenc{\expheight-1}{i^1_{\alpha_1}}
        }
        \numplus
        \cdots
        \numplus
        \brac{
            \tenc{\expheight-1}{i^\beta_1}
            \numeq
            \cdots
            \numeq
            \tenc{\expheight-1}{i^\beta_{\alpha_\beta}}
        }
        \numsep
    \]
    for some $\beta$, $\alpha_1$, \ldots, $\alpha_\beta$.
    Since
    $\goodnums{\expheight-1}{y}$
    will only accept such sequences of the form
    \[
        \brac{
            \tenc{\expheight-1}{1}
            \brac{
                \numeq \tenc{\expheight-1}{1}
            }^\ast
        }
        \numplus
        \cdots
        \numplus
        \brac{
            \tenc{\expheight-1}{\nmax{\expheight-1}}
            \brac{
                \numeq \tenc{\expheight-1}{\nmax{\expheight-1}}
            }^\ast
        }
        \numsep
    \]
    we know that $T$ must check each vertical column in turn, from $1$ to
    $\nmax{\expheight-1}$.
\end{itemize}

Thus, at the end of all passes, if $T$ has not output $\bot$ it has verified that $x$ is a correct encoding of a solution to
$\tup{\tilesnum{\expheight},
      \hrelnum{\expheight},
      \vrelnum{\expheight},
      \inittilenum{\expheight},
      \fintilenum{\expheight}}$.
That is, together with
$\goodnums{\expheight-1}{y}$
we know that
    the word is of the correct format,
    each row has a tile for each index and these indices appear in order,
    the horizontal relation is respected, and
    the vertical tiling relation is respected.
If $x$ is not a correct encoding then $T$ will not be able to produce a $y$ that satisfies
$\goodnums{\expheight-1}{y}$.


\subsection{Reducing from a Tiling Problem}

Now that we are able to encode large numbers, we can encode an $\expheight$-EXPSPACE-hard tiling problem as a satisfiability problem of $\strline[T]$ with two-way transducers.
In fact, most of the technical work has been done.

Thus, fix a tiling problem
$\tup{\tiles, \hrel, \vrel, \inittile, \fintile}$
that is $\expheight$-EXPSPACE-hard.
In particular, we allow a corridor $\nmax{\expheight}$ tiles wide.
We use the formula
\[
    \probfmla = \brac{y = \ap{T}{x} \land \goodnums{\expheight}{y}}
\]
where $T$ is defined exactly as in the inductive case of
$\goodnums{\expheight}{y}$
except the tiling problem used is
$\tup{\tiles, \hrel, \vrel, \inittile, \fintile}$
rather than
$\tup{\tilesnum{\expheight},
      \hrelnum{\expheight},
      \vrelnum{\expheight},
      \inittilenum{\expheight},
      \fintilenum{\expheight}}$.

A satisfying tiling
\[
    \begin{array}{c}
        \tile^1_1 \ldots \tile^1_{\nmax{\expheight}} \\
        \cdots \\
        \tile^\tileheight_1 \ldots \tile^\tileheight_{\nmax{\expheight}}
    \end{array}
\]
can be encoded
\[
    \tenc{\expheight}{1} \tile^1_1
    \cdots
    \tenc{\expheight}{\nmax{\expheight}} \tile^1_{\nmax{\expheight}}
    \numplus
    \cdots
    \numplus
    \tenc{\expheight}{1} \tile^\expheight_1
    \cdots
    \tenc{\expheight}{\nmax{\expheight}} \tile^\expheight_{\nmax{\expheight}}
    \numsep
\]
which will satisfy $\probfmla$ in the same way as a correct input to
$\goodnums{\expheight}{y}$.
To see this, note that
$\tenc{\expheight}{1} \tile^i_1
 \cdots
 \tenc{\expheight}{\nmax{\expheight}} \tile^i_{\nmax{\expheight}}$
acts like some
$\tenc{\expheight+1}{i}$.

In the opposite direction, assume some input satisfying $\probfmla$.
Arguing as in the encoding of large numbers, this input must be of the form
\[
    \brac{
        \brac{\tilerow_1 \brac{\numeq \tilerow_1}^\ast}
        \numplus
        \brac{\tilerow_2 \brac{\numeq \tilerow_2}^\ast}
        \numplus
        \ldots
        \numplus
        \brac{
            \tilerow_\tileheight
            \brac{\numeq \tilerow_\tileheight}^\ast
        }
        \numsep
    }^\ast
\]
where each $\tilerow_i$ is a row of a correct solution to the tiling problem.

Thus, with $\expheight+1$ transducers, we can encode a $\expheight$-EXPSPACE-hard problem.
