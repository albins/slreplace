\begin{abstract}
%Recent years have witnessed a lot of work in the development of 
%logics for reasoning about symbolic executions of programs with strings. 
    \OMIT{
    A symbolic execution can be construed as a sequence of assignments and
    conditionals in a program. 
    We study the problem of deciding the feasibility of a given symbolic
    execution (a.k.a. ``path feasibility'') of string-manipulating programs. 
    Such a problem can be mapped to the satisfiability problem of an appropriate
    string logic. Recent years saw a lot of work in the design of such a string
    logic.
    %a ``good'' 
    %string logic for symbolic execution analysis of programs with strings. 
    In addition to string concatenation, finite-state transductions have emerged
    as an expressive formalism for modelling ``built-in'' string functions 
    (e.g. sanitisers like htmlescape and implicit browser transductions like
    innerhtml) that
    are crucial for such an application as analysis of cross-site scripting
    (XSS) vulnerabilities in HTML5 applications. Although naively combining
    concatenation and transductions results in undecidability, the
    \emph{straight-line fragment} --- obtained by restricting the conditionals
    in symbolic executions (e.g. regular expression matching, length
    constraints, but not string equality) --- emerged as a reasonable 
    compromise between decidability and expressivity. 
}

    Finite-state transductions have emerged
    as an expressive formalism for modelling string functions including
    those that are crucial for %such an application as 
    analysis of cross-site scripting (XSS) vulnerabilities in HTML5 
    applications, e.g., sanitisers.
    %including sanitisation functions (e.g.~htmlescape) and 
    %implicit browser transductions (e.g.~innerhtml).  
    Thanks to their good algorithmic properties, %it was recently shown that 
    finite-state transductions have been incorporated into decidable
    theories over strings (with concatenation and regular expression pattern
    matching) in a way that is applicable for %for analysing
    symbolic execution analysis of string-manipulating programs. Despite this,
    such a formalism lacks the expressive power for modelling many
    useful string functions; three notable examples being the replaceall 
    function, %(replacing all occurrences of a ``pattern''
    %string constant/variable/regular expression by a ``replacement'' string 
    %constant/variable), 
    string reverse, and string concatenation. 
    In this paper we introduce \emph{parametric transducers}, an expressive
    formalism for modelling string functions that strictly 
    generalises the standard finite-state transducer model and captures the 
    three aforementioned string functions, among others. We provide an
    expressive decidable
    string logic with parametric transducers as primitive operations, and
    a detailed complexity analysis.
    %and provide a precise computational complexity analysis. 
    The logic 
    %a highly expressive theory over strings that 
    subsumes existing string 
    theories for symbolic execution analysis for programs 
    with strings. We show that string functions required
    for context-sensitive auto-sanitisation web templating can 
    be modelled in our constraint language. Finally, we lift our results
    to the world of symbolic finite automata/transducers, which is necessary for
    practical applications (e.g. large alphabets like UTF16).
    %can capture string functions including
    %that cannot be expressed by finite-state 
    %transductions, e.g., the
    %This paper starts with the observation that finite-state transductions 
    %lack the power of modelling many useful string functions including the
    %the replace-all function (replacing occurrence of replacing all 
    %occurrences of a ``pattern'' string constant/variable/regular expression 
    %by a ``replacement'' string constant/variable), the string reverse
    %function, and concatenation, to name a few. Finite-state transduction
    \OMIT{ 
    Our main result is the decidability of 
    path feasibility of string-manipulating programs whenever the string
    functions in the assignments can be modelled by parameterised transducers
    and the conditionals are regular expression matching. This strictly
    generalises existing decidability results in the literature, and provi. 
    }


    \OMIT{
The problem of reasoning about  
%Various constraint languages have been developed in the literature based on
%the primitive string operations adopted
%
One emerging theme is the need to incorporate string functions including 
replace-all and finite-state transductions; the latter are useful for capturing
sanitisation functions (e.g. htmlescape) and implicit browser transductions 
(e.g. innerhtml), which are crucial for analysing XSS vulnerabilities in
an HTML5 application. 


    In this paper, we introduce an expressive logic for reasoning about symbolic
execution of programs with strings. The logic unifies existing decidable logics 

	The purpose of the project is to generalise the replaceall function considered recently in the POPL'18 to transducers, along the line of the POPL'16 paper.
}
\end{abstract}
