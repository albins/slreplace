%!TEX root = main.tex

%\section{Algorithmic results}

\section{A generic decision procedure with recognisable relations} \label{sec:algo}

In this section,  we present a generic decision procedure for the path feasibility problem %$\straightline[\transet]$ 
of string manipulating program in the SSA form (defined in Section~\ref{subsec:symexe}), 
based on the notion of recognisable relations. We assume that for the assertion \ASSERT{$g(x_1,\ldots,x_n)$}, each variable $x_i$ is constrained regularly, i.e., it is of the form $x_i \in \Aut$ where $\Aut$ is an FA .  

%
%
%an $\straightline[\transet]$ formula $\varphi \wedge \psi$ where $\varphi$ and $\psi$ are relational and  regular constraints respectively.

For the string function $f(x_1,\ldots,x_n)$, we assume that its pre-image of a regular language is a recognisable relation.  

\begin{definition}[Recognisable relation]
	Given a finite alphabet $\Sigma$, a $k$-ary relation $R\subseteq \Sigma^*\times \ldots\times \Sigma^*$ is \emph{recognisable}  if $R=\bigcup_{i=1}^n L^{(i)}_1\times \ldots\times L^{(i)}_k$ where $L^{(i)}_j$ is regular for each $j\in [k]$.
%
%	[One can certainly generalise this to $n$-ary relations. ]
\end{definition}


For a recognisable relation $R$, a \emph{representation} of $R$ is a collection of tuples $(\Aut^{(i)}_1, \ldots, \Aut^{(i)}_k)_{i\in [n]}$  such that 
$R = \bigcup_{i=1}^n \Lang(\Aut^{(i)}_1) \times \ldots\times \Lang(\Aut^{(i)}_k)$, where each $\Aut^{(i)}_j$ is an NFA. The numbers $k, n$ are called the \emph{dimension} and \emph{width}  of the representation respectively, and each NFA $\Aut^{(i)}_j$ is called an \emph{atom} of the representation.
%
Furthermore, to obtain tight space complexity bounds (in some cases), we use an alternative  \emph{conjunctive} representation NFAs, %iof each atoms of the representation, 
More precisely, we encode each atom automaton as $((\controls, \transrel), S)$ such that $(\controls, \transrel)$ is a transition graph and $S \subseteq \controls \times \controls$ is the \emph{acceptance condition} such that a string is accepted by the NFA if for \emph{every} $(q, q') \in S$ there is a run of $(\controls, \transrel)$ on the the string, where $q$ and $q'$ are the first and last state of the run respectively. 
%
Note that $\Lang(((\controls, \transrel), S)) = \bigcap \limits_{(q,q') \in S} \Lang(\cB_{q,\{q'\}})$.
%, where $\cB_{q,q'} = (\controls, q, \{q'\}, \transrel)$. 
%\tl{previously, it was $\bigcap$, but I guess it was a typo?} \zhilin{no, indeed, it is intersection, instead of union.}
%\mat{Could we use the $\cB(q, \{q'\})$ notation introduced just before Section 3?}
We remark that the membership checking  can be done in NLOGSPACE for the standard representation of NFAs, but requires polynomial space for the conjunctive representation. (One needs to build up the product automaton which is of size $|Q|^{|S|}$.) 
%The advantage of the conjunctive encoding is %more \emph{succinct} than normal NFAs in the sense 
%that to recognise $\Lang((\controls, \transrel), S))$ with a normal NFA, the product automaton of  the NFAs $\cB_{q,q'}$ for $(q,q') \in S$, which is of size $|Q|^{|S|}$, is needed. 

% the atom is the product automaton of multiple NFAs whose transition graphs are \emph{the same} (isomorphic). 
 %\tl{why it is succinctly per se? the size does not change here} \zhilin{explained before, please check.}
%
Accordingly, a representation of $R$ is called a \emph{conjunctive representation} if every atom in the representation is encoded conjunctively.
Every (standard) representation of $R$  can be made conjunctively as follows: 
Suppose $\Aut^{(i)}_j=(\controls^{(i)}_j, q^{(i)}_{0,j}, \transrel^{(i)}_j, \finals^{(i)}_j)$ for $i \in [n]$ and $j \in [k]$. Then $\Lang(\Aut^{(i)}_j) = \bigcup \limits_{q \in \finals^{(i)}_j} \Lang(\Aut^{(i)}_j(q^{(i)}_{0,j}, \{q\}))$. Furthermore,  $\Aut^{(i)}_j(q^{(i)}_{0, j}, \{q\})$ can be decoupled as a pair $\cC^{(i)}_{j, (q^{(i)}_{0, j}, q)} = \left(\left(\controls^{(i)}_j,  \transrel^{(i)}_j \right), \{(q^{(i)}_{0,j}, q)\} \right)$. Therefore, $R$ is conjunctively represented by 
%
%$$\left(\cC^{(i)}_{1, (q^{(i)}_{0, 1}, q^{(i)}_1)}, \ldots, \cC^{(i)}_{k, (q^{(i)}_{0, k}, q^{(i)}_k)} \right)_{i \in [n], (q^{(i)}_1, \ldots, q^{(i)}_k) \in \finals^{(i)}_1 \times \ldots \times   \finals^{(i)}_k }.$$

$$\bigcup_{i \in [n], (q^{(i)}_1, \ldots, q^{(i)}_k) \in \finals^{(i)}_1 \times \ldots \times   \finals^{(i)}_k }  \cC^{(i)}_{1, (q^{(i)}_{0, 1}, q^{(i)}_1)}\times \ldots \times \cC^{(i)}_{k, (q^{(i)}_{0, k}, q^{(i)}_k)} .$$

%
%
%\footnote{Each representation can be turned into one satisfying the requirement as follows: Suppose $\Aut^{(i)}_j=(\controls^{(i)}_j, q^{(i)}_{0,j}, \transrel^{(i)}_j, \finals^{(i)}_j)$ for $i \in [n]$ and $j \in [k]$. Since $\Lang(\Aut^{(i)}_j) = \bigcup \limits_{q \in \finals^{(i)}_j} \Lang(\Aut^{(i)}_j(q^{(i)}_{0,j}, \{q\}))$, we have $R = \bigcup \limits_{i=1}^n \bigcup \limits_{\forall j \in [k]. q^{(i)}_j \in \finals^{(i)}_j } \Lang(\Aut^{(i)}_1(q^{(i)}_{0, 1}, \{q^{(i)}_1\})) \times \ldots \times \Lang(\Aut^{(i)}_k(q^{(i)}_{0, k}, \{q^{(i)}_k\}))$. Each NFA $\Aut^{(i)}_j(q^{(i)}_{0, j}, \{q^{(i)}_j\})$ can be encoded as a pair $((\controls^{(i)}_j,  \transrel^{(i)}_j), \{(q^{(i)}_{0,j}, q^{(i)}_j)\})$.} 
%
%
%%%%%%%%%%%%%%%%%%%%%%%%%%%%%%%%%%%%%%%%%
\hide{
It follows that one can reconstruct $R$ by using the alternative representation of automata by  
%we do the following transformation:
%
%Each representation can be turned into one satisfying the requirement as follows: 
%we have 
\begin{align*}
R  & =   \bigcup_{i=1}^n \prod_{j=1}^k \Lang(\Aut^{(i)}_j) \\
& =   \bigcup_{i=1}^n  \prod_{j=1}^k \bigcup \limits_{q_j \in \finals^{(i)}_j} \Lang(\Aut^{(i)}_j(q^{(i)}_{0,j}, \{q_j\}))\\
&=	\bigcup \limits_{i=1}^n \bigcup \limits_{\forall j \in [k]. q^{(i)}_j \in \finals^{(i)}_j } \prod_{j=1}^k \Lang(\Aut^{(i)}_j(q^{(i)}_{0, j}, \{q^{(i)}_j\})) 
%\times \ldots \times \Lang(\Aut^{(i)}_k(q^{(i)}_{0, k}, \{q^{(i)}_k\})).
\end{align*}
%
%
%that each atom $\Aut^{(i)}_j$ is \emph{succinctly} encoded into a pair $((\controls, \transrel), S)$, where $(\controls, \transrel)$ is a transition graph and $S \subseteq \controls \times \controls$, such that $\Lang(\Aut^{(i)}_j) = \bigcap \limits_{(q,q') \in S} \Lang(\cB_{q,q'})$ with $\cB_{q,q'} = (\controls, q, \transrel, \{q'\})$. 
%
\tl{shall we call it something like compact size, so later we can avoid "the size of the succinctly encoded NFA"? }
}
%%%%%%%%%%%%%%%%%%%%%%%%%%%%%%%%%%%%%%%%%

The \emph{size} of the conjunctive encoding of an atom $((\controls, \transrel), S)$ is defined as $|\controls|$. Moreover, we define the  \emph{atom size} of a conjunctive representation of $R$ to be the maximum size of the conjunctive encodings of the atoms therein.

%The maximum size of (the succinct encodings of) the atoms of a representation is called the \emph{atom size} of the representation. 
%=======
%Furthermore, in order to get better space complexity upper bounds (in some cases), we require\footnote{Each representation can be turned into one satisfying the requirement as follows: Suppose $\Aut^{(i)}_j=(\controls^{(i)}_j, q^{(i)}_{0,j}, \finals^{(i)}_j, \transrel^{(i)}_j)$ for $i \in [n]$ and $j \in [k]$. Since $\Lang(\Aut^{(i)}_j) = \bigcup \limits_{q \in \finals^{(i)}_j} \Lang(\Aut^{(i)}_j(q^{(i)}_{0,j}, \{q\}))$, we have $R = \bigcup \limits_{i=1}^n \bigcup \limits_{\forall j \in [k]. q^{(i)}_j \in \finals^{(i)}_j } \Lang(\Aut^{(i)}_1(q^{(i)}_{0, 1}, \{q^{(i)}_1\})) \times \ldots \times \Lang(\Aut^{(i)}_k(q^{(i)}_{0, k}, \{q^{(i)}_k\}))$. Each NFA $\Aut^{(i)}_j(q^{(i)}_{0, j}, \{q^{(i)}_j\})$ can be encoded as a pair $((\controls^{(i)}_j,  \transrel^{(i)}_j), \{(q^{(i)}_{0,j}, q^{(i)}_j)\})$.} 
%%
%that each atom $\Aut^{(i)}_j$ is \emph{succinctly} encoded into a pair $((\controls, \transrel), S)$, where $(\controls, \transrel)$ is a transition graph and $S \subseteq \controls \times \controls$, such that $\Lang(\Aut^{(i)}_j) = \bigcap \limits_{(q,q') \in S} \Lang(\cB_{q,q'})$ with $\cB_{q,q'} = (\controls, q, \{q'\}, \transrel)$. The size of $((\controls, \transrel), S)$ is defined as $|\controls|$. The maximum size of (the succinct encodings of) the atoms of a representation is called the \emph{atom size} of the representation. 
%>>>>>>> 104e69ae6204529f3ecd049a9405556039c70921

%it is the product of $\cB(q, \{q'\})$ for an NFA $\cB=(\controls, q_0, \transrel, \finals)$ and $(q, q') \in S$ with $S \subseteq Q \times Q$, then we encode $\Aut^{(i)}_j$ \emph{succinctly} as a pair $(\transrel, S)$. Note that the \emph{size} of this succinct encoding is $|\transrel|+|S|$, which is polynomial in $|\controls|$, while $|\Aut^{(i)}_j|$ is exponential in $|S|$.

In the sequel, %when we say a representation of $R$, 
the representation of $R$ simply refers to a \emph{conjunctive} representation of $R$. Moreover, a conjunctive FA refers to a pair $((\controls, \transrel), S)$ whereas an FA refers to an FA in the original form, i.e. $(\controls, q_0, \finals, \transrel)$.

\medskip

%Let $\Transducer$ be a parametric transducer %from $\transet$ with $k$ parameters ($k\geq 0$), and 
As we have discussed in Section~\ref{subsec:symexe}, a string function $f(x_1, \cdots, x_k)$ with $k$ parameters ($k\geq 1$) gives a relation $R_f\subseteq (\Sigma^*)^k\times \Sigma^*$. Let $\Aut$ be an FA. The \emph{pre-image} of $\Aut$ under $f$, denoted by $\Pre_{R_f}(\Aut)$, is 
\[\left\{(w_1,\ldots, w_k) \mid \exists w. w\in f(w_1, \cdots, w_k)\text{ and } w\in\Lang(\Aut) \right\}\]
%w is (one of) the output(s) of the string function $f$.  

%
%the set of tuples $(w, w_1,\ldots, w_k)$ such that there is an accepting run of $\Transducer$ on $w$ with output $w'\in \Lang(\Aut)$ when equipped with the parameters $w_1,\ldots, w_k$. 

The main result of this section is a generic decision procedure for the path feasibility problem of string manipulating program $S$ in the SSA form under the following assumptions:
\begin{description}
\item[A1] For each string function $f$ in $S$ and a conjunctive FA $\Aut$,  $\Pre_{R_f}(\Aut)$ is a recognisable relation. Furthermore, the space to compute such a representation and the atom size of $\Pre_{R_f}(\Aut)$\footnote{Usually the amount of space is at least as large as the size of the representation; here for simplicity we assume that they are asymptotically the same.} are both bounded by $\ell(|f|, |\Aut|)$ for some monotone function $\ell$. Here $|f|$ is the size of a representation of $f$; the concrete definition depends on the form of $f$, which will be given the generic decision procedure is instantiated.  

\item[A2] Each assertion $g$ in $S$ is given as a conjunction of atomic regular constraint $x\in \Aut$. 
\end{description} 

Given a string manipulating program $S$, we define $\rcdim(S)$ to be the maximum number of parameters of string functions in $S$, $\rcphi{S}$ to be the maximum size of the representation of string functions in $S$, $\rcpsi(S)$ to be the maximum size of the FA representing the assertions,  and $\rcdep(S)$  to be the number assignments appearing in $S$. 




For $\ell:\Nat^2\rightarrow \Nat$, we define $\ell^{\langle n \rangle}(j, k)$ ($n\geq 1$) as $\ell^{\langle 1 \rangle}(j,k)= \ell(j, k)$ and $\ell^{\langle n+1 \rangle }(j, k) = \ell(j, f^{\langle n \rangle}(j,k))$. 
\mat{
    Do we need to talk about transducers in this section?
    Could we just get away with any relation $R$ such that $Pre_R(A)$ can be constructed effectively?
    It's a bit sad to introduce a generic framework, then tie it unnecessarily to parametric transducers.
    The complexity would need adjusting i guess, though i think it has issues already:
    \begin{itemize}
    \item
        $|\varphi|$ is not defined
    \item
        there is no given space bound on effectively constructing a representation of Pre -- only the size of the output is considered.
        Presumably the complexity given below assumes that the amount of space required to compute the representation is no more than the size of the representation.
    \end{itemize}
}

\begin{theorem}\label{thm-generic-dec}
	Given a string manipulating program $S$ such that both assumptions A1 and A2 are satisfied. Then the path feasibility problem of $S$ can be decided in \emph{nondeterministic} 
	%\tl{why not remove  nondeterministic?} 
	$O((\rcdim(\varphi)+2)^{\rcdep(\varphi)}|\psi| \cdot (f^{\langle \rcdep(\varphi) \rangle}(|\varphi|, |\psi|))^2 \log f^{\langle \rcdep(\varphi) \rangle}(|\varphi|, |\psi|))$ space.
	%, where $K$ is the maximum number of parameters of 
	%transducers appearing in the constraints, 
	%$D$ is the maximum length of the paths in $\cG(\varphi)$.
\end{theorem}

%\begin{theorem}\label{thm-generic-dec}
%Suppose that %$\transet$ satisfies that 
%for each $\Transducer \in \transet$ and conjunctive NFA $\Aut$, $\Pre_\Transducer(\Aut)$ is a recognisable relation.  Moreover, a representation of $\Pre_\Transducer(\Aut)$, whose atom size is bounded by $f(|\Transducer|, |\Aut|)$ for some monotone function $f$, can be determined effectively. Then the satisfiability of a given $\straightline[\transet]$ formula $\varphi \wedge \psi$ can be decided in \emph{nondeterministic} 
%%\tl{why not remove  nondeterministic?} 
%$O((\rcdim(\varphi)+2)^{\rcdep(\varphi)}|\psi| \cdot (f^{\langle \rcdep(\varphi) \rangle}(|\varphi|, |\psi|))^2 \log f^{\langle \rcdep(\varphi) \rangle}(|\varphi|, |\psi|))$ space.
%%, where $K$ is the maximum number of parameters of 
%%transducers appearing in the constraints, 
%%$D$ is the maximum length of the paths in $\cG(\varphi)$.
%\end{theorem}

\begin{proof}
Let $\varphi \wedge \psi$ be an $\straightline[\transet]$ formula. 

We first \emph{nondeterministically} compute $\cE(x)$, a collection of conjunctive NFAs over the alphabet $\Sigma$, for each variable $x \in \vars(\varphi \wedge \psi)$, by utilising the dependency graph $\cG(\varphi)$. 
% , in a top-down manner.

Initially, let $\cG_0 := \cG(\varphi)$, and  
we construct $\cE_0(x)$ as follows: For each conjunct 
\mat{Can we use different terminology: we now have conjunctive NFA and conjuncts, which risks confusion}
$x \in \Aut$ of $\psi$, where $\Aut$ is an NFA, let $\Aut = (\controls, q_0, \transrel, \finals)$, \emph{nondeterministically} select one state $q \in \finals$ and include $((\controls, \transrel), \{(q_0, q)\})$ into $\cE_0(x)$.
  
 %$\Aut$ $\left\{\Aut \mid x\in \Aut \text{ is a conjunct of }\psi \right\}$ for each variable $x$. 

Starting from $\cG_0$ we repeat the following procedure until %we reach some $i$ where 
$\cG_i$ becomes empty, i.e., a graph without edges.
 
We select (nondeterministically) a vertex $x$ of $\cG_i$ such that $x$ has no predecessors and has $k+1$ outgoing edges ($k\geq 0$) via edges $(x, (\Transducer, 0), y_0)$ and $(x, (\Transducer, j), y_j)$ with $j \in [k]$ in $\cG_i$. 
(Intuitively this corresponds to a relational constraint $x=T(y_0, \vec{y})$ for $\vec{y}=(y_1, \ldots, y_k)$.
Note that two different outgoing edges of $x$ in $\cG_i$ may correspond to the same variable.)
Suppose $\cE_i(x)=\{\Aut_1, \ldots, \Aut_n\}$, 
where $\Aut_j$ is a conjunctive NFA for each $j \in [n]$.
%$(\Sigma, Q_j, q_{0,j}, F_j, \delta_j)$ 
By the premise of the theorem, $\Pre_{\Transducer}(\Aut_j)$ is a recognisable relation and a representation of which, say $(\Aut^{(j')}_{j, 0}, \Aut^{(j')}_{j, 1}, \ldots, \Aut^{(j')}_{j, k})_{ j'  \in [m_j]}$, can be computed effectively.
Then $\cE_{i+1}(z)$ for $z \in  \{y_0,\ldots, y_k\}$ and $\cG_{i+1}$ are computed as follows:
\begin{enumerate}
\item For each $j \in [n]$, nondeterministically select a tuple $\left(\Aut^{(r_j)}_{j, 0}, \ldots, \Aut^{(r_j)}_{j, k}\right)$ for some $r_j \in [m_j]$.
%
\item For each $z \in \{y_0,\ldots, y_k\}$, let
\[
    \cE_{i+1}(z):= \cE_{i}(z) \cup \left\{\Aut^{(r_j)}_{j, \ell} \mid  j \in [n], \ell \in \{0\} \cup [k], z = y_{\ell} \right\}.
\]
[For each vertex $z'  {\notin} \{y_0,\ldots, y_k\}$, let $\cE_{i+1}(z') := \cE_i(z')$.]
%We set $\cE_{i+1}(x) = \emptyset$.
%
\item Let $\cG_{i+1}:= \cG_i \backslash \{(x, (\Transducer, j), y_j) \mid j \in \{0\} \cup [k]\}$.
\end{enumerate}
For each iteration, $i$ is updated by  $i: = i+1$.
%\end{enumerate}
%
When exiting the loop, for each variable $x$, let $\cE(x)$ denote the set $\cE_i(x)$.

%\begin{example}
\tl{tbh, I think the algo is clear enough and the example does not add too much.}\zhilin{maybe, but the example is at least more concrete. Moreover, the description of the algorithm is short and dry, maybe just use the example to let the reader spend a bit more time on the algorithm and digest, although with some redundancy. We may remove it if we run out of space. May decide this later.}
Let us use an example to help the reader understand the computation of $\cE_{i+1}$ from $\cE_i$.  Suppose that in $\cG_i$, to compute $\cE_{i+1}$, we select a variable $x$, which has no predecessors and three outgoing edges, say $(x, (\Transducer, 0), y)$, $(x, (\Transducer, 1), z)$, and $(x, (\Transducer, 2), y)$, where $y$ and $z$ are two distinct variables, moreover, $\cE_i(x) = \{\Aut_1, \Aut_2\}$. Let us also assume that  $\Pre_{\Transducer}(\Aut_1)$ (resp. $\Pre_{\Transducer}(\Aut_2)$) is represented by $(\Aut^{(j')}_{1, 0}, \Aut^{(j')}_{1, 1}, \Aut^{(j')}_{1, 2})_{ j'  \in [2]}$ (resp. $(\Aut^{(j')}_{2, 0}, \Aut^{(j')}_{2, 1}, \Aut^{(j')}_{2, 2})_{ j'  \in [3]}$). If for $j = 1$ (resp. $j=2$), $(\Aut^{(1)}_{1, 0}, \Aut^{(1)}_{1, 1}, \Aut^{(1)}_{1, 2})$  (resp. $\left(\Aut^{(3)}_{2, 0}, \Aut^{(3)}_{2, 1}, \Aut^{(3)}_{2, 2}\right)$)  is selected, then $\cE_{i+1}(y) = \cE_i(y) \cup \{\Aut^{(1)}_{1, 0}, \Aut^{(1)}_{1, 2}, \Aut^{(3)}_{2, 0} \cup \Aut^{(3)}_{2, 2}\}$ and $\cE_{i+1}(z) = \cE_{i}(z) \cup \{\Aut^{(1)}_{1, 1}, \Aut^{(3)}_{2, 1}\}$. 
%\end{example}

To decide the satisfiability of $\varphi \wedge \psi$, we have the following nondeterministic algorithm: first (nondeterministically) construct the sets $\cE(x)$ for $x \in \vars(\varphi \wedge \psi)$, which has been detailed above, then 
%guessing an accepting run of the product of NFAs 
checking the emptiness of the product of NFAs in $\cE(x)$ for each $x \in \vars(\varphi \wedge \psi)$.

\paragraph{Complexity analysis.} For each $i$, 
%\begin{itemize}
%\item 
let $M_i$ be the maximum number of elements in $\cE_i(x)$ for $x  \in \vars(\varphi \wedge \psi)$,
%\item 
and $N_i$ be the maximum size of the conjunctive NFAs in $\bigcup \limits_{x \in \vars(\varphi \wedge \psi)} \cE_i(x)$.
%\end{itemize}
Then we have $M_{i+1} \le M_i + (\rcdim(\varphi)+1)M_i = (\rcdim(\varphi)+2) M_i$. Moreover,  because each $T \in \transet$ occurring in $\varphi$ satisfies   $|T| \le |\varphi|$, we have that $N_{i+1} \le f(|T|, N_i) \le f(|\varphi|, N_i)$ (note that we have assumed that $f$ is monotonic). Because for each $x \in \vars(\varphi \wedge \psi)$, $\cE_0(x)$ contains at most $|\psi|$ elements, and each conjunctive NFA in $\cE_0(x)$ is of size bounded by $|\psi|$, we conclude that for each $x \in \vars(\varphi \wedge \psi)$, $\cE(x)$ contains at most $(\rcdim(\varphi)+2)^{\rcdep(\varphi)} |\psi|$ elements, and each conjunctive NFA in $\cE(x)$ is of size bounded by $f^{\langle \rcdep(\varphi) \rangle}(|\varphi|, |\psi|)$. 
Since each conjunctive NFA $((Q, \delta), S)$ in $\cE(x)$ encodes an NFA of size $|Q|^{|S|} \le |Q|^{|Q|^2} \approx 2^{O(|Q|^2 \log |Q|)}$, 
%\tl{I do not quite understand here} \zhilin{explained in the definition of conjunctive encodings of atoms} 
it holds that the product automaton of NFAs in $\cE(x)$ is of size bounded by 
%
$$(\rcdim(\varphi)+2)^{\rcdep(\varphi)} |\psi| \cdot 2^{O((f^{\langle \rcdep(\varphi) \rangle}(|\varphi|, |\psi|))^2 \log f^{\langle \rcdep(\varphi) \rangle}(|\varphi|, |\psi|) )}.$$
%
Because the nonemptiness of an NFA 
can be decided in nondeterministic logarithmic space, we deduce that the nonemptiness checking for $\cE(x)$ can be done in nondeterministic 
$$O(\rcdep(\varphi) (\log ((\rcdim(\varphi)+2)|\psi|)) \cdot (f^{\langle \rcdep(\varphi) \rangle}(|\varphi|, |\psi|))^2 \log f^{\langle \rcdep(\varphi) \rangle}(|\varphi|, |\psi|)) \mbox{ space.}$$
%
Therefore, overall, the space used by the aforementioned nondeterministic algorithm is 
$$O((\rcdim(\varphi)+2)^{\rcdep(\varphi)} |\psi| \cdot  (f^{\langle \rcdep(\varphi) \rangle}(|\varphi|, |\psi|))^2 \log f^{\langle \rcdep(\varphi) \rangle}(|\varphi|, |\psi|)).$$ 
\zhilin{Please check the complexity}
%Therefore, the satisfiability can be checked by a nondeterministic Turing machine with 
%$O((\rcdim(\varphi)+2)^{\rcdep(\varphi)} f^{\langle \rcdep(\varphi) \rangle}(|\varphi|, |\psi|))$ space.
\end{proof}
