%!TEX root = main.tex

%\section{Algorithmic results}

\section{A generic decision procedure} \label{sec:algo}

In this section, we present a generic decision procedure for any
straight-line \emph{constraint language} that satisfies one further condition:
the pre-image of a regular language under any function 
$f: (\ialphabet^*)^k \to \ialphabet^*$ in the constraint satisfies a 
``regularity condition''.
In the case when $k=1$, this regularity condition would state that the pre-image
of a regular language under the function $f$ is effectively regular. This is
satisfied by \FT{} \cite{Berstel,BG07,BG08}, and there is
a decision procedure for satisfiability of the straight-line constraint
language %immediately follows from the decision procedure 
with $f: \ialphabet^* \to \ialphabet^*$ being a \FT{}, e.g., see 
\cite{LB16,BG07,BG08}. %This decision procedure, in fact, works when each
%function $f$ satisfies the regularity condition. 
We illustrate this decision procedure below.

\begin{example}
    Consider the constraint
    \[
        \ASSERT{x \in L_0}; \quad y := f(x); \quad z := g(x); 
        \quad \ASSERT{y \in L_1}; \quad \ASSERT{z \in L_2}
    \]
    where $f, g: \ialphabet^* \to \ialphabet^*$ satisfy the regularity 
    condition. To check this constraint, we repeatedly \emph{remove the last
    assignment} from the constraint by pre-image computation. More
    precisely, we compute the pre-image $g^{-1}(L_2)$
    of $L_2$ under $g$, which by assumption is a regular language $L_2'$,
    yielding the equi-satisfiable constraint:
    \[
        \ASSERT{x \in L_0}; \quad y := f(x); \quad \ASSERT{y \in L_1}; \quad 
        \ASSERT{x \in L_2'}
    \]
    Computing the regular pre-image $L_1' := f^{-1}(L_1)$ gives us
    the constraint:
    \[
        \ASSERT{x \in L_0}; \quad \ASSERT{x \in L_1'}; \quad \ASSERT{x \in L_2'}
    \]
    Checking this constraint amounts to checking emptiness of
    $L_0 \cap L_1' \cap L_2'$, which can be done by a simple automata algorithm.
    \qed
\end{example}
How do we generalise this generic procedure to partial functions $f: 
(\ialphabet^*)^k \to \ialphabet^*$ with multiple inputs, which at least
subsume parametric transducers? There are at least three generalisations
of the notion of ``regularity'' from sets to relations \cite{choffrut-survey}
(in the increasing order of expressivity):
(a) \emph{recognisable} relations, (b) \emph{regular} relations 
(a.k.a.~synchronised rational relations), and (c) \emph{rational} relations. 
Loosely speaking, regular and rational relations are generated by transducers
with $k$ input/output tapes (or variants thereof).
%Loosely speaking, rational relations can be generated by one-way finite 
%transducers with $k$ input/output tapes. Regular relations can be defined in 
%the same 
%way, but the $k$ heads
%pointing to $k$ tapes move in a ``synchronised'' way. 
They both have \emph{tight couplings} between the $k$ inputs tapes, which makes 
it impossible to generalise the above generic decision procedure for $k=1$. 
This tight coupling is, however, absent in recognisable relations.
%
%For each string function $f(x_1,\ldots,x_n)$, we assume that its pre-image of a regular language is a recognisable relation defined below.  

\begin{definition}[Recognisable relations]
	%Assume a finite alphabet $\Sigma$. 
    A $k$-ary relation $R\subseteq \Sigma^*\times \ldots\times \Sigma^*$ is \emph{recognisable}  if $R=\bigcup_{i=1}^n L^{(i)}_1\times \ldots\times L^{(i)}_k$ where $L^{(i)}_j$ is regular for each $j\in [k]$.
%
%	[One can certainly generalise this to $n$-ary relations. ]
\end{definition}
In other words, a recognisable relation is simply a finite union of 
cartesian products of regular languages. Fortunately, recognisability is
right notion of ``regularity'' that we need for pre-images of partial
functions $f: (\ialphabet^*)^k \to \ialphabet^*$, which will be satisfied by
parametric transducers as we will see in Section \ref{sec-instan}.


For a recognisable relation $R$, a \emph{representation} of $R$ is a collection of tuples $(\Aut^{(i)}_1, \ldots, \Aut^{(i)}_k)_{i\in [n]}$  such that 
$R = \bigcup_{i=1}^n \Lang(\Aut^{(i)}_1) \times \ldots\times \Lang(\Aut^{(i)}_k)$, where each $\Aut^{(i)}_j$ is an \FA. 
%The numbers $k, n$ are called the \emph{dimension} and \emph{width}  of the representation respectively, 
Each tuple $(\Aut^{(i)}_1, \ldots, \Aut^{(i)}_k)$ is called a \emph{disjunct} of the representation, and  each \FA{} $\Aut^{(i)}_j$ is called an \emph{atom} of the representation.
%
Furthermore, to obtain tight space complexity bounds (in some cases), we use an alternative  \emph{conjunctive} representation of \FA{}s, %iof each atoms of the representation, 
More precisely, we encode each atom automaton as $((\controls, \transrel), S)$ such that $(\controls, \transrel)$ is a transition graph and $S \subseteq \controls \times \controls$ is the \emph{conjunctive} acceptance condition satisfying that a string is accepted by $((\controls, \transrel), S)$ if for \emph{every} $(q, q') \in S$, there is a run of $(\controls, \transrel)$ on the string where $q$ and $q'$ are the first and last state of the run respectively. 
%
Note that $\Lang(((\controls, \transrel), S)) = \bigcap \limits_{(q,q') \in S} \Lang(\cB_{q,q'})$, where $\cB_{q,q'}$ is the \FA{} $(\controls, q, \{q'\}, \transrel)$.
%
%We remark that the nonemptiness checking  can be done in nondeterministic logarithmic space for the standard representation of \FA{}s, but requires polynomial space for the conjunctive representation. (One needs to build up the product automaton which is of size $|Q|^{|S|}$.) 
%
Accordingly, a representation of $R$ is called a \emph{conjunctive representation} if every atom in the representation is a conjunctive representation.
Every (standard) representation of $R$  can be transformed into a conjunctive one easily. For instance, let $(\Aut_1, \Aut_2)$ be a representation of $R$ with $\Aut_i = (\controls_i, q_{i,0}, \finals_i, \transrel_i)$ for $i=1,2$, then $R$ is conjunctively represented by 
%
$$\left(\left((\controls_1, \transrel_1), \{(q_{1,0}, q_1)\}\right), \left((\controls_2, \transrel_2), \{(q_{2,0}, q_2)\} \right)\right)_{(q_1, q_2) \in \finals_1 \times  \finals_2}.$$

%as follows: 
%Suppose $\Aut^{(i)}_j=(\controls^{(i)}_j, q^{(i)}_{0,j}, \transrel^{(i)}_j, \finals^{(i)}_j)$ for $i \in [n]$ and $j \in [k]$. Then $\Lang(\Aut^{(i)}_j) = \bigcup \limits_{q \in \finals^{(i)}_j} \Lang(\Aut^{(i)}_j(q^{(i)}_{0,j}, \{q\}))$. Furthermore,  $\Aut^{(i)}_j(q^{(i)}_{0, j}, \{q\})$ can be decoupled as a pair $\cC^{(i)}_{j, (q^{(i)}_{0, j}, q)} = \left(\left(\controls^{(i)}_j,  \transrel^{(i)}_j \right), \{(q^{(i)}_{0,j}, q)\} \right)$. Therefore, $R$ is conjunctively represented by 
%
%$$\bigcup_{i \in [n], (q^{(i)}_1, \ldots, q^{(i)}_k) \in \finals^{(i)}_1 \times \ldots \times   \finals^{(i)}_k }  \cC^{(i)}_{1, (q^{(i)}_{0, 1}, q^{(i)}_1)}\times \ldots \times \cC^{(i)}_{k, (q^{(i)}_{0, k}, q^{(i)}_k)} .$$

%
%
%\footnote{Each representation can be turned into one satisfying the requirement as follows: Suppose $\Aut^{(i)}_j=(\controls^{(i)}_j, q^{(i)}_{0,j}, \transrel^{(i)}_j, \finals^{(i)}_j)$ for $i \in [n]$ and $j \in [k]$. Since $\Lang(\Aut^{(i)}_j) = \bigcup \limits_{q \in \finals^{(i)}_j} \Lang(\Aut^{(i)}_j(q^{(i)}_{0,j}, \{q\}))$, we have $R = \bigcup \limits_{i=1}^n \bigcup \limits_{\forall j \in [k]. q^{(i)}_j \in \finals^{(i)}_j } \Lang(\Aut^{(i)}_1(q^{(i)}_{0, 1}, \{q^{(i)}_1\})) \times \ldots \times \Lang(\Aut^{(i)}_k(q^{(i)}_{0, k}, \{q^{(i)}_k\}))$. Each NFA $\Aut^{(i)}_j(q^{(i)}_{0, j}, \{q^{(i)}_j\})$ can be encoded as a pair $((\controls^{(i)}_j,  \transrel^{(i)}_j), \{(q^{(i)}_{0,j}, q^{(i)}_j)\})$.} 
%
%
%%%%%%%%%%%%%%%%%%%%%%%%%%%%%%%%%%%%%%%%%
\hide{
It follows that one can reconstruct $R$ by using the alternative representation of automata by  
%we do the following transformation:
%
%Each representation can be turned into one satisfying the requirement as follows: 
%we have 
\begin{align*}
R  & =   \bigcup_{i=1}^n \prod_{j=1}^k \Lang(\Aut^{(i)}_j) \\
& =   \bigcup_{i=1}^n  \prod_{j=1}^k \bigcup \limits_{q_j \in \finals^{(i)}_j} \Lang(\Aut^{(i)}_j(q^{(i)}_{0,j}, \{q_j\}))\\
&=	\bigcup \limits_{i=1}^n \bigcup \limits_{\forall j \in [k]. q^{(i)}_j \in \finals^{(i)}_j } \prod_{j=1}^k \Lang(\Aut^{(i)}_j(q^{(i)}_{0, j}, \{q^{(i)}_j\})) 
%\times \ldots \times \Lang(\Aut^{(i)}_k(q^{(i)}_{0, k}, \{q^{(i)}_k\})).
\end{align*}
%
%
%that each atom $\Aut^{(i)}_j$ is \emph{succinctly} encoded into a pair $((\controls, \transrel), S)$, where $(\controls, \transrel)$ is a transition graph and $S \subseteq \controls \times \controls$, such that $\Lang(\Aut^{(i)}_j) = \bigcap \limits_{(q,q') \in S} \Lang(\cB_{q,q'})$ with $\cB_{q,q'} = (\controls, q, \transrel, \{q'\})$. 
%
\tl{shall we call it something like compact size, so later we can avoid "the size of the succinctly encoded NFA"? }
}
%%%%%%%%%%%%%%%%%%%%%%%%%%%%%%%%%%%%%%%%%

The \emph{size} of a conjunctive representation $((\controls, \transrel), S)$ is defined as $|\controls|$. Moreover, we define the  \emph{atom size} of a conjunctive representation of $R$ to be the maximum size of the atoms therein.

%The maximum size of (the succinct encodings of) the atoms of a representation is called the \emph{atom size} of the representation. 
%=======
%Furthermore, in order to get better space complexity upper bounds (in some cases), we require\footnote{Each representation can be turned into one satisfying the requirement as follows: Suppose $\Aut^{(i)}_j=(\controls^{(i)}_j, q^{(i)}_{0,j}, \finals^{(i)}_j, \transrel^{(i)}_j)$ for $i \in [n]$ and $j \in [k]$. Since $\Lang(\Aut^{(i)}_j) = \bigcup \limits_{q \in \finals^{(i)}_j} \Lang(\Aut^{(i)}_j(q^{(i)}_{0,j}, \{q\}))$, we have $R = \bigcup \limits_{i=1}^n \bigcup \limits_{\forall j \in [k]. q^{(i)}_j \in \finals^{(i)}_j } \Lang(\Aut^{(i)}_1(q^{(i)}_{0, 1}, \{q^{(i)}_1\})) \times \ldots \times \Lang(\Aut^{(i)}_k(q^{(i)}_{0, k}, \{q^{(i)}_k\}))$. Each NFA $\Aut^{(i)}_j(q^{(i)}_{0, j}, \{q^{(i)}_j\})$ can be encoded as a pair $((\controls^{(i)}_j,  \transrel^{(i)}_j), \{(q^{(i)}_{0,j}, q^{(i)}_j)\})$.} 
%%
%that each atom $\Aut^{(i)}_j$ is \emph{succinctly} encoded into a pair $((\controls, \transrel), S)$, where $(\controls, \transrel)$ is a transition graph and $S \subseteq \controls \times \controls$, such that $\Lang(\Aut^{(i)}_j) = \bigcap \limits_{(q,q') \in S} \Lang(\cB_{q,q'})$ with $\cB_{q,q'} = (\controls, q, \{q'\}, \transrel)$. The size of $((\controls, \transrel), S)$ is defined as $|\controls|$. The maximum size of (the succinct encodings of) the atoms of a representation is called the \emph{atom size} of the representation. 
%>>>>>>> 104e69ae6204529f3ecd049a9405556039c70921

%it is the product of $\cB(q, \{q'\})$ for an NFA $\cB=(\controls, q_0, \transrel, \finals)$ and $(q, q') \in S$ with $S \subseteq Q \times Q$, then we encode $\Aut^{(i)}_j$ \emph{succinctly} as a pair $(\transrel, S)$. Note that the \emph{size} of this succinct encoding is $|\transrel|+|S|$, which is polynomial in $|\controls|$, while $|\Aut^{(i)}_j|$ is exponential in $|S|$.

In the sequel, %when we say a representation of $R$, 
the representation of $R$ simply refers to a \emph{conjunctive} representation of $R$. Moreover, a conjunctive \FA{} refers to a pair $((\controls, \transrel), S)$ whereas an \FA{} refers to an \FA{} in the original form, i.e. $(\controls, q_0, \finals, \transrel)$.
%
\begin{proposition}\label{prop-conj-fa-prod}
For a pair of conjunctive \FA{}s $\Aut_1 = ((\controls_1, \transrel_1), S_1)$ and $\Aut_2 = ((\controls_2, \transrel_2), S_2)$, a conjunctive \FA{} $\Aut = (\controls_1 \times \controls_2, \transrel_1 \times \transrel_2, S)$ such that $|S| = \max(|S_1|, |S_2|)$ can be constructed to recognise $\Lang(\Aut_1) \cap \Lang(\Aut_2)$. 
\end{proposition}


%Let $\Transducer$ be a parametric transducer %from $\transet$ with $k$ parameters ($k\geq 0$), and 
As we have discussed in Section~\ref{subsec:symexe}, a string function $f(x_1, \cdots, x_k)$ with $k$ parameters ($k\geq 1$) gives rise to a relation $R_f\subseteq (\Sigma^*)^k \times \Sigma^*$. Let $\Aut$ be an FA. The \emph{pre-image} of $\Aut$ under $f$, denoted by $\Pre_{R_f}(\Aut)$, is 
\[\left\{(w_1,\ldots, w_k) \in (\Sigma^*)^k \mid \exists w.\ w\in f(w_1, \cdots, w_k)\text{ and } w\in\Lang(\Aut) \right\}.\]
%w is (one of) the output(s) of the string function $f$.  

%
%the set of tuples $(w, w_1,\ldots, w_k)$ such that there is an accepting run of $\Transducer$ on $w$ with output $w'\in \Lang(\Aut)$ when equipped with the parameters $w_1,\ldots, w_k$. 

The main result of this section is a generic decision procedure for the path feasibility problem of string manipulating program $S$ in the SSA form, under the \prerec{} assumption defined as follows.
\begin{quote}
{\bf The \prerec{} assumption}. For each string function $f$ in $S$ and each conjunctive \FA{} $\Aut$,  $\Pre_{R_f}(\Aut)$ is a recognisable relation. Furthermore, 
a representation of $\Pre_{R_f}(\Aut)$, whose atom size is bounded by  $\ell(|f|, |\Aut|)$ for some monotone function $\ell$, can be computed effectively. We also assume that each disjunct of the representation can be nondeterministically computed in $\ell(|f|, |\Aut|)$ space.
%the space to compute a representation of $\Pre_{R_f}(\Aut)$ and the atom size of such a representation\footnote{Usually the amount of space is at least as large as the size of the representation; here for simplicity we assume that they are asymptotically the same.} are both bounded by $\ell(|f|, |\Aut|)$ for some monotone function $\ell$.  
%
%\item[A2] Each assertion $g$ in $S$ is given as a conjunction of atomic regular constraint $x\in \Aut$. 
\end{quote} 
%
Here $|f|$ is the size of a representation of $f$; the concrete definition depends on the form of $f$, which will be given when the generic decision procedure is instantiated in Section~\ref{sec-instan}. 

Given a string manipulating program $S$, 
%w.l.o.g., we assume that each variable appears at most once\footnote{If, for instance, $x\in \Aut_1$ and $x\in \Aut_2$ both appear in the assertions of $S$, then we can lump them using $x\in \Aut_1 \times \Aut_2$.} in the assertions of $S$. 
%Furthermore, 
we define $\rcdep(S)$ to be the number of assignments in $S$, $\rcdim(S)$ to be the maximum number of arguments of string functions in the assignments of $S$, $\rcphi(S)$ to be the maximum size of the representations of these string functions, $\rcpsi(S)$ to be the maximum size of the \FA{}s appearing in the assertions of $S$, and $\rcreg(S)$ to be the maximum number of atomic regular constraints for one variable in $S$.


For $\ell:\Nat^2\rightarrow \Nat$, we define $\ell^{\langle n \rangle}(j, k)$ ($n\geq 1$) as $\ell^{\langle 1 \rangle}(j,k)= \ell(j, k)$ and $\ell^{\langle n+1 \rangle }(j, k) = \ell(j, \ell^{\langle n \rangle}(j,k))$. 

%%%=========================================
\hide{
\mat{
    Do we need to talk about transducers in this section?
    Could we just get away with any relation $R$ such that $Pre_R(A)$ can be constructed effectively?
    It's a bit sad to introduce a generic framework, then tie it unnecessarily to parametric transducers.
    The complexity would need adjusting i guess, though i think it has issues already:
    \begin{itemize}
    \item
        $|\varphi|$ is not defined
    \item
        there is no given space bound on effectively constructing a representation of Pre -- only the size of the output is considered.
        Presumably the complexity given below assumes that the amount of space required to compute the representation is no more than the size of the representation.
    \end{itemize}
}
\tl{I have tried to accommodate this, though need to proof read carefully.}
}
%%%=========================================


%==== theorem ==============================
\begin{theorem}\label{thm-generic-dec}
	Given a string manipulating program $S$ satisfying the \prerec{} assumption, the path feasibility problem of $S$ can be decided in \emph{nondeterministic} 
	$|\vars(S)|(\rcdim(S)+1)^{\rcdep(S)}  \rcreg(S) \left(\ell^{\langle \rcdep(S) \rangle}(\rcphi(S), \rcpsi(S)) \right)^c$ space for some $c > 0$.
\end{theorem}

%\begin{theorem}\label{thm-generic-dec}
%Suppose that %$\transet$ satisfies that 
%for each $\Transducer \in \transet$ and conjunctive NFA $\Aut$, $\Pre_\Transducer(\Aut)$ is a recognisable relation.  Moreover, a representation of $\Pre_\Transducer(\Aut)$, whose atom size is bounded by $f(|\Transducer|, |\Aut|)$ for some monotone function $f$, can be determined effectively. Then the satisfiability of a given $\straightline[\transet]$ formula $\varphi \wedge \psi$ can be decided in \emph{nondeterministic} 
%%\tl{why not remove  nondeterministic?} 
%$O((\rcdim(\varphi)+2)^{\rcdep(\varphi)}|\psi| \cdot (f^{\langle \rcdep(\varphi) \rangle}(|\varphi|, |\psi|))^2 \log f^{\langle \rcdep(\varphi) \rangle}(|\varphi|, |\psi|))$ space.
%%, where $K$ is the maximum number of parameters of 
%%transducers appearing in the constraints, 
%%$D$ is the maximum length of the paths in $\cG(\varphi)$.
%\end{theorem}

%\begin{proof}
%Let $\varphi \wedge \psi$ be an $\straightline[\transet]$ formula. 

Let $S$ be a string manipulating program satisfying the \prerec{} assumption. 
We shall present a nondeterministic algorithm to decide the path feasibility problem of $S$. 
%\tl{we need to unify the name, path feasibility or satisfiability?}\zhilin{path feasibility}

We first \emph{nondeterministically} compute $\cE(x)$, a collection of conjunctive FAs over the alphabet $\Sigma$, for each variable $x \in \vars(S)$. %, by utilising the dependency graph $\cG(\varphi)$. 
% , in a top-down manner.

  
 %$\Aut$ $\left\{\Aut \mid x\in \Aut \text{ is a conjunct of }\psi \right\}$ for each variable $x$. 

%Starting from $\cG_0$ we repeat the following procedure until %we reach some $i$ where 
%$\cG_i$ becomes empty, i.e., a graph without edges.

Notice that $S$ has $\rcdep(S)$ assignments. we start from the last assignment of $S$,  set $i:=\rcdep(S)$, and construct $\cE_{\rcdep(S)}(x)$ as follows: For each %conjunct 
%\mat{Can we use different terminology: we now have conjunctive NFA and conjuncts, which risks confusion}
$x \in \Aut$ appearing in the assertion of $S$, where $\Aut$ is an FA, let $\Aut = (\controls, q_0, \transrel, \finals)$, we \emph{nondeterministically} select one state $q \in \finals$ and include $((\controls, \transrel), \{(q_0, q)\})$ into $\cE_{\rcdep(S)}(x)$. Then we iterate the following procedure to compute $\cE_{i}$ until $i=0$.  
%
%select (nondeterministically) a vertex $x$ of $\cG_i$ such that $x$ has no predecessors and has $k+1$ outgoing edges ($k\geq 0$) via edges $(x, (\Transducer, 0), y_0)$ and $(x, (\Transducer, j), y_j)$ with $j \in [k]$ in $\cG_i$. 
%(Intuitively this corresponds to a relational constraint $x=T(y_0, \vec{y})$ for $\vec{y}=(y_1, \ldots, y_k)$.
%Note that two different outgoing edges of $x$ in $\cG_i$ may correspond to the same variable.)

For the $i$-th iteration, suppose the $i$-th assignment of $S$ to be $y:= f(x_1, \cdots, x_k)$ with $k\geq 1$, and  $\cE_i(y)=\{\Aut_1, \ldots, \Aut_n\}$, 
where $\Aut_j$ is a conjunctive FA for each $j \in [n]$.
%$(\Sigma, Q_j, q_{0,j}, F_j, \delta_j)$ 
By the premise of the theorem, $\Pre_{R_f}(\Aut_j)$ is a recognisable relation and a representation of which, say $(\Aut^{(\iota)}_{j, 1}, \ldots, \Aut^{(\iota)}_{j, k})_{\iota \in [m_j]}$, 
%\tl{zhilin: why the index starts from 0? it is a $k$-ary relation, isn't it?} 
can be computed effectively. Then $\cE_{i-1}(x)$ for $x \in  \{x_1,\cdots, x_k\}$ is constructed as follows: %and $\cG_{i+1}$ are computed as follows:
\begin{enumerate}
\item For each $j \in [n]$, nondeterministically select a tuple $\left(\Aut^{(r_j)}_{j, 1}, \ldots, \Aut^{(r_j)}_{j, k}\right)$ for some $r_j \in [m_j]$.
%
\item For each $x \in \{x_1,\ldots, x_k\}$, let
\[
    \cE_{i-1}(x):= \cE_{i}(x) \cup \left\{\Aut^{(r_j)}_{j, h} \mid  j \in [n], h \in [k], x = x_h \right\}.
\]
[For each $x{\notin} \{x_1,\ldots, x_k\}$, let $\cE_{i-1}(x) := \cE_i(x)$.]
%We set $\cE_{i+1}(x) = \emptyset$.
%
%\item Let $\cG_{i+1}:= \cG_i \backslash \{(x, (\Transducer, j), y_j) \mid j \in \{0\} \cup [k]\}$.
\end{enumerate}
%For each iteration, $i$ is updated by  $i: = i+1$.
%\end{enumerate}
%
After all assignments of $S$ are scanned,  let $\cE(x):=\cE_{0}(x)$ for each variable $x$.


%%=================================================================
%%=================================================================
\hide{
%\begin{example}
\tl{tbh, I think the algo is clear enough and the example does not add too much.}\zhilin{maybe, but the example is at least more concrete. Moreover, the description of the algorithm is short and dry, maybe just use the example to let the reader spend a bit more time on the algorithm and digest, although with some redundancy. We may remove it if we run out of space. May decide this later.}
Let us use an example to help the reader understand the computation of $\cE_{i+1}$ from $\cE_i$.  Suppose that in $\cG_i$, to compute $\cE_{i+1}$, we select a variable $x$, which has no predecessors and three outgoing edges, say $(x, (\Transducer, 0), y)$, $(x, (\Transducer, 1), z)$, and $(x, (\Transducer, 2), y)$, where $y$ and $z$ are two distinct variables, moreover, $\cE_i(x) = \{\Aut_1, \Aut_2\}$. Let us also assume that  $\Pre_{\Transducer}(\Aut_1)$ (resp. $\Pre_{\Transducer}(\Aut_2)$) is represented by $(\Aut^{(j')}_{1, 0}, \Aut^{(j')}_{1, 1}, \Aut^{(j')}_{1, 2})_{ j'  \in [2]}$ (resp. $(\Aut^{(j')}_{2, 0}, \Aut^{(j')}_{2, 1}, \Aut^{(j')}_{2, 2})_{ j'  \in [3]}$). If for $j = 1$ (resp. $j=2$), $(\Aut^{(1)}_{1, 0}, \Aut^{(1)}_{1, 1}, \Aut^{(1)}_{1, 2})$  (resp. $\left(\Aut^{(3)}_{2, 0}, \Aut^{(3)}_{2, 1}, \Aut^{(3)}_{2, 2}\right)$)  is selected, then $\cE_{i+1}(y) = \cE_i(y) \cup \{\Aut^{(1)}_{1, 0}, \Aut^{(1)}_{1, 2}, \Aut^{(3)}_{2, 0} \cup \Aut^{(3)}_{2, 2}\}$ and $\cE_{i+1}(z) = \cE_{i}(z) \cup \{\Aut^{(1)}_{1, 1}, \Aut^{(3)}_{2, 1}\}$. 
\tl{I have not updated this part as likely the proof will go to appendix}
}
%%=================================================================
%%=================================================================


To decide the path feasibility of $S$, we have the following nondeterministic algorithm: first (nondeterministically) construct the sets $\cE(x)$ for $x \in \vars(S)$ as detailed above, then 
%guessing an accepting run of the product of NFAs 
checking the nonemptiness of the product of conjunctive \FA{}s in $\cE(x)$ for each $x \in \vars(S)$.

\begin{example}	
%	Let us use an example to help the reader understand the computation of $\cE_{i+1}$ from $\cE_i$.  Suppose that in $\cG_i$, to compute $\cE_{i+1}$, we select a variable $x$, which has no predecessors and three outgoing edges, say $(x, (\Transducer, 0), y)$, $(x, (\Transducer, 1), z)$, and $(x, (\Transducer, 2), y)$, where $y$ and $z$ are two distinct variables, moreover,
%	
%	
%	 $\cE_i(x) = \{\Aut_1, \Aut_2\}$. Let us also assume that  $\Pre_{\Transducer}(\Aut_1)$ (resp. $\Pre_{\Transducer}(\Aut_2)$) is represented by $(\Aut^{(j')}_{1, 0}, \Aut^{(j')}_{1, 1}, \Aut^{(j')}_{1, 2})_{ j'  \in [2]}$ (resp. $(\Aut^{(j')}_{2, 0}, \Aut^{(j')}_{2, 1}, \Aut^{(j')}_{2, 2})_{ j'  \in [3]}$). If for $j = 1$ (resp. $j=2$), $(\Aut^{(1)}_{1, 0}, \Aut^{(1)}_{1, 1}, \Aut^{(1)}_{1, 2})$  (resp. $\left(\Aut^{(3)}_{2, 0}, \Aut^{(3)}_{2, 1}, \Aut^{(3)}_{2, 2}\right)$)  is selected, then $\cE_{i+1}(y) = \cE_i(y) \cup \{\Aut^{(1)}_{1, 0}, \Aut^{(1)}_{1, 2}, \Aut^{(3)}_{2, 0} \cup \Aut^{(3)}_{2, 2}\}$ and $\cE_{i+1}(z) = \cE_{i}(z) \cup \{\Aut^{(1)}_{1, 1}, \Aut^{(3)}_{2, 1}\}$. 
%	
	We slightly extend the $\text{replaceAll}$ example given in Figure \ref{fig:SSA} to illustrate the generic decision procedure. Suppose that the String manipulating program $S$ is 
		$\ASSERT{x_0\in L_0}$, $x_1 = \text{replaceAll}(x_0,\texttt{a},y)$, $\ASSERT{x_1\in L_1}$; $x_2= \text{replaceAll}(x_1, \texttt{b}, y)$, $\ASSERT{x_2\in L_2}$. The variables are $x, x_1, x_2, y$, and clearly $\rcdep(S)=2$ and $\rcdim(S)=2$ (as the replaceAll function here has two arguments). 
		
		To run the algorithm, initially we have $ \cE_2(x_i)= \{\Aut_i\}$ for $i=0,1,2$, and $ \cE_2(y)= \emptyset$. 
		
		Then we start with the last assignment $x_2=\text{replaceAll}(x_1,\texttt{b},y)$. By the PreRec assumption, the pre-image of $\Aut_2$ under $\text{replaceAll}_b$ is a recognisable relation, which is 
		represented by  $(\Aut^{(k)}_{2, 1}, \Aut^{(k)}_{2, 2})_{ k=1,2,3}$. By nondeterministic selection (the machine happens to pick $k=1$), we update $\cE_2$ to $\cE_1$ with $\cE_1(x_0)=\{\Aut_0\}$ (remains unchanged), $\cE_1(x_1)=\{\Aut_1,\Aut^{(1)}_{2,1}\}$, $\cE_1(x_2)=\{\Aut_2\}$ (remains unchanged), and $\cE_1(y)=\{\Aut^{(1)}_{2,2}\}$. 
		
			%		In other
		%		words, we obtain the resulting regular constraints for $x_1$ and $y$. Assume that  $\Pre_{\Transducer}(\Aut_1)$ (resp. $\Pre_{\Transducer}(\Aut_2)$) is represented by $(\Aut^{(j')}_{1, 0}, \Aut^{(j')}_{1, 1}, \Aut^{(j')}_{1, 2})_{ j'  \in [2]}$  
		
		We continue with the assignment $x_1 = \text{replaceAll}(x_0,\texttt{a},y)$. Again by the PreRec assumption, the pre-image of both $\Aut_1,\Aut^{(1)}_{2,1}$ from $\cE_1(x_1)$ under  $\text{replaceAll}_a$ is a recognisable relation. In this case, suppose 
		the pre-image of $\Aut_1$ is represented by $(\mathcal{B}^{(k)}_{1, 1}, \mathcal{B}^{(k)}_{1, 2})_{k = 1,2}$, and the pre-imeage of $\Aut^{(1)}_{2,1})$ is represented by $( \mathcal{C}^{(\ell)}_{2, 1}, \mathcal{C}^{(\ell)}_{2, 2})_{ \ell=1,2,3}$. 
		
		%In other words, we obtain the resulting regular constraints for $x_0$ and $y$. 
		
		By nondeterministic selection (the machine happens to pick $k=2$ and $\ell=3$), we update $\cE_1$ to $\cE_0$ with $\cE_2(x_0)=\{\Aut_0, \mathcal{B}^{(2)}_{1,1}, \mathcal{C}^{(3)}_{1,1}\}$, $\cE_0(x_1)=\{\Aut_1,\Aut^{(1)}_{2,1}\}$, $\cE_0(x_2)=\{\Aut_2\}$ (these two remain unchanged), and $\cE_0(y)=\{\Aut^{(1)}_{2,2}, \mathcal{B}^{(2)}_{1,2},  \mathcal{C}^{(3)}_{1,2}\}$.
		
		The iteration now is done. We then only need to check nonemptiness of the product automata from $\cE_0(x_i)$ ($i=0,1,2$) and $\cE_0(y)$. \qed
\end{example}

%\zhilin{I simplified the complexity analysis. More details are put in the appendix.}
\paragraph{Complexity analysis.}
For each $i$, 
%\begin{itemize}
%\item 
let $M_i$ be the maximum number of elements in $\cE_i(x)$ for $x  \in \vars(S)$,
%\item 
and $N_i$ be the maximum size of the conjunctive FAs in $\bigcup \limits_{x \in \vars(S)} \cE_i(x)$.
%\end{itemize}
%<<<<<<< HEAD
%Then we have $M_{i-1} \le (k \cdot \ell_w(|f|, N_i)+1) M_i \le (\rcdim(S) \cdot \ell_w(|f|, N_i)+1) M_i$  (note that we have assumed that $\ell_w$ and $\ell_a$ are monotonic). Moreover, since each string function $f$ satisfies   $|f| \le \rcphi(S)$, we have that $N_{i-1} \le \ell_a(|f|, N_i) \le \ell_a(\rcphi(S), N_i)$. Because for each $x \in \vars(S)$, %$\cE_0(x)$ contains at most $|\psi|$ elements, and 
%each \FA{} in $\cE_{\rcdep(S)}(x)$ is of size bounded by $\rcpsi(S)$, we have that for each $x \in \vars(S)$, each \FA{} in $\cE(x)$ is of size bounded by $\ell_a^{\langle \rcdep(S) \rangle}(\rcphi(S), \rcpsi(S))$, and $\cE(x)$ contains at most 
%$$
%\begin{array}{l}
%|\cE_{\rcdep(S)}(x)  | \prod \limits_{i = 1}^m (\rcdim(S) \ell_w(|f|, N_i) + 1) \le \\
%\rcreg(S)  \prod \limits_{i = 1}^m \left(\rcdim(S) \ell_w^{\langle \rcdep(S)- i+1\rangle} (\rcphi(S), \rcpsi(S)) + 1\right)
%\end{array}
%$$ elements.
%We emphasise that, according to the \prerec{} assumption, the construction of these \FA{}s can be done in nondeterministic 
%$
%%k \cdot \ell^{\langle \rcdep(S)\rangle}_w(|f|, \rcpsi(S))\cdot \ell^{\langle \rcdep(S)\rangle}_a(|f|, \rcpsi(S)) \le 
%\rcdim(S) \cdot \ell^{\langle \rcdep(S)\rangle}_w(\rcphi(S), \rcpsi(S))\cdot \ell^{\langle \rcdep(S)\rangle}_a(\rcphi(S), \rcpsi(S)) $ space. 
%Therefore, for each $x \in \vars(S)$, the nonemptiness of the product of \FA{}s  in $\cE(x)$ can be solved in nondeterministic 
%\[
%\begin{array}{l}
%\rcreg^c(S)  \left(\prod \limits_{i = 1}^m \left(\rcdim(S) \ell_w^{\langle \rcdep(S)- i+1\rangle} (\rcphi(S), \rcpsi(S)) + 1\right)^c\right)\ \cdot \\
%\hspace{2cm} \log^c \ell_a^{\langle \rcdep(S) \rangle}(\rcphi(S), \rcpsi(S))
%\end{array}
%\]
%space for some constant $c > 0$.
%
%\begin{example}

%		
%		\end{example}
%
%%
%%Given a string manipulating program $S$, w.l.o.g., we assume that each variable appears at most once\footnote{If, for instance, $x\in \Aut_1$ and $x\in \Aut_2$ both appear in the assertions of $S$, then we can lump them using $x\in \Aut_1 \times \Aut_2$.} in the assertions of $S$. Furthermore, we define $\rcdep(S)$ to be the number of assignments in $S$, $\rcdim(S)$ to be the maximum number of arguments of string functions in the assignments of $S$, $\rcphi(S)$ to be the maximum size of the representations of these string functions, and $\rcpsi(S)$ to be the maximum size of the \FA{}s appearing in the assertions of $S$. 
%
%
%=======
Then we have $M_{i-1} \le (\rcdim(S)+1)M_i $. Moreover, since each string function $f$ satisfies   $|f| \le \rcphi(S)$, we have that $N_{i-1} \le \ell(|f|, N_i) \le \ell(\rcphi(S), N_i)$ (note that we have assumed that $\ell$ is monotonic). Because for each $x \in \vars(S)$, $\cE_{\rcdep(S)}(x)$ contains at most $\rcreg(S)$ elements, and 
each conjunctive \FA{} in $\cE_{\rcdep(S)}(x)$ is of size bounded by $\rcpsi(S)$, we have that for each $x \in \vars(S)$, $\cE(x)$ contains at most $(\rcdim(S)+1)^{\rcdep(S)}\rcreg(S)$ elements, and each conjunctive \FA{} in $\cE(x)$ is of size bounded by $\ell^{\langle \rcdep(S) \rangle}(\rcphi(S), \rcpsi(S))$. 
%We emphasise that, according to the \prerec{} assumption, the construction of these \FA{}s can be done in nondeterministic $\ell^{\langle \rcdep(S) \rangle}(\rcphi(S), \rcpsi(S))$ space as well. 
Then by utilising  the \prerec{} assumption and Proposition~\ref{prop-conj-fa-prod}, after some simple calculations, we can deduce that the aforementioned nondeterministic algorithm takes 
$|\vars(S)|(\rcdim(S)+1)^{\rcdep(S)}  \rcreg(S) \left(\ell^{\langle \rcdep(S) \rangle}(\rcphi(S), \rcpsi(S)) \right)^c$
 space for some constant $c > 0$.
 


%%%%===========================original complexity analysis removed ====================================
%%%%===========================original complexity analysis removed ==================================
\hide{
%
Since each conjunctive FA $((Q, \delta), S)$ in $\cE(x)$ encodes an FA of size $|Q|^{|S|} \le |Q|^{|Q|^2} \approx 2^{O(|Q|^2 \log |Q|)}$, 
%\tl{I do not quite understand here} \zhilin{explained in the definition of conjunctive encodings of atoms} 
it holds that the product automaton of FAs in $\cE(x)$ is of size bounded by 
%
$$(\rcdim(S)+1)^{\rcdep(S)} \cdot 2^{\mathcal{O}((\ell^{\langle \rcdep(S) \rangle}(\rcphi(S), \rcpsi(S)))^2 \log \ell^{\langle \rcdep(S) \rangle}(\rcphi(S), \rcpsi(S)) )}.$$
%
Because the nonemptiness of an FA 
can be decided in nondeterministic logarithmic space, we deduce that the nonemptiness checking for $\cE(x)$ can be done in nondeterministic space bounded by 
$$\mathcal{O}(\rcdep(S) (\log ((\rcdim(S)+1))) \cdot (\ell^{\langle \rcdep(S) \rangle}(\rcphi(S), \rcpsi(S)))^2 \log \ell^{\langle \rcdep(S) \rangle}(\rcphi(S), \rcpsi(S))).$$
%=======
%$$2^{O((\rcdim(\varphi)+2)^{\rcdep(\varphi)} |\psi|  (f^{\langle \rcdep(\varphi) \rangle}(|\varphi|, |\psi|))^2 \log f^{\langle \rcdep(\varphi) \rangle}(|\varphi|, |\psi|) )}.$$
%%
%Because the nonemptiness of an NFA 
%can be decided in nondeterministic logarithmic space, we deduce that the nonemptiness checking for $\cE(x)$ can be done in nondeterministic 
%$$O((\rcdim(\varphi)+2)^{\rcdep(\varphi)} |\psi|  (f^{\langle \rcdep(\varphi) \rangle}(|\varphi|, |\psi|))^2 \log f^{\langle \rcdep(\varphi) \rangle}(|\varphi|, |\psi|) ) \mbox{ space.}$$
%>>>>>>> e8a132a375e6d8f5c007d33427151d3386a5f8ee
%
Therefore, overall, the space used by the aforementioned nondeterministic algorithm is 
$$\mathcal{O}((\rcdim(S)+1)^{\rcdep(S)} \cdot  (\ell^{\langle \rcdep(S) \rangle}(\rcphi(S), \rcpsi(S)))^2 \log f^{\langle \rcdep(S) \rangle}(\rcphi(S), \rcpsi(S))).$$ 
%\zhilin{Please check the complexity}
%Therefore, the satisfiability can be checked by a nondeterministic Turing machine with 
%$O((\rcdim(\varphi)+2)^{\rcdep(\varphi)} f^{\langle \rcdep(\varphi) \rangle}(|\varphi|, |\psi|))$ space.
%\end{proof}
}
