
\paragraph{Google Closure Example}

Web pages are often constructed using templating systems such as Mustache.js or Google Closure.
In these systems, a template HTML file contains identifiers that are instantiated with values before the page is served.
A simple Google Closure template is give in Figure~\ref{fig:closure}.
In this example, the template will be instantiated by providing the values of the \linkvar\ and \linktextvar\ variables.
It is straightforward to see how this may be naively handled by a parametric transducer:
the input variables will be \linkvar\ and \linktextvar, and the input string will be the template.
Then, whenever the transducer reads the text \mintinline{html}|{$link}|, it outputs the value of the variable \linkvar, and similarly for \linktextvar.

\begin{figure}
    \begin{minted}{html}
            {template .closure-example}
                <a href="{$link}">{$linkText}</a>
            {/template}
    \end{minted}
    \caption{\label{fig:closure}A simple Google Closure template.}
\end{figure}

However, Closure is not quite so naive and provides an ``auto-escaping'' feature.
That is, the contents of the replaced variable will be ``escaped'', which means that special characters or character sequences which may allow code injection are replaced with safe alternatives to avoid attack.
One such example attack would be if \linkvar\ had the value \mintinline{js}{javascript: alert(1337)}.
In this case, the link in the instantiated template would run the JavaScript code \mintinline{js}{alert(1337)} when the user clicks on the link.
To prevent this, Closure would make the code safe by wrapping any URL beginning with \mintinline{html}{javascript:} in single quotes.
Providing the value \mintinline{html}{<script>alert(1337)</script>} to \linktextvar\ would have a similar effect.
Closure would make the instantiation safe by replacing \texttt{<} with \texttt{\&lt;} and \texttt{>} with \texttt{\&gt;}.

A key point is that the transduction Closure applies to an input variable depends on the context in which the variable occurs.
In this example, \linkvar\ appears in a URL context, while \linktextvar\ appears in an HTML context.
To be able to model the behaviour of Closure in our constraint language, we need to handle these context-aware transductions.

This can be done in two steps using two transducers.
In the first, the transducer replaces the variables with their supplied values, but surrounds them with special markers that indicate that the text inside needs to be subject to another transduction.
For our example, let these be \urlstarttag, \urlendtag, \htmlstarttag, and \htmlendtag.
The result of the first transduction is shown in Figure~\ref{fig:closure-step-one}.
Since transducers are a generalisation of finite automata, they are able to identify contexts using regular languages.

\begin{figure}
    \begin{minted}{html}
            <a href="[URL]javascript: alert(1337)[LRU]">
                [HTML]<script>alert(1337)</script>[HTML]
            </a>
    \end{minted}
    \caption{\label{fig:closure-step-one}The template after step 1 of the transductions.}
\end{figure}

The second step of the modelling applies a transducer which applies the appropriate safe transductions to the text inside the \urlstarttag, \urlendtag, \htmlstarttag, and \htmlendtag markers.
For the URL context, any text starting with \texttt{javascript:} will be surrounded by quotes, while in the HTML context, instances of \texttt{<} and \texttt{>} will be replaced by \texttt{\&lt;} and \texttt{\&gt;} respectively.
The result is shown in Figure~\ref{fig:closure-safe}.

\mat{
    One might worry that identifying text starting with \texttt{javascript:} means the transudcer needs to keep a buffer of the input string, leading to large transducers.
    In fact, we can avoid this by having the transducer non-deterministically guess that the following string will begin with \texttt{javascript:} and then verifying later that this was or was not the case.
    Let me know if you think i should put a footnote about this.
    It would also apply to contexts that need to see the text after the variable to confirm the (guessed) context is the right one.
}

\begin{figure}
    \begin{minted}{html}
            <a href="'javascript: alert(1337)'">
                &lt;script&gt;alert(1337)&lt;/script&gt;
            </a>
    \end{minted}
    \caption{\label{fig:closure-safe}A safely instantiated template.}
\end{figure}

\mat{Should i give a formula in our constraint language?}

