
\section{Definition}

We define $\numtapes$-tape transducers before introducing the nesting restriction.


\subsection{$\numtapes$-Tape Transducers}

\begin{definition}[$\numtapes$-Tape Transducers]
    A \emph{$\numtapes$-tape transducer} is a tuple
    $\ducer = \tup{\tstates, \alphabet, \tdelta, \tfinals}$
    where
        $\tstates$ is a finite set of control states,
        $\alphabet$ is a finite alphabet,
        $\tfinals \subseteq \tstates$ is a set of final states, and
        $\tdelta \subseteq
            \tstates \times
            \alphabet \times \interval{1}{\numtapes} \times
            \brac{\alphabet \cup \set{\eword}} \times
            \set{\tleft, \tright, \tnoop} \times \tstates$
        is the transition relation.
\end{definition}

A $\numtapes$-tape transducer takes $\numtapes$ input words over $\alphabet$ and (nondeterministically) produces an output word also over $\alphabet$.
Thus, a configuration is a tuple
\[
    \tconfig =
    \tup{
        \tstate,
        \iword_1, \tapepos_1,
        \ldots,
        \iword_\numtapes, \tapepos_\numtapes,
        \oword
    }
\]
where for each
$1 \leq \idxi \leq \numtapes$
we have
    the state $\tstate \in \tstates$ is the current control state,
    $\iword_\idxi \in \alphabet^\ast$
        is the word on the $\idxi$th input tape,
    $\tapepos_\idxi \in \interval{1}{\wlen{\iword_\idxi}}$
        is the position of the tape head on the $\idxi$th tape, and
    $\oword$ is the current output word.

We write
\[
    \ttran{\tstate}{\cha}{\tidx}{\chb}{\tdir}{\tstate'}
\]
for a transition
$\tup{\tstate, \cha, \tidx, \chb, \tdir, \tstate'} \in \tdelta$.
We have move
\[
    \tup{
        \tstate,
        \iword_1, \tapepos_1,
        \ldots,
        \iword_\numtapes, \tapepos_\numtapes,
        \oword
    }
    \tmove
    \tup{
        \tstate',
        \iword_1, \tapepos'_1,
        \ldots,
        \iword_\numtapes, \tapepos'_\numtapes,
        \oword'
    }
\]
whenever we have a transition
$\ttran{\tstate}{\cha}{\tidx}{\chb}{\tdir}{\tstate'}$
with
\begin{itemize}
\item
    $\wpos{\iword_\tidx}{\tapepos_\tidx} = \cha$,
\item
    $\tapepos'_\tidx = \tapepos_\tidx + 1$
        if $\tapepos_\tidx < \wlen{\iword_\tidx}$
        and $\tdir = \tright$,
\item
    $\tapepos'_\tidx = \tapepos_\tidx - 1$
        if $\tapepos_\tidx > 1$
        and $\tdir = \tleft$,
\item
    $\tapepos'_\tidx = \tapepos_\tidx$
        if $\tdir = \tnoop$,
\item
    $\tapepos'_\idxi = \tapepos_\idxi$
        for all $\idxi \neq \tidx$,
\item
    $\oword' = \oword \chb$.
\end{itemize}
We denote a run
\[
    \tconfig \trun \tconfig'
    \quad
    \text{whenever}
    \quad
    \exists
        \tconfig_0, \ldots, \tconfig_\runlen \ .\ %
    \tconfig = \tconfig_0 \tmove \cdots \tmove \tconfig_\runlen = \tconfig' \ .
\]
Then, we define
\begin{multline*}
    \tlangof{\ducer}{\tstate}{\iword_1, \ldots, \iword_\numtapes}
    = \\
    \setcomp{\oword}{
        \tup{
            \tstate,
            \iword_1, 1,
            \ldots,
            \iword_\numtapes, 1,
            \varepsilon
        }
        \trun
        \tup{
            \tstate',
            \iword_1, 1,
            \ldots,
            \iword_\numtapes, 1,
            \oword
        }
        \land
        \tstate' \in \tfinals
    } \ .
\end{multline*}

In the sequel, it will be convenient to refer to runs only using a restricted set of tapes.
Thus, we define
$\tconfig \tmoveidx{\tidxs} \tconfig'$
when there is a move from $\tconfig$ to $\tconfig'$ using a transition
$\ttran{\tstate}{\cha}{\tidx}{\chb}{\tdir}{\tstate'}$
with $\tidx \in \tidxs$.
Similarly, we define runs
$\tconfig \trunidx{\tidxs} \tconfig'$.
Finally, we define
\begin{multline*}
    \tlangofidx{\ducer}
               {\tstate}
               {\tstate'}
               {\tidxs}
               {\iword_1, \ldots, \iword_\numtapes}
    = \\
    \setcomp{\oword}{\begin{array}{c}
        \tup{
            \tstate,
            \iword_1, \tapepos_1,
            \ldots,
            \iword_\numtapes, \tapepos_\numtapes,
            \varepsilon
        }
        \trunidx{\tidxs}
        \tup{
            \tstate',
            \iword_1, \tapepos_1,
            \ldots,
            \iword_\numtapes, \tapepos_\numtapes,
            \oword
        }
        \\ \land \\
        \forall \tidx \in \tidxs\ .\ %
            \tapepos_\tidx = 1
    \end{array}} \ .
\end{multline*}
Note, in the above, we do not care about the head positions of tapes that are not read by any transitions.
Heads on tapes that may be used must start and finish at the initial position of the tape.


\begin{remark}
    An accepting run must return all tape heads to the initial position.
    This can easily be achieved using $\eword$-transitions.
    I.e.~transitions that add $\eword$ to $\oword$.
\end{remark}

\begin{remark}
    Each transition can only inspect the tape contents of a single tape.
    An alternative definition may allow transitions to depend on the character under each tape head.
    We can encode this by tracking the characters under the tape heads in the control state, at the cost of an exponential blow-up.
    However, to subsume existing results i don't think we need it.
    If we do, i suspect all proofs can easily be adapted.
\end{remark}

\begin{remark}
    If one wishes, one can keep in the control state a tape index $\tidx$, allowing the transducer to explicitly transfer control from one tape to another.
    We do not insist on this explicit index for the sake of simplicity of proofs.
\end{remark}


\subsection{Nesting Restriction}

Intuitively, the nesting restriction is a semantic restriction on runs of the transducer.
A move manipulating track $\tidx$ can only be applied if the tape heads on all tapes with index $\tidx' > \tidx$ are at the first position of their input words.

More formally, a nested $\numtapes$-tape transducer has a move
\[
    \tup{
        \tstate,
        \iword_1, \tapepos_1,
        \ldots,
        \iword_\numtapes, \tapepos_\numtapes,
        \oword
    }
    \tmove
    \tup{
        \tstate',
        \iword_1, \tapepos'_1,
        \ldots,
        \iword_\numtapes, \tapepos'_\numtapes,
        \oword'
    }
\]
via a transition
$\ttran{\tstate}{\cha}{\tidx}{\chb}{\tdir}{\tstate'}$
only if for all
$\tidx < \tidx' \leq \numtapes$
we have
$\tapepos_{\tidx'} = 1$.
All other definitions carry through as in the non-nested case above.

