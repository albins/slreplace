
\section{Recognisability}

In this section we prove the following theorem.

\begin{theorem}[Recognisable Pre-Image]
    Given a nested $\numtapes$-tape transducer
    \[
        \ducer = \tup{\tstates, \alphabet, \tdelta, \tfinals},
    \]
    some
    $\tstate \in \tstates$,
    and a regular language $\lang$, the relation
    \[
        \preimage{\ducer}{\tstate}{\lang} =
        \setcomp{
            \tup{\iword_1, \ldots, \iword_\numtapes}
        }{
            \exists \oword\ .\ %
                \oword \in
                    \tlangof{\ducer}
                            {\tstate}
                            {\iword_1, \ldots, \iword_\numtapes}
                    \cap
                    \lang
        }
    \]
    is recognisable and computable.
\end{theorem}

The proof is by induction over the number of tapes in the transducer.
In the base case we have a $1$-tape transducer and it is known that
$\preimage{\ducer}{\tstate}{\lang}$
is a regular language~\cite{??}.

Now suppose we have a $(\numtapes+1)$-tape transducer $\ducer$.
The idea is to summarise the $(\numtapes+1)$th tape as a set of tuples
\[
    \tup{\tstate, \astate, \tstate', \astate'}
    \in
    \tstates \times \astates \times \tstates \times \astates
\]
indicating that reading the $(\numtapes+1)$th input tape can take the transducer from $\tstate$ to $\tstate'$ and produce an output word that takes $\aut$ from $\astate$ to $\astate'$.
That is, such a tuple means there is a run
\[
    \tup{
        \tstate,
        \iword_1, \tapepos_1,
        \ldots,
        \iword_\numtapes, \tapepos_\numtapes,
        \iword_{\numtapes+1}, 1,
        \varepsilon
    }
    \trunidx{\set{\numtapes+1}}
    \tup{
        \tstate',
        \iword_1, \tapepos_1,
        \ldots,
        \iword_\numtapes, \tapepos_\numtapes,
        \iword_{\numtapes+1}, 1,
        \oword
    }
\]
with
$\astate \arun{\oword} \astate'$
for some $\oword$.
In fact, summaries will be sets of such tuples.
Let
$\summaries = 2^{\tstates \times \astates \times \tstates \times \astates}$.
We will show
\[
    \preimage{\ducer}{\tstate}{\lang} =
        \bigcup\limits_{\summary \in \summaries}
            \rel_\summary \ .
\]
where each $\rel_\summary$ is recognisable.
The finite union of a finite set of recognisable languages can easily be seen to be recognisable also.

First, we define a version of the pre-image where we can define exact initial and finite states, as well as which tapes are used.
\[
    \preimageidx{\ducer}{\tstate}{\tstate'}{\tidxs}{
        \alangofstates{\aut}{\astate}{\astate'}
    } =
    \setcomp{
        \tup{\iword_1, \ldots, \iword_\numtapes}
    }{
        \exists \oword\ .\ %
            \oword \in
                \tlangofidx{\ducer}
                           {\tstate}
                           {\tstate'}
                           {\tidxs}
                           {\iword_1, \ldots, \iword_\numtapes}
                \cap
                \alangofstates{\aut}{\astate}{\astate'}
    }
\]
Recall, in this definition, all tapes with indices in $\tidxs$ must start at their initial position, while we are agnostic about the position of heads on the other tapes (they cannot affect the computation).

For a summary $\summary$ we compute the set of values of
$\iword_{\numtapes+1}$
over which the transducer can move from $\tstate$ to $\tstate'$ while producing output which takes $\aut$ from $\astate$ to $\astate'$.
In particular, we need the regular language
\[
    \lang_\summary =
        \bigcap\limits_{
            \tup{\tstate, \astate, \tstate', \astate'}
            \in
            \summary
        }
        \ap{\proj{\numtapes+1}}{
            \preimageidx{\ducer}{\tstate}{\tstate'}{\set{\numtapes+1}}{
                \alangofstates{\aut}{\astate}{\astate'}
            }
        } \ .
\]
Note that such a language is computable because the transducer can only use the
$(\numtapes+1)$th
input tape during the run, making it effectively single-tape.
The projection to the
$(\numtapes+1)$th
component simply removes all values for the other elements, which have no effect on the computation and can thus be anything.
Hence, since we can compute the pre-image of a single-tape transducer given a particular regular language, we can compute $\lang_\summary$ above.

Next, we construct a $\numtapes$-tape transducer where the
$(\numtapes+1)$th
tape is replaced by $\summary$.
The transducer simply outputs a character
$\tup{\astate, \astate'}$
instead of beginning (and completing) a computation of the
$(\numtapes+1)$th
tape that would have moved the automaton from state $\astate$ to $\astate'$.
The automaton $\aut$ will be adjusted to interpret these characters correctly.
Formally, the transducer is
\[
    \ducer_\summary =
        \tup{\tstates, \alphabet_\summary, \tdelta_\summary, \tfinals}
\]
where
$\alphabet_\summary = \alphabet \uplus \brac{\astates \times \astates}$
and
\[
    \begin{array}{rcl}
        \tdelta_\summary
        &=&
        \tdelta \setminus
            \setcomp{
                \ttran{\tstate}{\cha}{\tidx}{\chb}{\tdir}{\tstate'}
                \in
                \tdelta
            }{\tidx = \numtapes + 1}
        \ \cup\ \\
        & &
        \setcomp{
            \ttran{\tstate}{\cha}{1}{\tup{\astate, \astate'}}{\tnoop}{\tstate'}
        }{
            \begin{array}{c}
                \tup{\tstate, \astate, \tstate', \astate'} \in \summary
                \ \land\ \\
                \cha \in \alphabet
            \end{array}
        }
    \end{array}
\]
The new automaton is
\[
    \aut_\summary =
        \tup{
            \astates,
            \alphabet_\summary,
            \adelta_\summary,
            \astateinit,
            \afinals
        }
\]
where $\alphabet_\summary$ is as above and
\[
    \begin{array}{rcl}
        \adelta_\summary
        &=&
        \adelta \cup
            \setcomp{\atran{\astate}{\tup{\astate, \astate'}}{\astate'}}
                    {\tup{\astate, \astate'} \in \astates \times \astates} \ .
    \end{array}
\]

By induction over the number of tapes, we know that
$\preimage{\ducer_\summary}{\tstate}{\alangof{\aut_\summary}}$
is recognisable and computable.
Let this relation be $\rel$.
From this, we construct
\[
    \rel_\summary = \rel \times \lang_\summary \ .
\]

We have thus far shown that
\[
    \bigcup\limits_{\summary \in \summaries}
        \rel_\summary
\]
is recognisable.
It remains to show that it is equal to
$\preimage{\ducer}{\tstate}{\lang}$.

\begin{remark}
    This is where i cop out.
    A proper proof requires some run surgery.
    However, i hope this so far is reminiscent enough of our POPL proof to be convincing.
\end{remark}
