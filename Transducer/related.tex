
\vspace{-2mm}

\section{Related Work and Future Work}
\label{sec:related}
\vspace{-2mm}


We conclude with a discussion of related work and future work, focussing on
(1) decidability, and (2) heuristics and implementation.

\smallskip
\noindent
\textbf{Decidability.}
\emph{Length constraints} --- i.e. an assertion 
$\varphi(\Len(x_1),\ldots,\Len(x_n))$, where $\varphi$ is a Presburger formula
and $\Len(x_i)$ is an integer variable interpreted as the length of the string
$x_i$ --- have been studied in the context of string solving. It is
a major open problem whether the theory of concatenation with length
constraints is decidable \cite{Vijay-length}. Several extensions of this 
theory is undecidable (e.g. with letter counting \cite{buchi} and
string-number conversion \cite{GB16}).
Several decidable restrictions, however,
have been proposed including solved form \cite{Vijay-length} and acyclicity 
\cite{Abdulla14}, both of which (like straight-line constraint) impose 
syntactic restrictions on the way in which string equality can be
used in the constraints. It was shown in \cite{LB16} the decidability of path 
feasibility for symbolic
executions allowing \FT{} and concatenation in the 
assignments, and regular constraints, and length constraints 
in the assertions. If we allow the functions
$\replaceall_p(sub,rep)$ in the assignments (instead of
\FT{}/concatenation) and length constraints as assertions, path
feasibility becomes undecidable \cite{CCHLW18}. This also implies undecidability
of allowing length constraints in our constraint language with parametric
transducers. Fortunately, decidability can be easily recovered in some
cases. One such case is when the length constraints $\varphi$ has only
one string variable $x_1$, e.g., $\Len(x_1) > 7$. In this case,
$\varphi(\Len(x_1))$ can be turned into a regular constraint $x_1 \in L$ for
some $L$. [This is because the set of integer solutions is effectively
a finite union of arithmetic progressions $\bigcup_{i=1}^n (a_i + b_i\N)$
(where $a_i + b_i\N := \{ a_i + b_in : n \in \N\}$), and each 
$\Len(x_i) \in (a_i + b_i\N)$ is equivalent to the regular constraint 
$x_i \in \ialphabet^{a_i} (\ialphabet^{b_i})^*$.] 
\OMIT{
The second case is when
the length constraints only relate input variables. In this case, running
our algorithm on the constraint (with length constraints ignored)
results in a conjunction of regular constraints on the input variables. Each
regular constraint $x \in L$ on an input variable $x$ yields a length
constraint $L_x$ on $x$ of the form a finite union of arithmetic progressions 
\cite{Kozen-automata}. We collect all the length constraints in a big
conjunction and run a solver for integer linear arithmetic.
%\cite{Vijay-length}, acyclicity \cite{Abdulla14,BFL13}, and straight-line
%\cite{LB16}) 
%on the shape of string constraints. 
}

The complexity of the theory of concatenation with regular constraints is
known to be \textsc{pspace}-complete \cite{Plandowski,J17}.
%, although the exact
%complexity for the theory of concatenation \emph{without} regular constraints is
%a long-standing open problem (only known to be in between NP and PSPACE). 
The complexity of the straight-line logic with concatenation, finite
transducers, and regular constraints is  \textsc{expspace}-complete \cite{LB16}. The same 
complexity holds when we swap finite transducers with replaceall 
\cite{CCHLW18}. Our logic %with sweeping parameterised transducers
%as primitive operations 
strictly subsumes these two logics (e.g. since it can
also express string reverse) and has precisely the same complexity  \textsc{expspace}.
%The straight-line logic of \cite{LB16} extended with length constraints is still
%EXPSPACE-complete. 
One future research avenue is to identify \emph{larger subclasses} of
constraints with parametric transducers that are still solvable in the same 
complexity class.
%
%, for which we 
%only have a nonelementary upper bound (and EXPSPACE lower bound).

In this paper, we have combined two powerful formalisms 
(two-way finite transducers and replaceall) into a single formalism. Since we 
are considering
only two-way transducers that define functions, they are equivalent to 
two-way deterministic finite transducers, streaming transducers, and
MSO-definable transductions 
\cite{EH01,AC10,AD11}. On the other hand, one-way transducers that define 
functions are strictly more expressive than deterministic one-way transducers
\cite{Berstel}. 
\OMIT{
In either case, it makes sense to consider nondeterministic
transducers for a succinctness reason. %For future work, it 
}
\OMIT{
It is also interesting to note that checking whether a \FFT{} defines a function
is decidable \cite{CK87} (in fact, in polynomial-time for \FT{} \cite{GI83}),
which implies decidability as well for \PPT{}.
}
%to consider 
%In this case, path feasibility of a
%symbolic execution $S$
%is defined in an \emph{angelic} way: each function in $S$
%should be able to produce \emph{some} output string that takes $S$ to the end of the
%program.
\OMIT{
Note that the generic decision procedure in the proof of
Theorem \ref{th:gen} still works when we admit relations, and the regularity
conditions for \PPT{} and \PT{} are still satisfied whether or not they define
functions. Therefore, our generic decision procedures would extend to ,
albeit with a worse complexity. 
}

\OMIT{
Symbolic automata were initially introduced in \cite{Watson96}, then
investigated in \cite{NG01}, with motivation from natural language processing.
The recent development of this topic was mainly driven by modern regular
expression analysis and advanced web security analysis \cite{Vea13}, which
require large alphabets (e.g. UTF16). Symbolic transducers were introduced in 
\cite{symbolic-transducer}. See \cite{symbolic-transducer-power} for an 
excellent survey. % on symbolic automata/transducers. 
}
%Prior to this work, symbolic 
%automata/transducers was never studied in the context of string solving.
%;
%the closest work related to this is the paper \cite{HJLRV18}, where the
%constraint language admits bitvectors in the alphabet.

%Symbolic automata/transducers were introduced by Veanes \emph{et al.}
%\cite{symbolic-transducer,symbolic-transducer-power}. \anthony{Say more}

%array theory? They can't concatenate and we can't do $a[i] = b[i]$.


\smallskip
\noindent
\textbf{Heuristics and implementation.}
Theoretical algorithms
(e.g. Makanin's algorithm \cite{Makanin}) typically do not directly lead to 
practical solvers. At the same time, the classes of constraints that are 
required in practice sometimes require string operations that are not covered by
decidable string constraint languages.
 %(e.g.~length constraints, replaceall, 
%transducers, etc.). 
For these reasons,
there is a large amount of work on heuristics for developing practical
(often incomplete) string solvers, e.g., 
\cite{BTV09,Berkeley-JavaScript,HAMPI,Stranger,YABI14,Abdulla14,fang-yu-circuits,Abdulla17,HJLRV18,S3,TCJ16,Z3-str,Z3-str2,cvc4,Saner,RVG12}.
%Some incomplete heuristics were developed that could work in practice,
Some practical heuristics include
bounding string lengths (e.g. \cite{HAMPI,Berkeley-JavaScript,BTV09}), 
induction, overapproximations \cite{Stranger,YABI14}, interpolation
\cite{Abdulla14}, and flat automata \cite{Abdulla17}, to name a few. 
Focusing on semi-algorithms also allows a highly expressive (but undecidable) 
string 
constraint languages, e.g., recursively defined functions \cite{S3,TCJ16}. 
Recently, the study of decidability of string constraints have also resulted in
automata-theoretic algorithms that are amenable to implementation, e.g.,
acyclic logic with concatenation, regular constraints, and length constraints 
\cite{Abdulla14}, and straight-line logic with finite transducers (or
replaceall), concatenation, and regular constraints 
\cite{HJLRV18,fang-yu-circuits,yan-tool}. 
%None of these techniques, however,
%would immediately lead to a fast string solver for string constraints with 
%parametric transducers as primitive operations.
 %(e.g. global representations
%of solutions employed in \cite{HJLRV18} will not work for parametric
%transducers). 
We leave the development of practical heuristics for our more expressive 
constraint language for future work.

\OMIT{
The development of theory and implementation of string solvers of course
has to go hand in hand with the development of string constraint benchmarks.
}
We mention some interesting benchmarks that are available for 
string constraints. The first one is Kaluza 
benchmarks
\cite{Berkeley-JavaScript}, which contain
string constraints with concatenation, regular constraints, and length
constraints. It was shown in \cite{Vijay-length}
that almost all the constraints are already in
\emph{solved form} (in particular, also straight-line). Some of these length
constraints are also expressible as regular constraints.
The second is SLOG benchmarks \cite{fang-yu-circuits}, which contain
string constraints with concatenation, (a restricted class of) replaceall, and 
regular constraints. The third is SLOTH benchmarks \cite{HJLRV18}, which
contains string constraints with concatenation, finite transducers, and
regular constraints. Most of these benchmarking examples are expressible in our
decidable constraint language.
\OMIT{
In this paper, we have described an application of
string constraints for modelling string functions required for context-sensitive
auto-sanitisation web templating \cite{SSS11}. An immediate future research 
direction is to develop string constraint benchmarks based on this application.
}

%Some of these techniques utilise the power of
%alternating automata (over bitvectors) 

