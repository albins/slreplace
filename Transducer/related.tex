\section{Related Work}
\label{sec:related}

\subsection{Theory}
\emph{Length constraints} --- i.e. an assertion of the form 
$\varphi(\Len(x_1),\ldots,\Len(x_n))$, where $\varphi$ is a Presburger formula
and $\Len(x_i)$ is an integer variable interpreted as the length of the string
$x_i$ --- have also been studied in the context of string solving. It is
currently a major open problem whether the theory of concatenation with length
constraints is decidable \cite{Vijay-length}. Several extensions of this 
theory is known to be undecidable (e.g. with letter counting \cite{buchi} and
string-number conversion \cite{GB16}).
Several decidable restrictions, however,
have been proposed including solved form \cite{Vijay-length} and acyclicity 
\cite{Abdulla14}, both of which (like straight-line constraints) impose 
syntactic restrictions on the way in which string equality/disequality can be
used in the constraints. It was shown in \cite{LB16} the decidability of path 
feasibility for symbolic
executions allowing finite-state transduction and concatenation in the 
assignments, and regular constraints, length constraints, and string disequality
in the assertions. Unfortunately, if we allow the functions
$\replaceall_p(sub,rep)$ in the assignments (instead of
finite transducer/concatenation) and length constraints as assertions, path
feasibility becomes undecidable \cite{CCHLW18}. This also implies undecidability
of allowing length constraints in our constraint language with parametric
transducers. Fortunately, decidability can be easily recovered in at least two
cases. The first case is when the length constraints $\varphi$ has only
one string variable $x_1$, e.g., $\Len(x_1) > 7$. In this case,
$\varphi(\Len(x_1))$ can be turned into a regular constraint $x_1 \in L$ for
some $L$. [This is because the set of integer solutions is effectively
a finite union of arithmetic progressions $\bigcup_{i=1}^n a_i + b_i\N$
(where $a_i + b_i\N := \{ a_i + b_in : n \in \N\}$), and each 
$\Len(x_i) \in (a_i + b_i\N)$ is equivalent to the regular constraint 
$x_i \in \ialphabet^{a_i} (\ialphabet^{b_i})^*$.] The second case is when
the length constraints only relate input variables. In this case, running
our algorithm on the constraint (with length constraints ignored)
results in a conjunction of regular constraints on the input variables. Each
regular constraint $x \in L$ on an input variable $x$ yields a length
constraint $L_x$ on $x$ of the form a finite union of arithmetic progressions 
\cite{Kozen-automata}. We collect all the length constraints in a big
conjunction and run a solver for integer linear arithmetic.
%\cite{Vijay-length}, acyclicity \cite{Abdulla14,BFL13}, and straight-line
%\cite{LB16}) 
%on the shape of string constraints. 

casting functions (e.g. conversions
between integers and strings) \cite{GB16}.

The complexity of the theory of concatenation with regular constraints is
known to be PSPACE-complete \cite{Plandowski,J17}, although the exact
complexity for the theory of concatenation without regular constraints is
still open (only known to be in between NP and PSPACE). 

\subsection{Practical}
Recursively defined functions.


