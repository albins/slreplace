%!TEX root = popl2018.tex

\section{Introduction}
\label{sec:intro}

The problem of %satisfiability of logical theories over strings
automatically solving string constraints (aka satisfiability of logical theories over
strings) has recently witnessed renewed interests %in the programming language and verification community
\cite{Berkeley-JavaScript,TCJ16,LB16,YABI14,S3,Abdulla14,Abdulla17,DV13,symbolic-transducer,BEK,HAMPI,cvc4,Z3-str,fang-yu-circuits,BTV09} 
because of important applications in the analysis of string-manipulating  programs. For example,
program analysis techniques like symbolic execution
\cite{king76,DART,EXE,jalangi} 
would
%. For example, program analysis techniques like
%symbolic execution \cite{king76} (or variants like dynamic symbolic execution
%\cite{DART,EXE}) will 
systematically explore executions in a program and collect symbolic path 
constraints, which could then be solved using a constraint solver and
used to determine which location in the program to continue exploring.
To successfully apply a constraint solver in this instance, it is
crucial that the constraint language precisely models the data types in the
program, along with the data-type operations used. In the context of
string-manipulating programs, this could include 
concatenation, regular constraints (i.e. pattern matching against a regular
expression), string-length functions, and the string-replace functions, among 
many others.

%For scripting
%languages (e.g. Python, PHP, and JavaScript), which have grown in popularity
%in recent years, numerous string 
%There are several well-established research threads on this satisfiability 
%problem depending on 
%which string operations are permitted in the theory. 
Perhaps the most well-known theory of strings for such applications as the
analysis of string-manipulating programs is the theory of strings with concatenation  (aka \emph{word equations}), whose decidability was shown by Makanin \cite{Makanin} in 1977 after it was open for many years. More importantly, 
this theory remains decidable even when regular constraints are incorporated into the 
language \cite{Schulz}. However, whether adding the string-length function
 preserves the decidability remains a long-standing open problem
\cite{Vijay-length,buchi}. 
%[See Section~\ref{sec-rel} for more recent results about solving word equations.]

%\OMIT{
%which is crucial for 
Another important string operation---especially in popular scripting
languages like Python, JavaScript, and PHP---is the \emph{string-replace function}, 
which may be used to replace either the \emph{first} occurrence or
\emph{all} occurrences of a string (a string constant/variable, or a regular expression) by 
another string (a string constant/variable). The replace function (especially 
the replace-all functionality) is omnipresent in HTML5 applications
\cite{LB16,TCJ16,YABI14}. 
%\mat{What does it mean for the replace function to be convincingly argued?}
For example, a standard industry defense against cross-site scripting 
(XSS) vulnerabilities includes sanitising untrusted strings before adding them
into the DOM (Document Object Model) or the HTML document. 
This is typically done by %replacing every occurrence of
various metacharacter-escaping mechanisms (see, for instance, 
\cite{Kern14,BEK,OWASP-XSS}). An example of such a mechanism is backslash-escape, which replaces \emph{every
occurrence} of quotes and double-quotes (i.e. \verb+'+ and \verb+"+) in the
string by \verb+\'+ and \verb+\"+. 
In addition to sanitisers, common JavaScript functionalities like \texttt{document.write()} 
and \texttt{innerHTML} apply an \emph{implicit browser transduction} --- which
decodes HTML codes (e.g. \verb+&#39;+ is replaced by \verb+'+) in the input 
string --- before inserting the input string into the DOM.
Both of these examples can be expressed by (perhaps multiple) 
applications of the string-replace function.
Moreover, although these examples replace constants by constants, the popularity of template systems such as Mustache~\cite{Mustache} and Closure Templates~\cite{Closure} demonstrate the need for replacements involving variables.
Using Mustache, a web-developer, for example, may define an HTML fragment with placeholders that is instantiated with user data during the construction of the delivered page.

%\mat{Need to say more what template systems are?}
%\anthony{Better mention closure templates I think, Matt. They do both
%mustache-kind of replace plus sanitisers. They call these strict autoescaping
%templates. See \cite{SSS11}.}
%\mat{Mentioned closure, and changed footnote to reflect the fact that neither autoescaping mechanism kills our example.}

%Concatenation: 8\%
%Replace-all: 8\%


%\anthony{I think the following example should appear earlier no? Perhaps	after Matt mentioned about Mustache? We need to get	the reader to be convinced first that general replaceall is useful.}
%which we address in this article.
%\mat{Am i right to say we tackle it here?}

%% If you're looking for all the omitted and hidden semi-paragraphs that were here, checkout version 407adea8bf0c0564f064f43ae3b355fc8ac901d7. -- Matt


\begin{example}
We give a simple example demonstrating a (naive) XSS vulnerability to illustrate the use of string-replace functions.
Consider the HTML fragment below.
\begin{minted}{html}
   <h1> User <span onMouseOver="popupText('{{bio}}')">{{userName}}</span> </h1>
\end{minted}
This HTML fragment is a template as might be used with systems such as Mustache to display a user on a webpage.
For each user that is to be displayed -- with their username and biography stored in variables \emph{user} and \emph{bio} respectively -- the string \verb+{{userName}}+ will be replaced by \emph{user} and the string \verb+{{bio}}+ will be replaced by \emph{bio}.
For example, a user \verb+Amelia+ with biography \verb+Amelia was born in 1979...+ would result in the HTML below.
\begin{minted}{html}
   <h1> User 
        <span onMouseOver="popupText('Amelia was born in 1979...')">
            Amelia </span> </h1>
\end{minted}
This HTML would display \verb+User Amelia+, and, when the mouse is placed over \verb+Amelia+, her biography would appear, thanks to the \verb+onMouseOver+ attribute in the \verb+span+ element.

Unfortunately, this template could be insecure if the user biography is not adequately sanitised: 
A user could enter a malicious biography, such as \verb+'); alert('Boo!'); alert('+ which would cause the following instantiation of the \verb+span+ element\footnote{
	Readers familiar with Mustache and Closure Templates may expect single quotes to be automatically escaped.
	However, we have tested our example with the latest versions of mustache.js~\cite{MustacheJS} and Closure Templates~\cite{Closure} (as of July 2017) and observed that the exploit is not disarmed by their automatic escaping features.
}.
\begin{minted}{html}
    <span onMouseOver="popupText(''); alert('Boo!'); alert('')">
\end{minted}
Now, when the mouse is placed over the user name, the malicious JavaScript \verb+alert('Boo!')+ is executed.

The presence of such malicious injections of code can be detected using string constraint solving and XSS \emph{attack patterns} given as regular expressions~\cite{BCFJKKV08,Berkeley-JavaScript,YABI14}.
For our example, given an attack pattern $P$ and template $temp$, we would generate the constraint
\[
    x_1 = \replaceall(temp, \verb+{{userName}}+, \mathit{user})
    \land
    x_2 = \replaceall(x_1, \verb+{{bio}}+, \mathit{bio})
    \land
    x_2 \in P
\]
which would detect if the HTML generated by instantiating the template is
    susceptible to the attack identified by $P$. \qed
\end{example}
%\mat{ANTHONY: can you verify/expand on my use of attack patterns?  I just copied it wholesale and blindly from your POPL 2016.}

In general, the string-replace function has three parameters, and in the current mainstream language such as Python and JavaScript, \emph{all of the three parameters can be inserted as string variables}. As result, when we perform program analysis for, for instance, detecting security vulnerabilities as described above, one often obtains string constraints of the form $z= \replaceall(x, p, y)$, where $x,y$ are string constants/variables, and $p$ is either a string constant/variable or a regular expression.
Such a constraint means that $z$ is obtained by replacing all occurrences of $p$
in $x$ with $y$. For convenience, we call $x, p, y$ as the \emph{subject}, the
\emph{pattern}, and the \emph{replacement} parameters respectively. 

%for instance $z=\replaceall(x,"aa", y)$, meaning that $z$ is obtained by replacing all occurrences of $aa$ in $x$ by $y$. Solving string constraints involving this type of constraints is crucial, but unfortunately, is not supported by the current technique. There are two reasons:

%The string-replace function is formalised as $\replaceall(x, p, y)$, where $x,y$ are string constants/variables, and $p$ is either a string constant/variable, or a regular expression. The semantics of $\replaceall(x, p, y)$ is to replace all occurrences of $p$ in $x$ with $y$. For convenience, we call $x, p, y$ as the subject, the pattern, and the replacement parameters respectively. 

%The string-replace function is a quite powerful string operation. To see this, let us consider the constraint $C \equiv z = \replaceall(x, 0, y) \wedge x \in (01)^* \wedge y \in 0^*$. Then the set of values of $z$ that make $C$ satisfied is $\{(0^n 1)^* \mid n \in \Nat \}$.


The $\replaceall$ function is a powerful string operation that goes beyond the 
expressiveness of concatenation. (On the contrary, as we will see later, concatenation can be expressed by the $\replaceall$ function easily.) 
It was shown in a recent POPL paper \cite{LB16} that any theory of strings containing the 
string-replace function (even the most restricted version where 
pattern/replacement strings are both constant strings) becomes undecidable if 
we do not impose some kind of
\emph{straight-line restriction}\footnote{Similar notions that appear in the 
literature of string constraints (without replace) include acyclicity 
\cite{Abdulla14} and solved form \cite{Vijay-length}} on 
the formulas. Nonetheless, as already noted in \cite{LB16},
the straight-line restriction is reasonable since it
is typically satisfied by constraints that are generated by symbolic 
execution, e.g., all constraints in the standard Kaluza benchmarks
\cite{Berkeley-JavaScript} with 50,000+ test cases
generated by symbolic execution on JavaScript applications were
noted in \cite{Vijay-length} to satisfy this condition. 
%This was noted already
%in \cite{LB16}, which provides other interesting examples of programs whose 
%program logic can be modelled by the straight-line fragment.
Intuitively, as elegantly described in \cite{BTV09}, constraints from symbolic 
execution on string-manipulating programs can be viewed as the problem of path 
feasibility over loopless string-manipulating programs $S$ with variable 
assignments and assertions, i.e., generated by the grammar
\begin{equation*}
    S ::= y := f(x_1,\ldots,x_n) \ |\ \text{\ASSERT{$g(x_1,\ldots,x_n)$}}\ |\ 
            S_1; S_2\ 
    %a ::= f(x_1,\ldots,x_n), \qquad b ::= g(x_1,\ldots,x_n) 
\end{equation*}
where $f: (\Sigma^*)^n \to \Sigma^*$ and $g: (\Sigma^*)^n \to \{0,1\}$ are
some string functions.
Straight-line programs with assertions can be obtained by turning such programs 
into a Static Single Assignment (SSA) form (i.e. introduce a new variable 
on the left hand side of each assignment).
%To see why the straight-line fragment can naturally model constraints from
%symbolic execution, we may view 
%In addition, the
%straight-line fragment naturally corresponds to the path feasibility problem for
%string-manipulating programs \cite{BTV09}. 
%arguments for straight-line restriction can be found in \cite{LB16}.
\OMIT{
As a matter of fact, it was shown in~\cite{LB16} that any 
incorporating a simple form of the $\replaceall$ function, where the pattern and the replacement are both string constants, into the theory of concatenations already results in an undecidable theory of strings.  
%that lies beyond 
%the expressiveness of the theory of concatenation, 
Therefore, it is challenging to reason about the string-replace function in its general form. 
}
%In~\cite{LB16}, a theory of strings involving concatenation and finite state transducers was considered. The $\replaceall$ function where the pattern is a string constant or regular expression, the third parameter is a constant can be seen as a special case of finite state transducers. It was shown therein that incorporating the simple form of the $\replaceall$ function into the theory of concatenations already results in an undecidable theory of strings. 
A partial decidability result 
can be deduced from \cite{LB16}
%from the following recent POPL paper in~\cite{LB16} provides decidability
for the straight-line fragment of the theory of strings, where (1) $f$ in the 
above
grammar is either a concatenation of string constants and variables, or 
the $\replaceall$ function where \emph{the 
pattern and the replacement are both string constants}, and (2) $g$ is
a boolean combination of regular constraints.
%concatenation, regular constraints, and 
%the $\replaceall$ function where \emph{the 
%pattern and the replacement are both string constants}. 
In fact, the decision procedure therein admits finite-state transducers, which
subsume only the aforementioned simple form of the $\replaceall$ function.
%, but
%\emph{not} its most general form.
%The straight-line restriction is natural in the sense that it reflects the shape of formulas typically generated by symbolic execution.  
The decidability boundary of the straight-line fragment involving the 
$\replaceall$ function in its general form (e.g., 
when the replacement parameter is a variable) remains open.


\paragraph{Contribution.} We investigate the decidability boundary of the theory
$\strline[\replaceall]$ of strings involving %concatenation, 
%regular constraints, and 
the $\replaceall$ function and regular constraints, with the straight-line
restriction introduced in \cite{LB16}. We provide a decidability result for a 
large fragment of $\strline[\replaceall]$, which is sufficiently powerful to
express the concatenation operator. We show that this decidability result is in a sense maximal
by showing that several important natural extensions of the logic result in undecidability.
We detail these results below:

\OMIT{
We first show that---as mentioned earlier---with the $\replaceall$ function in its general form, the concatenation operation is in fact \emph{redundant}, in the sense that it can be simulated by the $\replaceall$ function. This motivates us to consider a theory of strings where the $\replaceall$ function, instead of the concatenation operation, and the regular constraints, are the basic modalities. We focus on the straight-line fragment of this theory, denoted by $\strline[\replaceall]$.
}

%More precisely,
%we obtain the following results for the satisfiability problem of
%$\strline[\replaceall]$:
\begin{itemize}
\item If the pattern parameters of the $\replaceall$ function are allowed to be variables, then the satisfiability of $\strline[\replaceall]$ is undecidable (cf. Proposition~\ref{prop-und-pat-var}).
%
\item If the pattern parameters of the $\replaceall$ function are regular
    expressions, then the satisfiability of $\strline[\replaceall]$ is decidable
        and in EXPSPACE (cf. Theorem~\ref{thm-main}). In addition, we show that
        the satisfiability problem is PSPACE-complete for several cases that are
        meaningful in practice (cf. Corollary~\ref{cor-pspace}). This strictly
        generalises the decidability result in \cite{LB16} of the straight-line 
        fragment with concatenation, regular constraints, and the $\replaceall$ 
        function where
        patterns/replacement parameters are constant strings.
%
\item If $\strline[\replaceall]$, where the pattern parameter of the
    $\replaceall$ function is a constant letter, is extended with the 
        string-length constraint, then satisfiability becomes undecidable
        again. In fact, this undecidability can be obtained with
        either integer constraints, character constraints, or constraints 
        involving the $\indexof$ function
        %, then the satisfiability becomes undecidable again 
        (cf. Theorem~\ref{thm-ext-int} and 
        Proposition~\ref{prop-ext-ch-index}).
\end{itemize}

Our decision procedure for $\strline[\replaceall]$ where the pattern parameters
of the $\replaceall$ function are regular expressions follows an 
automata-theoretic approach. The key idea can be illustrated as follows. Let 
us consider the simple formula $C \equiv x = \replaceall(y, a, z) \wedge x \in e_1 \wedge y \in e_2 \wedge z \in e_3$. 
Suppose that $\cA_1,\cA_2,\cA_3$ are the nondeterministic finite state automata corresponding to $e_1,e_2,e_3$ respectively. 
We effectively eliminate the use of $\replaceall$ by nondeterministically
generating from $\cA_1$ a new regular constraint $\cA'_2$ for $y$ as well as a new regular constraint
$\cA'_3$ for $z$.   These constraints incorporate the effect of the
$\replaceall$ function (i.e.\ all regular constraints are on the ``source'' variables).
Then, the satisfiability of $C$ is turned into testing the nonemptiness of the intersection of $\cA_2$ and $\cA'_2$, as well as the nonemptiness of the intersection of $\cA_3$ and $\cA'_3$. When there are multiple occurrences of the $\replaceall$ function, this process can be iterated. 
Our decision procedure enjoys the following advantages:
\begin{itemize}
	\item It is automata-theoretic and built on clean automaton constructions, 
        moreover, when the formula is satisfiable, a solution can be 
        synthesised. For example, in the aforementioned XSS vulnerability detection example, one can synthesise the values of the variables $user$ and $bio$ for a potential attack. 
	\item The decision procedure is modular in
        that the $\replaceall$ terms are removed one by one to
        generate more and more regular constraints (emptiness of the
        intersection of regular constraints could be efficiently handled by
        state-of-the-art solvers like \cite{fang-yu-circuits}).  
    \item The decision procedure requires exponential space (thus double
        exponential time), but under assumptions that are reasonable in practice, 
        the decision procedure uses only polynomial space, which is not
        worse than other string logics (which can encode the PSPACE-complete
        problem of checking emptiness of the intersection of regular 
        constraints). 
\end{itemize}
%Finally, a corollary our result is a simpler decision procedure for
%the straight-line fragment of the theory of strings with concatenation,


%EXPSPACE algorithm
%
%a table to summarise the results.


\paragraph{Organisation.} 
This paper is organised as follows: Preliminaries are given in
Section~\ref{sec-prel}. The core string language is defined in
Section~\ref{sec-core}. The main results of this paper are summarised in
Section~\ref{sec-sat}. The decision procedure is presented in
Section~\ref{sec:replaceallsl}-\ref{sec:replaceallre}, case by case. The
extensions of the core string language are investigated in
Section~\ref{sec-ext}. The related work can be found in Section~\ref{sec-rel}.
The \shortlong{full version}{appendix} contains missing proofs and additional
examples.
