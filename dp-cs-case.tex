%!TEX root = popl2018.tex

\section{Decision procedure for $\strline[\replaceall]$: The constant-string case}

In this section, we consider the constant-string special case, that is, given an $\strline[\replaceall]$ formula $C = \varphi \wedge \psi$, it holds that every term of the form $\replaceall(z, e, z')$ in $\varphi$ satisfies that $e=u$ for $u \in \Sigma^+$.

Let us still start with the simple case that 
$$C \equiv x = \replaceall(y, u, z) \wedge x \in e_1 \wedge y \in e_2 \wedge z \in e_3,$$
where $|u| \ge 2$. For $i=1,2,3$, let $\cA_i = (Q_i, \delta_i, q_{0, i}, F_i)$ 
be the NFA corresponding to $e_i$. In addition, let $u = a_1 \cdots a_k$ with $a_i \in \Sigma$ for each $i \in [k]$.

From the semantics, $C$ is satisfiable iff $x, y, z$ can be assigned with  strings $v, w, w'$ such that: (1) $v = \replaceall(w, u, w')$, (2) $v,w, w'$ are accepted by $\cA_1, \cA_2, \cA_3$ respectively. Let $v, w, w'$ be the strings satisfying these two constraints. Since $v = \replaceall(w, u, w')$, we know that there are strings $w_1, w_2, \cdots, w_n$ such that $w= w_1 u w_2 \cdots u w_n$ and $v = w_1 w' w_2 \cdots w' w_n$. As $v$ is accepted by $\cA_1$, there is an accepting run of $\cA_1$ on $v$, say 
$$
q_{0,1} \xrightarrow[\cA_1]{w_1} q_1 \xrightarrow[\cA_1]{w'} q'_1 \xrightarrow[\cA_1]{w_2} q_2 \xrightarrow[\cA_1]{w'} q'_2 \cdots q_{n-1} \xrightarrow[\cA_1]{w'} q'_{n-1} \xrightarrow[\cA_1]{w_n} q_n.
$$
Let $T_z = \{(q_i, q'_i) \mid i \in [n]\}$. Then $w' \in \Ll(\cA_3) \cap\ \bigcap \limits_{(q,q') \in T_z} \Ll(\cA_1(q, q'))$. Therefore, $\Ll(\cA_3) \cap\ \bigcap \limits_{(q,q') \in T_z} \Ll(\cA_1(q, q')) \neq \emptyset$. Similarly to the single letter case, in the following, we will construct an NFA $\cB_{\cA_1, u, T_z}$ to characterise the satisfiability of $C$, more precisely, $C$ is satisfiable iff there is $T_z \subseteq Q_1 \times Q_1$ such that $\Ll(\cA_3) \cap\ \bigcap \limits_{(q,q') \in T_z} \Ll(\cA_1(q, q')) \neq \emptyset$ and 
$\Ll(\cA_2) \cap \Ll(\cB_{\cA_1, u, T_z}) \neq \emptyset$. Intuitively, when reading the string $w$, $\cB_{\cA_1, u, T_z}$ simulates the generation of $v$ from $w$ and $w'$, that is, simulates the replacement of  every occurrence of $u$ in $w$ with $w'$, and verifies that $v$ is accepted by $\cA_1$, by using $T_z$.
The construction of $\cB_{\cA_1, u, T_z}$ utilises the concepts of window profiles and parsing automata defined below.
%
%Intuitively, the window profiles encode the matching of the suffixes of a string of length $|u|$ w.r.t. the prefixes of $u$ and a parsing automaton for $u$ uses window profiles of $u$ to pinpoint the first, second, \dots, occurrence of $u$ in a string.


%Let us start with the simple case that $C \equiv x = \replaceall(y, u, z) \wedge \bigwedge \limits_{i \in [k]} x \in e_{i}$ such that $|u| \ge 2$.  In addition, for $i \in [k]$, suppose $\cA_i = (Q_i, \delta_i, q_{0, i}, F_i)$ 
%is the NFA corresponding to $e_i$. 
%\begin{enumerate}
%\item For each $i \in [k]$, we will guess a set $T_{i, z} \subseteq Q_i \times Q_i$, which intuitively means that in $\cA_i$, for each state pair $(q, q') \in T_{i, z}$, starting from $q$, after reading $z$, the state $q' $ can be reached. The functions $(T_{i, z})_{i \in [k]}$ induce regular constraints $\Ll(\cA_i(q,q'))$ on $z$, where $i \in [k]$ and $(q,q') \in T_{i, z}$.
%
%\item For each $i \in [k]$, we construct an NFA $\cB_{\cA_i, u,  T_{i, z}}$, which specifies some additional regular constraint on $y$. Intuitively, a string $v$ is accepted by $\cB_{\cA_i, u,  T_{i, z}}$ iff either $v \not \in \Sigma^\ast u \Sigma^\ast$ and $v$ is accepted by $\cA_i$, or otherwise, let $v = v'_1 u v'_2 u \dots v'_{k-1} u v'_{k}$ such that $v'_i u' \not \in \Sigma^\ast u \Sigma^\ast$ for each $i \in [k-1]$ and each strict prefix $u'$ of $u$ and $v'_k \not \in \Sigma^\ast u \Sigma^\ast$, then $q_{i,0} \xrightarrow{v'_1} q_1 \xrightarrow{T_{i,z}} q'_1 \xrightarrow{v'_2} q_2 \xrightarrow{T_{i,z}} q'_2 \dots \xrightarrow{v'_{k-1}} q_{k-1} \xrightarrow{T_{i,z}} q'_{k-1} \xrightarrow{v'_k} q_k$ for states $q_1, q'_1, \dots, q_{k-1}, q'_{k-1}, q_k \in Q_i$ with $q_k \in F_{i,z}$. 
%\end{enumerate}
%
%In order to construct the NFA $\cB_{\cA_i, u,  T_{i, z}}$, we introduce concepts of window profiles and parsing automata defined below.
%
%Let $u \in \Sigma^+$ and $k=|u| \ge 2$.


\begin{definition}[$k$-window profiles w.r.t. $u$]
A $k$-\emph{window profile $\overrightarrow{W}$ w.r.t. $u$} is an element of $\{\bot,\top\}^{k-1}$. Let $\wprof_{u, k}$ denote the set of $k$-window profiles w.r.t. $u$. 
\end{definition}

Intuitively, in a position $i$ of a string $v$, the $k$-window profile $\overrightarrow{W}$ of $i$ w.r.t. $u$ is an abstraction of the substring $v[i-k+2] \dots v[i]$ such that for each $j \in [k-1]$, $\overrightarrow{W}[j] = \top$ iff $v[i-j+1] \dots v[i] = u[1] \dots u[j]$. 

\begin{proposition}
$|\wprof_{u,k}|=k$.
\end{proposition}
\begin{proof}
The arguments for this fact proceed as follows: For each profile $\overrightarrow{W}$, let $v$ be a string and $i$ be a position of $v$ such that for each $j \in [k-1]$, $\overrightarrow{W}[j] = \top$ iff $v[i-j+1] \dots v[i] = u[1] \dots u[j]$. Define ${\sf idx}_{\overrightarrow{W}}$ as the maximum index $j \in [k-1]$ such that $\overrightarrow{W}[j]=\top$. Then 
\begin{itemize}
	\item for each $j': {\sf idx}_{\overrightarrow{W}} < j' < k$, $\overrightarrow{W}[j']=\bot$, 
	\item in addition, since $v[i-{\sf idx}_{\overrightarrow{W}}+1] \cdots v[i] = u[1] \cdots u[{\sf idx}_{\overrightarrow{W}}]$, the values of $\overrightarrow{W}[1],\cdots, \overrightarrow{W}[{\sf idx}_{\overrightarrow{W}}]$ are completely determined by $u[1] \cdots u[{\sf idx}_{\overrightarrow{W}}]$.
\end{itemize}
From the above arguments, we can  conclude that the number of $k$-window profile $\vec{W}$ w.r.t. $u$ is actually $k$.
\end{proof}

\begin{example}
Let $u = $.
\end{example}

From $u$, we will construct a parsing automaton $\cA_u$ which parses a string $v \in \Sigma^\ast u \Sigma^\ast$ into $v_1 u v_2 u \dots v_l u v_{l+1}$ such that $v_j u[1] \dots u[k-1] \not \in \Sigma^\ast u \Sigma^\ast$ for each $1 \le j \le l$, in addition, $v_{l+1} \not \in \Sigma^\ast u \Sigma^\ast$. 
The $k$-window profiles w.r.t. $u$ are used to check that a substring of $v$ is \emph{not} in $\Sigma^\ast u \Sigma^\ast$.

\begin{definition}[Parsing automata]
The \emph{parsing automaton} $\cA_u$ of $u$ is the NFA $(Q_u, \delta_u, q_{0,u}, F_u)$ defined as follows. 
\begin{itemize}
	\item  $Q_u =\left\{q_0 \right\} \cup \left\{ \left(\search, \overrightarrow{W} \right) \mid \overrightarrow{W} \in \wprof_{u, k} \right\} \cup  \left\{ \left(\verify, j, \overrightarrow{W} \right) \mid j \in [k-1], \overrightarrow{W} \in \wprof_{u,k} \right\}$, where $q_0$ is a distinguished state whose purpose will become clear later on,  the labels $\search$ and $\verify$ are used to denote whether $\cA_u$ is in the ``search'' mode to search for the next occurrence of $u$, or in the ``verify'' mode to verify that the current position is a part of an occurrence of $u$.
	%
	\item $q_{0,u}=q_0$.
	
	\item $\delta_{u}$ is defined as follows.
	%guesses over each position, one of the following holds, the substring comprising the next $k$-symbols (including the current one) is $u$ or not.
	\begin{itemize}
		\item The transition $\left(q_0, a, \left(\search, \overrightarrow{W}\right)\right) \in \delta_u$, where $\overrightarrow{W}[1]=\top$ iff $a = u[1]$, and for each $i: 2 \le i \le k-1$, $\overrightarrow{W}[i] = \bot$,
		%
		\item for each state $\left(\search, \overrightarrow{W} \right)$ and $a \in \Sigma$ such that $\overrightarrow{W}[k-1] = \bot$ or $a \neq u[k]$,
		\begin{itemize}
			\item the transition $\left(\left(\search, \overrightarrow{W} \right), a, \left(\search, \overrightarrow{W'} \right)\right) \in \delta_u$, where $\overrightarrow{W'}[1] = \top$ iff $a = u[1]$, and for each $i: 2 \le i \le k-1$, $\overrightarrow{W'}[i] =\top$ iff $\overrightarrow{W}[{i-1}] = \top$ and $a = u[i]$,
			%
			\item if $a = u[1]$, then the transition $\left(\left(\search, \overrightarrow{W} \right), a, \left(\verify, 1, \overrightarrow{W'} \right)\right) \in \delta_u$, where $\overrightarrow{W'}[1]=\top$,  and for each $i: 2 \le i \le k-1$, $\overrightarrow{W'}[i] =\top$ iff $\overrightarrow{W}[{i-1}] = \top$ and $a = u[i]$,
			%
		\end{itemize}
		%
		\item for each state $\left(\verify, i-1, \overrightarrow{W} \right)$ such that 
		\begin{itemize}
			\item $2 \le i \le k-1$,
			\item $\overrightarrow{W}[i-1]=\top$, $a = u[i]$, and
			\item either $\overrightarrow{W}[k-1]=\bot$ or $a \neq u[k]$, 
		\end{itemize}
		we have $\left(\left(\verify, i-1, \overrightarrow{W} \right), a, \left(\verify, i, \overrightarrow{W}' \right)\right) \in \delta_u$, where for each $j: 2 \le j \le k-1$, $\overrightarrow{W}'[j] = \top$ iff $\overrightarrow{W}[j-1]=\top$ and $a = u[j]$, 
		%
		\item for each state $\left(\verify, k-1, \overrightarrow{W} \right)$ such that $\overrightarrow{W}[k-1]=\top$, $\left(\left(\verify, k-1, \overrightarrow{W} \right), u[k], q_0\right) \in \delta_u$.
		%where $\bot^k$ in $(\search, \bot^k)$ is used to \emph{reinitialise} the $k$-window profile w.r.t. $u$.
		%
	\end{itemize}
Note that the constraint $\overrightarrow{W}[k-1] = \bot$ or $a \neq u[k]$ is used to guarantee that when parsing a string $v$ into $v_1 u v_2 u \dots v_{l} u v_{l+1}$, we have $v_j u[1] \dots u[k-1] \not \in \Sigma^\ast u \Sigma^\ast$ for each $j \in [l]$, in addition, $v_{l+1} \not \in  \Sigma^\ast u \Sigma^\ast$.
	%
	\item $F_u= \left\{q_0 \right\} \cup \left\{\left(\search, \overrightarrow{W} \right) \mid \overrightarrow{W} \in \wprof_{u, k} \right\} $. Note that the states $\left(\verify, j, \overrightarrow{W} \right)$ are not final states, since when in these states, the verification of the current occurrence of $u$ has not yet been complete.
\end{itemize}
\end{definition}

Let $Q_{\search}  = \left\{ \left(\search, \overrightarrow{W} \right) \mid \overrightarrow{W} \in \wprof_{u,k} \right\}$,  and $Q_{\verify, i} = \left\{ \left(\verify, i, \overrightarrow{W} \right) \mid \overrightarrow{W} \in \wprof_{u,k} \right\}$ for each $i \in [k-1]$. In addition, let $Q_{\verify} = \bigcup \limits_{i \in [k-1]} Q_{\verify,i}$.
In an accepting run $r$ of $\cA_u$ on a string $v = v_1 u v_2 u \dots v_l u v_{l+1}$, the state sequence in the run is of the form 
$$q_0\ r_1\ q_0\ r_2\ q_0\ \dots\ r_l\ q_0\ r_{l+1}$$ 
such that  for each $j \in [l]$, $r_j \in (Q_{\search})^+ Q_{\verify, 1}  \dots  Q_{\verify, k-1}$, and $r_{l+1} \in (Q_{\search})^+$. Intuitively, each occurence of $q_0$, except the first one, witnesses the \emph{first} occurrence of $u$ from the beginning or after its previous occurrence.

%The parsing automaton $\cA_u$ constructed above is \emph{unambiguous} in the sense that for each string $v \in \Sigma^+$, there is \emph{exactly one accepting run} of $\cA_u$ on $v$.

\begin{example}
	An example for $\cA_u$.
\end{example}

We are ready to present the construction of $\cB_{\cA_1, u,  T_{z}}$. The NFA $\cB_{\cA_1, u,  T_{z}}$ is constructed by the following three-step procedure.
\begin{enumerate}
\item Construct the product automaton $\cA_1 \times \cA_u$ of $\cA_1$ and $\cA_u$. Note that the set of final states of $\cA_1 \times \cA_u$ is $F_1 \times F_u$. 

\item Remove from $\cA_1 \times \cA_u$ all the states in $Q_1 \times Q_{\verify}$ as well as the transitions associated with them.

\item For each pair $(q,q') \in T_{z}$ and each sequence of transitions in $\cA_u$ of the form 
$$
\begin{array}{l}
\left( \left(\search, \overrightarrow{W} \right), u[1], \left(\verify, 1, \overrightarrow{W'_1} \right) \right), \left( \left(\verify, 1, \overrightarrow{W'_1} \right), u[2], 
 \left(\verify, 2, \overrightarrow{W'_2}\right) \right), \\
 \hspace{3cm} \cdots, \left(\left(\verify, k-1, \overrightarrow{W'_{k-1}} \right), u[k], q_0\right),
\end{array}
$$ 
add the transitions
$$
\begin{array}{c}
\left( \left(q, \left(\search, \overrightarrow{W} \right) \right), u[1], \left(q, \left(\verify, 1, \overrightarrow{W'_1} \right) \right) \right), \\
\left( \left(q, \left(\verify, 1, \overrightarrow{W'_1} \right) \right), u[2], \left(q, \left(\verify, 2, \overrightarrow{W'_2}\right)\right)\right),  \\
\cdots, \\
\left(\left(q, \left(\verify, k-2, \overrightarrow{W'_{k-2}} \right) \right), u[k-1], \left (q, \left (\verify, k-1, \overrightarrow{W'_{k-1}} \right) \right) \right),\\ \left( \left(q, \left (\verify, k-1, \overrightarrow{W'_{k-1}} \right) \right), u[k], \left(q', q_0 \right)\right).
\end{array}
$$
Note that the number of aforementioned sequences of transitions in $\cA_u$ is at most $|Q_{\search}|$, since  $ \overrightarrow{W'_1},\dots,  \overrightarrow{W'_{k-1}}$ are completely determined by $\overrightarrow{W} $ and $u$.
Intuitively, when $\cA_u$ identifies an occurrence of $u$, if the current state of $\cA_1$ is $q$, then after reading the occurrence of $u$, $\cB_{\cA_1, u, T_z}$ jumps from $q$ to some state $q'$ such that $(q,q') \in T_z$.
\end{enumerate}

\begin{example}
\end{example}

%Similarly to the single-letter case, we can define the dependency graph $G_C$. In addition, we can adapt $\dfs(z, z', a, f)$ into a procedure $\dfs(z, z', u, f)$, which integrates the automata $\cA_{u'}$ into the computation of the functions $f_{z', \cA_z}$, where $u,u'$ are constant strings occurring in the edge-labels in $G_C$.