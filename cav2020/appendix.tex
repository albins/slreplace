%!TEX root = main.tex
%
%In the next example, we are going to show that for a given string $v \in \Sigma^+$, $\indexof_v$ can be captured by a CEFA $\CEFA_{\indexof_v}$. 
\section{Construction of $\CEFA_{\indexof_v}$} \label{appendix:cefa_indexof}

\tl{to be checked...}
For this purpose, we need a concept of window profiles of  string positions w.r.t. $v$, which are elements of $\{\bot, \top\}^{n-1}$. The window profiles facilitate recognising the first occurrence of $v$ in the input string. 
Intuitively, given a string $u$, the window profile of a position $i$ in $u$ w.r.t. $v$ encodes the matchings of prefixes of $v$ to the suffixes of $u[0,i]$ (see \cite{CCH+18} for the details). For $\pi = \pi_1 \cdots \pi_{n-1} \in \{\bot, \top\}^{n-1}$ and $b \in \Sigma$, we use $\uwp(\vec{\pi}, b)$ to represent the window profile updated from $\pi$ after reading the letter $b$, specifically, $\uwp(\vec{\pi}, b) = \vec{\pi'}$ such that  
\begin{itemize}
\item $\pi'_1 = \top$ iff $b = a_1$, 
%
\item for each $i \in [n-2]$, $\pi'_{i+1} = \top$ iff $\pi_{i} = \top$ and $b = a_{i+1}$. 
\end{itemize}
Let $WP_v$ denote the set of window profiles of string positions w.r.t. $v$. From the result in \cite{CCH+18}, we know that $|WP_v| \le |v|$. 
%Then the set of window profiles of $v$, denoted by $WP_v$, is computed by setting $WP_0 := \{\bot^{n-1}\}$ and iterating the following procedure, until $WP_i = WP_{i+1}$:
%\[WP_{i+1}:=WP_i \cup \{\uwp(\vec{\pi}, b) \mid \vec{\pi} \in WP_i, b \in \Sigma\}.\] 
%Therefore, the aforementioned iteration terminates in at most $|v|$ steps.\\
%
%

Suppose $v = a_1 \cdots a_n$ with $n \ge 2$. 
Then $\indexof_v$ is captured by the CEFA $\CEFA_{\indexof_v}=(Q, \Sigma, R, \delta, I, F)$, such that 
\begin{itemize}
\item $Q = \{q_0, q_1\} \cup WP_v \cup WP_v \times [n]$, 
\item $R=(r_1, r_2)$ (where $r_1,r_2$ represent the input and output positions of $\indexof_v$ respectively), 
\item $I=\{q_0\}$, 
\item $F=\{q_1\}$, and 
\item $\delta$ comprises 
\begin{itemize}
\item the tuples $(q_0, a, q_0, \eta)$ such that $a \in \Sigma$, $\eta(r_1)=1$, and $\eta(r_2) = 1$,
%
\item the tuples $(q_0, a, \vec{\pi}, \eta)$ such that $a \in \Sigma$, $\vec{\pi} = \theta \bot^{n-2}$ where $\theta  = \top$ iff $a = a_1$, $\eta(r_1) = 0$, and $\eta(r_2)= 0$ (recall that the first position of a string is $0$),
% 
\item the tuples  $(\vec{\pi}, a, \uwp(\vec{\pi}, a), \eta)$ such that $\vec{\pi} \in WP_u$, $a \in \Sigma$, $\pi_{n-1} = \bot$ or $a \neq a_{n}$, $\eta(r_1) = 0$, and $\eta(r_2)= 1$,
%
\item the tuples $(\vec{\pi}, a, (\uwp(\vec{\pi}, a), 1), \eta)$ such that $\vec{\pi} \in WP_u$, $a = a_1$, $\pi_{n-1} = \bot$ or $a \neq a_{n}$, $\eta(r_1) = 0$, and $\eta(r_2)= 1$,
%
\item the tuples $((\vec{\pi}, i),  a, (\uwp(\vec{\pi}, a), i+1), \eta)$ such that $\vec{\pi} \in WP_u$, $i \in [n-2]$, $a = a_{i+1}$, $\pi_{n-1} = \bot$ or $a \neq a_{n}$, $\eta(r_1) = 0$, and $\eta(r_2)= 0$,
%
\item the tuples $((\vec{\pi}, n-1),  a, q_1, \eta)$ such that $\vec{\pi} \in WP_u$, $a = a_{n}$, $\eta(r_1) =0$, and $\eta(r_2)= 0$,
%
\item the tuples  $(q_1, a, q_1, \eta)$ such that $a \in \Sigma$, $\eta(r_1) = 0$, and $\eta(r_2)= 0$.
\end{itemize}
\end{itemize}

\section{Proof of Theorem~\ref{thm-la-sat-cefa}} \label{appendix:thm-la-sat-cefa}

%To prove Theorem~\ref{thm-la-sat-cefa}, we state and prove the following lemma. 

For a $k$-cost-enriched language $L$, we define 
\[
\prjnum(L) = \left\{(n_1, \cdots, n_k) \in \Int^k \mid \mbox{ there exist } w \in \Sigma^*.\ (w,(n_1,\cdots,n_k)) \in L \right\}.
\]

\begin{lemma}\label{lem-cefa-la}
	Let $\CEFA=(Q, \Sigma, R, \delta, I, F)$ be a CEFA with $R= (r_1, \cdots,  r_k)$. Then an existential LIA formula $\phi_\CEFA(r_1, \cdots, r_k)$ such that $\cM(\phi_\CEFA)= \prjnum(\Lang(\CEFA))$ can be computed in linear time from $\CEFA$.
\end{lemma}

\begin{proof}
	Suppose $\delta = \{\tau_1, \cdots, \tau_l\}$ such that $\tau_j = (q_j, a_j, q'_j, \eta_j)$ and $\eta_j(r_i) =  c_{j,i}$ for every $j \in [l]$ and $i \in [k]$.
	
	From the results on NFAs (Theorem~1 in \cite{SSMH04}), we know that for each pair of states $(q, q') \in I \times F$,  an existential LIA formula $\phi_{q,q'}(m_1, \cdots, m_l)$ can be computed in linear time such that $\cM(\phi_{q,q'})$ is the set of Parikh images of the runs of $\NFA$ starting from $q$ and ending at $q'$, where the variables $m_1, \cdots, m_l$ represent the numbers of occurrences of $\tau_1,\cdots, \tau_l$ respectively in the run. 
	
	Then the desired existential LIA formula $\phi_\NFA$ is constructed as follows,
	\[\phi_\NFA(r_1, \cdots, r_k) ::= \bigvee \limits_{(q,q') \in I \times F} \exists m_1 \cdots \exists m_l.\ \left(\varphi_{q,q'}(m_1, \cdots, m_l) \wedge \bigwedge \limits_{i \in [k]} r_i = \sum \limits_{j \in [l]} c_{j,i} m_j \right).\]
\end{proof}

We are ready to prove Theorem~\ref{thm-la-sat-cefa}.
\begin{proof}[Proof of Theorem~\ref{thm-la-sat-cefa}]
	The NP lower bound follows from the fact that the satisfiability problem of existential LIA formulas is NP-complete \cite{BT76,GS78} (see also \cite{Haase18}).
	
	For the upper bound, suppose that $\phi$ is a quantifier-free LIA formula and $\CEFA_1,\cdots,\CEFA_m$ are CEFAs such that 
	\begin{itemize}
		\item	$\CEFA_i=(Q_i, \Sigma, R_i, \delta_i, I_i, F_i)$  with $R_i = (r_{i, 1}, \cdots, r_{i, k_i})$, for every $i\in [m]$,
		\item $R_i \cap R_j = \emptyset$ for every $1 \le i < j \le m$, and
		\item the free variables of $\phi$ are from $\bigcup_{i\in [m]} R_i$.
	\end{itemize}
	From Lemma~\ref{lem-cefa-la}, for every $i \in [m]$, an existential LIA formula $\phi_{\CEFA_i}(r_{i,1}, \cdots, r_{i, k_i})$ such that $\cM(\phi_{\CEFA_i})= \prjnum(\Lang(\CEFA_i))$ can be computed in linear time from $\CEFA_i$.
	
	Then the satisfiability of $\phi$ w.r.t. $\CEFA_1,\cdots, \CEFA_m$ is reduced to the satisfiability problem of the  following existential LIA formula
	\[
	\phi' \equiv \phi \wedge \bigwedge \limits_{i \in [m]} \phi_{\CEFA_i}(r_{i,1}, \cdots, r_{i, k_i}).
	\]
	Since the size of $\phi'$ is linear in the size of $\phi$ and those of $\CEFA_1,\cdots,\CEFA_m$, and the satisfiability problem of existential LIA formulas is NP-complete, we conclude that the satisfiability of $\phi$ w.r.t.  $\CEFA_1,\cdots,\CEFA_m$ can be decided in nondeterministic polynomial time.
	
	It remains to prove the correctness of the reduction, namely, $\phi$ is satisfiable w.r.t. $\CEFA_1,\cdots, \CEFA_m$ iff $\phi'$ is satisfiable.
	
	\smallskip
	
	\noindent \emph{``Only if'' direction}. Suppose $\phi$ is satisfiable w.r.t. $\CEFA_1,\cdots, \CEFA_m$. Then there are an assignment function $\theta: \bigcup \limits_{i \in [m]} R_i \rightarrow \Int$ and strings $w_1, \cdots, w_m$  
	such that  $\phi[(\theta(R_i)/R_i)_{i \in [m]}]$ is evaluated to $true$ and $(w_i, \theta(R_i)) \in \Lang(\NFA_i)$ for every $i \in [m]$. For every $i \in [m]$, from $\cM(\phi_{\CEFA_i})=\prjnum(\Lang(\CEFA_i))$, we know that $\theta(R_i)$ satisfies $\phi_{\CEFA_i}$, namely, $\phi_{\CEFA_i}[\theta(R_i)/R_i]$ is evaluated to $true$. Therefore, the assignment $\theta$ makes $\phi'$ satisfied.
	
	\smallskip 
	
	\noindent \emph{``If'' direction}. Suppose $\phi'$ is satisfiable. Then there is an assignment $\theta: \bigcup \limits_{i \in [m]} R_i \rightarrow \Int$ such that $\phi[(\theta(R_i)/R_i)_{i \in [m]}]$, $\phi_{\CEFA_1}[\theta(R_1)/R_1]$, $\cdots$, and $\phi_{\CEFA_m}[\theta(R_m)/R_m]$ are all evaluated to $true$. For every $i \in [m]$, from $\cM(\phi_{\CEFA_i})=\prjnum(\Lang(\CEFA_i))$,  we know that there is a string $w_i$ such that $(w_i, \theta(R_i)) \in \Lang(\CEFA_i)$. From Definition~\ref{def-la-sat-cefa}, we conclude that $\phi$ is satisfiable w.r.t. $\CEFA_1,\cdots, \CEFA_m$.
\end{proof}

\section{Proof of Proposition~\ref{prop:pre-image}}

\begin{proof}
	Let $\CEFA=(Q, \Sigma, R, \delta, I, F)$ be a CEFA with $R= (r_1, \cdots, r_k)$. We show how to construct a CEFA representation of $f^{-1}_R(L)$ for each function $f$ in {\slint}.
	
	%%%%%%%%%%%%%%%%%%%%%%%%%%%%%%%%%%%%%%%%%%%%%%%%%%%%%%%%%%%%%%%%%%%%%%%%%%%%%%
	\paragraph*{$\concat^{-1}_R(L)$.}
	%
	A CEFA representation of $\concat^{-1}_R(L)$ is given by $((\CEFA_{I, q}, \NFA_{q, F})_{q \in Q}, \vec{t})$, where 
	\begin{itemize}
		\item $\CEFA_{I, q}=(Q, \Sigma, R^{(1)}, \delta^{(1)}, I, \{q\})$ and  $\CEFA_{q, F}=(Q, \Sigma, R^{(2)}, \delta^{(2)}, \{q\}, F)$ such that 
		\begin{itemize}
			\item $R^{(1)} = (r^{(1)}_1, \cdots, r^{(1)}_k)$, $R^{(2)} = (r^{(2)}_1, \cdots, r^{(2)}_k)$, 
			\item $\delta^{(1)}$ comprises the tuples $(q, a, q', \eta')$ satisfying that there exists $\eta$ such that $(q, a, q', \eta) \in \delta$ and for each $j \in [k]$, and $\eta'(r^{(1)}_j)=\eta(r_j)$,  similarly for $\delta^{(2)}$,
		\end{itemize}
		\item and $\vec{t} = (r^{(1)}_1 + r^{(2)}_1, \cdots, r^{(1)}_k + r^{(2)}_k)$.
	\end{itemize}
	Note that the size of $((\CEFA_{I, q}, \NFA_{q, F})_{q \in Q}, \vec{t})$ is $\bigO(|\CEFA|^2)$.
	%
	%%%%%%%%%%%%%%%%%%%%%%%%%%%%%%%%%%%%%%%%%%%%%%%%%%%%%%%%%%%%%%%%%%%%%%%%%%%%%%
	%
	\paragraph*{$\reverse^{-1}_R(L)$.} 
	%
	A CEFA representation of $\reverse^{-1}_R(L)$ is given by $(\CEFA^{(r)}, \vec{t})$, where 
	\begin{itemize}
		\item $\CEFA^{(r)}=(Q, \Sigma, R^{(1)}, \delta', F, I)$ such that 
		\begin{itemize}
			\item $R^{(1)}=(r^{(1)}_1,\cdots,r^{(1)}_k)$, and 
			\item $\delta'$ comprises the tuples $(q', a, q, \eta')$ satisfying that there exists $\eta$ such that $(q, a, q', \eta) \in \delta$, and $\eta'(r^{(1)}_i) = \eta(r_i)$ for each $i \in [k]$,
		\end{itemize}
		%
		\item and $\vec{t}=(r^{(1)}_1, \cdots, r^{(1)}_k)$. 
	\end{itemize}
	Note that $\Lang(\CEFA^{(r)}) = \{(w^{(r)}, \vec{n}) \mid (w, \vec{n}) \in \Lang(\CEFA)\}$, and the size of $(\CEFA^{(r)}, \vec{t})$ is $\bigO(|\CEFA|)$.
	
	%%%%%%%%%%%%%%%%%%%%%%%%%%%%%%%%%%%%%%%%%%%%%%%%%%%%%%%%%%%%%%%%%%%%%%%%%%%%%%
	
	\paragraph*{$\substring^{-1}_R(L)$.}
	A CEFA representation of $\substring^{-1}_R(L)$ is given by $(\cB, \vec{t})$, where 
	\begin{itemize}
		\item $\cB = (Q', \Sigma, R', \delta', I', F')$ such that 
		\begin{itemize}
			\item $Q' = Q \times \{p_0, p_1, p_2\}$, (intuitively, $p_0$, $p_1$, and $p_2$ denote that the current position is before the starting position, between the starting position and ending position, and after the ending position respectively)
			%
			\item $R' = \left(r^{(1)}_1,\cdots, r^{(1)}_k, r'_1, r'_2 \right)$, (intuitively, $r'_1$ denotes the starting position, and $r'_2$ denotes the length of the substring)
			%
			\item $I'=I \times \{p_0\}$, $F'=F \times \{p_2\}$,
			%
			\item and $\delta'$ comprises 
			\begin{itemize}
				\item the tuples $((q, p_0), a, (q, p_0), \eta')$ such that $q \in I$, $a \in \Sigma$, and $\eta'$ satisfies that $\eta'(r^{(1)}_j)=0$ for each $j \in [k]$, $\eta'(r'_1)= 1$, and $\eta'(r'_2) = 0$,
				%
				\item the tuples $((q, p_0), a, (q', p_1), \eta')$ such that there exists $\eta$ satisfying that $(q, a, q', \eta) \in \delta$, $\eta'(r^{(1)}_j)=\eta(r_j)$ for each $j \in [k]$,  $\eta'(r'_1)=0$ (recall that the positions of strings start at $0$), and $\eta'(r'_2) = 1$,
				%
				\item the tuples $((q, p_0), a, (q', p_2), \eta')$ such that there exists $\eta$ satisfying that $(q, a, q', \eta) \in \delta$, $q' \in F$, $\eta'(r^{(1)}_j)=\eta(r_j)$ for each $j \in [k]$,  $\eta'(r'_1)=0$ (recall that the positions of strings start at $0$), and $\eta'(r'_2) = 1$,
				%
				\item the tuples $((q, p_1), a, (q', p_1), \eta')$ such that there exists $\eta$ satisfying that $(q, a, q', \eta) \in \delta$ and $\eta'(r^{(1)}_j)=\eta(r_j)$ for each $j \in [k]$, $\eta'(r'_1) = 0$, and $\eta'(r'_2) = 1$,
				%
				\item the tuples $((q, p_1), a, (q', p_2), \eta')$ such that there exists $\eta$ satisfying that $(q, a, q', \eta) \in \delta$, $q' \in F$, and $\eta'(r^{(1)}_j)=\eta(r_j)$ for each $j \in [k]$, $\eta'(r'_1) = 0$, and $\eta'(r'_2) = 1$,
				%
				%\item the tuples $((q, p_1), a, (q, p_2), \eta')$ such that $q \in F$ and $\eta'(r^{(1)}_j)=0$ for each $j \in [k]$, $\eta'(r'_1) = 0$, and $\eta'(r'_2) = 1$,
				%
				\item the tuples $((q, p_2), a, (q, p_2), \eta')$ such that $q \in F$ and $\eta'(r^{(1)}_j)=0$ for each $j \in [k]$, $\eta'(r'_1) = 0$, and $\eta'(r'_2) = 0$,
				%
			\end{itemize}
		\end{itemize}
		\item $\vec{t}=(r^{(1)}_1, \cdots, r^{(1)}_k)$.
	\end{itemize}
	Note that the size of $(\cB, \vec{t})$ is $\bigO(|\CEFA|)$.
	%%%%%%%%%%%%%%%%%%%%%%%%%%%%%%%%%%%%%%%%%%%%%%%%%%%%%%%%%%%%%%%%%%%%%%%%%%%%%%
	%
	%
	\paragraph*{$(\Tran(\NFT))^{-1}_R(L)$.}
	%
	Suppose $\NFT = (Q', \Sigma, \delta', I', F')$. Then a CEFA representation of $(\Tran(\NFT))^{-1}_R(L)$ is given by 
	$(\cB, \vec{t})$, where 
	\begin{itemize}
		\item $\cB$ simulates the run of $\NFT$ on the input string, meanwhile, it simulates the run of $\CEFA$ on the output string of $\NFT$, formally, $\cB= (Q' \times Q, \Sigma, R^{(1)}, \delta'', I' \times I, F' \times F)$ such that 
		\begin{itemize}
			\item $R^{(1)}  = (r^{(1)}_1, \cdots, r^{(1)}_k)$, and
			\item $\delta''$ comprises the tuples $((q'_1, q_1), a, (q'_2, q_2), \eta')$ satisfying one of the following conditions,
			\begin{itemize}
				\item there exist $u = a_1 \cdots a_n \in \Sigma^+$ and a transition sequence $p_0 \xrightarrow[\delta]{a_1, \eta_1} p_2 \cdots p_{n-1} \xrightarrow[\delta]{a_n, \eta_n} p_{n}$ in $\CEFA$ such that $(q'_1, a, q'_2, u) \in \delta'$, $p_0 = q_1$, $p_{n}= q_2$, and for each $j \in [k]$,  $\eta'(r^{(1)}_j) = \eta_1(r_j) + \cdots + \eta_n(r_j)$,
				%
				\item $(q'_1, a, q'_2, \varepsilon) \in \delta'$, $q_1 = q_2$, and $\eta'(r^{(1)}_j) =0$ for each $j \in [k]$,
			\end{itemize}
		\end{itemize}
		%
		\item $\vec{t}=(r^{(1)}_1, \cdots, r^{(1)}_k)$.
	\end{itemize}
	Note that the number of transitions of $\cB$ can be exponential in the worst case, since it summarises the updates of cost registers of $\CEFA$ on the output strings of the transitions of $\NFT$. More precisely,  let
	\begin{itemize}
		\item $\ell$ be the maximum length of the output strings of transitions of $\NFT$, 
		\item $N$ be the maximum number of transitions between a given pair of states of $\CEFA$, and
		\item  $C$ be the maximum absolute value of the integer constants occurring in $\CEFA$,
	\end{itemize}
	then $|\delta''|$, the cardinality of $\delta''$, is bounded by $|\delta'| \times |Q|^2 \times N^\ell $, and the integer constants occurring in each transition of $\delta''$ are bounded by $\ell C$. Therefore, 
	the size of $(\cB, \vec{t})$ is 
	\[
	\bigO(|\delta'| \times |Q|^2 \times N^\ell \times k \log_2 (\ell C)).
	\] 
	Since $|\delta'|, \ell \le |\NFT|$, $|Q|, N, k \le |\CEFA|$, and $C \le 2^{|\CEFA|}$, we deduce that the size of $(\cB, \vec{t})$ is 
	$
	\bigO( |\NFT| \times  |\CEFA|^2 \times |\CEFA|^{|\NFT|} \times |\CEFA|^2 \log_2(|\NFT|))= |\CEFA|^{\bigO(|\NFT|)} |\NFT| \log_2(|\NFT|).$
	%
	
	%%%%%%%%%%%%%%%%%%%%%%%%%%%%%%%%%%%%%%%%%%%%%%%%%%%%%%%%%%%%%%%%%%%%%%%%%%%%%%
	\paragraph*{$(\replaceall_{e,u})^{-1}_R(L)$.}
	%
	From the result in \cite{CCH+18}, we know that  a NFT $\NFT_{e,u}=(Q', \Sigma, \delta', I', F')$ can be constructed to capture $\replaceall_{e,u}$.  Moreover, 
	\begin{itemize}
		\item $|Q'|$, as well as $|\delta'|$, is $2^{\bigO(|e|)}$,
		\item $\ell$, the maximum length of the output strings of transitions of $\NFT_{e,u}$, is $|u|$.
	\end{itemize}
	Then a CEFA representation of $(\replaceall_{e,u})^{-1}_R(L)$ can be constructed as that of $(\Tran(\NFT_{e,u}))^{-1}_R(L)$.
	Let $N$ denote the maximum number of transitions between a given pair of states of $\CEFA$, and $C$ be the maximum absolute value of the integer constants occurring in $\CEFA$, which is bounded by $2^{|\CEFA|}$. Then the CEFA representation of $(\replaceall_{e,u})^{-1}_R(L)$ is of size 
	\[
	\bigO(|\delta'| \times |Q|^2 \times N^\ell \times k \log_2 (\ell C)) = 2^{\bigO(|e|)} |\CEFA|^2 |\CEFA|^{|u|} |\CEFA|^2 \log_2 |u|=2^{\bigO(|e|)} |\CEFA|^{\bigO(|u|)}.
	\]
	%
	according to the aforementioned discussion for NFTs.
	% 
	%
\end{proof}




\smallskip

\noindent{\bf Proposition~\ref{prop:la-sat-cefa-inter}}.
\emph{Given a family of CEFAs $\{ \CEFA_i^{j} \}_{i\in I,j\in J_i}$ each of which carries a vector of registers $R_i^j$ and a quantifier-free LIA formula $\phi$ such that  $ R_i^{j} $ are pairwise disjoint and the variables of $\phi$ are from $R'=\bigcup_{i,j} R_i^j$. Deciding whether  %such that the free variables of $\phi$ are from $\bigcup_{i\in [m]} R_i$,  and here are 
	there are an assignment function $\theta: R' \rightarrow \Int$ and strings $(w_i)_{i \in I}$ such that  $\phi[\theta(R' )/R']$ holds and $(w_i, \theta(R_i^j)) \in \Lang(\CEFA_{i}^j)$ for every $i \in I$ and $j \in J_i$ is PSPACE-complete.} 
	
\begin{proof}
Use Proposition 16 in \cite{LB16}.
\zhilin{@Anthony, please write your proof here.}
\end{proof}
