%!TEX root = main.tex


This paper is concerned with strings and integers, two fundamental data types in virtually all programming languages.
String-manipulating programs are notoriously subtle and their potential bugs %Many subtle programming bugs are caused by , some with 
may bring severe security consequences. A typical example is cross-site scripting
(XSS), which is among the OWASP Top 10 Application Security Risks
\cite{owasp17}. Integer occurs naturally and extensively in string-manipulating programs. %For instance, access to lengths or positions of strings is frequently needed. 
Typically, mainstream programming languages provide, apart from standard string functions (e.g., concatenation, $\replace$ and $\replaceall$), %string functions involving the integer data type, e.g. 
functions such as $\length$, $\indexof$, and $\substring$, which can convert strings to integers and vice versa. %, are also widely used in string-manipulating programs. 
These functions are indeed heavily utilised in practice, for instance, it was reported \cite{Berkeley-JavaScript} that $\length$, $\indexof$, $\substring$ (or variants thereof) comprise over 80\% of string function occurrences in 18 popular JavaScript applications (notably more frequent than concatenation). Undoubtedly, the introduction of integers exacerbates the intricacy of string-manipulating programs, which calls for a unified framework to reason about them. 


%in programs it , typically in the form of lengths or positions of strings. Besides the classical string functions e.g. concatenation $\concat$ and $\replaceall$, string functions involving the integer data type, e.g. $\length$, $\indexof$, and $\substring$, are also widely used in string-manipulating programs. For instance, it was reported in \cite{Berkeley-JavaScript} that $\length$, $\indexof$, $\substring$, and their variants, occupy more than 80 percent of string function occurrences in their subject Javascript applications (even much more frequently used than concatenation). This motivates \emph{string constraint solvers that reason about strings and integers simultaneously}. 

One effective automatic analysis and testing method for identifying subtle programming errors is based on symbolic execution.
%\cite{king76} and combinations with dynamic analysis
%called \emph{dynamic symbolic execution} \cite{jalangi,DART,EXE,CUTE,KLEE}.
(See \cite{symbex-survey} for an excellent survey.)
%Unlike purely random testing,
%which runs only \emph{concrete} program executions on different
%inputs, the techniques of symbolic execution 
In a nutshell, this technique analyzes static paths
(a.k.a. symbolic executions) through the program being considered. %software system under test.
Such a path can be viewed as a constraint $\varphi$ (over
appropriate data domains) and one expects a fast (constraint)
solver is available for checking the satisfiability of $\varphi$ (i.e., to check
the \emph{feasibility} of the static path), which can be used for, e.g.,  generating
inputs that lead to certain parts of the program or an erroneous behavior.



%$\length: \Sigma^* \rightarrow \Int$ (where $\Sigma$ is the alphabet and $\Sigma^*$ is the set of strings over $\Sigma$), $\substring: \Sigma^* \times \Int \times \Int \rightarrow \Sigma^*$, and $\indexof: \Sigma^* \times \Int \rightarrow \Int$.

%%%%%%%%%%%%%%%%%%%%%%% The running example %%%%%%%%%%%%%%%%%%%%%%%
%%%%%%%%%%%%%%%%%%%%%%% The running example %%%%%%%%%%%%%%%%%%%%%%%
%%%%%%%%%%%%%%%%%%%%%%% The running example %%%%%%%%%%%%%%%%%%%%%%%
%!TEX root = main.tex

\begin{example}\label{exmp:running}
The following JavaScript program demonstrates the use of string functions coupled with integer data type as well as path feasibility of string-manipulating programs.
The program defines a function {\urlxsssanitise} (the input parameter url specifying an URL). A typical URL consists of a hierarchical sequence of components commonly referred to as protocol, host, path, query, and fragment. For instance, in ``\url{http://www.example.com/some/abc.html?name=john#print}'', the protocol is ``{\tt http}'', the host is ``{\tt www.example.com}'', the path is ``{\tt /some/abc.html}'', the query is ``{\tt name=john}'' (preceded by $?$), and the fragment is ``{\tt print}'' (preceded by $\#$). (Note that either query or fragment could be empty in an URL.) The aim of {\urlxsssanitise} is to mitigate \emph{URL reflection attacks} \cite{url-reflect}, by filtering out the dangerous substring ``\url{script}'' from the query and fragment components of  the input URL. URL reflection attacks are a type of cross-site-scripting (XSS) attacks that do not rely on saving malicious code in database, but rather on hiding it in the query or fragment component of URLs, e.g., ``\url{http://www.example.com/some/abc.html?name=<script>alert('xss!');</script>}''.
%\begin{example}
{\small
\begin{minted}[linenos]{javascript}
function urlXssSanitise(url)
{
    var prothostpath='', querfrag = '';
    url = url.trim();
    var qmarkpos = url.indexof('?'), sharppos = url.indexof('#');
    if(qmarkpos >= 0) 
    {   prothostpath = url.substr(0, qmarkpos);
        querfrag = url.substr(qmarkpos); }
    else if(sharppos >= 0)
    {   prothostpath = url.substr(0, sharppos);
        querfrag = url.substr(sharppos); }
    else prothostpath = url;
    querfrag = querfrag.replace(/script/g, '');
    url = prothostpath.concat(querfrag);
    return url;
}
\end{minted}
}
%\end{example}
%
\noindent Note that {\urlxsssanitise} uses the JavaScript sanitisation operation $\sf trim$ that removes whitespace from both ends of a string, which can be conveniently modeled by finite-state transducers. Two string functions involving integer data type, namely, $\sf indexof$ and $\sf substr$ ($\indexof$ and $\substring$ in this paper), as well as concatenation and $\sf replace$ (with the 'g'--global--flag, called $\replaceall$ in this paper), are present. 

%$\footnote{$\sf replace$ with the g flag in Javascript corresponds to the $\replaceall$ function in this paper.}

% \in [\backslash w | \backslash x2E]^*$
%We expect that the returned value of the ``host'' variable contains only the alphanumeric symbols as well as the dot symbol, but actually this is not the case. This question can be reduced to solving the path feasibility problem of the following program of the SSA (single static assignment) form.

The program analysis is to ascertain whether {\urlxsssanitise} indeed works, namely, after applying {\urlxsssanitise} to the input URL, ``\url{script}'' does disappear.  This problem can be reduced to checking whether there is an execution path of {\urlxsssanitise} that produces an output such that its query or fragment component contains occurrences of ``\url{script}''. For instance, assuming  the "if" branch is executed, %first execution path of {\urlxsssanitise} is chosen, then the problem is reduced to 
we will need to solve the path feasibility of the following JavaScript program in the single static assignment (SSA) form,
%
{\small
\begin{minted}{javascript}
    prothostpath =''; querfrag = '';
    url1 = url.trim(); qmarkpos = url1.indexof('?');
    sharppos = url1.indexof('#'); assert(qmarkpos >= 0); 
    prothostpath1 = url1.substr(0, qmarkpos);
    querfrag1 = url1.substr(qmarkpos);
    querfrag2 = querfrag1.replace(/script/g, '');
    url2 = prothostpath1.concat(querfrag2);
    assert(/script/.test(querfrag2))
\end{minted}
}
\noindent where the $\ASSERT{cond}$ statement checks that the condition $cond$ is satisfied. As one will see later, this can be directly encoded in our constraint language (cf.\ Section~\ref{sec:logic}) and handled by the decision procedure (cf. Section~\ref{sec:dec}).  \qed
\end{example}
%%%%%%%%%%%%%%%%%%%%%%% The running example %%%%%%%%%%%%%%%%%%%%%%%
%%%%%%%%%%%%%%%%%%%%%%% The running example %%%%%%%%%%%%%%%%%%%%%%%
%%%%%%%%%%%%%%%%%%%%%%% The running example %%%%%%%%%%%%%%%%%%%%%%%


Reasoning about strings and integers is important and challenging, but is not well-studied. 
%It is easy to end up with undecidability: even the slightest extension of this theory (e.g. the theory with concatenation and letter counting)  would render the theory undecidable \cite{buchi,GB16}. 
The problem is undecidable in general,  for instance, the string theory with concatenation and letter counting functions is undecidable \cite{buchi}. 
Even worse, although a great deal of research has shown that it is viable to reason about rather complex string operations without breaking decidability by %imposing proper 
restricting the form of constraints (e.g., \cite{CCH+18,CHL+19}; cf. related work for a brief survey), adding length constraints would immediately lead to undecidability \cite{CCH+18}. Moreover, it is still a major open problem whether the string theory with concatenation (arguably the simplest string operation) and length function (arguably the most common string-number conversion) is decidable \cite{Vijay-length}. 
%
%At the same time, the classes of constraints that are required in practice sometimes demand solving techniques which are not necessarily complete. 
As a result, a vast majority of  state-of-the-art string constraint solvers, e.g., {\cvc} \cite{cvc4}, {\zthree} \cite{Z3-str3}, Trau \cite{Abdulla17} (or its variants {\trauplus} \cite{AbdullaA+19} and {\zthreetrau} \cite{Z3-trau}),  
%ABC \cite{ABC}, and SLENT \cite{WC+18}, 
resort to heuristics without completeness guarantees to support reasoning about both strings and integers.
%Nevertheless, solving string constraints involving the integer data type is far from trivial and actually undecidable in general \cite{buchi,CCH+18}. Therefore, all the aforementioned string constraint solvers
%They however inevitably  resort to heuristics without completeness guarantees. 

Nevertheless, as already argued in \cite{CHL+19}, despite the excellent performance of some of these solvers on some existing benchmark suites, decision procedures for string constraints with stronger theoretical guarantees, e.g., in the form of decidability (perhaps accompanied by a complexity analysis), are still highly desirable. In a nutshell, there are at least two reasons: (1) they are \emph{theoretically intriguing} as a rich fragment of first-order theory of strings not least; %to deepen our understanding and increase our knowledge of the problem; 
and  %appealing %from a scientific point of view, and on the other hand, 
(2) they can be \emph{practically appealing} since they could potentially achieve a good balance between precision and scalability, if they are amenable to implementation. 
%they can provide a kind of robustness guarantees upon which a string constraint solver could further improve and optimise.if they are amenable to implementation, 

%\zhilin{this paragraph is copied from the popl paper} Despite the excellent performance of some of these solvers on several existing benchmarks, there are good reasons for designing decision procedures with stronger theoretical guarantees, e.g., in the form of decidability (perhaps accompanied by a complexity analysis). One such reason is that string constraint solving is a research area in its infancy with an insufficient range of benchmarking examples to convince us that if a string solver works well on existing benchmarks, it will also work well on future benchmarks. A theoretical result provides a kind of robustness guarantee upon which a practical solver could further improve and optimise.

Recently, researchers started investigating the decision procedures for string constraints involving the integer data type, for instance \cite{Vijay-length,LeH18,LinM18,LB16}. Nevertheless, those decision procedures are largely only of the theoretical value, since they either support limited string functions or are too complicated for  implementation. 
%(concatenation, length and regular constraints) 
We tackle this issue in this paper and aim at designing decision procedures for string constraints involving the integer data type, which, on the one hand, \emph{provide completeness guarantee}, and on the other hand, \emph{admit efficient implementation}.

The main contribution of this paper is a decision procedure for an expressive class of string constraints involving the integer data type, which includes not only concatenation, $\replaceall$, $\reverse$, finite transducers, and regular constraints, but also $\length$, $\indexof$ and $\substring$. The decision procedure is automata-theoretic and utilises a variant of cost-register automata introduced by Alur et al. \cite{RLJ+13}, called \emph{cost-enriched finite automata} (abbreviated as CEFA). The crux of the decision procedure is to compute the backward images of CEFAs under string functions, thus in the same flavour as that in \cite{CHL+19} for string constraints \emph{without} the integer data type and inherits its elegance by treating the aforementioned seemly diverse string functions in a \emph{simple and generic} way. To the best of our knowledge, the class of string constraints considered in this paper is currently \emph{the most expressive string theory involving the integer data type for which a decision procedure has been achieved}. 
%\emph{the first decision procedure for string constraints that supports the string functions} $\substring$, $\length$, and $\indexof$, as well as concatenation, $\replaceall$, and finite transducers. \zhilin{improve latter}

Most importantly, our decision procedure admits efficient implementation, as an extension of the {\ostrich} solver \cite{CHL+19}, resulting in a new solver {\ostrich}+.  We have done experiments on a wide range of benchmark suites, including the well-known {\kaluzabench} and {\pyexbench}, to evaluate the performance of {\ostrich}+. The experiment results show that  {\ostrich}+ achieves a nice balance between expressiveness and efficiency in the sense that 1) {\ostrich}+ is the \emph{only} string constraint solver capable of dealing with finite transducers and integer constraints, and 2) its performance is \emph{comparable} with the best state-of-the-art string constraint solvers (e.g. {\cvc} and {\zthreetrau}). In particular, {\ostrich}+ solves in \emph{several seconds} the path feasibility of the SSA program in Example~\ref{exmp:running}, which is impressive since the program contains not only complex string functions such as {\tt trim()} (modelled as finite transducers) and {\tt replaceAll}, but also those involving the integer data type, namely $\length$, $\indexof$, and $\substring$. 

%!TEX root = popl2018.tex

\section{Related work}\label{sec-rel}

 
In this section, we discuss some related work. We will classify our discussions into two main categories: (1) theoretical results in terms of decidability and complexity; (2) practical (but generally incomplete) approaches used in string solvers.  We shall emphasise work regarding $\replaceall$ functions as they are our main focus. 

\paragraph{Theoretical results}
We have discussed in Section \ref{sec:intro} works on string constraints with 
the theory of strings with concatenation. This research programme builds on
the question of solving satisfiability of \emph{word equations}, i.e., a
string equation $\alpha = \beta$ containing concatenation of string constants
and variables. Makanin showed decidability \cite{Makanin}, whose upper bound
was improved to PSPACE in \cite{P04} using a word compression technique. 
A simpler algorithm was in recent years proposed in \cite{J17} using
the recompression technique. The best lower bound for this problem is still NP,
and closing this complexity gap is a long-standing open problem. Decidability
(in fact, the PSPACE upper bound) can be retained in the presence of regular
constraints (e.g. see \cite{Schulz}). This can be extended to existential theory
of concatenation with regular constraints using the technique of \cite{buchi}.
The replace-all operator expressed by the concatenation operator alone. For 
this reason, our decidability of the fragment of $\strline[\replaceall]$ cannot
be derived from the results from the theory of concatenation alone.

%Our work is closely related to solving word equations, which are a conjunction of equations of $v=w$, where $v, w$ are concatenation of string constants and variables. The computational complexity of this problem remains unknown, with the best lower and upper bounds being NP and PSPACE respectively. Makanin %refuted a conjecture 
%showed, perhaps surprisingly, that the problem is decidable \cite{Makanin}, and %Plandowski's  on the decidability and complexity of satisfiability for word equations, i.e., 
%Plandowski explored an approach of compression and proposed a PSPACE algorithm \cite{P04}.  This is the best bound up to date, though a simpler PSPACE algorithm with smaller space requirement (nondeterministic $O(n \log n)$ space) was proposed by Jez \cite{J16}, based on compression. Very recently, Jez \cite{J17} further improves the complexity to (nondeterministic) $O(n )$, i.e.,  linear space. 

%The connection between compression and word equations was first observed by Plandowski and Rytter \cite{PR98}. %, who showed that a length-minimal solution of size N has a compressed representation
%of size poly(n, log N). 
\OMIT{
Essentially, the constraint language  $\strline[\replaceall]$ studied in this paper is word equations where the $\replaceall$ functions, which subsume the concatenation operation, are used. However, the additional straight-line restrictions are imposed.  
}

Regarding the extension with length constraints, it is still a long-standing
open problem whether word equations with length constraints is decidable, though
it is known that letter-counting (e.g. counting the number of occurrences of 0s
and 1s separately) yields undecidability \cite{buchi}. It was shown in
\cite{LB16} that the length constraints (in fact, letter-counting) can be
added to the subclass of $\strline[\replaceall]$ where the pattern/replacement 
are constants, while preserving decidability. In contrast, if we allow 
variables on the replacement strings of formulas in $\strline[\replaceall]$,
we can easily encode the Hilbert's 10th problem with length (integer) 
constraints. In fact, this undecidability holds even if we use the unary
alphabet $\Sigma = \{a\}$, and that the pattern string is
always fixed to the letter $a$.

%Solving word equation was an intriguing problem since the beginning of computer science, investigated
%initially due to its ties to Hilbert’s 10th problem. Initially it was conjectured that this
%problem is undecidable, which was disproved by Makanin [10]. At the beginning little attention
%was given to computational complexity of Makanin’s algorithm and the problem itself; these questions
%were reinvestigated in the ’90 [6, 18, 9], culminating in the EXPSPACE implementation of
%Makanin’s algorithm by Gutiérrez [5].


%\cite{J17}
%Word equations in linear space
%
%Word equations are an important problem on the intersection of formal languages and algebra. Given two sequences consisting of letters and variables we are to decide whether there is a substitution for the variables that turns this equation into true equality of strings. The computational complexity of this problem remains unknown, with the best lower and upper bounds being NP and PSPACE. Recently, a novel technique of recompression was applied to this problem, simplifying the known proofs and lowering the space complexity to (nondeterministic) O(n log n). In this paper we show that word equations are in nondeterministic linear space, thus the language of satisfiable word equations is context-sensitive. We use the known recompression-based algorithm and additionally employ Huffman coding for letters. The proof, however, uses analysis of how the fragments of the equation depend on each other as well as a new strategy for nondeterministic choices of the algorithm, which uses several new ideas to limit the space occupied by the letters.


\OMIT{
%A Decision Procedure for String Logic with Equations, Regular Membership and Length Constraints 
 Le \cite{L16} considered the satisfiability problem for string logic with word equations, regular membership and Presburger constraints over length functions. %The difficulty comes from multiple occurrences of string variables making state-of-the-art algorithms non-terminating. Our main contribution is to 
He showed that the satisfiability problem in a fragment where no string variable occurs more than twice in an equation is decidable. 
%In particular, he proposed a semi-decision procedure for arbitrary string formulae with word equations, regular membership and length functions, and showed that the algorithm always terminates for the aforementioned decidable fragment, with a complexity analysis. 
This work is largely distant from ours, as $\replaceall$ was not addressed. However, we mention that the fragment considered by Le allows Presburger constraints over length functions.
%The essence of our procedure is an algorithm to enumerate an equivalent set of solvable disjuncts for the formula. We further show that the algorithm always terminates for the aforementioned decidable fragment. Finally, we provide a complexity analysis of our decision procedure to prove that it runs, in the worst case, in factorial time.

%In \cite{BTV09} the authors discussed the problem of path feasibility for programs manipulating strings using a collection of standard string library functions. They prove results on the complexity of this problem, including its undecidability in the general case and decidability of some special cases. \tl{how would it connect to ours?}
}



The $\replaceall$ function can be seen as a special, yet expressive, string transformation function, aka string transducer. From this viewpoint, the closest work is~\cite{LB16}, which we discuss extensively in the introduction. Here, we discuss two further 
%the $\replaceall$ function is also related to two 
recent transducer models: streaming string transducers \cite{AC10} and symbolic transducers \cite{symbolic-transducer}. 

A streaming string transducer is a finite state machine where  a finite set of string variables are used to store the intermediate results for output. The $\replaceall(x, e, y)$ term can be modelled by an extension of streaming string transducers \emph{with parameters}, that is, a streaming string transducer which reads an input string (interpreted as the value of $x$), uses $y$ as a free string variable which is presumed to be read-only, and updates a string variable $z$, which stores the computation result, by a string term which may involve $y$. Nevertheless, to the best of our knowledge, this extension of streaming string transducers has not been investigated so far. 

Symbolic transducers are an extension of Mealy machine to infinite alphabets by using a variable $cur$ to represent the symbol in the current position, and replacing the input and output letters in transitions with unary predicates $\varphi(cur)$ and terms involving $cur$ respectively. Symbolic transducers can model $\replaceall$ functions \emph{when the third parameter is a constant}. Inspired by symbolic transducers, it is perhaps an interesting future work to consider an extension of the $\replaceall$ function by allowing predicates as patterns. %to sequences of numerical values. 
For instance, one may consider the term $\replaceall(x, cur \equiv 0 \bmod 2, y)$ which replaces every even number in $x$ with $y$. %This, however, is not addressed in the current paper. 

Finally, the $\replaceall$ function is related to Array Folds Logic introduced by Daca et al \cite{DHK16}. The authors considered an extension of the quantifier-free theory of integer arrays with counting. The main feature of the logic is the \emph{fold} terms, borrowed from the folding concept in functional programming languages. Intuitively, a fold term applies a function to every element of the array to compute an output. If strings are treated as arrays over a finite domain (the alphabet), the $\replaceall$ function can be seen as a fold term. Nevertheless, the $\replaceall$ function goes beyond the fold terms considered in \cite{DHK16}, since it outputs a string (an array), instead of an integer. Therefore, the results in \cite{DHK16} cannot be applied to our setting.

\paragraph{Practical Solvers}
There is a large amount of recent work on developing practical string solvers, Kaluza~\cite{Berkeley-JavaScript}, Hampi~\cite{HAMPI}, Z3-str~\cite{Z3-str}, CVC4~\cite{cvc4}, Stranger~\cite{YABI14}, Norn~\cite{Abdulla14}, S3 and S3P~\cite{S3,TCJ16}, and FAT~\cite{Abdulla17}.
Among them, only Stranger, S3, and S2P support $\replaceall$.  

%String solvers that support concatenations and the replace-all operator are available. \cite{BTV09,TCJ14,YABI14,S3}

%\cite{BTV09} considered an efficient finite model finding method for string constraints. They develop a two-tier finite model finding procedure. First, an integer abstraction of string constraints are passed to an SMT (Satisfiability Modulo Theories) solver. The abstraction is either unsatisfiable, or the solver produces a model that fixes lengths of enough strings to reduce the entire problem to be finite domain. The resulting fixed-length string constraints are then solved in a second phase. 

%We implemented the procedure in a symbolic execution framework, report on the encouraging results and discuss directions for improving the method further.


In the Stranger tool, %Verifying string manipulating programs is a crucial problem in computer security. String operations are used extensively within web applications to manipulate user input, and their erroneous use is the most common cause of security vulnerabilities in web applications. We 
an automata-based approach was provided for symbolic analysis of PHP programs, where two different semantics $\replaceall$ were considered, namely, the leftmost and longest matching as well as the leftmost and shortest matching. Nevertheless, they focused on the abstraction-interpretation based analysis of PHP programs and provided an \emph{over-approximation} of all the possible values of the string variables at each program point. Therefore, their string constraint solving algorithm is \emph{not} an exact decision procedure. In contrast, we provided a decision procedure for the straight-line fragment with the rather general $\replaceall$ function, where the pattern parameter can be arbitrary regular expressions and the replacement parameter can be variables. In the latter case,  we consider the leftmost and longest semantics mainly for simplicity, and the decision procedure can be adapted to the leftmost and shortest semantics easily.

%%%%%%%%%%%%%%%%%%%%%%%%%%%%%%%%%%%%%%%%%%
%%%%%%%%%%%%%%%%%%%%%%%%%%%%%%%%%%%%%%%%%%
\hide{
 For the string operations, the authors focus on two common ones: concatenation and replacement. The latter is close to---but not the same as---the $\replaceall$ function considered in this paper. However, in this paper, Yu et al provided   an \emph{over-approximation} of more %restricted replace 
commonly used semantics, i.e., the longest match and first match semantics. 
%
%\cite{SMV12} translating regular expression matching into transducer. 
%
They use deterministic finite automata (DFAs) to represent possible values of string variables. Using forward reachability analysis we compute an over-approximation of all possible values that string variables can take at each program point. They also implemented Stranger, an automata-based string analysis tool, with experiments. In general, this is essentially an abstract interpretation based approach.  In comparison, our algorithm is also automata-based, but we work on a semantics of $\replaceall$, but not its approximation. \tl{more need to be said here}
}
%%%%%%%%%%%%%%%%%%%%%%%%%%%%%%%%%%%%%%%%%%
%%%%%%%%%%%%%%%%%%%%%%%%%%%%%%%%%%%%%%%%%%

%Intersecting these with a given attack pattern yields the potential attack strings if the program is vulnerable. Based on the presented techniques, we have implemented Stranger, an automata-based string analysis tool for detecting string-related security vulnerabilities in PHP applications. We evaluated Stranger on several open-source Web applications including one with 350,000+ lines of code. Stranger is able to detect known/unknown vulnerabilities, and, after inserting proper sanitization routines, prove the absence of vulnerabilities with respect to given attack patterns.

 
The S3 and S3P tools also support the $\replaceall$ function, where some
progressive searching strategies were provided to deal with the non-termination
problem brought out by the recursively defined string operations (of which the
$\replaceall$ function is a special case). Nevertheless, the solvers are 
incomplete as reasoning about unbounded strings defined recursively is in 
general an undecidable problem.


%, the authors present S3 (which stands for Symbolic String Solver), a new symbolic string solver.
%The constraint language covers all the main string operations including the replace-all function. The authors 
%provided algorithms which make use of a symbolic representation so that membership in a set defined by a regular expression can be encoded as string equations. 

%To amplify this point, let us now state some statistics from a comprehensive
%study of practical JavaScript applications [28]. Constraints
%arising from the applications have an average (per benchmark
%query) of 63 JavaScript string operations, while the remaining
%are boolean, logical and arithmetic constraints. The largest fraction
%are for operations like indexOf, length (78%). A significant
%fraction of the operations, including substring (5%), replace
%(8%), and split, match (1%). Of the match, split and
%replace operations, 31% are based on regular expressions. Operations
%such as replace and split give rise to new strings
%from the original ones, thereby giving rise to constraints involving
%multiple string variables.



%. The algorithm first makes use of a symbolic representation
%so that membership in a set defined by a regular expression
%can be encoded as string equations. Secondly, there is a constraint based
%generation of instances from these symbolic expressions so
%that the total number of instances can be limited. 
%
%We evaluate S3 on a well-known set of practical benchmarks, demonstrating both
%its robustness (more definitive answers) and its efficiency (about 20
%times faster) against the state-of-the-art.



\hide{
%Progressive Reasoning over Recursively-Defined Strings
\cite{TCJ16} 
Trinh et al considered %the problem of reasoning over an expressive constraint language for unbounded strings. 
%In particular, they considered 
recursively defined string functions, a very expressive way to define functions manipulating strings. This includes a recursive definition of the replace-all function considered in this paper\footnote{\cite{TCJ16} used the notation \textbf{replace}}. The authors argue that ``the difficulty comes from ``recursively defined" functions such as replace, making state-of-the-art algorithms non-terminating." They proposed a progressive search algorithm, %to not only mitigate the problem of non-terminating reasoning but also guide the search towards a “minimal solution” when the input formula is in fact satisfiable. We have 
implemented within the state-of-the-art Z3 framework, with experimental evaluations. The algorithm is genetic and  applicable to all recursively defined string functions, but it is doomed to be incomplete as reasoning about unbounded strings defined recursively is in general an undecidable problem.   

The focus of our work is on the fundamental issue of decidability, and this is complementary to the work. Our result may be considered a completeness guarantee for existing string solver. 
}

%Importantly, we have enabled conflict clause learning for string theory so that our solver can be used effectively in the setting of program verification. Finally, our experimental evaluation shows leadership in a large benchmark suite, and a first deployment for another benchmark suite which requires reasoning about string formulas of a class that has not been solved before.



%===========================================================
%
%\subsection*{Other related work}
 



%Symbolic Finite State Transducers:
%Algorithms and Applications

%
%Finite automata and finite transducers are used in a wide
%range of applications in software engineering, from regular
%expressions to specification languages. We extend these
%classic objects with symbolic alphabets represented as parametric
%theories. Admitting potentially infinite alphabets
%makes this representation strictly more general and succinct
%than classical finite transducers and automata over strings.
%Despite this, the main operations, including composition,
%checking that a transducer is single-valued, and equivalence
%checking for single-valued symbolic finite transducers are
%effective given a decision procedure for the background theory.
%We provide novel algorithms for these operations and
%extend composition to symbolic transducers augmented with
%registers. Our base algorithms are unusual in that they are
%nonconstructive, therefore, we also supply a separate model
%generation algorithm that can quickly find counterexamples
%in the case two symbolic finite transducers are not
%equivalent. The algorithms give rise to a complete decidable
%algebra of symbolic transducers. Unlike previous work, we
%do not need any syntactic restriction of the formulas on the
%transitions, only a decision procedure. In practice we leverage
%recent advances in satisfiability modulo theory (SMT)
%solvers. We demonstrate our techniques on four case studies,
%covering a wide range of applications. Our techniques
%can synthesize string pre-images in excess of 8, 000 bytes
%in roughly a minute, and we find that our new encodings
%significantly outperform previous techniques in succinctness
%and speed of analysis
% 


