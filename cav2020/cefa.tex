%!TEX root = main.tex

%\subsection{Cost-enriched regular languages and recognisable relations}

Let $k \in \Nat$ with $k > 0$. A \emph{$k$-cost-enriched string} is $(w, (n_1, \cdots, n_k))$ where $w$ is a string and $n_i \in \intnum$ for all $i \in [k]$. 
A \emph{$k$-cost-enriched language} $L$ is a subset of $\Sigma^* \times \intnum^k$. 
%Note that all the cost-enriched strings in $L$ are associated with the same number of costs (i.e., $k$).

\begin{definition}[Cost-enriched regular languages]
Let $k \in \Nat$ with $k > 0$. A $k$-cost-enriched language is \emph{regular} (abbreviated as CERL) if it can be defined by a \emph{cost-enriched finite automaton}. A cost-enriched finite automaton (CEFA) $\CEFA$ is a tuple $(Q, \Sigma, R, \delta, I, F)$ where 
\begin{itemize}
\item $Q, \Sigma, I, F$ are defined as in NFAs, 
%
\item $R=(r_1, \cdots, r_k)$ is a vector of (mutually distinct) \emph{cost registers}, 
%
\item $\delta$ is the transition relation which is a finite set of tuples $(q, a, q', \eta)$ where $q, q' \in Q$, $a \in \Sigma$, and $\eta: R \rightarrow \intnum$ is a cost register update function. \\
For convenience, we usually write $(q, a, q', \eta) \in \Delta$ as $q \xrightarrow{a, \eta} q'$.
\end{itemize}
%
A \emph{run} of $\CEFA$ on a $k$-cost-enriched string $(a_1 \cdots a_m, (n_1, \cdots,n_k))$ is a  transition sequence $q_0 \xrightarrow{a_1, \eta_1} q_1 \cdots q_{m-1} \xrightarrow{a_m, \eta_m} q_m$ such that $q_0 \in I$ and $n_i = \sum \limits_{1\leq j\leq m}\eta_j(r_i)$ for each $i \in [k]$ (Note that the initial values of cost registers are zero). The run is \emph{accepting} if $q_m \in F$. A $k$-cost-enriched string $(w, (n_1, \cdots,n_k))$ is accepted by $\CEFA$ if there is an accepting run of $\CEFA$ on $(w, (n_1, \cdots,n_k))$. In particular, $(\varepsilon, n)$ is accepted by $\CEFA$ if $n=0$ and $I\cap F \neq \emptyset$.
The $k$-cost-enriched language defined by $\CEFA$, denoted by $\Lang(\CEFA)$, is the set of $k$-cost-enriched strings accepted by $\CEFA$. 
%A cost-enriched language $L \subseteq \Sigma^* \times \intnum^k$ is called a cost-enriched regular language (CERL) if there is a CEFA $\NFA$ such that $L = \Lang(\NFA)$.
\end{definition}

\begin{remark}
CEFAs can be seen as a variant of Cost Register Automata \cite{RLJ+13}, by admitting nondeterminism and discarding partial final cost functions. 
\end{remark}

\begin{example}[CEFA for the function $\length$]\label{exm:len}
Let $L$ be a regular language defined by an NFA $\NFA=(Q, \Sigma, \delta, I, F)$. 
Then the $1$-cost-enriched regular language $L' = \{(w, |w|) \mid w \in L\}$ is a CERL, defined by the CEFA $\CEFA' = (Q, \Sigma, R=(r_1), \delta', I, F)$, where $\delta'$ comprises the tuples $(q, a, q', \eta)$ such that $(q,a,q') \in \delta$ and $\eta(r_1) = 1$.  In particular, if $L=\Sigma^*$, then $L' = \{(w, |w|) \mid w \in \Sigma^*\}$ is defined by the CEFA $\CEFA'$ with $Q =I=F= \{q_0\}$ and $\delta' = \{(q_0, a, q_0, \eta)\}$.
\end{example}

Given two CEFAs $\NFA_1 = ( Q_1, \Sigma, R_1, \delta_1, I_1, F_1)$ and $\NFA_2 = (Q_2, \Sigma, \delta_2, R_2, I_2, F_2)$ with $R_1 \cap R_2 = \emptyset$ (where the notation is abused a bit, viewing $R_1$ and $R_2$ as sets),  the product of $\NFA_1$ and $\NFA_2$, denoted by $\NFA_1 \times \NFA_2$, is defined as $(Q_1 \times Q_2, \Sigma, R_1 \cup R_2, \delta, I_1 \times I_2, F_1 \times F_2)$, where $\delta$ comprises the tuples $((q_1, q_2), \sigma, (q'_1, q'_2), \eta)$ such that $(q_1, \sigma, q'_1, \eta_1) \in \delta_1$, $(q_2, \sigma, q'_2, \eta_2) \in \delta_2$, and $\eta = \eta_1\cup \eta_2$.  %for some $\eta_1, \eta_2$.


For a CEFA $\CEFA$, we use $R(\CEFA)$ to denote the vector of cost registers occurring in $\CEFA$. Note that cost registers of $\CEFA$ are simply integer variables to store costs in $\CEFA$.
%Moreover, for a CEFA $\NFA$ and 
Suppose $\CEFA$ is  CEFA with $R(\CEFA)=(r_1,\cdots, r_k)$ and $\vec{i} = (i_1,\cdots, i_k)$ with $R(\NFA) \cap \vec{i} = \emptyset$. We use $\CEFA[\vec{i}/R(\CEFA)]$ to denote the CEFA obtained from $\CEFA$ by simultaneously replacing $r_j$ with $i_j$ for every $j \in [k]$ in $\CEFA$. 

\smallskip

%Let $(k_1,\cdots, k_l) \in \Nat^l$ with $k_j > 0$ for every $j \in [l]$. A  $(k_1,\cdots, k_l)$-cost-enriched relation $\cR$ is a subset of $(\Sigma^* \times \intnum^{k_1}) \times \cdots (\Sigma^* \times \intnum^{k_l})$.
\begin{definition}[Cost-enriched recognisable relations]
Let $(k_1,\cdots, k_l) \in \Nat^l$ with $k_j > 0$ for every $j \in [l]$. Intuitively, a cost-enriched recognisable relation (CERR)  $\cR \subseteq (\Sigma^* \times \intnum^{k_1}) \times \cdots  \times (\Sigma^* \times \intnum^{k_l})$ is a finite union of products of CERLs. Formally,
	\[\cR = \bigcup \limits_{i=1}^n L_{i,1 } \times \cdots \times L_{i, l},\]
	where for every $j \in [l]$, $L_{i,j} \subseteq \Sigma^* \times \intnum^{k_j}$ is a CERL. 
	A CEFA representation of $\cR$ is a collection of CEFA tuples $(\CEFA_{i,1}, \cdots, \CEFA_{i,l})_{i \in [n]}$ such that $\Lang(\CEFA_{i,j}) = L_{i,j}$ for every $i \in [n]$ and $j \in [l]$.
\end{definition}

\begin{example}
The relation 
\[\cR=\{((w_1, |w_1|), (w_2, |w_2|)) \mid  w_1 \in \Lang((aa)^*), w_2 \in \Lang(b(bb)^*), |w_1|+|w_2| \ge 2\}\] 
is a CERR since it is represented by CEFA tuples $(\CEFA_{i,1}, \CEFA_{i,2})_{i \in [2]}$, where 
\begin{itemize}
\item $\CEFA_{1,1} = (\{p_0\}, \{a,b\}, (r_{1,1}), \emptyset, \{p_0\}, \{p_0\})$ recognises only the empty string $\varepsilon$, $\CEFA_{1,2}=(\{q_0,q_1,q_2\}, \{a,b\}, (r_{1,2}), \{(q_0, b, q_1), (q_1, b, q_2), (q_2, b, q_1)\}, \{q_0\}, \{q_1\})$,
\item 
\end{itemize}
\end{example}

%Note that according to the definition, over each word $w$, $\NFA^{(r)}(w) = \NFA(w)$. 

\begin{definition}[LA-SAT w.r.t. CEFA]\label{def-la-sat-cefa}
	Given $k$ CEFAs $\NFA_i=(Q_i, \Sigma, R_i, \delta_i, I_i, F_i)$ for $i\in [k]$, % , $\cdots$, $\NFA_k=(Q_k, \Sigma, R_k, \delta_k, I_k, F_k)$ be CEFAs
	%  with $R_i = (r_{i,1}, \cdots, r_{i, l_i})$ for each $i \in [k]$, 
	and  a quantifier-free linear arithmetic formula $\phi$ 
	%    
	whose free variables contain $\bigcup_{i\in [k]} R_i$. Then $\phi$ is said to be satisfiable w.r.t. $\NFA_1, \cdots, \NFA_k$ if  there are words $w_1, \cdots, w_k$ and an assignment function $\eta$ %: R_1 \cup \cdots R_k \cup X \rightarrow \intnum$  %integers $\vec{c_1} \in \intnum^{|R_1|}, \cdots, \vec{c_k} \in \intnum^{|R_k|}$, $\vec{d} \in \intnum^{|X|}$
	such that  $(w_i, \eta(R_i)) \in \Lang(\NFA_i)$ for $1\leq i\leq k$, and $\phi$ holds under $\eta$.
	%
	%  
	%  whose free variables are from  $R_1 \cup \cdots \cup R_k \cup X$ for some $X$ such that $X \cap (R_1 \cup \cdots \cup R_k) = \emptyset$. Then $\phi$ is said to be satisfiable w.r.t. $\NFA_1, \cdots, \NFA_k$ if  there are words $w_1, \cdots, w_k$ and an assignment function $\eta: R_1 \cup \cdots R_k \cup X \rightarrow \intnum$  %integers $\vec{c_1} \in \intnum^{|R_1|}, \cdots, \vec{c_k} \in \intnum^{|R_k|}$, $\vec{d} \in \intnum^{|X|}$
	% such that  $(w_1, \eta(R_1)) \in \Lang(\NFA_1)$, $ \cdots$, $(w_k, \eta(R_k)) \in \Lang(\NFA_k)$, and $\phi[\eta(R_1)/R_1, \cdots,\eta(R_k)/R_k, \eta(X)/X]$ holds.
\end{definition}
Note that in Definition~\ref{def-la-sat-cefa}, registers in $\NFA_i$'s may intersect. %it may happen that $R_i \cap R_j \neq \emptyset$ for some $i, j \in [k]$ with $i \neq j$.

\begin{example}
\end{example}

\begin{theorem}\label{thm-incra-la-sat}
	The LA-SAT w.r.t. CEFA problem is decidable.
\end{theorem}

For the proof of Theorem~\ref{thm-incra-la-sat}, we state and prove the following lemma. 

\begin{lemma}\label{lem-incra-la}
	Let $\NFA=(Q, \Sigma, R, \delta, I, F)$ be a CEFA with $R= (r_1, \cdots,  r_m)$. Then there is an existential linear arithmetic formula $\varphi_\NFA(r_1, \cdots, r_m)$ such that $\cM(\varphi_\NFA)= \{(c_1, \cdots, c_m) \mid \exists w.  (w, c_1, \cdots, c_m) \in \Lang(\NFA)\}$.
\end{lemma}

\begin{proof}
	Let $\delta = \{\tau_1, \cdots, \tau_l\}$ such that $\tau_j = (p_j, \sigma_j, p'_j, \eta_j)$ and $\eta_j(r_i) =  c_{i,j}$ for each $j \in [l]$ and $i \in [m]$.
	According the results on FA, we know that for each pair of states $(q, q') \in I \times F$,  an existential linear arithmetic formula $\varphi_{q,q'}(j_1, \cdots, j_l)$ can be computed in linear time such that $\cM(\varphi_{q,q'})$ is the set of Parikh images of the transition sequences of $\NFA$ starting from $q$ and ending at $q'$. 
	
	Then 
	\[\varphi_\NFA(r_1, \cdots, r_m) ::= \bigvee \limits_{(q,q') \in I \times F} \exists j_1 \cdots \exists j_l.\ \left(\varphi_{q,q'}(j_1, \cdots, j_l) \wedge \bigwedge \limits_{i \in [m]} r_i = \sum \limits_{j \in [l]} c_{i,j} j_j \right).\]
	\qed
\end{proof}

\begin{proof}[Theorem~\ref{thm-incra-la-sat}]
	Let $\NFA_1=(Q_1, \Sigma, R_1, \delta_1, I_1, F_1)$, $\cdots$, $\NFA_k=(Q_k, \Sigma, R_k, \delta_k, I_k, F_k)$ be CEFAs and $\phi$ be a quantifier-free linear arithmetic formula whose free variables are from  $R_1 \cup \cdots \cup R_k \cup X$ for some $X$ such that $X \cap (R_1 \cdots R_k) = \emptyset$.
	Suppose that for each $i \in [k]$, $R_i = (r_{i, 1}, \cdots, r_{i, l_i})$. Then the satisfiability of $\phi$ w.r.t. $\NFA_1,\cdots, \NFA_k$ can be reduced to the satisfiability problem of the  existential linear arithmetic formula
	$
	\phi \wedge \bigwedge \limits_{i \in [k]} \varphi_{\NFA_i}(r_{i,1}, \cdots, r_{i, l_i}).
	$
\end{proof}

