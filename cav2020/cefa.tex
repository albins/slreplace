%!TEX root = main.tex


In this section, we introduce cost-enriched regular languages and recognisable relations, as the extensions of regular languages and recognisable relations, moreover, we investigate the decidability and complexity of a related decision problem, thus laying down the theoretical foundations of the decision procedure in the next section. 

\subsection{Cost-Enriched Regular Languages and Recognisable Relations}

Let $k \in \Nat$ with $k > 0$. A \emph{$k$-cost-enriched string} is $(w, (n_1, \cdots, n_k))$ where $w$ is a string and $n_i \in \intnum$ for all $i \in [k]$. 
A \emph{$k$-cost-enriched language} $L$ is a subset of $\Sigma^* \times \intnum^k$. For our purpose, we identify a ``regular" fragment of cost-enriched languages as follows. 
%Note that all the cost-enriched strings in $L$ are associated with the same number of costs (i.e., $k$).

\begin{definition}[Cost-enriched regular languages]
Let $k \in \Nat$ with $k > 0$. A $k$-cost-enriched language is \emph{regular} (abbreviated as CERL) if it can be accepted by a \emph{cost-enriched finite automaton}. 

A cost-enriched finite automaton (CEFA) $\CEFA$ is a tuple $(Q, \Sigma, R, \delta, I, F)$ where 
\begin{itemize}
\item $Q, \Sigma, I, F$ are defined as in NFAs, 
%
\item $R=(r_1, \cdots, r_k)$ is a vector of (mutually distinct) \emph{cost registers}, 
%
\item $\delta$ is the transition relation which is a finite set of tuples $(q, a, q', \eta)$ where $q, q' \in Q$, $a \in \Sigma$, and $\eta: R \rightarrow \intnum$ is a cost register update function. \\
For convenience, we usually write $(q, a, q', \eta) \in \Delta$ as $q \xrightarrow{a, \eta} q'$.
\end{itemize}
%
A \emph{run} of $\CEFA$ on a $k$-cost-enriched string $(a_1 \cdots a_m, (n_1, \cdots,n_k))$ is a  transition sequence $q_0 \xrightarrow{a_1, \eta_1} q_1 \cdots q_{m-1} \xrightarrow{a_m, \eta_m} q_m$ such that $q_0 \in I$ and $n_i = \sum \limits_{1\leq j\leq m}\eta_j(r_i)$ for each $i \in [k]$ (Note that the initial values of cost registers are zero). The run is \emph{accepting} if $q_m \in F$. A $k$-cost-enriched string $(w, (n_1, \cdots,n_k))$ is accepted by $\CEFA$ if there is an accepting run of $\CEFA$ on $(w, (n_1, \cdots,n_k))$. In particular, $(\varepsilon, n)$ is accepted by $\CEFA$ if $n=0$ and $I\cap F \neq \emptyset$.
The $k$-cost-enriched language defined by $\CEFA$, denoted by $\Lang(\CEFA)$, is the set of $k$-cost-enriched strings accepted by $\CEFA$. 
%A cost-enriched language $L \subseteq \Sigma^* \times \intnum^k$ is called a cost-enriched regular language (CERL) if there is a CEFA $\NFA$ such that $L = \Lang(\NFA)$.
\end{definition}
The \emph{size} of a CEFA $\CEFA=(Q, \Sigma, R, \delta, I, F)$, denoted by $|\CEFA|$, is defined as the sum of the sizes of its transitions, where the size of each transition $(q, a, q', \eta)$ is $\sum \limits_{r \in R} \lceil \log_2 (|\eta(r)|) \rceil +3$. Note here  the integer constants in $\CEFA$ are encoded in binary.

\begin{remark}
CEFAs can be seen as a variant of Cost Register Automata \cite{RLJ+13}, by admitting nondeterminism and discarding partial final cost functions. 
\end{remark}

\begin{example}[CEFA for $\length$]\label{exm:len}
The string function $\length$ can be captured by CEFA. For any NFA $\NFA=(Q, \Sigma,  \delta, I, F)$, it is not difficult to see that the cost-enriched language $\{(w, \length(w)) \mid w\in \Lang(\NFA)\}$ is accepted by a $1$-cost-enriched automaton, i.e.
$\CEFA_{\rm len}=(Q, \Sigma, (r_1), \delta', I, F)$ such that %$Q =I=F= \{q_0\}$, $R=(r_1)$, and 
such that for each $(q, a, q')\in \delta$, we let $(q, a, q', \eta)\in \delta'$, where $\eta(r_1) = 1$.
%\tl{it is a bit vague; you mean for any RE $e$,  is a cerl?}
\end{example}

We can show that the function $\indexof_v(\cdot, \cdot)$ can be captured by CEFA as well, in the sense that, for any NFA $\NFA$ and constant string $v$, we can construct a $2$-cost-enriched automaton $\CEFA_{\indexof_v}$ such that $\Lang(\CEFA_{\indexof_v})=\{(w, (c, \indexof_v(w, c))\mid w\in \Lang(\NFA) \}$. The details are slightly technical and will be given in Appendix, Section~\ref{appendix:cefa_indexof}. 



Given two CEFAs $\NFA_1 = ( Q_1, \Sigma, R_1, \delta_1, I_1, F_1)$ and $\NFA_2 = (Q_2, \Sigma, \delta_2, R_2, I_2, F_2)$ with $R_1 \cap R_2 = \emptyset$ (where %the notation is abused a bit, viewing 
$R_1$ and $R_2$ are treated as sets), the product of $\NFA_1$ and $\NFA_2$, denoted by $\NFA_1 \times \NFA_2$, is defined as $(Q_1 \times Q_2, \Sigma, R_1 \cup R_2, \delta, I_1 \times I_2, F_1 \times F_2)$, where $\delta$ comprises the tuples $((q_1, q_2), \sigma, (q'_1, q'_2), \eta)$ such that $(q_1, \sigma, q'_1, \eta_1) \in \delta_1$, $(q_2, \sigma, q'_2, \eta_2) \in \delta_2$, and $\eta = \eta_1\cup \eta_2$.  %for some $\eta_1, \eta_2$.


For a CEFA $\CEFA$, we use $R(\CEFA)$ to denote the vector of cost registers occurring in $\CEFA$. %Note that cost registers of $\CEFA$ are simply integer variables to store costs in $\CEFA$.
%Moreover, for a CEFA $\NFA$ and 
Suppose $\CEFA$ is  CEFA with $R(\CEFA)=(r_1,\cdots, r_k)$ and $\vec{i} = (i_1,\cdots, i_k)$ with $R(\NFA) \cap \vec{i} = \emptyset$. We use $\CEFA[\vec{i}/R(\CEFA)]$ to denote the CEFA obtained from $\CEFA$ by simultaneously replacing $r_j$ with $i_j$ for $j \in [k]$. 

\smallskip

%Let $(k_1,\cdots, k_l) \in \Nat^l$ with $k_j > 0$ for every $j \in [l]$. A  $(k_1,\cdots, k_l)$-cost-enriched relation $\cR$ is a subset of $(\Sigma^* \times \intnum^{k_1}) \times \cdots (\Sigma^* \times \intnum^{k_l})$.

%%%%%%%%%%%%%%%%%%%%%%%%%%%%%%%%%%%%%%%%%%%%%%%%%%%%%%%%%%%%%%%%%%%%%%%%%%%%%%%%%%%%%%%%%%%%%%%
\begin{definition}[Cost-enriched recognisable relations]
Let $(k_1,\cdots, k_l) \in \Nat^l$ with $k_j > 0$ for every $j \in [l]$. A cost-enriched recognisable relation (CERR)  $\cR \subseteq (\Sigma^* \times \intnum^{k_1}) \times \cdots  \times (\Sigma^* \times \intnum^{k_l})$ is a finite union of products of CERLs. Formally,
	\[\cR = \bigcup \limits_{i=1}^n L_{i,1 } \times \cdots \times L_{i, l},\]
	where for every $j \in [l]$, $L_{i,j} \subseteq \Sigma^* \times \intnum^{k_j}$ is a CERL. 
	A CEFA representation of $\cR$ is a collection of CEFA tuples $(\CEFA_{i,1}, \cdots, \CEFA_{i,l})_{i \in [n]}$ such that $\Lang(\CEFA_{i,j}) = L_{i,j}$ for every $i \in [n]$ and $j \in [l]$.
\end{definition}

\begin{example}\label{exm:CERR}
The relation 
\[\cR=\left\{((w_1, |w_1|), (w_2, |w_2|)) \mid  w_1 \in \Lang((aa)^*), w_2 \in \Lang(b(bb)^*), |w_1|+|w_2| \ge 2\right\}\] 
is a CERR since 
$\cR = L_{1,1} \times L_{1,2} \cup L_{2,1} \times L_{2,2}$, where 
\begin{itemize}
\item $L_{1,1}=\{(w_1, |w_1|) \mid w_1 \in \Lang((aa)^*)\}$, 
\item $L_{1,2}=\{(w_2, |w_2|) \mid  w_2 \in \Lang(bbb(bb)^*)\}$ , 
\item $L_{2,1}=\{(w_1, |w_1|) \mid w_1 \in \Lang(aa(aa)^*)\}$,  and
\item $L_{2,2}=\{(w_2, |w_2|) \mid  w_2 \in \Lang(b(bb)^*)\}$. 
\end{itemize}
Note that $L_{i,j}$ for $i,j\in[2]$ are CERLs, with corresponding CEFAs $\CEFA_{i,j}$ by  Example~\ref{exm:len}. It follows that %Moreover, 
$(\CEFA_{i,1}, \CEFA_{i,2})_{i \in [2]}$ gives a CEFA representation of $\cR$. %, where 
%\tl{to be simplified.}
%\begin{itemize}
%\item  $\CEFA_{1,1} = (\{p_0, p_1\}, \{a,b\}, (r_{1}), \delta_{1,1}, \{p_0\}, \{p_0\})$ defines $L_{1,1}$, such that $\delta_{1,1}= \{(p_0, a, p_1,\eta_1)$, $(p_1, a, p_0, \eta_1)\}$, with $\eta_1(r_{1})=1$,
%%
%\item $\CEFA_{1,2}=(\{q_0,q_1,q_2\}, \{a,b\}, (r_{2}), \delta_{1,2}, \{q_0\}, \{q_1\})$ defines $L_{1,2}$, such that $\delta_{1,2}=\{(q_0, b, q_1,\eta_2), (q_1, b, q_2, \eta_2), (q_2, b, q_1,\eta_2)\}$, with $\eta_2(r_{2})=1$,
%%
%\item $\CEFA_{2,1} = (\{p_0, p_1,p_2, p_3\}, \{a,b\}, (r_{1}), \delta_{2,1}, \{p_0\}, \{p_2\})$ defines $L_{2,1}$, such that $\delta_{2,1} = \{(p_0, a, p_1, \eta_1), (p_1, a, p_2, \eta_1), (p_2, a, p_3, \eta_1), (p_3, a, p_2, \eta_1)\}$, with $\eta_1(r_{1})=1$, and
%%
%\item $\CEFA_{2,2}=(\{q_0,q_1,q_2\}, \{a,b\}, (r_{2}), \delta_{2,2}, \{q_1\})$ defines $L_{2,2}$, such that $\delta_{2,2}= (q_0, b, q_1,\eta_2), (q_1, b, q_2, \eta_2), (q_2, b, q_1,\eta_2)\}$, with $\eta_2(r_{2})=1$.
%\end{itemize}
\end{example}

\subsection{Satisfiability of Linear Integer Arithmetic Formula with respect to CEFAs}

In the sequel, we consider a decision problem for CEFAs which will be used for the decision procedure of the path feasibility problem for {\slint} in the next section.

\begin{definition}[{\lasat} Problem]\label{def-la-sat-cefa}
	The satisfiability problem of LIA formulas w.r.t. CEFAs (abbreviated as {\lasat} problem) is defined as follows.
	
	\textbf{Input}: a given quantifier-free LIA formula $\phi$ and CEFAs $\CEFA_1,\cdots,\CEFA_m$, such that $\CEFA_i=(Q_i, \Sigma, R_i, \delta_i, I_i, F_i)$ for every $i\in [m]$, 
  $R_i \cap R_j = \emptyset$ for every $1 \le i < j \le m$, and
 the free variables of $\phi$ are from $\bigcup_{i\in [m]} R_i$, 
 
Decide whether %$\phi$ is satisfiable w.r.t. $\CEFA_1, \cdots, \CEFA_m$, namely, whether 
there are an assignment function $\theta: \bigcup \limits_{i \in [m]} R_i \rightarrow \Int$ and strings $w_1, \cdots, w_m$  
	such that  $\phi[(\theta(R_i)/R_i)_{i \in [m]}]$ hold and $(w_i, \theta(R_i)) \in \Lang(\NFA_i)$ for every $i \in [m]$.
\end{definition}
%Note that in Definition~\ref{def-la-sat-cefa}, registers in $\NFA_i$'s may intersect. 

\begin{example}
Let $\phi \equiv r_1 = r_2$ and $\CEFA_{1,1}, \CEFA_{1,2}$ be two CEFAs in Example~\ref{exm:CERR} defining $L_{1,1}, L_{1,2}$ respectively. (Recall that $R(\CEFA_{1,1})=(r_1)$ and $R(\CEFA_{1,2})=(r_2)$.) Then it is easy to see that $\phi$ is unsatisfiable w.r.t. $\CEFA_{1,1}$ and $\CEFA_{1,2}$, since for each $(w_1, n_1) \in L_{1,1}$, $n_1$ must be even, while for each $(w_2, n_2) \in L_{1,2}$, $n_2$ must be odd. Hence $n_1$ and $n_2$ cannot be equal.
\end{example}

\begin{theorem}\label{thm-la-sat-cefa}
	The {\lasat} problem is NP-complete.
\end{theorem}
The proof is given in Appendix, Section~\ref{appendix:thm-la-sat-cefa}.