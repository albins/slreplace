%!TEX root = main.tex

A cost-enriched string is $(w, (n_1, \cdots, n_k))$ with $w$  a string and $n_i \in \intnum$ for all $i \in [k]$. 
A cost-enriched language $L$ is a subset of $\Sigma^* \times \intnum^k$ for some $k$. Note that all the cost-enriched strings in $L$ have the same number of costs, namely $k$.
A cost-enriched relation $\cR$ is a subset of $\Sigma^* \times \intnum^{k_1} \times \cdots \Sigma^* \times \intnum^{k_l}$.

\begin{definition}[Cost-enriched finite automata and regular languages]
A cost-enriched finite automaton (CEFA) $\Aut$ is a tuple $(Q, \Sigma, R, \delta, I, F)$ where $Q, \Sigma, I, F$ are as in FA, $R=(r_1, \cdots, r_k)$ is a vector of (mutually distinct) registers, $\delta$ is the transition relation which is a finite set of tuples $(q, \sigma, q', \eta)$ where $q, q' \in Q$, $\sigma \in \Sigma$, and $\eta: R \rightarrow \intnum$ is the cost-update function. For convenience, we usually write $(q, \sigma, q', \eta) \in \Delta$ as $q \xrightarrow{\sigma, \eta} q'$.
\\
A cost-enriched string $(w, (n_1, \cdots,n_k)) \in \Sigma^* \times \intnum^k$ with $w=\sigma_1 \cdots \sigma_m$ is accepted by $\Aut$ if there is a sequence of transitions $q_0 \xrightarrow{\sigma_1, \eta_1} q_1 \cdots q_{m-1} \xrightarrow{\sigma_m, \eta_m} q_m$ such that $n_i = \eta_1(r_i) + \cdots + \eta_m(r_i)$ for each $i \in [k]$. The set of cost-enriched strings accepted by $\Aut$ is denoted as $\Lang(\Aut)$. A cost-enriched language $L \subseteq \Sigma^* \times \intnum^k$ is called a cost-enriched regular language (CERL) if there is a CEFA $\Aut$ such that $L = \Lang(\Aut)$.
\end{definition}
CEFA can be seen as a variant of CRA in \cite{RLJ+13}, by adding the nondeterminism and discarding the partial final cost function. 

For a CEFA $\Aut$, we use $R(\Aut)$ to denote the vector of registers occurring in $\Aut$. Moreover, for a CEFA $\Aut$ and a vector of integer variables $\vec{i}$ such that $R(\Aut) \cap \vec{i} = \emptyset$, we use $\Aut[\vec{i}/R(\Aut)]$ to denote the CEFA obtained from $\Aut$ by replacing the registers in $R(\Aut)$ with those from $\vec{i}$. 

%Let $\Aut = (Q, \Sigma, R, \delta, I, F)$ be a CEFA and $I',F' \in Q$. Then we use $\Aut_{I', F'}$ to denote the CEFA $(Q, \Sigma, R, \delta, I', F')$.

Given two CEFAs $\Aut_1 = ( Q_1, \Sigma, R_1, \delta_1, I_1, F_1)$ and $\Aut_2 = (Q_2, \Sigma, \delta_2, R_2, I_2, F_2)$ with $R_1 \cap R_2 = \emptyset$, we define the product of $\Aut_1$ and $\Aut_2$, denoted by $\Aut_1 \times \Aut_2$, as $(Q_1 \times Q_2, \Sigma, R_1 \cup R_2, \delta, I_1 \times I_2, F_1 \times F_2)$ such that $\delta$ comprises the tuples $((q_1, q_2), \sigma, (q'_1, q'_2), \eta)$ satisfying that $(q_1, \sigma, q'_1, \eta_1) \in \delta_1$, $(q_2, \sigma, q'_2, \eta_2) \in \delta_2$, and $\eta = \eta_1\cup \eta_2$ for some $\eta_1, \eta_2$.

%Note that according to the definition, over each word $w$, $\Aut^{(r)}(w) = \Aut(w)$. 

\begin{definition}[LA-SAT w.r.t. CEFA]\label{def-la-sat-cefa}
Let $\Aut_1=(Q_1, \Sigma, R_1, \delta_1, I_1, F_1)$, $\cdots$, $\Aut_k=(Q_k, \Sigma, R_k, \delta_k, I_k, F_k)$ be CEFAs
%  with $R_i = (r_{i,1}, \cdots, r_{i, l_i})$ for each $i \in [k]$, 
  and $\phi$ be a quantifier-free linear arithmetic formula whose free variables are from  $R_1 \cup \cdots \cup R_k \cup X$ for some $X$ such that $X \cap (R_1 \cup \cdots \cup R_k) = \emptyset$. Then $\phi$ is said to be satisfiable w.r.t. $\Aut_1, \cdots, \Aut_k$ if  there are words $w_1, \cdots, w_k$ and an assignment function $\eta: R_1 \cup \cdots R_k \cup X \rightarrow \intnum$  %integers $\vec{c_1} \in \intnum^{|R_1|}, \cdots, \vec{c_k} \in \intnum^{|R_k|}$, $\vec{d} \in \intnum^{|X|}$
 such that  $(w_1, \eta(R_1)) \in \Lang(\Aut_1)$, $ \cdots$, $(w_k, \eta(R_k)) \in \Lang(\Aut_k)$, and $\phi[\eta(R_1)/R_1, \cdots,\eta(R_k)/R_k, \eta(X)/X]$ holds.
\end{definition}
Note that in Definition~\ref{def-la-sat-cefa}, it may happen that $R_i \cap R_j \neq \emptyset$ for some $i, j \in [k]$ with $i \neq j$.

\begin{theorem}\label{thm-incra-la-sat}
The LA-SAT w.r.t. CEFA problem is decidable.
\end{theorem}

For the proof of Theorem~\ref{thm-incra-la-sat}, we state and prove the following lemma. 

\begin{lemma}\label{lem-incra-la}
Let $\Aut=(Q, \Sigma, R, \delta, I, F)$ be a CEFA with $R= (r_1, \cdots,  r_m)$. Then there is an existential linear arithmetic formula $\varphi_\Aut(r_1, \cdots, r_m)$ such that $\cM(\varphi_\Aut)= \{(c_1, \cdots, c_m) \mid \mbox{there exists } w \mbox{ such that } (w, c_1, \cdots, c_m) \in \Lang(\Aut)\}$.
\end{lemma}

\begin{proof}
Let $\delta = \{\tau_1, \cdots, \tau_l\}$ such that $\tau_j = (p_j, \sigma_j, p'_j, \eta_j)$ and $\eta_j(r_i) =  c_{i,j}$ for each $j \in [l]$ and $i \in [m]$.
According the results on FA, we know that for each pair of states $(q, q') \in I \times F$,  an existential linear arithmetic formula $\varphi_{q,q'}(j_1, \cdots, j_l)$ can be computed in linear time such that $\cM(\varphi_{q,q'})$ is the set of Parikh images of the transition sequences of $\Aut$ starting from $q$ and ending at $q'$. 

Then 
\[\varphi_\Aut(r_1, \cdots, r_m) ::= \bigvee \limits_{(q,q') \in I \times F} \exists j_1 \cdots \exists j_l.\ \left(\varphi_{q,q'}(j_1, \cdots, j_l) \wedge \bigwedge \limits_{i \in [m]} r_i = \sum \limits_{j \in [l]} c_{i,j} j_j \right).\]
\qed
\end{proof}

\begin{proof}[Theorem~\ref{thm-incra-la-sat}]
Let $\Aut_1=(Q_1, \Sigma, R_1, \delta_1, I_1, F_1)$, $\cdots$, $\Aut_k=(Q_k, \Sigma, R_k, \delta_k, I_k, F_k)$ be CEFAs and $\phi$ be a quantifier-free linear arithmetic formula whose free variables are from  $R_1 \cup \cdots \cup R_k \cup X$ for some $X$ such that $X \cap (R_1 \cdots R_k) = \emptyset$.
Suppose that for each $i \in [k]$, $R_i = (r_{i, 1}, \cdots, r_{i, l_i})$. Then the satisfiability of $\phi$ w.r.t. $\Aut_1,\cdots, \Aut_k$ can be reduced to the satisfiability problem of the  existential linear arithmetic formula
$
\phi \wedge \bigwedge \limits_{i \in [k]} \varphi_{\Aut_i}(r_{i,1}, \cdots, r_{i, l_i}).
$
\qed
\end{proof}

\begin{definition}[Cost-enriched recognisable relations]
A cost-enriched relation $\cR \subseteq \Sigma^* \times \intnum^{k_1} \times \cdots  \times \Sigma^* \times \intnum^{k_l}$ is a cost-enriched recognisable relation (CERR)  if it is a finite union of products of cost-enriched regular languages, namely, 
\[\cR = \bigcup \limits_{i=1}^n L_{i,1 } \times \cdots \times L_{i, l},\]
where $L_{i,1} \subseteq \Sigma^* \times \intnum^{k_1}, \cdots, L_{i, l} \subseteq \Sigma^* \times \intnum^{k_l}$ are CERL. 
A CEFA representation of $\cR$ is a collection of CERA tuples $(\Aut_{i,1}, \cdots, \Aut_{i,l})_{i \in [n]}$ such that $\Lang(\Aut_{i,j}) = L_{i,j}$ for each $i \in [n]$ and $j \in [l]$.
\end{definition}



%\subsection{The two semantic conditions}
%The first semantic condition we put is as follows: The relation defined by the integer functions $g(x_1, \vec{i_1}, \cdots, x_k, \vec{i_k})$ is 

