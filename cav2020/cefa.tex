%!TEX root = main.tex


In this section, we introduce cost-enriched regular languages and recognisable relations, as the extensions of regular languages and recognisable relations, moreover, we investigate the decidability and complexity of a related decision problem, thus laying down the theoretical foundations of the decision procedure in the next section. 

\subsection{Cost-Enriched Regular Languages and Recognisable Relations}

Let $k \in \Nat$ with $k > 0$. A \emph{$k$-cost-enriched string} is $(w, (n_1, \cdots, n_k))$ where $w$ is a string and $n_i \in \intnum$ for all $i \in [k]$. 
A \emph{$k$-cost-enriched language} $L$ is a subset of $\Sigma^* \times \intnum^k$. 
%Note that all the cost-enriched strings in $L$ are associated with the same number of costs (i.e., $k$).

\begin{definition}[Cost-enriched regular languages]
Let $k \in \Nat$ with $k > 0$. A $k$-cost-enriched language is \emph{regular} (abbreviated as CERL) if it can be defined by a \emph{cost-enriched finite automaton}. A cost-enriched finite automaton (CEFA) $\CEFA$ is a tuple $(Q, \Sigma, R, \delta, I, F)$ where 
\begin{itemize}
\item $Q, \Sigma, I, F$ are defined as in NFAs, 
%
\item $R=(r_1, \cdots, r_k)$ is a vector of (mutually distinct) \emph{cost registers}, 
%
\item $\delta$ is the transition relation which is a finite set of tuples $(q, a, q', \eta)$ where $q, q' \in Q$, $a \in \Sigma$, and $\eta: R \rightarrow \intnum$ is a cost register update function. \\
For convenience, we usually write $(q, a, q', \eta) \in \Delta$ as $q \xrightarrow{a, \eta} q'$.
\end{itemize}
%
A \emph{run} of $\CEFA$ on a $k$-cost-enriched string $(a_1 \cdots a_m, (n_1, \cdots,n_k))$ is a  transition sequence $q_0 \xrightarrow{a_1, \eta_1} q_1 \cdots q_{m-1} \xrightarrow{a_m, \eta_m} q_m$ such that $q_0 \in I$ and $n_i = \sum \limits_{1\leq j\leq m}\eta_j(r_i)$ for each $i \in [k]$ (Note that the initial values of cost registers are zero). The run is \emph{accepting} if $q_m \in F$. A $k$-cost-enriched string $(w, (n_1, \cdots,n_k))$ is accepted by $\CEFA$ if there is an accepting run of $\CEFA$ on $(w, (n_1, \cdots,n_k))$. In particular, $(\varepsilon, n)$ is accepted by $\CEFA$ if $n=0$ and $I\cap F \neq \emptyset$.
The $k$-cost-enriched language defined by $\CEFA$, denoted by $\Lang(\CEFA)$, is the set of $k$-cost-enriched strings accepted by $\CEFA$. 
%A cost-enriched language $L \subseteq \Sigma^* \times \intnum^k$ is called a cost-enriched regular language (CERL) if there is a CEFA $\NFA$ such that $L = \Lang(\NFA)$.
\end{definition}
The \emph{size} of  a CEFA $\CEFA=(Q, \Sigma, R, \delta, I, F)$, denoted by $|\CEFA|$, is defined as the sum of the sizes of its transitions, where the size of a transition $(q, a, q', \eta)$ is $\sum \limits_{r \in R} \lceil \log_2 (|\eta(r)|) \rceil +3$. Note here the binary encoding is used to represent the integer constants occurring in $\CEFA$.

\begin{remark}
CEFAs can be seen as a variant of Cost Register Automata \cite{RLJ+13}, by admitting nondeterminism and discarding partial final cost functions. 
\end{remark}

\begin{example}[CEFA for $\length$]\label{exm:len}
The string function $\length$ can be captured by the CEFA $\CEFA_{\rm len}=(Q, \Sigma, R, \delta, I, F)$ such that $Q =I=F= \{q_0\}$, $R=(r_1)$, and $\delta = \{(q_0, a, q_0, \eta)\}$, where $\eta(r_1) = 1$.
\tl{it is a bit vague; you mean for any RE $e$, $\{(w, \length(w)) \mid w\in e\}$ is a cerl?}
\end{example}
%
In the next example, we are going to show that for a given string $v \in \Sigma^+$, $\indexof_v$ can be captured by a CEFA $\CEFA_{\indexof_v}$. 
For this purpose, we need a concept of window profiles of  string positions w.r.t. $v$, which are elements of $\{\bot, \top\}^{n-1}$. The window profiles facilitate recognising the first occurrence of $v$ in the input string. 
Intuitively, given a string $u$, the window profile of a position $i$ in $u$ w.r.t. $v$ encodes the matchings of prefixes of $v$ to the suffixes of $u[0,i]$ (see \cite{CCH+18} for the details). For $\pi = \pi_1 \cdots \pi_{n-1} \in \{\bot, \top\}^{n-1}$ and $b \in \Sigma$, we use $\uwp(\vec{\pi}, b)$ to represent the window profile updated from $\pi$ after reading the letter $b$, specifically, $\uwp(\vec{\pi}, b) = \vec{\pi'}$ such that  
\begin{itemize}
\item $\pi'_1 = \top$ iff $b = a_1$, 
%
\item for each $i \in [n-2]$, $\pi'_{i+1} = \top$ iff $\pi_{i} = \top$ and $b = a_{i+1}$. 
\end{itemize}
Let $WP_v$ denote the set of window profiles of string positions w.r.t. $v$. From the result in \cite{CCH+18}, we know that $|WP_v| \le |v|$. 
%Then the set of window profiles of $v$, denoted by $WP_v$, is computed by setting $WP_0 := \{\bot^{n-1}\}$ and iterating the following procedure, until $WP_i = WP_{i+1}$:
%\[WP_{i+1}:=WP_i \cup \{\uwp(\vec{\pi}, b) \mid \vec{\pi} \in WP_i, b \in \Sigma\}.\] 
%Therefore, the aforementioned iteration terminates in at most $|v|$ steps.\\
%
%
\begin{example}[CEFA for $\indexof_v$]\label{exm:indexof}
Suppose $v = a_1 \cdots a_n$ with $n \ge 1$. Then $\indexof_v$ is captured by the CEFA $\CEFA_{\indexof_v}=(Q, \Sigma, R, \delta, I, F)$, such that 
\begin{itemize}
\item $Q = \{q_0, q_1\} \cup WP_v \cup WP_v \times [n]$, 
\item $R=(r_1, r_2)$ (where $r_1,r_2$ represent the input and output positions of $\indexof_v$ respectively), 
\item $I=\{q_0\}$, 
\item $F=\{q_1\}$, and 
\item $\delta$ comprises 
\begin{itemize}
\item the tuples $(q_0, a, q_0, \eta)$ such that $a \in \Sigma$, $\eta(r_1)=1$, and $\eta(r_2) = 1$,
%
\item the tuples $(q_0, a, \vec{\pi}, \eta)$ such that $a \in \Sigma$, $\vec{\pi} = \theta \bot^{n-2}$ where $\theta  = \top$ iff $a = a_1$, $\eta(r_1) = 0$, and $\eta(r_2)= 0$ (recall that the first position of a string is $0$),
% 
\item the tuples  $(\vec{\pi}, a, \uwp(\vec{\pi}, a), \eta)$ such that $\vec{\pi} \in WP_u$, $a \in \Sigma$, $\pi_{n-1} = \bot$ or $a \neq a_{n}$, $\eta(r_1) = 0$, and $\eta(r_2)= 1$,
%
\item the tuples $(\vec{\pi}, a, (\uwp(\vec{\pi}, a), 1), \eta)$ such that $\vec{\pi} \in WP_u$, $a = a_1$, $\pi_{n-1} = \bot$ or $a \neq a_{n}$, $\eta(r_1) = 0$, and $\eta(r_2)= 1$,
%
\item the tuples $((\vec{\pi}, i),  a, (\uwp(\vec{\pi}, a), i+1), \eta)$ such that $\vec{\pi} \in WP_u$, $i \in [n-2]$, $a = a_{i+1}$, $\pi_{n-1} = \bot$ or $a \neq a_{n}$, $\eta(r_1) = 0$, and $\eta(r_2)= 0$,
%
\item the tuples $((\vec{\pi}, n-1),  a, q_1, \eta)$ such that $\vec{\pi} \in WP_u$, $a = a_{n}$, $\eta(r_1) =0$, and $\eta(r_2)= 0$,
%
\item the tuples  $(q_1, a, q_1, \eta)$ such that $a \in \Sigma$, $\eta(r_1) = 0$, and $\eta(r_2)= 0$.
\end{itemize}
\end{itemize}
\end{example}

Given two CEFAs $\NFA_1 = ( Q_1, \Sigma, R_1, \delta_1, I_1, F_1)$ and $\NFA_2 = (Q_2, \Sigma, \delta_2, R_2, I_2, F_2)$ with $R_1 \cap R_2 = \emptyset$ (where the notation is abused a bit, viewing $R_1$ and $R_2$ as sets),  the product of $\NFA_1$ and $\NFA_2$, denoted by $\NFA_1 \times \NFA_2$, is defined as $(Q_1 \times Q_2, \Sigma, R_1 \cup R_2, \delta, I_1 \times I_2, F_1 \times F_2)$, where $\delta$ comprises the tuples $((q_1, q_2), \sigma, (q'_1, q'_2), \eta)$ such that $(q_1, \sigma, q'_1, \eta_1) \in \delta_1$, $(q_2, \sigma, q'_2, \eta_2) \in \delta_2$, and $\eta = \eta_1\cup \eta_2$.  %for some $\eta_1, \eta_2$.


For a CEFA $\CEFA$, we use $R(\CEFA)$ to denote the vector of cost registers occurring in $\CEFA$. Note that cost registers of $\CEFA$ are simply integer variables to store costs in $\CEFA$.
%Moreover, for a CEFA $\NFA$ and 
Suppose $\CEFA$ is  CEFA with $R(\CEFA)=(r_1,\cdots, r_k)$ and $\vec{i} = (i_1,\cdots, i_k)$ with $R(\NFA) \cap \vec{i} = \emptyset$. We use $\CEFA[\vec{i}/R(\CEFA)]$ to denote the CEFA obtained from $\CEFA$ by simultaneously replacing $r_j$ with $i_j$ for every $j \in [k]$ in $\CEFA$. 

\smallskip

%Let $(k_1,\cdots, k_l) \in \Nat^l$ with $k_j > 0$ for every $j \in [l]$. A  $(k_1,\cdots, k_l)$-cost-enriched relation $\cR$ is a subset of $(\Sigma^* \times \intnum^{k_1}) \times \cdots (\Sigma^* \times \intnum^{k_l})$.
\begin{definition}[Cost-enriched recognisable relations]
Let $(k_1,\cdots, k_l) \in \Nat^l$ with $k_j > 0$ for every $j \in [l]$. Intuitively, a cost-enriched recognisable relation (CERR)  $\cR \subseteq (\Sigma^* \times \intnum^{k_1}) \times \cdots  \times (\Sigma^* \times \intnum^{k_l})$ is a finite union of products of CERLs. Formally,
	\[\cR = \bigcup \limits_{i=1}^n L_{i,1 } \times \cdots \times L_{i, l},\]
	where for every $j \in [l]$, $L_{i,j} \subseteq \Sigma^* \times \intnum^{k_j}$ is a CERL. 
	A CEFA representation of $\cR$ is a collection of CEFA tuples $(\CEFA_{i,1}, \cdots, \CEFA_{i,l})_{i \in [n]}$ such that $\Lang(\CEFA_{i,j}) = L_{i,j}$ for every $i \in [n]$ and $j \in [l]$.
\end{definition}

\begin{example}\label{exm:CERR}
The relation 
\[\cR=\{((w_1, |w_1|), (w_2, |w_2|)) \mid  w_1 \in \Lang((aa)^*), w_2 \in \Lang(b(bb)^*), |w_1|+|w_2| \ge 2\}\] 
is a CERR since 
$\cR = L_{1,1} \times L_{1,2} \cup L_{2,1} \times L_{2,2}$, where 
\begin{itemize}
\item $L_{1,1}=\{(w_1, |w_1|) \mid w_1 \in \Lang((aa)^*)\}$, 
\item $L_{1,2}=\{(w_2, |w_2|) \mid  w_2 \in \Lang(bbb(bb)^*)\}$ , 
\item $L_{2,1}=\{(w_1, |w_1|) \mid w_1 \in \Lang(aa(aa)^*)\}$,  and
\item $L_{2,2}=\{(w_2, |w_2|) \mid  w_2 \in \Lang(b(bb)^*)\}$. 
\end{itemize}
Note that $L_{1,1}$, $L_{1,2}$, $L_{2,1}$, and $L_{2,2}$ are CERLs, according to Example~\ref{exm:len}. Moreover, $(\CEFA_{i,1}, \CEFA_{i,2})_{i \in [2]}$ is a CEFA representation of $\cR$, where 
\begin{itemize}
\item  $\CEFA_{1,1} = (\{p_0, p_1\}, \{a,b\}, (r_{1}), \delta_{1,1}, \{p_0\}, \{p_0\})$ defines $L_{1,1}$, such that $\delta_{1,1}= \{(p_0, a, p_1,\eta_1)$, $(p_1, a, p_0, \eta_1)\}$, with $\eta_1(r_{1})=1$,
%
\item $\CEFA_{1,2}=(\{q_0,q_1,q_2\}, \{a,b\}, (r_{2}), \delta_{1,2}, \{q_0\}, \{q_1\})$ defines $L_{1,2}$, such that $\delta_{1,2}=\{(q_0, b, q_1,\eta_2), (q_1, b, q_2, \eta_2), (q_2, b, q_1,\eta_2)\}$, with $\eta_2(r_{2})=1$,
%
\item $\CEFA_{2,1} = (\{p_0, p_1,p_2, p_3\}, \{a,b\}, (r_{1}), \delta_{2,1}, \{p_0\}, \{p_2\})$ defines $L_{2,1}$, such that $\delta_{2,1} = \{(p_0, a, p_1, \eta_1), (p_1, a, p_2, \eta_1), (p_2, a, p_3, \eta_1), (p_3, a, p_2, \eta_1)\}$, with $\eta_1(r_{1})=1$, and
%
\item $\CEFA_{2,2}=(\{q_0,q_1,q_2\}, \{a,b\}, (r_{2}), \delta_{2,2}, \{q_1\})$ defines $L_{2,2}$, such that $\delta_{2,2}= (q_0, b, q_1,\eta_2), (q_1, b, q_2, \eta_2), (q_2, b, q_1,\eta_2)\}$, with $\eta_2(r_{2})=1$.
\end{itemize}
\end{example}

\subsection{Satisfiability of Linear Integer Arithmetic Formula with respect to CEFAs}

In the sequel, we consider a decision problem for CEFAs, for the decision procedure of the path feasibility problem of {\slint} in the next section.

\begin{definition}[{\lasat} Problem]\label{def-la-sat-cefa}
	The satisfiability problem of LIA formulas w.r.t. CEFAs (abbreviated as {\lasat} problem) is to decide for a given quantifier-free LIA formula $\phi$ and CEFAs $\CEFA_1,\cdots,\CEFA_m$, such that 
\begin{itemize}
\item	$\CEFA_i=(Q_i, \Sigma, R_i, \delta_i, I_i, F_i)$ for every $i\in [m]$, 
\item $R_i \cap R_j = \emptyset$ for every $1 \le i < j \le m$, and
\item the free variables of $\phi$ are from $\bigcup_{i\in [m]} R_i$, 
\end{itemize}
whether $\phi$ is satisfiable w.r.t. $\CEFA_1, \cdots, \CEFA_m$, namely, whether there are an assignment function $\theta: \bigcup \limits_{i \in [m]} R_i \rightarrow \Int$ and strings $w_1, \cdots, w_m$  
	such that  $\phi[(\theta(R_i)/R_i)_{i \in [m]}]$ is evaluated to $true$ and $(w_i, \theta(R_i)) \in \Lang(\NFA_i)$ for every $i \in [m]$.
\end{definition}
%Note that in Definition~\ref{def-la-sat-cefa}, registers in $\NFA_i$'s may intersect. 

\begin{example}
Let $\phi \equiv r_1 = r_2$ and $\CEFA_{1,1}, \CEFA_{1,2}$ be two CEFAs in Example~\ref{exm:CERR} defining $L_{1,1}, L_{1,2}$ respectively (Recall that $R(\CEFA_{1,1})=(r_1)$ and $R(\CEFA_{1,2})=(r_2)$). Then it is easy to see that $\phi$ is unsatisfiable w.r.t. $\CEFA_{1,1}$ and $\CEFA_{1,2}$, since for each $(w_1, n_1) \in L_{1,1}$, we have that $n_1$ is even, while for each $(w_2, n_2) \in L_{1,2}$, we have that $n_2$ is odd, thus $n_1$ and $n_2$ cannot be equal.
\end{example}

\begin{theorem}\label{thm-la-sat-cefa}
	The {\lasat} problem is NP-complete.
\end{theorem}

For the proof of Theorem~\ref{thm-la-sat-cefa}, we state and prove the following lemma. 

For a $k$-cost-enriched language $L$, we define 
\[
\prjnum(L) = \left\{(n_1, \cdots, n_k) \in \Int^k \mid \mbox{ there exist } w \in \Sigma^*.\ (w,(n_1,\cdots,n_k)) \in L \right\}.
\]

\begin{lemma}\label{lem-cefa-la}
	Let $\CEFA=(Q, \Sigma, R, \delta, I, F)$ be a CEFA with $R= (r_1, \cdots,  r_k)$. Then an existential LIA formula $\phi_\CEFA(r_1, \cdots, r_k)$ such that $\cM(\phi_\CEFA)= \prjnum(\Lang(\CEFA))$ can be computed in linear time from $\CEFA$.
\end{lemma}

\begin{proof}
	Suppose $\delta = \{\tau_1, \cdots, \tau_l\}$ such that $\tau_j = (q_j, a_j, q'_j, \eta_j)$ and $\eta_j(r_i) =  c_{j,i}$ for every $j \in [l]$ and $i \in [k]$.
	
	From the results on NFAs (Theorem~1 in \cite{SSMH04}), we know that for each pair of states $(q, q') \in I \times F$,  an existential LIA formula $\phi_{q,q'}(m_1, \cdots, m_l)$ can be computed in linear time such that $\cM(\phi_{q,q'})$ is the set of Parikh images of the runs of $\NFA$ starting from $q$ and ending at $q'$, where the variables $m_1, \cdots, m_l$ represent the numbers of occurrences of $\tau_1,\cdots, \tau_l$ respectively in the run. 
	
	Then the desired existential LIA formula $\phi_\NFA$ is constructed as follows,
	\[\phi_\NFA(r_1, \cdots, r_k) ::= \bigvee \limits_{(q,q') \in I \times F} \exists m_1 \cdots \exists m_l.\ \left(\varphi_{q,q'}(m_1, \cdots, m_l) \wedge \bigwedge \limits_{i \in [k]} r_i = \sum \limits_{j \in [l]} c_{j,i} m_j \right).\]
\end{proof}

We are ready to prove Theorem~\ref{thm-la-sat-cefa}.
\begin{proof}[Proof of Theorem~\ref{thm-la-sat-cefa}]
The NP lower bound follows from the fact that the satisfiability problem of existential LIA formulas is NP-complete \cite{BT76,GS78} (see also \cite{Haase18}).

For the upper bound, suppose that $\phi$ is a quantifier-free LIA formula and $\CEFA_1,\cdots,\CEFA_m$ are CEFAs such that 
\begin{itemize}
\item	$\CEFA_i=(Q_i, \Sigma, R_i, \delta_i, I_i, F_i)$  with $R_i = (r_{i, 1}, \cdots, r_{i, k_i})$, for every $i\in [m]$,
\item $R_i \cap R_j = \emptyset$ for every $1 \le i < j \le m$, and
\item the free variables of $\phi$ are from $\bigcup_{i\in [m]} R_i$.
\end{itemize}
From Lemma~\ref{lem-cefa-la}, for every $i \in [m]$, an existential LIA formula $\phi_{\CEFA_i}(r_{i,1}, \cdots, r_{i, k_i})$ such that $\cM(\phi_{\CEFA_i})= \prjnum(\Lang(\CEFA_i))$ can be computed in linear time from $\CEFA_i$.

Then the satisfiability of $\phi$ w.r.t. $\CEFA_1,\cdots, \CEFA_m$ is reduced to the satisfiability problem of the  following existential LIA formula
\[
	\phi' \equiv \phi \wedge \bigwedge \limits_{i \in [m]} \phi_{\CEFA_i}(r_{i,1}, \cdots, r_{i, k_i}).
\]
Since the size of $\phi'$ is linear in the size of $\phi$ as well as those of $\CEFA_1,\cdots,\CEFA_m$, and the satisfiability problem of existential LIA formulas is NP-complete, we conclude that the satisfiability of $\phi$ w.r.t.  $\CEFA_1,\cdots,\CEFA_m$ can be decided in nondeterministic polynomial time.

It remains to prove the correctness of the reduction, namely, $\phi$ is satisfiable w.r.t. $\CEFA_1,\cdots, \CEFA_m$ iff $\phi'$ is satisfiable.

\smallskip

\noindent \emph{``Only if'' direction}. Suppose $\phi$ is satisfiable w.r.t. $\CEFA_1,\cdots, \CEFA_m$. Then there are an assignment function $\theta: \bigcup \limits_{i \in [m]} R_i \rightarrow \Int$ and strings $w_1, \cdots, w_m$  
	such that  $\phi[(\theta(R_i)/R_i)_{i \in [m]}]$ is evaluated to $true$ and $(w_i, \theta(R_i)) \in \Lang(\NFA_i)$ for every $i \in [m]$. For every $i \in [m]$, from $\cM(\phi_{\CEFA_i})=\prjnum(\Lang(\CEFA_i))$, we know that $\theta(R_i)$ satisfies $\phi_{\CEFA_i}$, namely, $\phi_{\CEFA_i}[\theta(R_i)/R_i]$ is evaluated to $true$. Therefore, the assignment $\theta$ makes $\phi'$ satisfied.
	
\smallskip 

\noindent \emph{``If'' direction}. Suppose $\phi'$ is satisfiable. Then there is an assignment $\theta: \bigcup \limits_{i \in [m]} R_i \rightarrow \Int$ such that $\phi[(\theta(R_i)/R_i)_{i \in [m]}]$, $\phi_{\CEFA_1}[\theta(R_1)/R_1]$, $\cdots$, and $\phi_{\CEFA_m}[\theta(R_m)/R_m]$ are all evaluated to $true$. For every $i \in [m]$, from $\cM(\phi_{\CEFA_i})=\prjnum(\Lang(\CEFA_i))$,  we know that there is a string $w_i$ such that $(w_i, \theta(R_i)) \in \Lang(\CEFA_i)$. From Definition~\ref{def-la-sat-cefa}, we conclude that $\phi$ is satisfiable w.r.t. $\CEFA_1,\cdots, \CEFA_m$.
\end{proof}

