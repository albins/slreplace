%!TEX root = main.tex

%In this section, we show the main result of this paper. 

\begin{theorem}\label{thm-main}
The path feasibility of {\slint} programs is $\expspace$-complete. \tl{to be examined}
\end{theorem}
%
The lower bound of Theorem~\ref{thm-main} follows from Theorem~5 in \cite{LB16}, where the satisfiability of straight-line sting constraints with concatenation and finite transducers is shown to be the $\expspace$-complete.

For the upper bound, we will design a decision procedure for the path feasibility problem of {\slint} programs, based on the concepts of CERLs and CERRs introduced Section~\ref{sec:cefa}.


%Before presenting the decision procedure, we introduce an additional concept, i.e., cost-enriched pre-images of CERLs under string operations. Moreover, 
The crux of the procedure is to %we will 
show that the %cost-enriched 
pre-images of CERLs under the string operations in {\slint} (i.e., concatenation $\concat$, $\replaceall_{e,u}$, $\reverse$, NFTs $\NFT$, and $\substring$) are CERRs. 

To simplify the presentation of the decision procedure, in this section, we usually keep the string operations abstract by only mentioning the input and output data types, namely, we consider string operations $f: (\Sigma^* \times \Int^{k_1}) \times \cdots \times (\Sigma^* \times \Int^{k_l}) \rightarrow 2^{\Sigma^*}$ (if there is no integer input parameter, then $k_1,\cdots,k_l$ are zero), where each integer input parameter (if there is any) is assumed to be affiliated to a unique string input parameter. Note that  in general $f$ can be nondeterministic, namely, on one input, $f$ may output several  strings.


\subsection{Pre-images of CERLs under string operations}

\begin{definition}[Cost-enriched pre-images of CERLs]
Suppose that $f: (\Sigma^* \times \Int^{k_1}) \times \cdots \times (\Sigma^* \times \Int^{k_l}) \rightarrow 2^{\Sigma^*}$ is a string operation, $L \subseteq \Sigma^* \times \Int^{k_0}$ is a CERL defined by a CEFA $\CEFA=(Q, \Sigma, R, \delta, I, F)$ with $R= (r_1, \cdots, r_{k_0})$. Then the $R$-cost-enriched pre-image of $L$ under $f$, denoted by $f^{-1}_R(L)$, is a pair $(\cR, \vec{t})$ such that 
\begin{itemize}
\item $\cR \subseteq (\Sigma^* \times \Int^{k_1 + k_0}) \times \cdots \times (\Sigma^* \times \Int^{k_l + k_0})$,
\item $\vec{t} = (t_1, \cdots ,t_{k_0})$ is a vector of linear integer terms where for each $i \in [k_0]$, $t_i$ is a term whose variables are from $\{r^{(1)}_i, \cdots, r^{(l)}_i\}$ 

%[intuitively, each cost register $r_i$ is split into $l$ cost registers $r^{(1)}_i, \cdots,r^{(l)}_i$, one for each string input parameter, and $t_i$ tells how to compute $r_i$ from $r^{(1)}_i, \cdots,r^{(l)}_i$]
\item $L$ is equal to the language comprising the $k_0$-cost-enriched strings
%
\[\left(w_0, t_1\left[d^{(1)}_{1}/r^{(1)}_1, \cdots, d^{(l)}_{1}/r^{(l)}_1\right], \cdots, t_{k_0}\left[d^{(1)}_{k_0}/r^{(1)}_{k_0}, \cdots, d^{(l)}_{k_0}/r^{(l)}_{k_0}\right]
\right), \]
%
such that 
\[w_0 \in f\left((w_1, \vec{c_1}), \cdots, (w_l, \vec{c_l}\right)) \mbox{ for some } ((w_1, (\vec{c_1}, \vec{d_1})), \cdots, (w_l, (\vec{c_l}, \vec{d_l}))) \in \cR,\]
where $\vec{c_j} \in \Int^{k_j}$, $\vec{d_j} = (d^{(j)}_{1}, \cdots, d^{(j)}_{k_0}) \in \Int^{k_0}$ for $j\in [l]$.


%
%$\vec{c_1} \in \Int^{k_1}$, $\cdots$, $\vec{c_l} \in \Int^{k_l}$, $\vec{d_1} = (d^{(1)}_{1}, \cdots, d^{(1)}_{k_0}) \in \Int^{k_0}$, $\cdots$, and $\vec{d_l} = (d^{(l)}_{1},\cdots, d^{(l)}_{k_0}) \in \Int^{k_0}$.
\end{itemize}
The $R$-cost-enriched pre-image of $L$ under $f$, say $f^{-1}_R(L)=(\cR, \vec{t})$, is said to be CERR-definable if $\cR$ is a CERR. 

A CEFA representation of a CERR-definable $f^{-1}_R(L)=(\cR, \vec{t})$ is a tuple $((\CEFA_{i,1}, \cdots, \CEFA_{i, l})_{i \in [n]}, \vec{t})$ such that $(\CEFA_{i,1}, \cdots, \CEFA_{i, l})_{i \in [n]}$ is a CEFA representation of $\cR$, where $R(\CEFA_{i,j})=(r'_{j,1}, \cdots, r'_{j,k_j}, r^{(j)}_1, \cdots,r^{(j)}_{k_0})$ for each $i \in [n]$ and $j \in [l]$. (The cost registers $r'_{1,1}, \cdots, r'_{1,k_1},\cdots, r'_{l,1}, \cdots, r'_{l,k_l}, r^{(1)}_1, \cdots,r^{(1)}_{k_0}, \cdots, r^{(l)}_1, \cdots,r^{(l)}_{k_0}$ are mutually distinct and freshly introduced.) \tl{$r^{(1)}_1, \cdots,r^{(1)}_{k_0}, \cdots, r^{(l)}_1, \cdots,r^{(l)}_{k_0}$ are actually introduced above?}
\end{definition}

\begin{example}\label{exm:pre-image}
Let $L = \{(w, |w|) \mid w \in \Lang((aa)^*b(bb)^*) \}$. Evidently $L$  is a CERL defined by a CEFA $\CEFA = (Q, \{a,b\}, (r_1), \delta, I, F)$. Since the concatenation operation $\concat$  is a string function from $\Sigma^* \times \Sigma^*$ to $\Sigma^*$, $\concat^{-1}_R(L)$, the $R$-cost-enriched pre-image of $L$ under concatenation $\concat$, is the pair $(\cR, t)$, where $t=r^{(1)}_1+r^{(2)}_1$ (note that in this case $k_0=1$ and $l=2$) and 
\[\cR = L_{1,1} \times L_{1,2} \cup L_{2,1} \times L_{2,2} \cup L_{3,1} \times L_{3,2} \cup L_{4,1} \times L_{4,2} \cup L_{5,1} \times L_{5,2},\]
such that
\begin{itemize}
\item $L_{1,1} = \{(w_1, |w_1|) \mid w_1 \in \Lang((aa)^*)\}$ and $L_{1,2} = \{(w_2, |w_2|) \mid w_2 \in \Lang(b(bb)^*)\}$,
%
\item $L_{2,1} = \{(w_1, |w_1|) \mid w_1 \in \Lang((aa)^*)\}$ and $L_{2,2} = \{(w_2, |w_2|) \mid w_2 \in \Lang((aa)^*b(bb)^*)\}$,
%
\item $L_{3,1} = \{(w_1, |w_1|) \mid w_1 \in \Lang(a(aa)^*)\}$ and $L_{3,2} = \{(w_2, |w_2|) \mid w_2 \in \Lang(a(aa)^*b(bb)^*)\}$,
%
\item $L_{4,1} = \{(w_1, |w_1|) \mid w_1 \in \Lang((aa)^*b(bb)^*)\}$ and $L_{4,2} = \{(w_2, |w_2|) \mid w_2 \in \Lang((bb)^*)\}$,
%
\item $L_{5,1} = \{(w_1, |w_1|) \mid w_1 \in \Lang((aa)^*(bb)^*)\}$ and $L_{5,2} = \{(w_2, |w_2|) \mid w_2 \in \Lang(b(bb)^*)\}$.
\end{itemize}
\tl{is this compatible with the def above?}
It is easy to see that $\cR$ is a CERR. Thus $\concat^{-1}_R(L)$ is CERR-definable.
\end{example}

It turns out that for each string operation $f$ in {\slint}, the cost-enriched pre-images of CERLs under $f$ are CERR-definable.

\begin{proposition}\label{prop:pre-image}
Let $L$ be a CERL defined by a CEFA $\CEFA = (Q, \Sigma, R, \delta, I, F)$. Then for each string operation $f$ ranging over $\concat$, $\replaceall_{e,u}$, $\reverse$, NFTs $\NFT$, and $\substring$, $f^{-1}_R(L)$ is CERR-definable. In addition,
\begin{itemize}
\item a CEFA representation of $\concat^{-1}_R(L)$ can be computed in time $\bigO(|\CEFA|^2)$, 
%
\item a CEFA representation of $\reverse^{-1}_R(L)$ (resp. $\substring^{-1}_R(L)$) can be computed in time $\bigO(|\CEFA|)$,
%
\item a CEFA representation of  $(\Tran(\NFT))^{-1}_R(L)$ can be computed in time polynomial in $|\CEFA|$ and exponential in $|\NFT|$,
%
\item a CEFA representation of  $(\replaceall_{e,u})^{-1}_R(L)$ can be computed in time polynomial in $|\CEFA|$ and exponential in $|e|$ and $|u|$.
\end{itemize}
\end{proposition}

\begin{proof}
Let $\CEFA=(Q, \Sigma, R, \delta, I, F)$ be a CEFA with $R= (r_1, \cdots, r_k)$. We show how to construct a CEFA representation of $f^{-1}_R(L)$ for each function $f$ in {\slint}.

%%%%%%%%%%%%%%%%%%%%%%%%%%%%%%%%%%%%%%%%%%%%%%%%%%%%%%%%%%%%%%%%%%%%%%%%%%%%%%
\paragraph*{$\concat^{-1}_R(L)$.}
%
A CEFA representation of $\concat^{-1}_R(L)$ is given by $((\CEFA_{I, q}, \NFA_{q, F})_{q \in Q}, \vec{t})$, where 
\begin{itemize}
\item $\CEFA_{I, q}=(Q, \Sigma, R^{(1)}, \delta^{(1)}, I, \{q\})$ and  $\CEFA_{q, F}=(Q, \Sigma, R^{(2)}, \delta^{(2)}, \{q\}, F)$ such that 
\begin{itemize}
\item $R^{(1)} = (r^{(1)}_1, \cdots, r^{(1)}_k)$, $R^{(2)} = (r^{(2)}_1, \cdots, r^{(2)}_k)$, 
\item $\delta^{(1)}$ comprises the tuples $(q, a, q', \eta')$ satisfying that there exists $\eta$ such that $(q, a, q', \eta) \in \delta$ and for each $j \in [k]$, and $\eta'(r^{(1)}_j)=\eta(r_j)$,  similarly for $\delta^{(2)}$,
\end{itemize}
\item and $\vec{t} = (r^{(1)}_1 + r^{(2)}_1, \cdots, r^{(1)}_k + r^{(2)}_k)$.
\end{itemize}
Note that the size of $((\CEFA_{I, q}, \NFA_{q, F})_{q \in Q}, \vec{t})$ is $\bigO(|\CEFA|^2)$.
%
%%%%%%%%%%%%%%%%%%%%%%%%%%%%%%%%%%%%%%%%%%%%%%%%%%%%%%%%%%%%%%%%%%%%%%%%%%%%%%
%
\paragraph*{$\reverse^{-1}_R(L)$.} 
%
A CEFA representation of $\reverse^{-1}_R(L)$ is given by $(\CEFA^{(r)}, \vec{t})$, where 
\begin{itemize}
\item $\CEFA^{(r)}=(Q, \Sigma, R^{(1)}, \delta', F, I)$ such that 
\begin{itemize}
\item $R^{(1)}=(r^{(1)}_1,\cdots,r^{(1)}_k)$, and 
\item $\delta'$ comprises the tuples $(q', a, q, \eta')$ satisfying that there exists $\eta$ such that $(q, a, q', \eta) \in \delta$, and $\eta'(r^{(1)}_i) = \eta(r_i)$ for each $i \in [k]$,
\end{itemize}
%
\item and $\vec{t}=(r^{(1)}_1, \cdots, r^{(1)}_k)$. 
\end{itemize}
Note that $\Lang(\CEFA^{(r)}) = \{(w^{(r)}, \vec{n}) \mid (w, \vec{n}) \in \Lang(\CEFA)\}$, and the size of $(\CEFA^{(r)}, \vec{t})$ is $\bigO(|\CEFA|)$.

%%%%%%%%%%%%%%%%%%%%%%%%%%%%%%%%%%%%%%%%%%%%%%%%%%%%%%%%%%%%%%%%%%%%%%%%%%%%%%

\paragraph*{$\substring^{-1}_R(L)$.}
A CEFA representation of $\substring^{-1}_R(L)$ is given by $(\cB, \vec{t})$, where 
\begin{itemize}
\item $\cB = (Q', \Sigma, R', \delta', I', F')$ such that 
\begin{itemize}
\item $Q' = Q \times \{p_0, p_1, p_2\}$, (intuitively, $p_0$, $p_1$, and $p_2$ denote that the current position is before the starting position, between the starting position and ending position, and after the ending position respectively)
%
\item $R' = \left(r^{(1)}_1,\cdots, r^{(1)}_k, r'_1, r'_2 \right)$, (intuitively, $r'_1$ denotes the starting position, and $r'_2$ denotes the length of the substring)
%
\item $I'=I \times \{p_0\}$, $F'=F \times \{p_2\}$,
%
\item and $\delta'$ comprises 
\begin{itemize}
\item the tuples $((q, p_0), a, (q, p_0), \eta')$ such that $q \in I$, $a \in \Sigma$, and $\eta'$ satisfies that $\eta'(r^{(1)}_j)=0$ for each $j \in [k]$, $\eta'(r'_1)= 1$, and $\eta'(r'_2) = 0$,
%
\item the tuples $((q, p_0), a, (q', p_1), \eta')$ such that there exists $\eta$ satisfying that $(q, a, q', \eta) \in \delta$, $\eta'(r^{(1)}_j)=\eta(r_j)$ for each $j \in [k]$,  $\eta'(r'_1)=0$ (recall that the positions of strings start at $0$), and $\eta'(r'_2) = 1$,
%
\item the tuples $((q, p_0), a, (q', p_2), \eta')$ such that there exists $\eta$ satisfying that $(q, a, q', \eta) \in \delta$, $q' \in F$, $\eta'(r^{(1)}_j)=\eta(r_j)$ for each $j \in [k]$,  $\eta'(r'_1)=0$ (recall that the positions of strings start at $0$), and $\eta'(r'_2) = 1$,
%
\item the tuples $((q, p_1), a, (q', p_1), \eta')$ such that there exists $\eta$ satisfying that $(q, a, q', \eta) \in \delta$ and $\eta'(r^{(1)}_j)=\eta(r_j)$ for each $j \in [k]$, $\eta'(r'_1) = 0$, and $\eta'(r'_2) = 1$,
%
\item the tuples $((q, p_1), a, (q', p_2), \eta')$ such that there exists $\eta$ satisfying that $(q, a, q', \eta) \in \delta$, $q' \in F$, and $\eta'(r^{(1)}_j)=\eta(r_j)$ for each $j \in [k]$, $\eta'(r'_1) = 0$, and $\eta'(r'_2) = 1$,
%
%\item the tuples $((q, p_1), a, (q, p_2), \eta')$ such that $q \in F$ and $\eta'(r^{(1)}_j)=0$ for each $j \in [k]$, $\eta'(r'_1) = 0$, and $\eta'(r'_2) = 1$,
%
\item the tuples $((q, p_2), a, (q, p_2), \eta')$ such that $q \in F$ and $\eta'(r^{(1)}_j)=0$ for each $j \in [k]$, $\eta'(r'_1) = 0$, and $\eta'(r'_2) = 0$,
%
\end{itemize}
\end{itemize}
\item $\vec{t}=(r^{(1)}_1, \cdots, r^{(1)}_k)$.
\end{itemize}
Note that the size of $(\cB, \vec{t})$ is $\bigO(|\CEFA|)$.
%%%%%%%%%%%%%%%%%%%%%%%%%%%%%%%%%%%%%%%%%%%%%%%%%%%%%%%%%%%%%%%%%%%%%%%%%%%%%%
%
%
\paragraph*{$(\Tran(\NFT))^{-1}_R(L)$.}
%
Suppose $\NFT = (Q', \Sigma, \delta', I', F')$. Then a CEFA representation of $(\Tran(\NFT))^{-1}_R(L)$ is given by 
$(\cB, \vec{t})$, where 
\begin{itemize}
\item $\cB$ simulates the run of $\NFT$ on the input string, meanwhile, it simulates the run of $\CEFA$ on the output string of $\NFT$, formally, $\cB= (Q' \times Q, \Sigma, R^{(1)}, \delta'', I' \times I, F' \times F)$ such that 
\begin{itemize}
\item $R^{(1)}  = (r^{(1)}_1, \cdots, r^{(1)}_k)$, and
\item $\delta''$ comprises the tuples $((q'_1, q_1), a, (q'_2, q_2), \eta')$ satisfying one of the following conditions,
\begin{itemize}
\item there exist $u = a_1 \cdots a_n \in \Sigma^+$ and a transition sequence $p_0 \xrightarrow[\delta]{a_1, \eta_1} p_2 \cdots p_{n-1} \xrightarrow[\delta]{a_n, \eta_n} p_{n}$ in $\CEFA$ such that $(q'_1, a, q'_2, u) \in \delta'$, $p_0 = q_1$, $p_{n}= q_2$, and for each $j \in [k]$,  $\eta'(r^{(1)}_j) = \eta_1(r_j) + \cdots + \eta_n(r_j)$,
%
\item $(q'_1, a, q'_2, \varepsilon) \in \delta'$, $q_1 = q_2$, and $\eta'(r^{(1)}_j) =0$ for each $j \in [k]$,
\end{itemize}
\end{itemize}
%
\item $\vec{t}=(r^{(1)}_1, \cdots, r^{(1)}_k)$.
\end{itemize}
Note that the number of transitions of $\cB$ can be exponential in the worst case, since it summarises the updates of cost registers of $\CEFA$ on the output strings of the transitions of $\NFT$. More precisely,  let
\begin{itemize}
\item $\ell$ be the maximum length of the output strings of transitions of $\NFT$, 
\item $N$ be the maximum number of transitions between a given pair of states of $\CEFA$, and
\item  $C$ be the maximum absolute value of the integer constants occurring in $\CEFA$,
\end{itemize}
then $|\delta''|$, the cardinality of $\delta''$, is bounded by $|\delta'| \times |Q|^2 \times N^\ell $, and the integer constants occurring in each transition of $\delta''$ are bounded by $\ell C$. Therefore, 
the size of $(\cB, \vec{t})$ is 
\[
\bigO(|\delta'| \times |Q|^2 \times N^\ell \times k \log_2 (\ell C)).
\] 
Since $|\delta'|, \ell \le |\NFT|$, $|Q|, N, k \le |\CEFA|$, and $C \le 2^{|\CEFA|}$, we deduce that the size of $(\cB, \vec{t})$ is 
$
\bigO( |\NFT| \times  |\CEFA|^2 \times |\CEFA|^{|\NFT|} \times |\CEFA|^2 \log_2(|\NFT|))= |\CEFA|^{\bigO(|\NFT|)} |\NFT| \log_2(|\NFT|).$
%

%%%%%%%%%%%%%%%%%%%%%%%%%%%%%%%%%%%%%%%%%%%%%%%%%%%%%%%%%%%%%%%%%%%%%%%%%%%%%%
\paragraph*{$(\replaceall_{e,u})^{-1}_R(L)$.}
%
From the result in \cite{CCH+18}, we know that  a NFT $\NFT_{e,u}=(Q', \Sigma, \delta', I', F')$ can be constructed to capture $\replaceall_{e,u}$.  Moreover, 
\begin{itemize}
\item $|Q'|$, as well as $|\delta'|$, is $2^{\bigO(|e|)}$,
\item $\ell$, the maximum length of the output strings of transitions of $\NFT_{e,u}$, is $|u|$.
\end{itemize}
Then a CEFA representation of $(\replaceall_{e,u})^{-1}_R(L)$ can be constructed as that of $(\Tran(\NFT_{e,u}))^{-1}_R(L)$.
Let $N$ denote the maximum number of transitions between a given pair of states of $\CEFA$, and $C$ be the maximum absolute value of the integer constants occurring in $\CEFA$, which is bounded by $2^{|\CEFA|}$. Then the CEFA representation of $(\replaceall_{e,u})^{-1}_R(L)$ is of size 
\[
\bigO(|\delta'| \times |Q|^2 \times N^\ell \times k \log_2 (\ell C)) = 2^{\bigO(|e|)} |\CEFA|^2 |\CEFA|^{|u|} |\CEFA|^2 \log_2 |u|=2^{\bigO(|e|)} |\CEFA|^{\bigO(|u|)}.
\]

%
 according to the aforementioned discussion for NFTs.
% 
%
\end{proof}

\subsection{The Decision Procedure}

%
Let $S$  be a {\slint} program. We show how to decide the path feasibility of $S$. The decision procedure is nondeterministic and divided into three steps. 
%\begin{description}
%\item[Step I: Preprocessing.] 
%
%\item 

\medskip
\noindent {\bf Step I: Removing $\length$ and $\indexof$}.

\smallskip

Repeat the following procedure until there are no occurrences of $\length$ and $\indexof$.
\begin{itemize}
\item For each term $\length(x)$ occurring in $S$, introduce a \emph{fresh} integer variable $i$, replace every occurrence of $\length(x)$ by $i$, and add the statement $\ASSERT{x \in \CEFA_{\rm len}[i/r_1]}$ to $S$. (See Example~\ref{exm:len} for the definition of $\CEFA_{\rm len}$.)  
%
\item 
For each term $\indexof_v(x, i)$ occurring in $S$, introduce two fresh integer variables $i_1$ and $i_2$, replace every occurrence of $\indexof_v(x, i)$ by $i_2$, and add the statements $\ASSERT{i=i_1}; \ASSERT{x \in \CEFA_{\indexof_v}[i_1/r_1, i_2/r_2]}$ to $S$.  (See Example~\ref{exm:indexof} for the definition of $\CEFA_{\rm \indexof_v}$.)
\end{itemize}
%
%\item 

\medskip
\noindent {\bf Step II: Removing the assignment statements}.

\smallskip

Repeat the following procedure until $S$ contains no assignment statements.
%
\begin{quote}
Suppose $y := f(x_1, \vec{i_1}, \cdots, x_l, \vec{i_l})$ is the last assignment of $S$, where $f: (\Sigma^* \times \Int^{k_1}) \times \cdots \times (\Sigma^* \times \Int^{k_l}) \rightarrow 2^{\Sigma^*}$ is any function from $\concat$, $\replaceall_{e,u}$, $\reverse$, $\NFT$, and $\substring$.
\\
Let $\rho := \{\CEFA_1, \cdots, \CEFA_s\}$ be the set of all CEFAs such that $\ASSERT{y \in \CEFA_j}$ occurs in $S$ for every $j \in [s]$. Construct $\NFA = \NFA_1 \times \cdots \times \NFA_s$ (Since the cost registers of CEFAs are always freshly introduced, it holds that $R(\CEFA_1)$, $\cdots$, and $R(\CEFA_s)$ are mutually disjoint). 
%
Suppose $R(\CEFA) = (r_1, \cdots, r_{k_0})$. Then from Proposition~\ref{prop:pre-image}, 
a CEFA representation of $f^{-1}_{R(\CEFA)}(\Lang(\CEFA))$, say $((\cB_{j, 1}, \cdots, \cB_{j, l})_{j \in [m]}, \vec{t})$, can be effectively computed from $\NFA$ and $f$, where 
\[
R(\cB_{j,j'})=\left(r'_{j',1}, \cdots, r'_{j',k_j}, r^{(j')}_1, \cdots,r^{(j')}_{k_0} \right)
\]
for each $j \in [m]$ and $j' \in [l]$, and $\vec{t}=(t_1,\cdots, t_{k_0})$. Note that the cost registers $r'_{1,1}, \cdots, r'_{1,k_1},\cdots, r'_{l,1}, \cdots, r'_{l,k_l}, r^{(1)}_1, \cdots,r^{(1)}_{k_0}, \cdots, r^{(l)}_1, \cdots,r^{(l)}_{k_0}$ are mutually distinct and freshly introduced.\\
%
Remove $y := f(x_1, \vec{i_1}, \cdots, x_k, \vec{i_k})$ from $S$, moreover, nondeterministically choose $j \in [m]$ and add the following statements to $S$, 
%
\[
\begin{array}{l}
\ASSERT{x_1 \in \cB_{j, 1}};\ \cdots;\ \ASSERT{x_l \in \cB_{j, l}}; \\
\ASSERT{\bigwedge \limits_{j' \in [l], j'' \in [k_j]} i_{j', j''} = r'_{j', j''}}; \ASSERT{\bigwedge \limits_{j' \in [k_0]} r_{j'} = t_{j'}}.
\end{array}
\]
%
\end{quote}

We would like to remark that if all the string operations $f$ in $S$ are \emph{deterministic}, then the product of  the CEFAs in $\rho$ can be avoided and the pre-image can be computed \emph{distributively} for every CEFA in $\rho$.

\medskip
\noindent{\bf Step III: Solving the {\lasat} problem}.

\smallskip

Suppose $S$ is a {\slint} program containing no assignment statements and all the integer terms are linear integer arithmetic terms. 

Let $X$ denote the set of string variables occurring in $S$.
For each string variable $x \in X$, let $\CEFA_x$ denote the product of all CEFAs $\CEFA$ such that $\ASSERT{x \in \CEFA}$ appears in $S$. Let $\phi$ denote the conjunction of all the LIA formulas occurring in $S$. Then the path feasibility of $S$ is reduced to deciding whether $\phi$ is satisfiable w.r.t. $(\CEFA_x)_{x \in X}$. From Theorem~\ref{thm-la-sat-cefa}, we know that there is a nondeterministic polynomial time algorithm to solve this problem.


\paragraph*{Complexity analysis.} Step I can be fulfilled in linear time. The complexity analysis of Step II is similar to that in \cite{CHL+19}. Roughly speaking, during Step II, for each input string variable $x$ in $S$, at most exponentially many CEFAs are introduced for $x$, with each of them of at most exponential size. Therefore, Step II can be fulfilled in nondeterministic exponential space. For Step III, \zhilin{remained to be done.}






%%%%%%%%%%%%%%%%%%%%%%%%%%%%%%%%%%%%%%%%%%%%%%%%
%%%%%%%%%%%%%%%%CERR linear integer functions removed%%%%%%%%%%%%%
%%%%%%%%%%%%%%%%CERR linear integer functions removed%%%%%%%%%%%%%
%%%%%%%%%%%%%%%%%%%%%%%%%%%%%%%%%%%%%%%%%%%%%%%%

\hide{
\begin{definition}[CERR linear integer functions]
An integer function $g: \Sigma^* \times \Int^{k_1} \times \Sigma^* \times \Int^{k_l} \rightarrow 2^\Int$ is  \emph{linear} if there is a pair $(\cR, t)$ such that $\cR \subseteq \Sigma^* \times \Int^{k_1+1} \times \Sigma^* \times \Int^{k_l+1}$ is a CERR and $t$ a linear integer term over $r^{(1)}, \cdots, r^{(l)}$ such that for all $\vec{c_1} \in \Int^{k_1}, \cdots, \vec{c_l} \in \Int^{k_l}$, and $d_1, \cdots, d_l \in \Int$, it holds that $(w_1, (\vec{c_1}, d_1), \cdots, w_l, (\vec{c_l}, d_l)) \in \cR$ iff $t[d_1/r^{(1)}, \cdots, d_l/r^{(l)}] \in g(w_1, \vec{c_1}, \cdots, w_l, \vec{c_l})$.  

For a CERR linear integer function $g$ witnessed by the pair $(\cR, t)$, a CEFA representation of $g$ is a tuple $((\NFA_{i,1}, \cdots, \NFA_{i, l})_{i \in [n]}, t)$, where $(\NFA_{i,1}, \cdots, \NFA_{i, l})_{i \in [n]}$ is a CEFA representation of $\cR$.

\end{definition}

\begin{example}
The string functions $\length$ and $\indexof_u$ are CERR linear integer functions, whose CEFA representations can be found in Section~\ref{sec-cslint}.
\end{example}

Suppose $A = A_r \wedge A_i$, where $A_r$ is a conjunction of atomic formulae of the form $x \in \NFA$, and $A_i$ is linear arithmetic formula (containing no integer functions). By computing the product construction of CEFAs, $A_r$ can be rewritten as $x_1 \in \NFA_1 \wedge \cdots \wedge x_n \in \NFA_n$, where $x_1,\cdots, x_n$ are mutually distinct. Therefore, the path feasibility of $S \wedge A$ is exactly the satisfiability of $A_i$ w.r.t. the CEFAs $\NFA_1, \cdots, \NFA_n$. From Theorem~\ref{thm-incra-la-sat}, we conclude that the path feasibility of  {\slint} is decidable.




Without loss of generality, we assume that string operations only apply to string variables.


Let $S':=S$ and $A':=A$. Moreover, let $A'':= \ltrue$. Then execute the following procedure to (partially) flatten the integer terms.
\begin{description}
\item[Step 1.] Recursively apply the following transformation until $S' \wedge A'$ contains no more occurrences of integer functions: Select an occurrence of integer functions, say $g(x_1, \vec{t_1}, \cdots, x_k, \vec{t_k})$, such that 
%it is a \emph{proper} subterm of the other integer term and 
{\it none} of $\vec{t_1}, \cdots, \vec{t_k}$ contains occurrences of integer functions, introduce a fresh integer variable $i$, let $S' \wedge A'$ be the formula obtained by replacing $g(x_1, \vec{t_1}, \cdots, x_k, \vec{t_k})$ with $i$, moreover, let $A'':= A'' \wedge i = g(x_1, \vec{t_1}, \cdots, x_k, \vec{t_k})$.
%
\item[Step 2.] It comprises the following two substeps. 
\begin{enumerate}
\item For each occurrence of string functions in $S'$, say $f(x_1, \vec{t_1}, \cdots, x_k, \vec{t_k})$, suppose $\vec{t_j} = (t_{j,1}, \cdots, t_{j, l_j})$ for each $j \in [k]$, introduce fresh integer variables $i_{j, j'}$ for $j \in [k]$ and $j' \in [l_j]$, replace $f(x_1, \vec{t_1}, \cdots, x_k, \vec{t_k})$ with $f(x_1, \vec{i_1}, \cdots, x_k, \vec{i_k})$ in $S'$, where $\vec{i_j} = (i_{j,1}, \cdots, i_{j, l_j})$ for each $j \in [k]$, and let $A':=A' \wedge \bigwedge \limits_{j \in [k], j' \in [l_j]} i_{j, j'} = t_{j, j'}$. 
\item For each occurrence of integer functions in $A''$, say $g(x_1, \vec{t_1}, \cdots, x_k, \vec{t_k})$, suppose $\vec{t_j} = (t_{j,1}, \cdots, t_{j, l_j})$ for each $j \in [k]$, introduce fresh integer variables $i_{j, j'}$ for $j \in [k]$ and $j' \in [l_j]$, replace $g(x_1, \vec{t_1}, \cdots, x_k, \vec{t_k})$ with $g(x_1, \vec{i_1}, \cdots, x_k, \vec{i_k})$ in $A''$, where $\vec{i_j} = (i_{j,1}, \cdots, i_{j, l_j})$ for each $j \in [k]$, and let $A':=A' \wedge \bigwedge \limits_{j \in [k], j' \in [l_j]} i_{j, j'} = t_{j, j'}$. 
\end{enumerate}
%
\item[Step 3.] Let $S:=S'$ and $A:=A'' \wedge A' $.
\end{description}
The aforementioned flattening procedure is a bit technical, for simplicity, we may assume that the integer terms are fully flattened, including the arithmetic operations.

Note that after the aforementioned flattening procedure, the resulting formula $S \wedge A$ satisfies the following property: 
\begin{quote}
The integer terms in all the occurrences of string and integer functions  are integer variables, moreover, each integer variable occurs at most once in these string and integer functions.  \hfill ($*$)
\end{quote}
Therefore, in the sequel, we assume that $S \wedge A$ satisfies the property ($*$).

\begin{theorem}\label{thm-sl-int-dec}
Path feasibility of {\slint} satisfying the semantic conditions is decidable.
\end{theorem}

\begin{proof}
In the following, we extend the generic decision procedure in \cite{CHL+19}, where NFA is replaced by CEFA.

Let $S \wedge A$ be an {\slint} formula (satisfying the property ($*$)).

For each occurrence of $i = g(x_1, \vec{i'_1}, \cdots, x_k, \vec{i'_k})$ in $A$ with $g$ an integer function, apply the following nondeterministic transformation to $A$: 
\begin{quote}
According to the 1st semantic condition, $g$ is a CERR linear integer function and a CEFA representation of $g$, say $((\NFA_{j,1}, \cdots, \NFA_{j, k})_{j \in [m]}, t)$, can be computed effectively from $g$. Consider $((\NFA'_{j,1}, \cdots, \NFA'_{j, k})_{j \in [m]}, t')$, where $\NFA'_{j,1}=\NFA_{j,1}[\vec{i'_1}/R(\NFA_{j,1})]$, $\cdots$, $\NFA'_{j,k}=\NFA_{j,k}[\vec{i'_k}/R(\NFA_{j,k})]$, and $t' = t[i^{(1)}/r^{(1)}, \cdots, i^{(k)}/r^{(k)}]$.
Nondeterministically choose $j \in [m]$, and replace $i = g(x_1, \vec{i'_1}, \cdots, x_k, \vec{i'_k})$ by $x_1 \in \NFA'_{j,1} \wedge \cdots \wedge x_k \in \NFA'_{j,k} \wedge i = t'$ in $A$.
\end{quote}
Note that after this transformation, $S \wedge A$ contains no occurrences of integer functions, moreover, as a result of the property ($*$), for every variable $x$, all the CEFAs to which $x$ belongs satisfy that their sets of registers are  mutually disjoint.

Then repeat the following procedure until $S$ becomes empty.
%
\begin{quote}
Suppose $y := f(x_1, \vec{i_1}, \cdots, x_k, \vec{i_k})$ is the last assignment of $S$. 
\\
Let $\rho := \{\NFA_1, \cdots, \NFA_s\}$ be the set of all CEFAs such that $y \in \NFA_j$ occurs in $A$ for each $j \in [s]$. Construct $\NFA = \NFA_1 \times \cdots \times \NFA_s$ (Recall that the sets of registers of $\NFA_1$, $\cdots$, $\NFA_s$ are mutually disjoint). Let  the vector of registers in $\NFA$ be $R = (r'_1, \cdots, r'_n)$. Then according to the 2nd semantic condition, 
a CEFA representation of the $R$-cost enriched pre-image of $\Lang(\NFA)$ under $f$, say $((\cB_{j, 1}, \cdots, \cB_{j, k})_{j \in [\ell]}, \vec{t})$, can be effectively computed from $\NFA$ and $f$. Consider $((\cB'_{j, 1}, \cdots, \cB'_{j, k})_{j \in [\ell]}, \vec{t'})$, where $\cB'_{j, 1} = \cB_{j, 1}[\vec{i_1}/R(\cB_{j,1}), \vec{(r')^{(1)}}/\vec{r^{(1)}}]$, $\cdots$, $\cB'_{j,k}=\cB_{j,k}[\vec{i_k}/R(\cB_{j,k}), \vec{(r')^{(k)}}/\vec{r^{(k)}}]$ (with $\vec{r^{(1)}}= (r^{1}_1, \cdots, r^{(1)}_n)$, similarly for $\vec{r^{(2)}}$ and so on), and $\vec{t'} = \vec{t}[\vec{r'_1}/\vec{r_1}, \cdots, \vec{r'_n}/\vec{r_n}]$ (with $\vec{r_1} = (r^{(1)}_1, \cdots, r^{(k)}_1)$, similarly for $\vec{r^{(2)}}$ and so on). 
\\
Nondeterministically choose $j \in [\ell]$ and let 
$$A:= A \wedge x_1 \in \cB'_{j, 1} \wedge \cdots \wedge x_k \in \cB'_{j, k}  \wedge \bigwedge \limits_{j' \in [n]} r'_{j'} = t'_{j'}.$$
%
Remove $y := f(x_1, \vec{i_1}, \cdots, x_k, \vec{i_k})$ from $S$.
\end{quote}

We would like to remark that if all the string functions $f$ in $S \wedge A$ are \emph{deterministic}, then the product of CEFAs before the pre-image computation can be avoided and the pre-image can be computed \emph{distributively} for CEFAs in $\rho$.

In the end, we get a formula $S \wedge A$ where $S$ is empty. Suppose $A = A_r \wedge A_i$, where $A_r$ is a conjunction of atomic formulae of the form $x \in \NFA$, and $A_i$ is linear arithmetic formula (containing no integer functions). By computing the product construction of CEFAs, $A_r$ can be rewritten as $x_1 \in \NFA_1 \wedge \cdots \wedge x_n \in \NFA_n$, where $x_1,\cdots, x_n$ are mutually distinct. Therefore, the path feasibility of $S \wedge A$ is exactly the satisfiability of $A_i$ w.r.t. the CEFAs $\NFA_1, \cdots, \NFA_n$. From Theorem~\ref{thm-incra-la-sat}, we conclude that the path feasibility of  {\slint} is decidable.
\qed
\end{proof}

\begin{corollary}
Path feasibility of {\cslint} is decidable.
\end{corollary}
}