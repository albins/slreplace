%!TEX root = main.tex

Strings are widely used in programs, especially in Web applications. Incorrectly handled string manipulation is a frequent cause of security vulnerabilities, which makes analysis and verification of string-manipulating programs an important security goal. Unfortunately, problems involving strings are hard to analyse, and to make matters worse, such programs often combine operations on strings with integer data types, used to refer to lengths of, or positions in, strings.

A common approach for program verification is to symbolically execute a program under analysis, and rely on a constraint solver to solve path feasability constraints arising from operations in the program. A solution to the path feasibility problem finds an assignment to inputs (if any) that would yield a successful execution along a given path, such as one leading to an unauthorised execution of code.

%For instance, the inputs or outputs of widely used string operations including length, substring, and indexof involve the integer data type.
%a given execution path is feasible, namely, 
%, which plays a central role in the static analysis and verification, e.g. symbolic execution, of programs. 
%Path feasibility problem of string-manipulating programs is very challenging and undecidable in general, especially when involving the integer data type. 
%Aiming at solving the path feasibility of string-manipulating programs in practice, 
%on the one hand, most of the 
Although state-of-the-art string constraint solvers usually provide support for both string and integer data types,   
 %strong support of complex string operations, in conjunction with limited support of integer data type. 
they mainly resort to heuristics without completeness guarantees. \\ %, partially because it is a generally undecidable problem. \\
%
In this paper, we propose a decision procedure 
%of the path feasibility problem 
for a class of string-manipulating programs
which includes  not only a wide range of complex string operations such as concatenation, replaceAll, reverse, and finite transducers, but also those involving the integer data type such as length, indexof, and substring. To the best of our knowledge, this represents one of the most expressive string constraint languages that is currently known to be decidable.  Our decision procedure is automata-theoretic and based on a variant of cost register automata (Alur et al. LICS 2013). 
%To the best of our knowledge, %this is %the first time that 
%the first (complete) decision procedure which covers the most expressive string constraint language %have been achieved for the path feasibility problem of such an expressive class of string-manipulating programs. Furthermore, 
%Different from most string constraint solving techniques which provide completeness guarantees, 
%Distinguished features of 
It is simple, generic, and, most importantly, amenable to implementation, which 
%We implement the decision procedure on top of OSTRICH, resulting into a new solver called 
is built on top of a recent solver OSTRICH,
%, which focuses on string constraints without integer data type, 
giving rise to OSTRICH+.
%, which is based on a recent solver OSTRICH which could not tackle integer data types. 
We evaluate the performance of OSTRICH+ on  a wide range of existing and new benchmarks. The experimental results show that OSTRICH+ is the first string decision procedure capable of tackling finite transducers and integer constraints, whilst
%analysing properties (e.g. idempotence) of HTML sanitizers (e.g. htmlEscape) 
its performance is comparable with some of the best state-of-the-art string constraint solvers.
