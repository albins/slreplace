%!TEX root = popl2018.tex

\section{Preliminaries}

\subsection*{General notations} 
$\Nat$ stands for the set of natural numbers. For $k \in \Nat$, let $[k] = \{1,\dots, k\}$. For a vector $\vec{x}=(x_1,\dots, x_n)$, let $|\vec{x}|$ denote the length of $\vec{x}$ (i.e., $n$), and for each $i \in [n]$, let $\vec{x}(i)$ denote $x_i$. For a vector $\vec{x} = (x_1, \dots, x_n)$, let $\red(\vec{x})$ denote $(x_{i_1},\dots, x_{i_m})$ such that for each $j \in [m]$, $x_{i_j}$ is different from all $x_1, \dots, x_{i_j-1}$.
\tl{red is for reduced?}
The term DAG stands for \emph{directed acylic graphs}.

\subsection*{Regular language}
Fix a finite alphabet $\Sigma$. Elements in $\Sigma^*$ are called words or strings interchangeably. $\Sigma^+$ comprises words that are not empty. For a finite word $u \in \Sigma^+$, let $|u|$ denote the length of $u$, in addition, for $i \in [|u|]$, let $u[i]$ the $i$-th letter of $u$.

\begin{definition}[Regular expressions $\regexp$]
	\[e \eqdef a \mid e + e \mid e \concat e \mid e^\ast, \mbox{ where } a \in \Sigma. \]
	We also use the abbreviations $\Sigma \equiv \cup_{a \in \Sigma}\ a$ and $\Sigma^\ast \equiv (\cup_{a \in \Sigma}\ a)^\ast$.
\end{definition}

We use $\Ll(e)$ to denote the set of strings that match $e$.

%%%%%%%%%%%%%%%%%%%%%%%%%%%%%%%%%%%%%%%%%%%%
\hide{
	A regular expression $e$ is said to be \emph{bounded} if $e$ is defined by the rules, $e \eqdef a \mid u^\ast \mid e \concat e$, where $u$ is a constant string. For instance, $a (bc)^\ast$ is bounded, while $(ab^\ast)^\ast$ is not. A regular expression $e$ is a union of bounded regular expressions if $e = e_1 + \dots + e_k$ such that each $e_i$ is a bounded regular expression. We use UBR to abbreviate the set of regular expressions which are a union of bounded regular expressions.
}
%%%%%%%%%%%%%%%%%%%%%%%%%%%%%%%%%%%%%%%%%%%%

A nondeterministic finite automata (NFA) $\cA$ on $\Sigma$ is a tuple $(Q, \delta, q_0, F)$, where $Q$ is a finite set of states, $q_0 \in Q$ is the initial state, $F \subseteq Q$ is the set of final states, and $\delta \subseteq Q \times \Sigma \times Q$ is the transition relation. An NFA $\cA$ is deterministic if for each $(q, \sigma) \in Q \times \Sigma$, there is at most one $q' \in Q$ such that $(q, a, q') \in \delta$. An NFA $\cA$ is complete if for each $(q, \sigma) \in Q \times \Sigma$, there is at least one $q' \in Q$ such that $(q, a, q') \in \delta$. We assume that all NFA considered in this paper are complete. For a string $w = \sigma_1 \dots \sigma_n$, a run of $\cA$ on $w$ is a sequence $q_0 \dots q_n$ such that for each $i \in [n]$, $(q_{i-1}, \sigma_i, q_i) \in \delta$. A run $q_0 \dots q_n$ is accepting if $q_n \in F$. A string $w$ is accepted by $\cA$ if there is an accepting run of $\cA$ on $w$. An NFA $\cA$ is \emph{unambiguous} if for each word $w$, there is \emph{at most one accepting} run of $\cA$ on $w$.
We also use the notation $q_1 \xrightarrow[\cA]{w} q_{n+1}$ to denote the fact that there are $q_2,\dots, q_n \in Q$ such that for each $i \in [n]$, $(q_i, \sigma_i, q_{i+1}) \in \delta$.  

\subsection*{Computational complexity}
In this paper, we study not only decidability but also the complexity of string logics. 

%=====================================================================================================

\section{The core constraint language}

In this section, we define a general string constraint language that supports concatenations and the replaceall function.  

An alphabet $\Sigma$ is fixed.

We consider the following data types: String data type $\str$, Integer data type $\intnum$, and Array of strings $\str [\ ]$.


We assume a countably set of variables, of data types form $\dtypes$. We will use $x, y, z, \dots$ to denote the variables of data type $\str$, and $n, n', \dots$ to denote the variables of type $\intnum$. In addition, we use $X, Y, Z, \dots$ to denote the variables of data type $\str[\ ]$.
We use $u, v, w, \dots$ to denote the constant strings, and $c, c',\dots$ to denote the constant integers.


\subsection{Semantics of the function $\replaceall$}


The semantics of $\replaceall(x, e, y)$, where $\Ll(e) \subseteq \Sigma^+$, is defined inductively as follows:
\begin{itemize}
	\item if $x \not \in \Ll(\Sigma^\ast e \Sigma^\ast)$, then $\replaceall(x, e, y) = x$, 
	%
	\item otherwise, let $x = x' u z$ such that $u$ is the \emph{first and longest} match of $e$ in $x$, more specifically, 1) $u \in \Ll(e)$, and $x' u' \not \in  \Ll(\Sigma^\ast e \Sigma^\ast)$ for every strict prefix $u'$ of $u$, 2) for every nonempty prefix $z'$ of $z$, $u z' \not \in \Ll(e)$, then $\replaceall(x, e, y) = x' y \cdot \replaceall(z, e, y)$.
\end{itemize}
For instance, let $e = a^+$, then $\replaceall(aaabaac, a^+, y)=ybyc$.

\begin{remark}
Follows the POSIX standard. Different from Javascript.
\end{remark}

\tl{Anthony mentioned that there might be alternatives, for instance,  $\replaceall$ can be made nondeterministic ($\replaceall(aaaaa, aaa, y) = \{yaa, aya, aay\}$). This MIGHT be considered as well.}


%%%%%%%%%%%%%%%%%%%%%%%%%%%%%%%%%%%%%%%%%%%%%%
%%%%%%%%%%%%%%%%%%%%%%%%%%%%%%%%%%%%%%%%%%%%%%
\hide{
The semantics of $\replaceall(x, u, y)$ is defined inductively as follows: let $u = u_1 \dots u_k$ (where $k \ge 1$),
\begin{itemize}
	\item if $x \not \in \Sigma^\ast u \Sigma^\ast$, then $\replaceall(x, u, y) = x$, 
	%
	\item otherwise, let $x = x' u z$ such that $x' u_1 \dots u_{k-1} \not \in \Sigma^\ast u \Sigma^\ast$, then $\replaceall(x, u, y) = x' v \cdot \replaceall(z, u, y)$.
\end{itemize}
For instance, $\replaceall(aaaaa, aaa, y) = yaa$, and $\replaceall("Jeve", e, a) = Java$.

\tl{Anthony mentioned that there might be alternatives, for instance,  $\replaceall$ can be made nondeterministic ($\replaceall(aaaaa, aaa, y) = \{yaa, aya, aay\}$). This MIGHT be considered as well.}
}
%%%%%%%%%%%%%%%%%%%%%%%%%%%%%%%%%%%%%%%%%%%%%%
%%%%%%%%%%%%%%%%%%%%%%%%%%%%%%%%%%%%%%%%%%%%%%
\hide{
	\smallskip
	
	\noindent The semantics of $split(x, u)$ is defined inductively as follows:
	\begin{itemize}
		\item if $x \not \in \Sigma^\ast u \Sigma^\ast$, then $split(x,u) = x$, 
		%
		\item otherwise, let $x = y u z$ such that $y \not \in \Sigma^\ast u \Sigma^\ast$, then $split(x, u) = concat([y], split(z, u))$, where $[y]$ is the array comprising one element $y$ and $concat$ operation concatenates two arrays into one.  
	\end{itemize} 
	
	For instance, $split("Java", a) = [``J", ``v", ``"]$.
	
	\smallskip
	
	\noindent The semantics of $join(X)$ is defined inductively as follows: 
	\begin{itemize}
		\item if $X$ is the empty array, then $join(X) = ``''$,  
		\item otherwise, let $X= concat([x], Y)$, then $join(X) = x\ \cdot\ ``," \ \cdot\ join(Y)$.
	\end{itemize}
	For instance, $join([``Tom", ``John", ``Henry"]) =``Tom,John,Henry"$.
}
%%%%%%%%%%%%%%%%%%%%%%%%%%%%%%%%%%%%%%%%%%%%%%
%%%%%%%%%%%%%%%%%%%%%%%%%%%%%%%%%%%%%%%%%%%%%%

\subsection{The straight-line string fragment}


\begin{definition}[Relational and regular constraints with $\replaceall$ function]
	Relational constraints and regular constraints are defined by the following rules,
	\[
	\begin{array}{r c l cr}
	s &\eqdef & x \mid u & \ \ & \mbox{(string terms)}\\
	%t &\eqdef & s \mid e & \ \ & \mbox{(terms)}\\
	\varphi &\eqdef & x = s \concat \dots \concat s  \mid  x = \replaceall(s, e, s) \mid \varphi \wedge \varphi & \ \ & \mbox{(relational constraints)}\\
	\psi & \eqdef & x \in e \mid \psi \wedge \psi \mid \psi \vee \psi \mid \neg \psi   & \ \ & \mbox{(regular constraints)} \\
	\end{array}
	\]
	where $u \in \Sigma^\ast$ and $e$ is a regular expression. 
	%A regular constraint $\psi$ is a UBR constraint if for each atom $x \in e$ occurring in $\psi$, $e$ is in UBR.
\end{definition}

\begin{remark}\label{rem-concat}
The concatenation modality $x = s \concat \dots \concat s$ is redundant in the sense that $z = x \concat y$ can be rewritten as $z' = \replaceall(ab, a, x), z = \replaceall(z', b, y)$, where $a,b$ are two fresh letters.
\end{remark}


For a formula $\varphi$, let $\vars(\varphi)$ denote the set of variables occurring in $\varphi$. Similarly for $\vars(\psi)$.


\tl{can the regular constraints be simplified to just a conjunction of $x\in e$?}

Given a relational constraint $\varphi$, a variable $x$ (of type $\str$) is called a \emph{source variable} of $\varphi$ if $\varphi$ \emph{does not} contain a conjunct of the form $x = s_1 \concat \dots \concat s_n$ or $x = \replaceall(\dots)$.

\tl{note that here the pattern $u$ is a constant, but we may also introduce a variable here, though its decidability is not clear.} 

%%%%%%%%%%%%%%%%%%%%%%%%%%%%%%%%%%%%%%%%%%
\hide{
	\begin{definition}[Dependency graph]
		Let $\varphi$ be a relational constraint with $\replaceall$ function. Then the dependency graph of $\varphi$, denoted by $\cG_\varphi = (\vars(\varphi), E_\varphi)$, where $E_\varphi$ comprises the following edges,
		\begin{itemize}
			\item for each atomic formula $y= s_1 \concat \dots \concat s_n$ and each $i \in [n]$ such that $s_i$ is a variable, $(s_i, y) \in E_\varphi$,
			\item for each atomic formula $x = \replaceall(y, u, s)$, $(y, x) \in E_\varphi$, in addition, if $s$ is a variable, then $(s, x) \in E_\varphi$, $(s, y) \in E_\varphi$, 
		\end{itemize}
	\end{definition}
}
%%%%%%%%%%%%%%%%%%%%%%%%%%%%%%%%%%%%%%%%%%%

\begin{definition}[Straight-line relational constraints with $\replaceall$ function]
	A relational constraint $ \varphi$ with $\replaceall$ function is straight-line, if $\varphi \eqdef \bigwedge \limits_{1 \le i \le m} x_i = P_i$ such that
	\begin{itemize}
		\item $x_1,\dots, x_m$ are mutually distinct,
		\item for each $i \in [m]$, all the variables in $P_i$ are either source variables, or variables from $\{x_1,\dots, x_{i-1}\}$,
	\end{itemize}
\end{definition}
Intuitively, in a straight-line relational constraint, the dependency graph of the string variables is acyclic.


\begin{definition}[Straight-line string constraints with $\replaceall$ function]
	A straight-line string constraint $C$ with $\replaceall$ function (denoted by $\strline[\replaceall]$)  is defined as $ \varphi \wedge \psi$,  where 
	\begin{itemize}
		\item $\varphi$ is a straight-line relational constraint,  and
		%
		\item $\psi$ is a regular constraint.
		%
	\end{itemize}
	%Let us use $\Cc$ to denote the set of straight-line string constraints with $\replaceall$ function.
\end{definition}

By Remark~\ref{rem-concat}, without loss of generality, we assume that 
in each $\strline[\replaceall]$ constraint $\varphi \wedge \psi$, the concatenation symbol $\concat$ does not occur in $\varphi$. 

%Let us use $\pstrline[\replaceall]$ to denote the set of pure $\strline[\replaceall]$ constraints.

%\medskip

\paragraph*{Satisfiability problem.} Given an $\strline[\replaceall]$ constraint $C$, decide whether $C$ is satisfiable.

\smallskip

The questions we plan to consider is summarized in the following table. Note that for $x=\replaceall (y,u,z)$, $y$ is referred to as a \emph{pattern} and $z$ is referred to as a \emph{replacement}.

\[
\begin{tabular}{c|c|c|c}
pattern (e)  &   replacement (z)        & complexity \\
\hline
constant string  &   constant                       & PSPACE-c (\cite{LB16})     \\
\hline
constant string  &   variable                       & PSPACE-c (Section \ref{sec:replaceallsl})       \\
\hline
regular expression  &   constant                       &    PSPACE-c (\cite{LB16}) ?      \\
\hline
regular expression  &   variable                       &           \\
\end{tabular}
\]